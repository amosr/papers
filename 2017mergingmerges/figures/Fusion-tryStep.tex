%!TEX root = ../Main.tex


% Settle on a syntax for this later.
% Might be too many nested pairs.
\newcommand\nextStep[5]{\big((#1,~#2),~(#3,~#4),~#5 \big)}


\begin{figure*}
\begin{tabbing}
M \= M \= M \= M \kill
$\ti{tryStep} ~:~ \ChanTypeMap \to \Label_1 \to \Instr \to \Label_1 \to \Maybe~\Instr$ \\
$\ti{tryStep} ~\cs~(l_p,s_p)~i_p~(l_q,s_q)~=~@match@~i_p~@with@$ \\

\\
\> $@jump@~(l',u')$ \\
\> \> \note{LocalJump} \\
\> \> $\to~@jump@~
      \nextStep
        {l'}{s_p}
        {l_q}{s_q}
        {u'}
      $ \\

\\


\> $@case@~e~(l'_t,u'_t)~(l'_f,u'_f)$ \\
\> \> \note{LocalCase} \\
\> \> $\to~@case@~e~
      \nextStep
        {l'_t}{s_p}
        {l_q}{s_q}
        {u'_t}
      ~
      \nextStep
        {l'_f}{s_p}
        {l_q}{s_q}
        {u'_f}
      $ \\

\\

\> $@push@~c~e~(l',u')$ \\
\> \> \note{LocalPush}\\
\> \> $~|~\cs[c]=@out1@$ \\
\> \> $\to~@push@~c~e~
      \nextStep
        {l'}
          {s_p}
        {l_q}
          {s_q}
        {u'}
      $ \\

\> \> \note{SharedPush}\\
\> \> $~|~\cs[c]=@in1out1@ ~\wedge~ s_q[c]=@none@_F$ \\
\> \> $\to~@push@~c~e~
      \nextStep
        {l'}
          {s_p}
        {l_q}
          {\HeapUpdateOne{c}{@pending@_F}{s_q}}
        {\HeapUpdateOne{@chan@~c}{e}{u'}}
      $ \\

\\

\> $@pull@~c~x~(l',u')$ \\
\> \> \note{LocalPull}\\
\> \> $~|~\cs[c]=@in1@$ \\
\> \> $\to~@pull@~c~x~
      \nextStep
        {l'}{s_p}
        {l_q}{s_q}
        {u'}
    $ \\

\> \> \note{SharedPull} \\
\> \> $~|~(\cs[c]=@in2@ \vee \cs[c]=@in1out1@) ~\wedge~ s_p[c]=@pending@_F$ \\
\> \> $\to~@jump@~
      \nextStep
        {l'}
          {\HeapUpdateOne{c}{@have@_F}{s_p}}
        {l_q}
          {s_q}
        {\HeapUpdateOne{x}{@chan@~c}{u'}}
        $ \\

\> \> \note{SharedPullInject}\\
\> \> $~|~\cs[c]=@in2@ ~\wedge~ s_p[c]=@none@_F ~\wedge~ s_q[c]=@none@_F$ \\
\> \> $\to~@pull@~c~(@chan@~c)~
      \nextStep
        {l_p}
          {\HeapUpdateOne{c}{@pending@_F}{s_p}}
        {l_q}
          {\HeapUpdateOne{c}{@pending@_F}{s_q}}
        {[]}
  $ \\
\\

\> $@drop@~c~(l',u')$ \\
\> \> \note{LocalDrop} \\
\> \> $~|~\cs[c]=@in1@$ \\
\> \> $\to~@drop@~c~
      \nextStep
        {l'}
          {s_p}
        {l_q}
          {s_q}
        {u'}
      $ \\

\> \> \note{ConnectedDrop}\\
\> \> $~|~\cs[c]=@in1out1@$ \\
\> \> $\to~@jump@~
      \nextStep
        {l'}
          {\HeapUpdateOne{c}{@none@_F}{s_p}}
        {l_q}
          {s_q}
        {u'}
      $ \\

\> \> \note{SharedDropOne}\\
\> \> $~|~\cs[c]=@in2@ ~\wedge~ (s_q[c]=@have@_F \vee s_q[c]=@pending@_F)$ \\
\> \> $\to~@jump@~
      \nextStep
        {l'}
          {\HeapUpdateOne{c}{@none@_F}{s_p}}
        {l_q}
          {s_q}
        {u'}
      $ \\


\> \> \note{SharedDropBoth}\\
\> \> $~|~\cs[c]=@in2@ ~\wedge~ s_q[c]=@none@_F$ \\
\> \> $\to~@drop@~c~
      \nextStep
        {l'}
          {\HeapUpdateOne{c}{@none@_F}{s_p}}
        {l_q}
          {s_q}
        {u'}
      $
\end{tabbing}

\caption{Fusion step for a single process of the pair.} 

% Given the state of both processes, compute the instruction this process can perform. This is analogous to statically evaluating the pair of processes. If this process cannot execute, the other process may still be able to.
\label{fig:Fusion:Def:Step}
\end{figure*}

