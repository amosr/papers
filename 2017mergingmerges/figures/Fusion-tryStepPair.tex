%!TEX root = ../Main.tex

% Settle on a syntax for this later.
% Might be too many nested pairs.
\newcommand\nextStep[5]{\big((#1,~#2),~(#3,~#4),~#5 \big)}


% I tried this colour in a colour blindness simulator and it seems to be OK.
% Should still be readable when converted to grayscale.
\definecolor{notec}{HTML}{C03020}
\newcommand\note[1]{\textcolor{notec}{(#1)}}

\begin{figure}
\begin{tabbing}
M \= M \= M \= M \kill
$\ti{tryStepPair} ~:~ \ChanTypeMap \to \Label_1 \to \Instr \to \Label_1 \to \Instr \to \Maybe~\Instr$ \\
$\ti{tryStepPair} ~\cs~l_p~i_p~l_q~i_q$ \\

\> \note{PreferJump1} \\
\> $~|~i_p'~\in~\ti{tryStep}~\cs~l_p~i_p~l_q ~\wedge~@jump@~(l,u)~\in~i_p'$ 
   ~~~~~ $\to~i_p'$ \\
\> \note{PreferJump2} \\
\> $~|~i_q'~\in~\ti{tryStep}~\cs~l_q~i_q~l_p ~\wedge~@jump@~(l,u)~\in~i_q'$
   ~~~~~ $\to~\ti{swaplabels}~i_q'$ 
\\[0.5em]

\> \note{DeferPull1} \\
\> $~|~i_p'~\in~\ti{tryStep}~\cs~l_p~i_p~l_q ~\wedge~ i_q'~\in~\ti{tryStep}~\cs~l_q~i_q~l_p$ 
   ~~ $\wedge~@pull@~c~x~(l,u)~\not\in~i_p'$ 
   ~~~~~ $\to~i_p'$ \\
\> \note{DeferPull2} \\
\> $~|~i_p'~\in~\ti{tryStep}~\cs~l_p~i_p~l_q ~\wedge~i_q'~\in~\ti{tryStep}~\cs~l_q~i_q~l_p$
   ~~ $\wedge~@pull@~c~x~(l,u)~\not\in~i_q'$ 
   ~~~~~ $\to~\ti{swaplabels}~i_q'$ 
\\[0.5em]

\> \note{Run1} \\
\> $~|~i_p'~\in~\ti{tryStep}~\cs~l_p~i_p~l_q$ ~~~~~ $\to~i_p'$ \\
\> \note{Run2} \\
\> $~|~i_q'~\in~\ti{tryStep}~\cs~l_q~i_q~l_p$ ~~~~~ $\to~\ti{swaplabels}~i_q'$
\end{tabbing}
\caption{Fusion step coordination for a pair of processes.}
% Statically compute the instruction to perform at a particular fused label. Try to execute either process, preferring jumps and other instructions over pulling, as pulling can block while other instructions may perform ``useful work'' without blocking. If neither machine can execute, fusion fails.
\label{fig:Fusion:Def:StepPair}
\end{figure}

