%!TEX root = ../Main.tex

\begin{figure}

$$
\arrLR{
  \boxed{\ProcInject{\Proc}{\Chan}{\Value}{\Proc}}
}{
  \boxed{\ProcsInject{\sgl{\Proc}}{\Chan}{\Value}{\sgl{\Proc}}}
}
$$

$$
\ruleIN{
  c=@none@ \in @ins@~p
}{
  \ProcInject{p}{c}{v}{\HeapUpdateOne{c}{@pending@~v}{@ins@~p}}
}{InjectValue}
\ruleIN{
  c \not\in @ins@~p
}{
  \ProcInject{p}{c}{v}{p}
}{InjectIgnore}
$$

$$
\ruleIN{
  \forall i.~ \ProcInject{p_i}{c}{v}{p'_i}
}{
  \ProcsInject{\sgl{p_i}}{c}{v}{\sgl{p'_i}}
}{ProcessesInject}
$$

\caption{Evaluation: values are injected into a channel's input buffer which makes them available to pull.
Injecting requires the input buffer to be empty or the process to ignore that channel.
For networks, all processes must successfully inject the value.}
\label{fig:Process:Eval:Inject}
\end{figure}


\begin{figure}

$$
\alpha~@:=@~ \Push~\Chan~\Value ~|~ \tau
$$

$$
  \boxed{
    \ProcBlockShake
      {\Instr}{\MapType{\Chan}{\InputState}}{\Sigma}
      {\alpha}
      {\Label}{\MapType{\Chan}{\InputState}}{\MapType{\Var}{\Exp}}
  }
$$


$$
\ruleIN{
  c=@pending@~v \in i
}{
  \ProcBlockShake{@pull@~c~x~(l,u)}{i}{\Sigma}{\tau}{l}{\HeapUpdateOne{c}{@have@}{i}}{u,~x = v}
}{Pull}
\ruleIN{
  c=@have@ \in i
}{
  \ProcBlockShake{@drop@~c~(l,u)}{i}{\Sigma}{\tau}{l}{\HeapUpdateOne{c}{@none@}{i}}{u}
}{Drop}
$$

$$
\ruleIN{
  \ExpEval{\Sigma}{e}{v}
}{
  \ProcBlockShake{@push@~c~e~(l,u)}{i}{\Sigma}{\Push~c~v}{l}{i}{u}
}{Push}
\ruleIN{
}{
  \ProcBlockShake{@jump@~(l,u)}{i}{\Sigma}{\tau}{l}{i}{u}
}{Jump}
$$

$$
\ruleIN{
  \ExpEval{\Sigma}{e}{@true@}
}{
  \ProcBlockShake{@case@~e~(l_t,u_t)~(l_f,u_f)}{i}{\Sigma}{\tau}{l_t}{i}{u_t}
}{CaseT}
\ruleIN{
  \ExpEval{\Sigma}{e}{@false@}
}{
  \ProcBlockShake{@case@~e~(l_t,u_t)~(l_f,u_f)}{i}{\Sigma}{\tau}{l_f}{i}{u_f}
}{CaseF}
$$

$$
  \boxed{\ProcShake{\Proc}{\alpha}{\Proc}}
  \quad
  \boxed{\ProcsShake{\sgl{\Proc}}{\alpha}{\sgl{\Proc}}}
$$

$$
@let@~@block@~p~=~@blocks@~p~(@label@~p)
$$
$$
\ruleIN{
  \ProcBlockShake
    {@block@~p} {@ins@~p}{@heap@~p}
    {\alpha}
    {l}{i}{u}
  \quad
    \ExpEval{@heap@~p}{u}{\Sigma}
}{
  \ProcShake{p}{\alpha}{p~\sgl{@label@~=~l,~@heap@~=~(\HeapUpdates{\Sigma}{@heap@~p}),~@ins@~=~i}}
}{Shake}
$$




$$
\ruleIN{
  \ProcShake{p_i}{\tau}{p'_i}
}{
  \ProcsShake{
    \sgl{p_0 \ldots p_i \ldots p_n}
  }{\tau}{
    \sgl{p_0 \ldots p'_i \ldots p_n}
  }
}{ProcessesInternal}
$$

$$
\ruleIN{
  \ProcShake{p_i}{\Push~c~v}{p'_i}
  \quad
  \forall j~|~j \neq i.~
  \ProcInject{p_j}{c}{v}{p'_j}
}{
  \ProcsShake{
    \sgl{p_0 \ldots p_i \ldots p_n}
  }{\Push~c~v}{
    \sgl{p'_0 \ldots p'_i \ldots p'_n}
  }
}{ProcessesPush}
$$


\caption{Evaluation: shaking allows proceses to take a step from one label to another as well as produce an output message.
If the message is a push, the value is injected to all other processes in the network; otherwise it is an internal step.}
\label{fig:Process:Eval:Shake}
\end{figure}

