%!TEX root = ../Main.tex

\clearpage{}
\section{Process definitions}

%% This figure is referenced below, in 'Process definitions', but putting it up here before the big fused process stops the fused one from splitting across multiple pages.
%!TEX root = ../Main.tex

\begin{figure}
\begin{minipage}[t]{0.4\textwidth}
\begin{tabbing}
\Instr \TABDEF @MMMM@  \TABSKIP $\Exp$ \TABSKIP $\Exp$ \TABSKIP $\Exp$ \kill

\Exp,~$e$ \> $\to$ \> $x~|~v~|~e~e $ \\
  \> $\enskip|~$ \> $ (e~@||@~e) ~|~ e+e ~|~ e~@/=@~e ~|~ e < e$ \\
\Value,~$v$ \> $\to$ \> $\mathbb{N}~|~\mathbb{B}~|~(\lambda{}x.~e)$ \\
$\Sigma$ \> $\to$ \> $\cdot~|~\Sigma,~x~=~v$ \\
\\

\Proc \>:=\> @process@ \\
M \= M \= \kill
\> \> @ins:   @  $\MapType{\Chan}{\InputState}$ \\
\> \> @outs:  @  $\sgl{\Chan}$ \\
\> \> @heap:  @  $\Sigma$ \\
\> \> @label: @  \Label \\
\> \> @instrs:@  $\MapType{Label}{\Instr}$ \\
\\
\Instr \TABDEF \kill
\InputState \> := \> @pending@~\Value~$|$~@have@~$|$~@none@

\end{tabbing}
\end{minipage}
\begin{minipage}[t]{0.05\textwidth}
\quad
\end{minipage}
\begin{minipage}[t]{0.4\textwidth}
\begin{tabbing}
\Instr \TABDEF @MMMM@  \TABSKIP $\Chan$ \TABSKIP $\Chan$ \TABSKIP $\Exp$ \kill

\Var,~$x$ \> $\to$ \> (value variable) \\
\Chan,~$c$ \> $\to$ \> (channel/stream name) \\
\Label,~$l$ \> $\to$ \> (label name) \\
\\
\\

\Instr
    \> :=\> @pull@  \> \Chan  \> \Var  \> \Next \\
    \TABALT @drop@  \> \Chan  \>       \> \Next \\
    \TABALT @push@  \> \Chan  \> \Exp  \> \Next \\
    \TABALT @case@  \> \Exp   \> \Next \> \Next \\
    \TABALT @jump@  \>        \>       \> \Next \\
\\
\\
\Next \> := \> $\Label~\times~\MapType{\Var}{\Exp}$ \\
\end{tabbing}
\end{minipage}

\caption{Grammar and types for defining processes. Each process is a sequential stream computation, networks of which are evaluated concurrently with a single-element bounded buffer for each stream.}
\label{fig:Process:Def}
\end{figure}




The grammar for processes is shown in Fig.~\ref{fig:Process:Def}.
Channels, labels and variables are specified by some external, globally unique set of names.
For values and expressions we use an untyped lambda calculus with a few primitives chosen to facilitate the examples.

We refer to the \emph{endpoint} of a stream as a channel.
A stream may flow through many input channels on different processes, but can only be produced from at most one output channel.

Each input channel is paired with an input state, which can be either @none@ (no value), @have@ (a value has been pulled), or ($@pending@~\Value$) (a value is available to be pulled).
These input states are used for evaluation to ensure that communication between processes does not require unbounded buffers.
These should be initialised to @none@ for a new process.
Input states are explained in detail later in \S\ref{s:Process:Eval}.

The output streams are in some sense ``owned'' by the process that produces them: while a stream may be consumed by any number of processes, each stream can only appear as the output for one process.
This ensures a sort of determinism in the scheduling of multiple processes; if different processes could push to the same stream, the order of values would depend on the scheduled order.
A process may, however, produce multiple output streams.

The heap is used for evaluation of expressions.
Each process has its own private heap, therefore the only communication between processes occurs by streams.

The instructions (@instrs@) are a mapping from label to instruction, and label points to the current instruction.
Instructions can pull from a channel, drop an already pulled value, push a value, perform an if/case analysis on a boolean, or perform an internal jump.

As usual with Kahn processes~\cite{kahn1976coroutines}, pulling from a channel is blocking.
Unlike normal Kahn processes however, pushing to a channel can also block: each consumer has a single-element buffer and pushing only succeeds when all buffers are empty.
After values have been pulled, they must be disposed of with @drop@: this empties the value from the buffer and allows the producer to push to the channel.

All instructions take as an argument the next label to jump to, as well as any variable updates that should be performed on the private heap at the same time.

A process network is a set of multiple processes that can be evaluated concurrently.
Any inputs that are not produced as outputs of processes are assumed to be external inputs --- their values will be provided by the environment.
Processes form the essence of stream computation, and a single process can be given a straightforward sequential semantics by mapping to an imperative language.
By fusing multiple processes into a single one, we are effectively giving a sequential interpretation for concurrent processes.


% \subsection{Map/map}
% \label{s:Process:MapMap}
% 
% One of the simplest combinators is @map@.
% This might need to go elsewhere.
% As well as needing a better example than map/map.
% Let's start with the process definition for @map@.
% The inputs here is actually a map with @as=none@, but we leave the value off when it is @none@.
% The next labels for the instructions are also shortened as @map0@ instead of writing the empty heap update afterwards.
% Mention that the initial heap has the names but no values, but they could be initialised to whatever.
% It doesn't matter since they'll be written before anything is read.
% 
% \begin{code}
% map f = process (map f)
%      ins: as
%     outs: bs
%     heap: {a = 0}
%    label: map0
%   blocks: map0 = pull as    a  map1
%           map1 = push bs (f a) map2
%           map2 = drop as       map0
% \end{code}
% 
% As well as a ``combinator network'', a function comprised exclusively of process combinators.
% The input streams are supplied as arguments, and output streams as return values.
% \begin{code}
% mapMap f g xs
%  = let ys = map f xs
%        zs = map g ys
%    in  zs
% \end{code}
% 
% It is not hard to assume that, given the process definitions and a combinator network, we can produce a process network.
% This is simple enough for a paragraph prose description.
% It's just a bit of inlining and renaming everything to be unique.
% 
% \begin{code}
% process (map f)
%      ins: xs
%     outs: ys
%     heap: {x}
%    label: p0
%   blocks: p0 = pull xs    x  p1
%           p1 = push ys (f x) p2
%           p2 = drop xs       p0
% process (map g)
%      ins: ys
%     outs: zs
%     heap: {y}
%    label: q0
%   blocks: q0 = pull ys    y  q1
%           q1 = push zs (g y) q2
%           q2 = drop ys       q0
% \end{code}
% 
% Now we can perform some kind of fusion on this network, resulting in one process that computes, as output, both @ys@ and @zs@.
% Later, when producing imperative code for this, the output pushes to @ys@ can be ignored and changed to jumps, as the original combinator network did not return them.
% 
% \begin{code}
% process (map f / map g)
%      ins: xs
%     outs: ys zs
%     heap: {x, y, _ys}
%    label: p0q0
%   blocks: p0q0            = pull xs    x  p1q0
%           p1q0            = push ys (f x) p2q0-pending-ys { _ys = f x }
%           p2q0-pending-ys = drop xs       p0q0-pending-ys
%           p0q0-pending-ys = jump          p0q1-have-ys    { y = _ys }
%           p0q1-have-ys    = push zs (g y) p0q2-have-ys
%           p0q2-have-ys    = jump          p0q0
% \end{code}


% -----------------------------------------------------------------------------
\subsection{Execution}
\label{s:Process:Eval}

%!TEX root = ../Main.tex

\begin{figure}

$$
\arrLR{
  \boxed{\ProcInject{\Proc}{\Chan}{\Value}{\Proc}}
}{
  \boxed{\ProcsInject{\sgl{\Proc}}{\Chan}{\Value}{\sgl{\Proc}}}
}
$$

$$
\ruleIN{
  c=@none@ \in @ins@~p
}{
  \ProcInject{p}{c}{v}{p ~@ins@~ \{ c = @pending@~v\}}
}{InjectValue}
\ruleIN{
  c \not\in @ins@~p
}{
  \ProcInject{p}{c}{v}{p}
}{InjectIgnore}
$$

$$
\ruleIN{
  \forall i.~ \ProcInject{p_i}{c}{v}{p'_i}
}{
  \ProcsInject{\sgl{p_i}}{c}{v}{\sgl{p'_i}}
}{ProcessesInject}
$$

\caption{Process evaluation: inject}
\label{fig:Process:Eval:Inject}
\end{figure}


\begin{figure}

$$
\alpha~@:=@~ \Push~\Chan~\Value ~|~ \tau
$$

$$
  \boxed{
    \ProcBlockShake
      {\Instr}{\MapType{\Chan}{\InputState}}{\Sigma}
      {\alpha}
      {\Label}{\MapType{\Chan}{\InputState}}{\phi}
  }
$$


$$
\ruleIN{
  c=@pending@~v \in i
}{
  \ProcBlockShake{@pull@~c~x~l[\phi]}{i}{\Sigma}{\tau}{l}{i[c=@have@]}{\phi,~x = v}
}{Pull}
\ruleIN{
  c=@have@ \in i
}{
  \ProcBlockShake{@drop@~c~l[\phi]}{i}{\Sigma}{\tau}{l}{i[c=@none@]}{\phi}
}{Drop}
$$

$$
\ruleIN{
  \ExpEval{\Sigma}{e}{v}
}{
  \ProcBlockShake{@push@~c~e~l[\phi]}{i}{\Sigma}{\Push~c~v}{l}{i}{\phi}
}{Push}
\ruleIN{
}{
  \ProcBlockShake{@jump@~l[\phi]}{i}{\Sigma}{\tau}{l}{i}{\phi}
}{Jump}
$$

$$
\ruleIN{
  \ExpEval{\Sigma}{e}{@true@}
}{
  \ProcBlockShake{@case@~e~l_t[\phi_t]~l_f[\phi_f]}{i}{\Sigma}{\tau}{l_t}{i}{\phi_t}
}{CaseT}
\ruleIN{
  \ExpEval{\Sigma}{e}{@false@}
}{
  \ProcBlockShake{@case@~e~l_t[\phi_t]~l_f[\phi_f]}{i}{\Sigma}{\tau}{l_f}{i}{\phi_f}
}{CaseF}
$$

$$
  \boxed{\ProcShake{\Proc}{\alpha}{\Proc}}
  \quad
  \boxed{\ProcsShake{\sgl{\Proc}}{\alpha}{\sgl{\Proc}}}
$$

$$
@let@~@block@~p~=~@blocks@~p~(@label@~p)
$$
$$
\ruleIN{
  \ProcBlockShake
    {@block@~p} {@ins@~p}{@heap@~p}
    {\alpha}
    {l}{i}{\phi}
  \quad
  \forall u~|~x_u=e_u \in \phi.~
    \ExpEval{@heap@~p}{e_u}{v_u}
}{
  \ProcShake{p}{\alpha}{p~@label@~=~l,~@heap@~=~@heap@[\sgl{x_u=v_u}],~@ins@~=~i}
}{Shake}
$$




$$
\ruleIN{
  \ProcShake{p_i}{\tau}{p'_i}
}{
  \ProcsShake{
    \sgl{p_0 \ldots p_i \ldots p_n}
  }{\tau}{
    \sgl{p_0 \ldots p'_i \ldots p_n}
  }
}{ProcessesInternal}
$$

$$
\ruleIN{
  \ProcShake{p_i}{\Push~c~v}{p'_i}
  \quad
  \forall j~|~j \neq i.~
  \ProcInject{p_j}{c}{v}{p'_j}
}{
  \ProcsShake{
    \sgl{p_0 \ldots p_i \ldots p_n}
  }{\Push~c~v}{
    \sgl{p'_0 \ldots p'_i \ldots p'_n}
  }
}{ProcessesPush}
$$


\caption{Process evaluation: shake}
\label{fig:Process:Eval:Shake}
\end{figure}


%!TEX root = ../Main.tex

\begin{figure}

% ---------------------------------------------------------
$$
  \boxed{\ProcsShake{\sgl{\Proc}}{\Action}{\sgl{\Proc}}}
$$

$$
\ruleIN{
  \ProcShake{p_i}{\cdot}{p'_i}
}{
  \ProcsShake{
    \sgl{p_0 \ldots p_i \ldots p_n}
  }{\cdot}{
    \sgl{p_0 \ldots p'_i \ldots p_n}
  }
}{ProcessesInternal}
$$

$$
\ruleIN{
  \ProcShake{p_i}{\Push~c~v}{p'_i}
  \quad
  \forall j~|~j \neq i.~
  \ProcInject{p_j}{c}{v}{p'_j}
}{
  \ProcsShake{
    \sgl{p_0 \ldots p_i \ldots p_n}
  }{\Push~c~v}{
    \sgl{p'_0 \ldots p'_i \ldots p'_n}
  }
}{ProcessesPush}
$$


% ---------------------------------------------------------
\vspace{1em}

\newcommand\vs {\ti{vs}}
\newcommand\accs {\ti{accs}}
\newcommand\network {\ti{ps}}

$$
  \boxed{
    \ProcsFeed
      {(\Chan \mapsto \overline{Value})~}
      {\sgl{\Proc}}
      {(\Chan \mapsto \overline{Value})~}
      {\sgl{\Proc}}
  }
$$
$$
\ruleIN{
  \ProcsShake
    {ps}
    {\cdot}
    {ps'}
}{
  \ProcsFeed
    {cvs}
    {ps}
    {cvs}
    {ps'}
}{FeedInternal}
$$


% $$
% \ruleIN{
%   \forall c \in \accs.~
%   \accs~c~=~[]
% }{
%   \ProcsFeed
%     {\accs}
%     {\network}
%     {\accs}
%     {\network}
% }{FeedStart}
% $$

% $$
% \ruleIN{
%   \ProcsFeed
%     {\accs}
%     {\network}
%     {\accs'}
%     {\network'}
% \quad
%   \ProcsShake
%     {\network'}
%     {\tau}
%     {\network''}
% }{
%   \ProcsFeed
%     {\accs}
%     {\network}
%     {\accs'}
%     {\network''}
% }{FeedInternal}
% $$



% $$
% \ruleIN{
%   \ProcsFeed
%     {\accs}
%     {\network}
%     {\accs'}
%     {\network'}
% \quad
%   \ProcsShake
%     {\network'}
%     {\Push~c~v}
%     {\network''}
% }{
%   \ProcsFeed
%     {\accs}
%     {\network}
%     {c=\accs'~c \listappend [v], \accs'}
%     {\network''}
% }{FeedPush}
% $$


$$
\ruleIN{
  \ProcsShake
    {ps}
    {\Push~c~v}
    {ps'}
}{
  \ProcsFeed
    {cvs}
    {ps}
    {cvs[c \mapsto (cvs[c] \listappend v)]}
    {ps'}
}{FeedPush}
$$


% $$
% \ruleIN{
%   (\forall p \in \network.~c \not\in @outs@~p)
% \quad
%   \ProcsFeed
%     {c=\vs, \accs}
%     {\network}
%     {\accs'}
%     {\network'}
% \quad
%   \ProcsInject
%     {\network'}
%     {c}{v}
%     {\network''}
% }{
%   \ProcsFeed
%     {c=\vs \listappend [v], \accs}
%     {\network}
%     {c=\vs \listappend [v], \accs'}
%     {\network''}
% }{FeedExternal}
% $$


$$
\ruleIN{
  (\forall p \in \network.~c \not\in p[@outs@])
\quad
  \ProcsInject
    {ps}
    {c}
    {v}
    {ps'}
}{
  \ProcsFeed
    {cvs[c \mapsto ([v] \listappend vs)]}
    {ps}
    {cvs[c \mapsto vs~]}
    {ps'}
}{FeedExternal}
$$



\caption{Feeding Process Networks}
\label{fig:Process:Eval:Feed}
\end{figure}



Evaluation of processes and process networks is split into three parts:
\begin{itemize}
\item Injection, where a single value for a stream is inserted into a process' input buffer.
Injection is only possible when the process' input buffer is empty, or the process ignores that stream.
Injection for a process network proceeds by injecting the same value into all processes in the network.

\item Shaking, where a process takes a step from one label to another.
Shaking a process results in an updated process and an optional pushed value.
When a process pushes to a channel, the value is injected to other processes.
Shaking a process network chooses any process that can be shaked, and updates it in the process network.

\item Feeding, where an environment of input values are fed to the process network, and output values are collected.
Feeding alternates between injecting and shaking: injecting values from the environment when processes are ready to receive values, and repeatedly shaking the processes to collect their outputs.
\end{itemize}

Evaluation of a process network is non-deterministic, in that at any point there are many possible processes that can take a step.
However, because each process itself is deterministic and has blocking reads, overall evaluation is deterministic as per Kahn process networks.
That is: the order in which values are pushed to different output streams is not deterministic, but the order and values for a particular output stream \emph{are} deterministic.

% Note that while process network evaluation is non-deterministic and concurrent, evaluating a single process is sequential and deterministic: code generation for fused processes only needs to deal with the sequential case.

\subsubsection{Injection}
Fig.~\ref{fig:Process:Eval:Inject} defines the judgment forms and rules for injection.
Injection is used by the other parts of evaluation to \emph{inject} stream values into a process' input buffer.

A process has a single element buffer for each input, stored in its input state.
This can be either @none@, in which case the buffer is empty and a value may be injected; @pending@, where a single value has been injected to the buffer but has not been pulled yet; or @have@, where a value has already been injected and is currently being used.

The judgment form ($\ProcInject{p}{c}{v}{p'}$) means that the process $p$ has the value $v$ inserted to the input buffer for channel $c$, resulting in the updated process $p'$.

Rule (InjectValue) allows a value to be injected only when the input state is @none@, meaning the buffer is empty.
An attempt to inject a value while the buffer is @pending@ or @have@ would require an unbounded, or at least multiple element, buffer.
We use the syntax ($\HeapUpdateOne{@ins@~p}{c}{@pending@~v}$) to perform a map update, in this case removing the ($c=@none@$) from ($@ins@~p$) and then inserting ($c=@pending@~v$).
This sets the input buffer as pending, meaning that any future attempts to inject to this channel will fail, until the process pulls and then drops the value.

Rule (InjectIgnore) allows processes that do not use a particular input stream to ignore an injected input.

Rule (ProcessesInject) performs injection over a process network.
Every process in the network must have the value injected into it.
This means if multiple processes read from that stream, all input buffers for that stream must be empty.

\subsubsection{Shaking}
Fig.~\ref{fig:Process:Eval:Shake} defines the judgment forms and rules for shaking a process, allowing it to take a step, and produce an output message.
% I don't like them, but alpha and tau are standard names in process calculi. 
The type of messages is denoted by $\alpha$, which can either be ($\Push~\Chan~\Value$) for pushing a value to a particular channel, or $\tau$ for an `internal' message which only performs local state updates.


The judgment form for shaking a single instruction $\ProcBlockShake{x}{i}{\Sigma}{\alpha}{l'}{i'}{u'}$
executes an instruction $x$ with the input states $i$ and the heap $\Sigma$.
The result contains an output message $\alpha$, as well as the new label $l'$, the new input state $i'$, and the update substitution to apply to the heap $u'$.

Rule (Pull) takes an already injected value from the input buffer, which changes its state from @pending@ to @have@.
The result substitution sets the variable to the pulled value, as well as any substitutions in the \Next~ of the instruction.

Rule (Drop) changes the input buffer state from @have@ to @none@. A drop can only be executed after pull has set the input buffer to @have@.

Rule (Push) evaluates the push expression $e$ under the heap, and sends that value as output the message.

Rule (Jump) simply returns the new label and substitution.

Rules (CaseT) and (CaseF) evaluate the case expression $e$ and jump to the true or false label depending on the value.

The judgment forms for shaking processes and process networks are $\ProcShake{p}{\alpha}{p'}$ and $\ProcsShake{\sgl{p}}{\alpha}{\sgl{p'}}$.
The process shaking shakes a single instruction and updates the process.
Shaking a process network chooses a single process to shake and updates it in the network.
If shaking the process resulted in an output value, that value is injected into all the other processes in the network.

Rule (Shake) takes a single process and shakes its instruction.
The judgment ($\ExpEval{@heap@~p}{u}{\Sigma}$) evaluates all the expressions in the update substitution $u$, under the process heap ($@heap@~p$).
The process steps to the new label with the new input state, and the values are updated in the heap.

Rule (ProcessesInternal) allows the process network to shake, when one of the processes inside it produces an internal message ($\tau$).

Rule (ProcessesInternal) allows the process network to shake, when one of the processes inside it produces a push message.
The emitted push message is then injected into all other processes in the network, meaning they must either ignore the channel or be ready to receive it in their buffer.
If the process tries to emit a push message but it cannot be injected into all other processes, the push fails and another process will be tried.
The entire network then emits the same message.

\subsubsection{Feeding}
Fig.~\ref{fig:Process:Eval:Feed} defines the judgment forms and rules for feeding, where external input values are fed into a process network and output values are accumulated.
The judgment form for feeding is $\ProcsFeed{\ti{inputs}}{\ti{network}}{\ti{streams}}{\ti{network}'}$.
The input map $\ti{inputs}$ contains values for the network inputs: network outputs are not allowed, but ignored channels can have values.
The result $\ti{streams}$ contains the original inputs as well as accumulated output values.
Feeding evaluates the process network until all input values have been injected.

% Note that the result stream and network are not canonical, as an infinite @push@ loop has an infinite number of evaluations.
% The feed form does not ensure that the processes themselves have finished evaluating, only that all input values have been injected.

Rule (FeedStart) applies when all input values have been injected and there are no input values left.
In this case, the output values are the same as the input values.

Rule (FeedInternal) allows the process network to take an internal step.
It first feeds its input accumulator and process network, then allows the resulting network to take an internal step.

Rule (FeedPush) allows the process network to emit a push message.
As with (FeedInternal), it first feeds its input accumulator, then allows the resulting network to emit a push message.
The pushed value is collected in the accumulator list for that stream.

Rule (FeedExternal) allows inputs to be injected into the process network.
For any channel $c$ which is not an output of one of the processes, we take the last value off its list.
The recursive feed is evaluated with the last value removed from the accumulators.
The last value is then injected into the network, and added back to the result accumulators.



% -- cuts ---------------------------------------------------------------------
% BL: I don't think describing iota works at this point. This combinator is not used in the motivating example, so skipping to it seems disjointed.

% Before describing the @group@ process, we start by looking at one of the simplest combinators, @iota@, which produces a stream of increasing numbers.
% It takes no inputs, and produces one output stream @xs@.
% Each process has its own local heap where the values are stored, and in this case we initialise the local variable @i@ to @0@.
% This variable will be incremented and pushed.
% Each process also has a current label, which denotes the instruction to perform next.
% The initial label for @iota@ is @L0@.
% Each process has a mapping from labels to instructions.
% In this case we have two instructions, @L0@ and @L1@.
% The instruction for @L0@ pushes the current value of variable @i@ onto the output stream, then proceeds to move %to label @L1@. Instruction @L1@ moves back to @L0@, while also incrementing the variable @i@.

% \begin{code}
% process (iota)
%      ins: 
%     outs: xs
%     heap: {i = 0}
%    label: L0
%   blocks: L0 = push xs i  L1
%           L1 = jump       L0{i = i + 1}
% \end{code}

% When executed, this program produces an infinite stream of increasing numbers: $0, 1, 2\ldots$ while the label alternates between @L0@ and @L1@.

% The @uniques = group file1@ is a more interesting example.