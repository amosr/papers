%!TEX root = ../Main.tex

\section{Fully abstract case interpretation}

This cannot be fused because it requires an unbounded buffer.
\begin{code}
zipltgts :: [a] -> [a*a]
zipltgts as =
  let as1 = filter (<0) as
      as2 = filter (>0) as
      aas = zip as1 as2
  in  aas
\end{code}

You might think the following can be fused.
It cannot because we treat @case@ conditions as fully abstract and make no attempt to filter out impossible combinations.
So while it cannot be true that the first filter reaches @>0 = true@ case and the second filter reaches @>0 = false@, we try both combinations and treat it as unfusable.
\begin{code}
zipgts :: [a] -> [a*a]
zipgts as =
  let as1 = filter (>0) as
      as2 = filter (>0) as
      aas = zip as1 as2
  in  aas
\end{code}

\section{Finite streams}
\label{s:Finite}
So far we have only treated infinite streams.
Finite streams could easily be encoded as infinite streams padded with a sentinel ``end of file'' element.
However, this plays poorly with the fully abstract case statements, as we would require a case for each element read to check whether it is the end of stream.
Adding these cases, we would end up with similar problems as @zipgts@ above.

One solution is to embed this special case into @pull@ itself, by giving it two actions to perform; one after successful read, one after end of stream.
It may be simpler and more useful to instead perform more sophisticated tracing of the @case@ conditionals though, as this would also solve the @zipgts@ example above.


