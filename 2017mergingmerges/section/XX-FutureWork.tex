%!TEX root = ../Main.tex
\section{Conclusions and future work}
\label{s:FutureWork}

We now discuss some of the shortcomings of the system, and extensions and future work to ameliorate this.

\subsection{Finite streams}
\label{s:Finite}

The processes we have seen so far deal with infinite streams, but in practice most streams are finite.
Certain combinators such as @fold@ and @append@ only make sense on finite streams, and others like @take@ produce inherently finite output.
We have focussed on the infinite stream version because it is somewhat simpler to explain and prove, but the extensions required to support finite streams do not require substantial conceptual changes.

We now describe the extensions required to support finite streams.
We add a new @closed@ constructor to the \InputState~ to encode the end of the stream.
Once an input stream is in the closed state, it can never change to another state: it remains closed thereafter.

We modify the @pull@ instruction so that it has two output labels (like @case@).
The first label, the read branch, is executed as before when the pull succeeds and a value is read from the stream.
The second label, the close branch, is executed when the stream is closed, and no more values will ever be available.
After a pull takes the close branch, any subsequent pulls from that stream will also take the close branch.

We add two new instructions for closing output streams and disconnecting from input streams.
Closing an output stream $(@close@~\Chan~\Goto)$ is similar to pushing an end-of-file marker to all readers.
As with @push@, the evaluation semantics of @close@ can only proceed if all readers are in a position to accept the end-of-file, but instead of setting the new \InputState~ to @pending@ with a value, the \InputState~ is set to @closed@.
After a stream has been closed, no further values can be pushed.

Disconnecting from input streams $(@disconnect@~\Chan~\Goto)$ signals that a process is no longer interested in the values of a stream.
This can be used when a process requires the first values of a stream, but does not require the whole stream.
If a process read the first values of a stream and then stopped pulling, its \InputState~ buffer would fill up and never be cleared, so no other process would be able to continue pulling from that stream.
Disconnecting the stream allows other processes to use the stream without the disconnected process getting in the way of computation.
The evaluation semantics for @disconnect@ remove the channel from the inputs of the process.
After removing the channel from the inputs, when a writing process tries to inject values, this process will just be ignored rather than inserting into the \InputState~ buffer and potentially causing writing to block.
After a process disconnects from an input channel, it can no longer pull from that channel.

We also add an instruction for terminating the process (@done@).
After all input streams have been read to completion or disconnected and output streams closed, the process may execute @done@ to signal that processing is complete.

The fusion definition must be extended to deal with these new instructions.
The static input state has a @closed@ constructor added and disconnection is encoded by removal from the input state, and the \ti{tryStep} changes more or less follow the evaluation changes.
Shared and connected pulls now deal with two more possibilities in the input state: the input may be closed in which case the close branch of the pull is taken; or the other process may have disconnected in which case the pull is executed as in the non-shared non-connected case.
Connected pushes must also deal with when the other process has disconnected in which case the push is executed as if it were non-connected.
For @in1@ and @out1@ channels, the new @close@ and @disconnect@ instructions are used as normal with no coordination required.
For connected @close@, as with @push@, the receiving process must have @none@ and the next step performs the @close@ and sets the input state to @closed@.
For shared @disconnect@, the @disconnect@ is only performed after both processes have disconnected; otherwise the entry is just removed from the input state.
For connected @disconnect@, the @disconnect@ is not performed and the entry is removed from the input state.

Finally, \ti{tryStepPair} is modified so that @done@ is performed when both machines are @done@.

These modifications allow our system to fuse finite streams as well as infinite.
We have implemented an initial prototype that supports finite streams, but future work is required to prove them correct.

\subsection{Fully abstract case interpretation}
\label{s:FullyAbstractCase}

The fusion algorithm treats all @case@ conditions as fully abstract, by exploring all possible combinations for both processes.
This can cause an issue with coordination and buffering, as the processes may dynamically require only a bounded buffer, the fusion algorithm statically tries every combination and wrongly asserts that unbounded buffering is required.

For example, suppose two processes have the same case condition @x > 0@.
The fusion algorithm will try all possibilities including contradictory ones, such as where the first process has @x > 0 = true@ and the second has @x > 0 = false@.
If any of these possibilities require an unbounded buffer, the fusion algorithm fails.
This means that the following program which pairs all positive elements with themselves, while unlikely to be written by a human, is not able to be fused.
\begin{code}
zipgts as =
  let as1 = filter (>0) as
      as2 = filter (>0) as
      aas = zip as1 as2
  in  aas
\end{code}

In general, if the example above used two different predicates for each filter, the fusion system would be right to outlaw it as requiring unbounded buffers.
If the same program is rewritten to use only a single filter and use its result twice, the program is able to be fused.

A possible extension is to somehow cull these contradictory states so that if it is statically known that a state is unreachable, it does not matter if it requires unbounded buffers.
It may be possible to achieve this with relatively little change to the fusion algorithm itself, by having it emit some kind of failure instruction rather than failing to produce a process.
A separate postprocessing step could then perform analyses and remove statically unreachable process states.
After postprocessing, if any failure instructions are reachable, the fusion process would fail as before.

\subsection{Non-determinism and fusion order}
\label{s:FusionOrder}

The main fusion algorithm here works on pairs of processes.
When there are more than two processes, there are multiple orders in which the pairs of processes can be fused.
The order in which pairs of processes are fused does not affect the output values, but it does affect the access pattern: the order in which outputs are produced and inputs read.
Importantly, the access pattern also affects whether fusion succeeds or fails to produce a process.
In other words, while evaluating multiple processes is non-deterministic, the act of fusing two processes \emph{commits} to a particular deterministic interleaving of the two processes.
The simplest example of this has two input streams, a function applied to both, then zipped together. 

\begin{code}
zipMap as bs =
  let as' = map (+1) as
      bs' = map (+1) bs
      abs = zip as' bs'
  in  abs
\end{code}

There are three combinators here, so after converting each combinator to its process there are three orders we can fuse.
The two main options are to fuse the two maps together and then add the zip, or to fuse the zip with one of the maps, then add the other map.
If we start by fusing the zip with one of its maps, the zip ensures that its inputs are produced in lock-step pairs, and then adding the other map will succeed.
However if we try to fuse the two maps together, there are many possible interleavings: the fused program could read all of @as'@ first; it could read all of @bs'@ first; it read the two in lock-step pairs; or any combination of these.
When the zip is added, fusion will fail if the wrong interleaving was chosen.

% The example above can be solved by fusing connected processes first, but it is possible to construct a connected process that still relies on the order of fusion.
% 
% \begin{code}
% zipApps as bs cs =
%   let as' = as ++ bs
%       bs' = as ++ cs
%       abs = zip as' bs'
%   in  abs
% \end{code}

Our current solution to this is to try all permutations of fusing processes and use the first one that succeeds.
A more principled solution may be to allow non-determinism in a single process by adding a non-deterministic choice instruction.
Then when fusing two processes together, if both processes are pulling from unrelated streams, the result would be a non-deterministic choice between pulling from the first process and executing the first, or pulling from the second process and executing the second.
In this way we could defer committing to a particular evaluation order until the required order is known.
This may produce larger intermediate programs, but the same deterministic program could be extracted at the end.

