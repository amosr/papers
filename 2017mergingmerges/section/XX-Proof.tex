%!TEX root = ../Main.tex

\section{Proofs}
\label{s:Proofs}

Our fusion system is formalised in Coq, where we have proved soundness of \ti{fusePair}: if the fused program evaluates to a particular output, then the two original programs also evaluate to that output.
It is interesting to note that the converse is not necessarily true: just because two programs can evaluate to a particular output does not mean the fused program will evaluate to that.
This is because, as explained in~\S\ref{s:EvaluationOrder}, evaluation of a process network is non-deterministic, and fusion commits to a particular evaluation order.

\begin{code}
Theorem Soundness (P1 : Program L1 C V1) (P2 : Program L2 C V2) (ss : Streams) (h : Heap)
    (l1 : L1) (is1 : InputStates) (l2 : L2) (is2 : InputStates) :
  EvalBs (fuse P1 P2) ss h (LX l1 l2 is1 is2)
  -> EvalOriginal Var1 P1 P2 is1 ss h l1 /\ EvalOriginal Var2 P2 P1 is2 ss h l2.
\end{code}

The @Soundness@ theorem above uses @EvalBs@ to evaluate the fused program, and @EvalOriginal@ ensures that the original program evaluates with that program's subset of the fused heap, using @Var1@ and @Var2@ to extract the variables.
Care must be taken to remove stream values that the other process has pulled but this one has not yet.
%%% AR: Really would like to say something about this but no room to explain properly
% For shared inputs when one program has pulled, the other program must be evaluated with the other value removed from the end of the stream.

The system described here has some differences to our Coq formalisation.
First, the Coq formalisation has a separate @update@ instruction which modifies a variable in the local heap, rather than allowing heap updates in the output \Next~label of any instruction.
This causes the fusion definition to be slightly more complicated, as two output instructions must be emitted when performing a push or pull followed by an update.
This is a fairly minor difference, and we have made this change in the paper version for ease of exposition.
% Ideally, a future version of the formalisation would include this change.
Secondly, our formalisation does not implement the concurrent evaluation semantics for processes, only sequential evaluation for a single process.
Instead we sequentially evaluate both processes with the same input values and outputs.
Despite these differences, the Coq formalisation gives us sufficient confidence in the correctness of the version presented here.

