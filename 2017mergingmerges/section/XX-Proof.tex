%!TEX root = ../Main.tex

\section{Proofs}
\label{s:Proofs}

Our fusion system is formalized in Coq, and we have proved soundness of \ti{fusePair}: if the fused result process produces a particular sequence of values on its output channels then the two source processes may also produce that same sequence. Note that due to non-determinism of process execution the converse is not true in practice: just because the two concurrent processes can produce a particular output sequence does not mean the fused result process will as well --- the fused result process uses only one of the many possible orders. 

% Note that the converse is not necessarily true: just because two processes can evaluate to a particular output does not mean the fused program will evaluate to that. This is because, as explained in~\S\ref{s:EvaluationOrder}, evaluation of a process network is non-deterministic, and fusion commits to a particular evaluation order.

\begin{code}
Theorem Soundness (P1 : Program L1 C V1) (P2 : Program L2 C V2) (ss : Streams) (h : Heap)
                  (l1 : L1) (is1 : InputStates) (l2 : L2) (is2 : InputStates) 
  :  EvalBs (fuse P1 P2) ss h (LX l1 l2 is1 is2)
  -> EvalOriginal Var1 P1 P2 is1 ss h l1 /\ EvalOriginal Var2 P2 P1 is2 ss h l2.
\end{code}

The @Soundness@ theorem uses @EvalBs@ to evaluate the fused program, and @EvalOriginal@ ensures that the original program evaluates with that program's subset of the result heap, using @Var1@ and @Var2@ to extract the variables.

% Care must be taken to remove stream values that the other process has pulled but this one has not yet.
%%% AR: Really would like to say something about this but no room to explain properly
% For shared inputs when one program has pulled, the other program must be evaluated with the other value removed from the end of the stream.

To aid mechanization, the Coq formalization has some small differences from the system presented in this paper. Firstly, the Coq formalization uses a separate @update@ instruction to modify variables in the local heap, rather than attaching heap updates to the \Next~ label of every instruction. Performing this desugaring makes the low level lemmas easier to prove, but we find attaching the updates to each instruction makes for an easier exposition. Secondly, our formalization only implements sequential evaluation for a single process, rather than non-deterministic evaluation for whole process groups as per Fig.\ref{fig:Process:Eval:Feed}. Instead, we sequentially evaluate each source processes independently, and compare the output values to the ones produced by sequential evaluation of the fused result process. This is sufficient for our purposes because we are mainly interested in the value correctness of the fused program, rather than making a statement about the possible execution orders of the source processes when run concurrently.

% This causes the fusion definition to be slightly more complicated, as two output instructions must be emitted when performing a push or pull followed by an update. This is a fairly minor difference, and we have made this change in the paper version for ease of exposition.
% Ideally, a future version of the formalisation would include this change.
% Secondly, our formalisation does not implement the concurrent evaluation semantics for processes, only sequential evaluation for a single process. Instead we sequentially evaluate both processes with the same input values and outputs. Despite these differences, the Coq formalisation gives us sufficient confidence in the correctness of the version presented here.

