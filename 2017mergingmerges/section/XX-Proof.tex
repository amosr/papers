%!TEX root = ../Main.tex

\section{Proofs}
\label{s:Proofs}

Our fusion system is formalised in Coq, where we have proved soundness of \ti{fusePair} if the fused program evaluates to a particular output, then the two original programs also evaluate to that output.
It is important to note that the converse is not necessarily true: just because two programs can evaluate to a particular output does not mean the fused program will evaluate to that.
This is because evaluation of a process nest is non-deterministic, and fusion commits to a particular evaluation order.

The problem with commiting to an evaluation order is most easily explained with a process nest containing two infinitely pushing processes.
Process @A@ is repeatedly pushing to a stream called @X@, while process @B@ repeatedly pushes to @Y@.
When evaluating this pair as a process nest, there are an infinite number of possible interleavings: all @X@s, all @Y@s, pairs of @X@s followed by @Y@s and so on.

When fusion is performed on this process pair, an arbitrary but \emph{particular} order will be chosen.
Thus, the chosen order will be one of the valid ones, but not all valid orders will be the chosen one.

We know of no `realistic' examples where combinators have multiple evaluation orders, so believe this is not an issue in practice.

The system described here has some differences to our Coq formalisation.
First, the Coq formalisation has a separate @update@ instruction which modifies a variable in the local heap, rather than allowing heap updates in the output \Goto~label of any instruction.
This causes the fusion definition to be slightly more complicated, as two output instructions must be emitted when performing a push or pull followed by an update.
This is a fairly minor difference, and we have made this change in the paper version for ease of exposition.
Ideally, a future version of the formalisation would have this change.
Secondly, our formalisation does not implement the concurrent evaluation semantics for processes, only sequential evaluation for a single process.
Instead we sequentially evaluate both processes separately with the same input values and outputs.
Finally, we have not implemented \ti{fuseNest} for fusing multiple processes at a time, only \ti{fusePair} for fusing pairs of processes.

Despite these differences, we believe the current formalisation gives sufficient confidence in correctness.

