%!TEX root = ../Main.tex
\section{Fusion}
\label{s:Fusion}


%!TEX root = ../Main.tex

% Settle on a syntax for this later.
% Might be too many nested pairs.
\newcommand\nextStep[5]{\big((#1,~#2),~(#3,~#4),~#5 \big)}

\begin{figure}

\begin{tabbing}
@MMMMMMMMMMMM@   \TABDEF \kill

$\InputState_1$ \> := \> $@pending@ ~|~ @have@ ~|~ @none@$
\\
$\Label_1$ \> := \> $\Label~\times~\MapType{\Chan}{\InputState_1}$ \\
$\Label$   \> := \> $\ldots ~|~\Label_1~\times~\Label_1 ~|~ \ldots$ \\
$\Var$     \> := \> $\ldots ~|~@chan@~\Chan ~|~ \ldots$ \\
\\

$\ChanType_2$   \> := \> $@in2@~|~@in1@~|~@in1out1@~|~@out1@$ \\
\end{tabbing}

\caption{Fusion types}
\label{fig:Fusion:Types}
\end{figure}

\begin{figure}

\begin{tabbing}
\ti{fuseNetwork}~@M M@   \TABDEF \kill

\amos{deal with maybes} \\
\amos{try permutations of process orders} \\
\\
\ti{fuseNetwork} \> $~:$ \> $\sgl{\Proc} \to  \Maybe~\Proc$ \\
\ti{fuseNetwork}~$\sgl{p}$ \> $=$ \> $p$ \\
\ti{fuseNetwork}~$\ti{ps}$
    \> $~|$      \> $p \in \ti{ps} ~\wedge~ q \in \ti{ps} ~\wedge~ p \not\eq q ~\wedge~ \ti{connected}~p~q$ \\
    \> $=$ \> $\ti{fuseNetwork}~(\sgl{\ti{fusePair}~p~q}~\cup~\ti{ps} \setminus \sgl{p,q})$ \\
\\
\ti{fusePair} \> $~:$ \> $\Proc \to \Proc \to  \Maybe~\Proc$ \\
$\ti{fusePair}~p~q$ \> $=$ \\
@    process@ \\
@        ins: @ $\sgl{c~|~c=t \in \cs,~t \in \sgl{@in1@,@in2@}} $ \\
@       outs: @ $\sgl{c~|~c=t \in \cs,~t \in \sgl{@in1out1@,@out1@}} $ \\
@       heap: @ $@heap@~p~\cup~@heap@~q$ \\
@      label: @ $l_0$ \\
@     blocks: @ $\ti{go}~\sgl{}~l_0$ \\
@ where@ \\
MM\=MM\=~=~\=\kill
 \> \cs \> $=$ \> $\ti{channels}~p~q$ \\
 \> $l_0$   \> $=$ \> $
      \big( 
      (@label@~p,~\sgl{c=none~|~c~\in~@ins@~p}),~
      (@label@~q,~\sgl{c=none~|~c~\in~@ins@~q})
      \big)$ \\
 \\
 \> $\ti{go}~\ti{bs}~l$ \\
 \> \> $~|$ \> $l~\in~\ti{bs}$ \\
 \> \> $=$  \> $\ti{bs}$ \\
 \> \> $~|$     \> $@label@'~l_p~l_q~\in~l$ \\
 \> \> $\wedge$ \>
        $b~\in~\ti{tryStepPair}~\cs~l_p~(@blocks@~p~l_p)~l_q~(@blocks@~q~l_q)$ \\ 
 \> \> $=$ \> $\ti{fold}~\ti{go}~(\ti{bs}~\cup~\sgl{l=b})~(\ti{outlabels}~b)$ \\
\end{tabbing}
\caption{Fusion top-level definitions}
\label{fig:Fusion:Def:Top}
\end{figure}


% I tried this colour in a colour blindness simulator and it seems to be OK.
% Should still be readable when converted to grayscale.
\definecolor{notec}{HTML}{C03020}
\newcommand\note[1]{\textcolor{notec}{(#1)}}

\begin{figure}

\amos{Is this syntax too weird a mixture of haskell and ml?}
\begin{tabbing}
M \= M \= M \= M \kill
$\ti{tryStep} ~:~ \ChanTypeMap \to \Label_1 \to \Instr \to \Label_1 \to \Maybe~\Instr$ \\
$\ti{tryStep} ~\cs~(l_p,s_p)~i_p~(l_q,s_q)~=~@match@~i_p~@with@$ \\

\> $@pull@~c~x~(l',u')$ \\
\> \> \note{Single input}\\
\> \> $~|~c=@in1@\in\cs$ \\
\> \> $\to~@pull@~c~x~
      \nextStep
        {l'}{s_p}
        {l_q}{s_q}
        {u'}
    $ \\

\> \> \note{Shared input and in/out are both the same if we already have an injected value}\\
\> \> $~|~(c=@in2@\in\cs \vee c=@in1out1@\in\cs) ~\wedge~ c=@pending@ \in s_p$ \\
\> \> $\to~@jump@~
      \nextStep
        {l'}
          {\HeapUpdateOne{c}{@have@}{s_p}}
        {l_q}
          {s_q}
        {\HeapUpdateOne{x}{@chan@~c}{u'}}
        $ \\

\> \> \note{Shared input where neither have it, so inject into both. Note this machine does not jump to $l'$}\\
\> \> $~|~c=@in2@\in\cs ~\wedge~ c=@none@ \in s_p ~\wedge~ c=@none@ \in s_q$ \\
\> \> $\to~@pull@~c~(@chan@~c)~
      \nextStep
        {l_p}
          {\HeapUpdateOne{c}{@pending@}{s_p}}
        {l_q}
          {\HeapUpdateOne{c}{@pending@}{s_q}}
        {\sgl{}}
  $ \\
\\

\> $@drop@~c~(l',u')$ \\
\> \> \note{Single input, no coordination required} \\
\> \> $~|~c=@in1@\in\cs$ \\
\> \> $\to~@drop@~c~
      \nextStep
        {l'}
          {s_p}
        {l_q}
          {s_q}
        {u'}
      $ \\

\> \> \note{Pulled from other, so fake drop}\\
\> \> $~|~c=@in1out1@\in\cs$ \\
\> \> $\to~@jump@~
      \nextStep
        {l'}
          {\HeapUpdateOne{c}{@none@}{s_p}}
        {l_q}
          {s_q}
        {u'}
      $ \\

\> \> \note{Shared input, but other is still using it so only fake drop}\\
\> \> $~|~c=@in2@\in\cs ~\wedge~ (c=@have@ \in s_q \vee c=@pending@ \in s_q)$ \\
\> \> $\to~@jump@~
      \nextStep
        {l'}
          {\HeapUpdateOne{c}{@none@}{s_p}}
        {l_q}
          {s_q}
        {u'}
      $ \\


\> \> \note{Shared input, other has already fake dropped so we real drop}\\
\> \> $~|~c=@in2@\in\cs ~\wedge~ c=@none@ \in s_q$ \\
\> \> $\to~@drop@~c~
      \nextStep
        {l'}
          {\HeapUpdateOne{c}{@none@}{s_p}}
        {l_q}
          {s_q}
        {u'}
      $ \\
\\

\> $@push@~c~e~(l',u')$ \\
\> \> \note{Single output}\\
\> \> $~|~c=@out1@\in\cs$ \\
\> \> $\to~@push@~c~e~
      \nextStep
        {l'}
          {s_p}
        {l_q}
          {s_q}
        {u'}
      $ \\

\> \> \note{Shared output, other is ready to inject}\\
\> \> $~|~c=@in1out1@\in\cs ~\wedge~ c=@none@ \in s_q$ \\
\> \> $\to~@push@~c~e~
      \nextStep
        {l'}
          {s_p}
        {l_q}
          {\HeapUpdateOne{c}{@pending@}{s_q}}
        {\HeapUpdateOne{@chan@~c}{e}{u'}}
      $ \\

\\

\> $@case@~e~(l'_t,u'_t)~(l'_f,u'_f)$ \\
\> \> $\to~@case@~e~
      \nextStep
        {l'_t}{s_p}
        {l_q}{s_q}
        {u'_t}
      ~
      \nextStep
        {l'_f}{s_p}
        {l_q}{s_q}
        {u'_f}
      $ \\

\\

\> $@jump@~(l',u')$ \\
\> \> $\to~@jump@~
      \nextStep
        {l'_f}{s_p}
        {l_q}{s_q}
        {u'_f}
      $ \\
\end{tabbing}

\caption{Fusion step definition}
\label{fig:Fusion:Def:Step}
\end{figure}

\begin{figure}

\begin{tabbing}
M \= M \= M \= M \kill
$\ti{tryStepPair} ~:~ \ChanTypeMap \to \Label_1 \to \Instr \to \Label_1 \to \Instr \to \Maybe~\Instr$ \\
$\ti{tryStepPair} ~\cs~l_p~i_p~l_q~i_q$ \\
\\

\> \note{If both processes can run, prefer ``useful work now'' over pulls} \\
\> $~|~i_p'~\in~\ti{tryStep}~\cs~l_p~i_p~l_q ~\wedge~ i_q'~\in~\ti{tryStep}~\cs~l_q~i_q~l_p$ \\
\> $\wedge~@pull@~c~x~(l,u)~\not\in~i_p'$ \\
\> $\to~i_p'$ \\
\> $~|~i_p'~\in~\ti{tryStep}~\cs~l_p~i_p~l_q ~\wedge~i_q'~\in~\ti{tryStep}~\cs~l_q~i_q~l_p$ \\
\> $\wedge~@pull@~c~x~(l,u)~\not\in~i_q'$ \\
\> $\to~\ti{fliplabels}~i_q'$ \\
\\

\> \note{Otherwise only one process can run or both are pulls} \\
\> \note{If both are pulls, take the first, otherwise take whichever can run} \\
\> $~|~i_p'~\in~\ti{tryStep}~\cs~l_p~i_p~l_q$ \\
\> $\to~i_p'$ \\
\> $~|~i_q'~\in~\ti{tryStep}~\cs~l_q~i_q~l_p$ \\
\> $\to~\ti{fliplabels}~i_q'$ \\

\end{tabbing}

\caption{Fusion step coordination rules}
\label{fig:Fusion:Def:StepPair}
\end{figure}


Figure~\ref{fig:Fusion:Types} shows the type definitions for fusion.
Fusion proceeds by taking two processes in a network, and creating a single process that computes both.
The new process uses labels made of the product of the two input processes, as well as the static part of the input state - that is, whether the input buffer is pending after injection, has been pulled, or is empty, but without the actual injected value.


$\InputState_1$ defines the statis input state. $\Label_1$ is a single process label with the set of input states. We add the constructor $@label@'$ to the $\Label$ type which is the pair of process labels with input states.

$\ChanType_2$ classifies the kind of channels and the communications between two processes.
Two processes can read from the same channel (@in2@), in which case pulling must be coordinated together.
When one process reads from a channel and the other ignores it (@in1@) no coordination is required.
When one process writes to a channel and the other reads (@in1out1@) the reading process must wait for the other to write.
Finally, when one process writes and the other ignores (@out1@) no coordination is necessary.
As explained in~\S\ref{s:Process:Eval}, it is not allowed for two processes to write to the same channel.

Before looking at the definitions, it is worth noting the types and purposes of important functions:
\begin{itemize}
\item
\ti{fuseNetwork} takes a process network and repeatedly fuses pairs together. 
It chooses arbitrary processes with shared channels and fuses them together.

\item
\ti{fusePair} fuses a pair of processes together.
The heap variables must be distinct (this can be ensured by renaming).
The two processes must have some shared channels; otherwise no coordination is required at all, and the fusion process will choose an arbitrary interleaving of the processes.
This restriction is not an issue in practice since there is no benefit to fusing unrelated processes.
\end{itemize}

Figure~\ref{fig:Fusion:Def:Top} contains the definitions of top-level fusion functions.

Figures~\ref{fig:Fusion:Def:Step} and~\ref{fig:Fusion:Def:StepPair} define the judgment rules for fusing a single instruction at a time.

%!TEX root = ../Main.tex
\begin{figure*}

\begin{minipage}{0.45\textwidth}
\begin{tabbing}
$\ChanTypeTwo$   \TABDEF \kill

\ti{channels} \> $:$ \> $\Proc \to \Proc \to \MapType{\Chan}{\ChanTypeTwo}$ \\

  \> $=$    \> $\{ c=@MMMMMMM@~$\= \kill
$\ti{channels}~p~q$
  \> $=$    \> $\{ c=@in2@$
            \> $|~c$ \= $\in$ \= $(@ins@~p~\cap~@ins@~q) \}$ 
            \\

  \> $\cup$ \> $\{ c=@in1@$
            \> $ |~c$ \> $\in$ \> $(@ins@~p~\cup~@ins@~q)~\wedge~c~\not\in$ \= $(@outs@~p~\cup~@outs@~q) \}$ \\

  \> $\cup$ \> $\{ c=@in1out1@$
            \> $|~c$ \> $\in$ \> $(@ins@~p~\cup~@ins@~q)~\wedge~c~\in$ \> $(@outs@~p~\cup~@outs@~q) \}$ \\

  \> $\cup$ \> $\{ c=@out1@$
            \> $ |~c$ \> $\not\in$ \> $(@ins@~p~\cup~@ins@~q)~\wedge~c~\in$ \> $(@outs@~p~\cup~@outs@~q) \}$ 
\end{tabbing}
\end{minipage}

\newcommand\funClauseDef[3]
{ $\ti{#1}~(#2)$ \> $=$ \> $#3$
}
\newcommand\outlabelsDef[2]
{ \funClauseDef{outlabels}{#1}{\sgl{#2}} 
}

\vspace{1ex}
\begin{minipage}{0.45\textwidth}
\begin{tabbing}
$\ti{outlabels}~(@case@~e~(l_t,u_t)~(l_t,u_f))$ \TABSKIP $=$ \TABSKIP \kill
$\ti{outlabels} ~~:~~ \Instr \to \sgl{\Label}$ \\
\outlabelsDef{@pull@~c~x~(l,u)}{l}              \\
\outlabelsDef{@drop@~c~(l,u)}{l}                \\
\outlabelsDef{@push@~c~e~(l,u)}{l}              \\
\outlabelsDef{@case@~e~(l,u)~(l',u')}{l, l'}    \\
\outlabelsDef{@jump@~(l,u)}{l}
\end{tabbing}
\end{minipage}
\begin{minipage}{0.1\textwidth}
~
\end{minipage}
\begin{minipage}{0.45\textwidth}
\begin{tabbing}
$\ti{swaplabels}~(@case@~e~((l_1,l_2),u)~((l'_1,l'_2),u'))$ \TABSKIP $=$ \TABSKIP \kill
$\ti{swaplabels} ~~:~~ \Instr \to \Instr$ \\
\funClauseDef{swaplabels}
  {@pull@~c~x~((l_1,l_2),u)}
  {@pull@~c~x~((l_2,l_1),u)}    \\
\funClauseDef{swaplabels}
  {@drop@~c~((l_1,l_2),u)}
  {@drop@~c~((l_2,l_1),u)}      \\
\funClauseDef{swaplabels}
  {@push@~c~e~((l_1,l_2),u)}
  {@push@~c~e~((l_2,l_1),u)}    \\
\funClauseDef{swaplabels}
  {@case@~e~((l_1,l_2),u)~((l'_1,l'_2),u')}
  {@case@~e~((l_2,l_1),u)~((l'_2,l'_1),u')}     \\
\funClauseDef{swaplabels}
  {@jump@~((l_1,l_2),u)}
  {@jump@~((l_2,l_1),u)}
\end{tabbing}
\end{minipage}
\caption{Utility functions}
\label{fig:Fusion:Utils}
\end{figure*}



Figure~\ref{fig:Fusion:Utils} contains definitions of some utility functions which are not specific to fusion.
\ti{channels} computes the $\ChanType_2$ map for a pair of processes.
\ti{connected} checks whether two processes have any shared inputs or outputs - basically whether it is worth fusing them.
\ti{outlabels} gets the set of output labels for an instruction - this is used when computing the fixpoint of the blocks map.
\ti{swaplabels} flips the order of the compound labels in an instruction.
