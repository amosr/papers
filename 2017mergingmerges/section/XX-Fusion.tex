%!TEX root = ../Main.tex
\section{Fusion}
\label{s:Fusion}


%!TEX root = ../Main.tex

\begin{figure}

\begin{tabbing}
@MMMMMMMMMMMM@   \TABDEF \kill

$\InputState_1$ \> := \> @pending@ $~|~$ @have@ $~|~$ @none@
\\
$\Label_1$ \> := \> $@label@_1~\Label~\MapType{\Chan}{\InputState_1}$ \\
$\Label$   \> := \> $\ldots ~|~@label@'~\Label_1~\Label_1 ~|~ \ldots$ \\
\\

$\ChanType_2$   \> := \> $@in2@~|~@in1@~|~@in1out1@~|~@out1@$ \\
\\
\ti{fuseNest} \> $:$ \> $\sgl{\Proc} \to  \Maybe~\Proc$ \\
\ti{fusePair} \> $:$ \> $\Proc \to \Proc \to  \Maybe~\Proc$ \\
\ti{tryStepPair} \> $:$ \> $\MapType{\Chan}{\ChanType_2} \to \Instr \to \Label_1 \to \Instr \to \Label_1 \to  \Maybe~\Instr$ \\
\ti{tryStep} \> $:$ \> $\MapType{\Chan}{\ChanType_2} \to \Instr \to \Label_1 \to \Label_1 \to  \Maybe~\Instr$ \\
\end{tabbing}

\caption{Fusion types}
\label{fig:Fusion:Types}
\end{figure}

\begin{figure}

\begin{tabbing}
$\ChanType_2$   \TABDEF \kill

\TODO{deal with maybes} \\
\\
\ti{fuseNest} \> $~:$ \> $\sgl{\Proc} \to  \Maybe~\Proc$ \\
\ti{fuseNest}~$\sgl{p}$ \> $=$ \> $p$ \\
\ti{fuseNest}~$\ti{ps}$
    \> $~|$      \> $p \in \ti{ps} ~\wedge~ q \in \ti{ps} ~\wedge~ p \not\eq q ~\wedge~ \ti{connected}~p~q$ \\
    \> $=$ \> $\ti{fuseNest}~(\sgl{\ti{fusePair}~p~q}~\cup~\ti{ps} \setminus \sgl{p,q})$ \\
    \> $~|$      \> @otherwise@ \\
    \> $=$ \> $\bot$ \\
\\
\ti{fusePair} \> $~:$ \> $\Proc \to \Proc \to  \Maybe~\Proc$ \\
$\ti{fusePair}~p~q$ \> $=$ \\
@    process@ \\
@        ins: @ $\sgl{c~|~c=t \in \ti{cs},~t \in \sgl{@in1@,@in2@}} $ \\
@       outs: @ $\sgl{c~|~c=t \in \ti{cs},~t \in \sgl{@in1out1@,@out1@}} $ \\
@       heap: @ $@heap@~p~\cup~@heap@~q$ \\
@      label: @ $l_0$ \\
@     blocks: @ $\ti{go}~\sgl{}~l_0$ \\
@ where@ \\
MM\=MM\=~=~\=\kill
 \> \ti{cs} \> $=$ \> $\ti{channels}~p~q$ \\
 \> $l_0$   \> $=$ \> $@label@'~l_p~l_q$ \\
 \> $l_p$   \> $=$ \> $@label@_1~(@label@~p)~\sgl{c=none~|~c~\in~cs}$ \\
 \> $l_q$   \> $=$ \> $@label@_1~(@label@~q)~\sgl{c=none~|~c~\in~cs}$ \\
 \\
 \> $\ti{go}~\ti{bs}~l$ \\
 \> \> $~|$ \> $l~\in~\ti{bs}$ \\
 \> \> $=$  \> $\ti{bs}$ \\
 \> \> $~|$ \> $l~\in~\ti{bs}$ \\
 \> \> $=$ \> $@label@_1~(@label@~q)~\sgl{c=none~|~c~\in~cs}$ \\
\end{tabbing}
\caption{Fusion top-level definitions}
\label{fig:Fusion:Def:Top}
\end{figure}
\begin{figure}

\begin{tabbing}
$\ChanType_2$   \TABDEF \kill

\ti{tryStep} \> $:$ \> $\MapType{\Chan}{\ChanType_2} \to \Instr \to \Label_1 \to \Label_1 \to  \Maybe~\Instr$ \\
\end{tabbing}

\caption{Fusion step definitions}
\label{fig:Fusion:Def:Step}
\end{figure}


The core fusion algorithm works by constructing a static schedule for a pair of processes, which can be repeated over the whole network until only one process remains.
The static schedule mirrors the dynamic evaluation by statically performing injection and shaking on the two processes.
Figure~\ref{fig:Fusion:Types} shows the type definitions for fusion.
Because the fused process is \emph{statically} evaluating both processes, we use its labels to encode static information about the dynamic state of the original processes.
The fused label contains the pair of both original labels, as well as the static part of the $\InputState$ for each input channel.
If the static $\InputState_S$ is $@pending@_S$, there is a value waiting to be pulled, but rather than knowing its actual value as in the dynamic evaluation, we know its value is stored in the $@chan@$ variable for that channel.

$\ChanType_2$ classifies the kind of channels and the communications between two processes.
Two processes can read from the same channel (@in2@), in which case pulling must be coordinated together.
When one process reads from a channel and the other ignores it (@in1@) no coordination is required.
When one process writes to a channel and the other reads (@in1out1@) the reading process must wait for the other to write.
Finally, when one process writes and the other ignores (@out1@) no coordination is necessary.
Recall that each output channel is uniquely owned and cannot be written by another process, so it is not possible for both processes to write to the same output.

Figure~\ref{fig:Fusion:Def:Top} contains the definitions of top-level fusion functions to fuse a network and a pair of processes.
\ti{fuseNetwork} takes a process network and repeatedly fuses pairs together until all processes in the network have been fused into a single process.
The order processes are fused can affect whether fusion succeeds or fails, so all permutations are tried.
We address this in more detail in~\S\ref{s:FusionOrder}.

\ti{fusePair} fuses a pair of processes together, constructing a new process that computes the outputs of both.
The blocks are constructed by computing a fixpoint, starting at the initial labels of each process with empty input states.
Each block is computed with \ti{tryStepPair}, which statically chooses one of the two processes to execute, and any reachable blocks are added recursively until fixpoint is reached.
This implementation assumes the heap variables are distinct, which can easily be ensured by renaming.



Figure~\ref{fig:Fusion:Def:Step} defines the \ti{tryStep} function which checks if one of the processes can take a step.
It takes the channel types, the label with static input states and instruction at that label, as well as the other process' label with static input states.
The \ti{tryStepPair} function defined in figure~\ref{fig:Fusion:Def:StepPair} calls this for both processes, and if either machine can execute it will execute that process.
If both processes can execute, \ti{tryStepPair} uses a simple priority heuristic of deferring pulls: this is because pulls may block, while other actions are more likely to produce immediate results.

\ti{tryStep} statically performs shaking for the current process, as well as injection for both processes.
If the process is pulling or pushing a non-shared channel categorised as @in1@ or @out1@, no coordination is required and injection is ignored.
For shared and connected channels, the process can only pull or push if the other process would be willing to accept the injection at the same time.
Simple instructions such as @case@ and @jump@ require no coordination and can be executed at any point.


\begin{figure}

\begin{tabbing}
$\ChanType_2$   \TABDEF \kill

\ti{channels} \> $:$ \> $\Proc \to \Proc \to \MapType{\Chan}{\ChanType_2}$ \\

  \> $=$    \> $\{ c=@MMMMMMM@~$\= \kill
$\ti{channels}~p~q$
  \> $=$    \> $\{ c=@in2@$
            \> $|~c~\in~(@ins@~p~\cap~@ins@~q) \}$ \\

  \> $\cup$ \> $\{ c=@in1@$
            \> $ |~c~\in~(@ins@~p~\cup~@ins@~q)~\wedge~c~\not\in(@outs@~p~\cup~@outs@~q) \}$ \\

  \> $\cup$ \> $\{ c=@in1out1@$
            \> $|~c~\in~(@ins@~p~\cup~@ins@~q)~\wedge~c~\in(@outs@~p~\cup~@outs@~q) \}$ \\

  \> $\cup$ \> $\{ c=@out1@$
            \> $ |~c~\not\in~(@ins@~p~\cup~@ins@~q)~\wedge~c~\in(@outs@~p~\cup~@outs@~q) \}$ \\
\\

\ti{connected} \> $:$ \> $\Proc \to \Proc \to  \mathbb{B}$ \\
$\ti{connected}~p~q$
  \> $=$ \> $(@ins@~p~\cup~@outs@~p)~\cap~(@ins@~q~\cup~@outs@~q)~\not\eq~\sgl{}$ \\
\end{tabbing}

\newcommand\funClauseDef[3]
{ $\ti{#1}~(#2)$ \> $=$ \> $#3$ \\
}
\newcommand\outlabelsDef[2]
{ \funClauseDef{outlabels}{#1}{\sgl{#2}} 
}

\begin{tabbing}
$\ti{outlabels}~(@case@~e~l_t[u_t]~l_t[u_f])$ \TABSKIP $=$ \TABSKIP \kill
\ti{outlabels} \> $~:$ \> $\Instr \to \sgl{\Label}$ \\
\outlabelsDef{@pull@~c~x~l[u]}{l}
\outlabelsDef{@drop@~c~l[u]}{l}
\outlabelsDef{@push@~c~e~l[u]}{l}
\outlabelsDef{@case@~e~l[u]~l'[u']}{l, l'}
\outlabelsDef{@jump@~l[u]}{l}
\end{tabbing}

\begin{tabbing}
$\ti{swaplabel}~(@case@~e~l_t[u_t]~l_t[u_f])$ \TABSKIP $=$ \TABSKIP \kill
\ti{swaplabel} \> $~:$ \> $\Label \to \Label$ \\
$\ti{swaplabel}~(@label@'~l_1~l_2)$ \> $=$ \> $@label@'~l_2~l_1$ \\
\end{tabbing}

\begin{tabbing}
$\ti{swaplabels}~(@case@~e~l_t[u_t]~l_t[u_f])$ \TABSKIP $=$ \TABSKIP \kill
\ti{swaplabels} \> $~:$ \> $\Instr \to \Instr$ \\
\funClauseDef{swaplabels}{@pull@~c~x~l[u]}{@pull@~c~x~(\ti{swaplabel}~l)[u]}
\funClauseDef{swaplabels}{@drop@~c~l[u]}{@drop@~c~(\ti{swaplabel}~l)[u]}
\funClauseDef{swaplabels}{@push@~c~e~l[u]}{@push@~c~e~(\ti{swaplabel}~l)[u]}
\funClauseDef{swaplabels}{@case@~e~l[u]~l'[u']}{@case@~e~(\ti{swaplabel}~l)[u]~(\ti{swaplabel}~l')[u']}
\funClauseDef{swaplabels}{@jump@~l[u]}{@jump@~(\ti{swaplabel}~l)[u]}
\end{tabbing}

\caption{Utility functions}
\label{fig:Fusion:Utils}
\end{figure}



Figure~\ref{fig:Fusion:Utils} contains definitions of some utility functions which are not specific to fusion.
\ti{channels} computes the $\ChanType_2$ map for a pair of processes.
\ti{connected} checks whether two processes have any shared inputs or outputs - basically whether it is worth fusing them.
\ti{outlabels} gets the set of output labels for an instruction - this is used when computing the fixpoint of the blocks map.
\ti{swaplabels} flips the order of the compound labels in an instruction.
