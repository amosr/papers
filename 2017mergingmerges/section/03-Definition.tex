%!TEX root = ../Main.tex

\clearpage{}
% -----------------------------------------------------------------------------
\section{Process definitions}

%!TEX root = ../Main.tex

\begin{figure}
\begin{minipage}[t]{0.4\textwidth}
\begin{tabbing}
\Instr \TABDEF @MMMM@  \TABSKIP $\Exp$ \TABSKIP $\Exp$ \TABSKIP $\Exp$ \kill

\Exp,~$e$ \> $\to$ \> $x~|~v~|~e~e $ \\
  \> $\enskip|~$ \> $ (e~@||@~e) ~|~ e+e ~|~ e~@/=@~e ~|~ e < e$ \\
\Value,~$v$ \> $\to$ \> $\mathbb{N}~|~\mathbb{B}~|~(\lambda{}x.~e)$ \\
$\Sigma$ \> $\to$ \> $\cdot~|~\Sigma,~x~=~v$ \\
\\

\Proc \>:=\> @process@ \\
M \= M \= \kill
\> \> @ins:   @  $\MapType{\Chan}{\InputState}$ \\
\> \> @outs:  @  $\sgl{\Chan}$ \\
\> \> @heap:  @  $\Sigma$ \\
\> \> @label: @  \Label \\
\> \> @instrs:@  $\MapType{Label}{\Instr}$ \\
\\
\Instr \TABDEF \kill
\InputState \> := \> @pending@~\Value~$|$~@have@~$|$~@none@

\end{tabbing}
\end{minipage}
\begin{minipage}[t]{0.05\textwidth}
\quad
\end{minipage}
\begin{minipage}[t]{0.4\textwidth}
\begin{tabbing}
\Instr \TABDEF @MMMM@  \TABSKIP $\Chan$ \TABSKIP $\Chan$ \TABSKIP $\Exp$ \kill

\Var,~$x$ \> $\to$ \> (value variable) \\
\Chan,~$c$ \> $\to$ \> (channel/stream name) \\
\Label,~$l$ \> $\to$ \> (label name) \\
\\
\\

\Instr
    \> :=\> @pull@  \> \Chan  \> \Var  \> \Next \\
    \TABALT @drop@  \> \Chan  \>       \> \Next \\
    \TABALT @push@  \> \Chan  \> \Exp  \> \Next \\
    \TABALT @case@  \> \Exp   \> \Next \> \Next \\
    \TABALT @jump@  \>        \>       \> \Next \\
\\
\\
\Next \> := \> $\Label~\times~\MapType{\Var}{\Exp}$ \\
\end{tabbing}
\end{minipage}

\caption{Grammar and types for defining processes. Each process is a sequential stream computation, networks of which are evaluated concurrently with a single-element bounded buffer for each stream.}
\label{fig:Process:Def}
\end{figure}




The grammar for process definitions is given in Fig.~\ref{fig:Process:Def}. Variables, Channels and Labels are specified by unique names. We refer to the \emph{endpoint} of a stream as a channel. A particular stream may flow into the input channels of several different processes, but can only be produced by a single output channel. For values and expressions we use an untyped lambda calculus with a few primitives chosen to facilitate the examples. The `$||$' operator is boolean-or, `+' addition, `/=' not-equal, and `$<$' less-then.

A $\Proc$ is a record with five fields: the @ins@ field specifies the input channels; the @outs@ field the output channels; the @heap@ field the process-local heap; the @label@ field the label of the currently being executed instruction, and the @instrs@ a map of labels to instructions. We use the same record when specifying both the definition of a particular process, as well as when giving the evaluation semantics. When specifying a process the @label@ field gives the entry-point to the process code, though during evaluation it is the label of the instruction currently being executed. Likewise, when specifying a process we usually only list channel names in the @ins@ field, though during evaluation they are also paired with their current $\InputState$. If an $\InputState$ is not specified we assume it is `none'.

In the grammar of Fig.~\ref{fig:Process:Def} the $\InputState$ has three options: @none@, which means no value is currently stored in the associated stream buffer variable, $(@pending@~\Value)$ which gives the current value in the stream buffer variable and indicates that it has not yet been copied into a process-local variable, and @have@ which means the pending value has been copied into a process-local variable. The $\Value$ attached to the @pending@ state is used when specifying the evaluation semantics of processes. During fusion, the $\Value$ itself will not be known, but we can still reason statically that a process must be in the @pending@ state. We will use a different version of $\InputState$ in \S\ref{s:Fusion} when we define the fusion algorithm.

The @instrs@ field of the $\Proc$ maps labels to instructions. The possible instructions are: @pull@, which pulls the next value from a channel into a given heap variable; @push@, which pushes the value of an expression to an output channel; @case@ which branches depending on the result of a boolean expression; @jump@ which causes control to move to a new instruction, and @drop@ which indicates that the current value pulled from a channel is no longer needed. 

All instructions include a $\Next$ field which is a pair of the label of the next instruction to execute, as well as a list of $\Var \times \Exp$ bindings used to update the heap. The list of update bindings is attached directly to instructions to make the fusion algorithm easier to specify, in contrast to a presentation with a separate @update@ instruction. 

As with Kahn processes~\cite{kahn1976coroutines}, pulling from a channel is blocking. Unlike Kahn processes, pushing to a channel can also block. Each consumer has a single element buffer, which is the stream buffer variable, and pushing can only succeed when that buffer is empty.

When lowering process code to a target language, such as C, LLVM, or some sort of assembly code, we can safely ignore the @drop@ instructions. The @drop@ instructions are used to help control how processes being fused should be synchronized, but do not affect the execution of a single process. We will discuss @drop@s further in \S\ref{s:Optimisation}.


% -----------------------------------------------------------------------------
\subsection{Execution}
\label{s:Process:Eval}

%!TEX root = ../Main.tex

\begin{figure}

$$
\arrLR{
  \boxed{\ProcInject{\Proc}{\Chan}{\Value}{\Proc}}
}{
  \boxed{\ProcsInject{\sgl{\Proc}}{\Chan}{\Value}{\sgl{\Proc}}}
}
$$

$$
\ruleIN{
  c=@none@ \in @ins@~p
}{
  \ProcInject{p}{c}{v}{p ~@ins@~ \{ c = @pending@~v\}}
}{InjectValue}
\ruleIN{
  c \not\in @ins@~p
}{
  \ProcInject{p}{c}{v}{p}
}{InjectIgnore}
$$

$$
\ruleIN{
  \forall i.~ \ProcInject{p_i}{c}{v}{p'_i}
}{
  \ProcsInject{\sgl{p_i}}{c}{v}{\sgl{p'_i}}
}{ProcessesInject}
$$

\caption{Process evaluation: inject}
\label{fig:Process:Eval:Inject}
\end{figure}


\begin{figure}

$$
\alpha~@:=@~ \Push~\Chan~\Value ~|~ \tau
$$

$$
  \boxed{
    \ProcBlockShake
      {\Instr}{\MapType{\Chan}{\InputState}}{\Sigma}
      {\alpha}
      {\Label}{\MapType{\Chan}{\InputState}}{\phi}
  }
$$


$$
\ruleIN{
  c=@pending@~v \in i
}{
  \ProcBlockShake{@pull@~c~x~l[\phi]}{i}{\Sigma}{\tau}{l}{i[c=@have@]}{\phi,~x = v}
}{Pull}
\ruleIN{
  c=@have@ \in i
}{
  \ProcBlockShake{@drop@~c~l[\phi]}{i}{\Sigma}{\tau}{l}{i[c=@none@]}{\phi}
}{Drop}
$$

$$
\ruleIN{
  \ExpEval{\Sigma}{e}{v}
}{
  \ProcBlockShake{@push@~c~e~l[\phi]}{i}{\Sigma}{\Push~c~v}{l}{i}{\phi}
}{Push}
\ruleIN{
}{
  \ProcBlockShake{@jump@~l[\phi]}{i}{\Sigma}{\tau}{l}{i}{\phi}
}{Jump}
$$

$$
\ruleIN{
  \ExpEval{\Sigma}{e}{@true@}
}{
  \ProcBlockShake{@case@~e~l_t[\phi_t]~l_f[\phi_f]}{i}{\Sigma}{\tau}{l_t}{i}{\phi_t}
}{CaseT}
\ruleIN{
  \ExpEval{\Sigma}{e}{@false@}
}{
  \ProcBlockShake{@case@~e~l_t[\phi_t]~l_f[\phi_f]}{i}{\Sigma}{\tau}{l_f}{i}{\phi_f}
}{CaseF}
$$

$$
  \boxed{\ProcShake{\Proc}{\alpha}{\Proc}}
  \quad
  \boxed{\ProcsShake{\sgl{\Proc}}{\alpha}{\sgl{\Proc}}}
$$

$$
@let@~@block@~p~=~@blocks@~p~(@label@~p)
$$
$$
\ruleIN{
  \ProcBlockShake
    {@block@~p} {@ins@~p}{@heap@~p}
    {\alpha}
    {l}{i}{\phi}
  \quad
  \forall u~|~x_u=e_u \in \phi.~
    \ExpEval{@heap@~p}{e_u}{v_u}
}{
  \ProcShake{p}{\alpha}{p~@label@~=~l,~@heap@~=~@heap@[\sgl{x_u=v_u}],~@ins@~=~i}
}{Shake}
$$




$$
\ruleIN{
  \ProcShake{p_i}{\tau}{p'_i}
}{
  \ProcsShake{
    \sgl{p_0 \ldots p_i \ldots p_n}
  }{\tau}{
    \sgl{p_0 \ldots p'_i \ldots p_n}
  }
}{ProcessesInternal}
$$

$$
\ruleIN{
  \ProcShake{p_i}{\Push~c~v}{p'_i}
  \quad
  \forall j~|~j \neq i.~
  \ProcInject{p_j}{c}{v}{p'_j}
}{
  \ProcsShake{
    \sgl{p_0 \ldots p_i \ldots p_n}
  }{\Push~c~v}{
    \sgl{p'_0 \ldots p'_i \ldots p'_n}
  }
}{ProcessesPush}
$$


\caption{Process evaluation: shake}
\label{fig:Process:Eval:Shake}
\end{figure}


%!TEX root = ../Main.tex

\begin{figure}

% ---------------------------------------------------------
$$
  \boxed{\ProcsShake{\sgl{\Proc}}{\Action}{\sgl{\Proc}}}
$$

$$
\ruleIN{
  \ProcShake{p_i}{\cdot}{p'_i}
}{
  \ProcsShake{
    \sgl{p_0 \ldots p_i \ldots p_n}
  }{\cdot}{
    \sgl{p_0 \ldots p'_i \ldots p_n}
  }
}{ProcessesInternal}
$$

$$
\ruleIN{
  \ProcShake{p_i}{\Push~c~v}{p'_i}
  \quad
  \forall j~|~j \neq i.~
  \ProcInject{p_j}{c}{v}{p'_j}
}{
  \ProcsShake{
    \sgl{p_0 \ldots p_i \ldots p_n}
  }{\Push~c~v}{
    \sgl{p'_0 \ldots p'_i \ldots p'_n}
  }
}{ProcessesPush}
$$


% ---------------------------------------------------------
\vspace{1em}

\newcommand\vs {\ti{vs}}
\newcommand\accs {\ti{accs}}
\newcommand\network {\ti{ps}}

$$
  \boxed{
    \ProcsFeed
      {(\Chan \mapsto \overline{Value})~}
      {\sgl{\Proc}}
      {(\Chan \mapsto \overline{Value})~}
      {\sgl{\Proc}}
  }
$$
$$
\ruleIN{
  \ProcsShake
    {ps}
    {\cdot}
    {ps'}
}{
  \ProcsFeed
    {cvs}
    {ps}
    {cvs}
    {ps'}
}{FeedInternal}
$$


% $$
% \ruleIN{
%   \forall c \in \accs.~
%   \accs~c~=~[]
% }{
%   \ProcsFeed
%     {\accs}
%     {\network}
%     {\accs}
%     {\network}
% }{FeedStart}
% $$

% $$
% \ruleIN{
%   \ProcsFeed
%     {\accs}
%     {\network}
%     {\accs'}
%     {\network'}
% \quad
%   \ProcsShake
%     {\network'}
%     {\tau}
%     {\network''}
% }{
%   \ProcsFeed
%     {\accs}
%     {\network}
%     {\accs'}
%     {\network''}
% }{FeedInternal}
% $$



% $$
% \ruleIN{
%   \ProcsFeed
%     {\accs}
%     {\network}
%     {\accs'}
%     {\network'}
% \quad
%   \ProcsShake
%     {\network'}
%     {\Push~c~v}
%     {\network''}
% }{
%   \ProcsFeed
%     {\accs}
%     {\network}
%     {c=\accs'~c \listappend [v], \accs'}
%     {\network''}
% }{FeedPush}
% $$


$$
\ruleIN{
  \ProcsShake
    {ps}
    {\Push~c~v}
    {ps'}
}{
  \ProcsFeed
    {cvs}
    {ps}
    {cvs[c \mapsto (cvs[c] \listappend v)]}
    {ps'}
}{FeedPush}
$$


% $$
% \ruleIN{
%   (\forall p \in \network.~c \not\in @outs@~p)
% \quad
%   \ProcsFeed
%     {c=\vs, \accs}
%     {\network}
%     {\accs'}
%     {\network'}
% \quad
%   \ProcsInject
%     {\network'}
%     {c}{v}
%     {\network''}
% }{
%   \ProcsFeed
%     {c=\vs \listappend [v], \accs}
%     {\network}
%     {c=\vs \listappend [v], \accs'}
%     {\network''}
% }{FeedExternal}
% $$


$$
\ruleIN{
  (\forall p \in \network.~c \not\in p[@outs@])
\quad
  \ProcsInject
    {ps}
    {c}
    {v}
    {ps'}
}{
  \ProcsFeed
    {cvs[c \mapsto ([v] \listappend vs)]}
    {ps}
    {cvs[c \mapsto vs~]}
    {ps'}
}{FeedExternal}
$$



\caption{Feeding Process Networks}
\label{fig:Process:Eval:Feed}
\end{figure}



Execution of process networks consists of three aspects:

\begin{enumerate}
\item \emph{Injection}, which attempts to provide a single value gained from a stream to a process, or a set of processes. Each individual process only needs to accept the value when it is ready for it, and injection of a value into a set of processes succeeds only when they \emph{all} accept it.

\item \emph{Shaking}, which advances a single process from one instruction to another. Shaking a set of processes succeeds when \emph{any} of the processes in the set can advance.

\item \emph{Feeding}, which manages communication between separate processes in the network. Feeding alternates between Injecting and Shaking. When a process pushes a value to an output channel we attempt to inject this value into all processes that have that same channel as an input. If they all accept it then we then advance their programs as far as they will go, which may cause more values to be pushed to output channels, and so on.

\end{enumerate}

Evaluation of a process network is non-deterministic. At any moment several processes may be able to take a step, while others are blocked trying to pull or push to their channels. However, because each process itself is deterministic, and uses blocking reads, the sequence of values pushed to each stream is deterministic, as per Kahn process networks. 

Importantly, it is the order in which values are \emph{pushed to each particular output channel} which is deterministic, whereas the order in which different processes execute their instructions, is not. When we fuse two processes together we exploit this fact by choosing one particular instruction ordering that enables the process network to advance without requiring unbounded buffering.

Each output channel may be pushed to by a single process only, so in a sense each output channel is ``owned'' by a single process. The only intra-process communication is via channels and streams. Our model is ``pure data flow'' (or perhaps ``functional data flow'') as there are no side-channels between the processes. This is in contrast to systems such as StreamIt~\cite{thies2002streamit}, where the processes are also able to send asynchronous messages to each other, in addition to the formal input and output streams.


% -----------------------------------------------------------------------------
\eject{}
\subsubsection{Injection}
Fig.~\ref{fig:Process:Eval:Inject} gives the rules for injecting values into processes. A judgment of form $(\ProcInject{p}{v}{c}{p'})$ reads ``given process $p$, injecting value $v$ into channel $c$ yields a new process $p'$''. The @injects@ form is similar, but operates on sets of processes rather than a single one.

Rule (InjectValue) injects of a single value into a single process. The value is stored as a (@pending@~ v) binding in the $\InputState$ for the associated channel of the process. The $\InputState$ acts as a single element buffer, and must be empty (set to @none@) for the injection to succeed.

Rule (InjectIgnore) allows processes that do not use a particular named channel to ignore values injected into that channel.

Rule (InjectMany) attempts to inject a single value into a set of processes. We use the single process judgment form to inject the value into all processes in the set, which must succeed for all of the. Once a value has been injected into all consuming processes that require it, it is then safe for a producing process to discard it.


% -----------------------------------------------------------------------------
\subsubsection{Shaking}
Fig.~\ref{fig:Process:Eval:Shake} gives the rules for advancing a single process. The first set of rules handle specific instructions. A judgment of form $(\ProcBlockShake{i}{is}{\Sigma}{\alpha}{l}{is'}{us})$ reads ``instruction $i$, given channel states $is$ and the heap $\Sigma$, passes control to instruction at label $l$ and yields new channel states $is$, heap update expressions $us'$, and performs an output action $\alpha$.'' An output action $\alpha$ is a list of messages of the form $(\Push~\Chan~\Value)$, which encode the values a process pushes to its output channels. We write ~$\cdot$~ for an empty action. 

\eject{}
Rule (Pull) takes the @pending@ value $v$ from the channel state and produces a heap update that will copy this value into the variable $x$ named in the @pull@ instruction. We use the syntax $us \rhd x=v$ to mean that the list of updates $us$ is extended with the new binding $x=v$. In the result channel states, the state of the channel $c$ that was pulled from is set to @have@, to indicate the value has been copied.

Rule (Push) evaluates the expression $e$ under heap $\Sigma$ to a value $v$, and produces a corresponding action which carries this value. The judgment $(\Sigma \vdash e \Downarrow v)$ is standard simply typed lambda calculus reduction using the heap $\Sigma$ for the values of free variables. As this evaluation is completely standard we omit it to save space.

Rule (Drop) changes the input buffer state from @have@ to @none@. A drop can only be executed after pull has set the input buffer to @have@. Rule (Jump) simply produces a new label and update expressions. Rules (CaseT) and (CaseF) evaluate the scrutinee $e$ and jump to the appropriate label.

A judgment of form $\ProcShake{p}{\alpha}{p'}$ reads ``process $p$ advances to new process $p'$, yielding action $\alpha$''. Rule (Shake) advances a whole process. We lookup the current instruction pointed to by the processes @label@ and pass it, along with the current channel states and heap to the previous single-instruction judgment. The update expressions @us@ that the single-instruction judgment yields are first reduced to values before updating the heap. The update expressions themselves are all pure, so the evaluation can safely be done in parallel (or in arbitrary order).


% -----------------------------------------------------------------------------
\subsubsection{Feeding}
Fig.~\ref{fig:Process:Eval:Feed} gives the rules for collecting the output actions and feeding output values to other processes. The first set of rules concerns feeding values to other processes within the same process network, while the second exchanges input and output values with the environment the process network is running in.

\eject{}
A judgment of form $\ProcShake{ps}{\alpha}{ps'}$ reads ``the processes group $ps$ advances to the new process group $ps'$ yielding output action $\alpha$. A process ``group'' is just a set of processes. 

Rule (ProcessInternal) allows an arbitrary process in the group to advance to a new state at any time, provided it does not produce an output action. This allows processes to perform internal computation, without needing to synchronize with the rest of the group.

Rule (ProcessPush) allows an arbitrary process in the group to advance to a new state, while producing an output action (@push@ c v). For this to happen it must be possible to inject the output value @v@ into any processes that has channel @c@ as one of its inputs. As all consuming processes must accept the output value at the time it is created, there is no need to buffer it further in the producing process. When any process in the group produces an output action then we take that as the action of the whole group, as it might also need to be sent to the environment. 

\smallskip

The judgment form for feeding is $\ProcsFeed{\ti{inputs}}{\ti{network}}{\ti{streams}}{\ti{network}'}$.
The input map $\ti{inputs}$ contains values for the network inputs: network outputs are not allowed, but ignored channels can have values.
The result $\ti{streams}$ contains the original inputs as well as accumulated output values.
Feeding evaluates the process network until all input values have been injected.

% Evaluation: feeding evaluates a process network on a list of input values and collects the outputs. Feeding alternates between injecting input values and shaking the processes to collect the output messages.

% Note that the result stream and network are not canonical, as an infinite @push@ loop has an infinite number of evaluations.
% The feed form does not ensure that the processes themselves have finished evaluating, only that all input values have been injected.

Rule (FeedStart) applies when all input values have been injected and there are no input values left.
In this case, the output values are the same as the input values.

Rule (FeedInternal) allows the process network to take an internal step.
It first feeds its input accumulator and process network, then allows the resulting network to take an internal step.

Rule (FeedPush) allows the process network to emit a push message.
As with (FeedInternal), it first feeds its input accumulator, then allows the resulting network to emit a push message.
The pushed value is collected in the accumulator list for that stream.

Rule (FeedExternal) allows inputs to be injected into the process network.
For any channel $c$ which is not an output of one of the processes, we take the last value off its list.
The recursive feed is evaluated with the last value removed from the accumulators.
The last value is then injected into the network, and added back to the result accumulators.



% -- cuts ---------------------------------------------------------------------
% BL: Discuss this during def of evaluation.

% These input states are used for evaluation to ensure that communication between processes does not require unbounded buffers.

% Claim "functional dataflow" or some such. Each process only updates values in its local heap, the only intra-process communication is via streams. This is unlike StreamIt which can send out-of-band messages between its processes.

% The output streams are in some sense ``owned'' by the process that produces them: while a stream may be consumed by any number of processes, each stream can only appear as the output for one process. This ensures a sort of determinism in the scheduling of multiple processes; if different processes could push to the same stream, the order of values would depend on the scheduled order. A process may, however, produce multiple output streams.

% Each process has its own private heap, therefore the only communication between processes occurs by streams.

% The instructions (@instrs@) are a mapping from label to instruction, and label points to the current instruction. Instructions can pull from a channel, drop an already pulled value, push a value, perform an if/case analysis on a boolean, or perform an internal jump.


% After values have been pulled, they must be disposed of with @drop@: this empties the value from the buffer and allows the producer to push to the channel.

% A process network is a set of multiple processes that can be evaluated concurrently. Any inputs that are not produced as outputs of processes are assumed to be external inputs --- their values will be provided by the environment. Processes form the essence of stream computation, and a single process can be given a straightforward sequential semantics by mapping to an imperative language. By fusing multiple processes into a single one, we are effectively giving a sequential interpretation for concurrent processes.

