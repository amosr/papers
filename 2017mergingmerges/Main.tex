\documentclass[acmlarge,review,anonymous]{acmart}\settopmatter{printfolios=true}

\usepackage{amssymb}
\usepackage{amsthm}
\usepackage{graphicx}
\usepackage{amsmath}
\usepackage{mathptmx}
\usepackage{mathtools}
\usepackage{stmaryrd}
\usepackage{hyperref}
\usepackage{alltt}
\usepackage{url}
\usepackage{float}
\usepackage{style/utils}
\usepackage{style/code}
\usepackage{style/proof}
\usepackage{style/keywords}
\usepackage{style/layout}
\usepackage{style/judgements}

% for combinator pictures
\usepackage{tikz}
\usetikzlibrary{shapes,arrows}
\usepackage[outputdir={out/}]{dot2texi}

\setcopyright{none}
\bibliographystyle{ACM-Reference-Format}

% -----------------------------------------------------------------------------
\begin{document}

\title{Machine fusion}
\subtitle{Merging merges, more or less}

\author{Amos Robinson}
\affiliation{
  \position{UNSW (Australia)}
}
\email{amosr@cse.unsw.edu.au}

\author{Ben Lippmeier}
\affiliation{
  \position{Vertigo Technology (Australia)}
}
\email{benl@vergo.co}

\makeatactive
\begin{abstract}

Fusion is an essential part of optimising high-level array and streaming computations, as it reduces memory traffic and intermediate arrays.
The benefits of removing intermediate arrays are even more important as data sizes approach the size of memory.
When comparing fusion systems, three important criteria to consider are: whether the system supports splits, where a stream is used multiple times; whether it supports joins, where a combinator has multiple inputs; and whether arbitrary combinators such as @merge@ and segmented appends can be encoded.
Existing fusion systems support one or two of these, but not three.
We present a fusion system based on process calculus that supports all three: splits, joins and arbitrary combinators.

\amos{``Arbitrary combinators'' is not quite true. How can we distinguish combinators we support from 2013 data flow fusion paper?}

We encode each combinator as a separate process with any number of input and output channels.
Each process is sequential but multiple processes can be executed concurrently.
We give a concurrent execution semantics for multiple processes, but these are used only as a specification for how the fused program must behave.
The fused program itself is sequential and can easily be converted to simple imperative code.

Our fusion algorithm takes two concurrently executable processes and creates a sequential interleaving of the two such that they execute with no unbounded buffers.
If fusion would require unbounded buffers (or the fusion algorithm wrongly infers that it would) then fusion fails.
If fusion fails, the user can be presented with an error message telling them which combinators could not be fused.
For scenarios where fusion is required, this is a great advantage over fragile shortcut fusion systems.

Our system has been formalised in Coq where we have proved soundness of the fusion algorithm.
It is expressive enough to encode a wide range of combinators including operations on segmented arrays.
The version presented here deals with infinite streams, and we informally describe the extensions required to support finite streams.
\end{abstract}


\maketitle


%!TEX root = ../Main.tex
\section{Introduction}
\label{s:Introduction}

Suppose we have two input streams of numeric identifiers, and wish to perform some analysis on these identifiers. The identifiers from both streams arrive sorted, but may include duplicates. We wish to produce an output stream of unique identifiers from the first input stream, as well as produce the unique union of identifiers from both streams. Can we perform both of these tasks at once, without needing to read through the stream data multiple times, and without needing unbounded buffering? Here is how we might write the source code, where @S@ is for @S@-tream.
\begin{code}
  uniquesUnion : S Nat -> S Nat -> (S Nat, S Nat)
  uniquesUnion sIn1 sIn2
   = let  sUnique = group sIn1
          sMerged = merge sIn1 sIn2
          sUnion  = group sMerged
     in   (sUnique, sUnion)
\end{code}

In this implementation the @group@ operator filters out consecutive duplicates, while @merge@ combines two sorted streams so that the output remains sorted. This example has a few interesting properties. Firstly, the data-access pattern of @merge@ is \emph{value-dependent}, meaning that the order in which this operator pulls values from @sIn1@ and @sIn2@ depends on the values themselves. If all the values from @sIn1@ are smaller than the values in @sIn2@, then @merge@ will pull all values from @sIn1@ before pulling the rest from @sIn2@, and vice versa. Secondly, although @sIn1@ occurs twice in the program, at runtime we only want to handle the elements of each stream once. To achieve this, the compiled program must coordinate between the two uses of @sIn1@, so that values are only read when both the @group@ and @merge@ operators are ready to receive a new value. Finally, as the stream length is assumed to be unbounded, we cannot buffer an arbitrary number of elements read from either stream, or risk running out of local storage space.

For an implementation which does \emph{not} use stream fusion, we might implement each of the operators as a separate concurrent process, and send each identifier value using an intra-process communication mechanism. Developing such an implementation could be easy or hard, depending on what language features are available for concurrency. However, worrying about the \emph{performance tuning} of such a system, such as whether we need back-pressure, or how to chunk the stream data to reduce the amount of communication overhead, is invariably a headache. 

We might instead define some sort of uniform interface for data sources, with a single `pull' function that provides the next value in each stream. Each operator could be given this interface, so that the next value from each result stream is computed on demand. This is approach is commonly taken with implementations of physical operators in data base systems. However, this `pull only' model does not support operators with multiple outputs, such as our derived @uniquesUnion@ operator, at least not without unbounded buffering. Suppose a consumer pulls many elements from the result @sUnique@ stream. The implementation needs to pull the corresponding source elements from @sIn1@ \emph{as well} as buffering an arbitrary number of matching elements from @sIn2@. It needs to buffer an aribrary number of elements from @sIn2@ because there is no guarantee of when a consumer will also pull from the @sUnion@ result stream. Once that happens the elements from @sIn2@ no longer need to be retained, but not before.

Instead, for a single threaded program, we want to perform \emph{stream fusion}, which takes the dataflow network and produces a simple sequential loop that gets the job done without requiring extra process-control abstractions and without requiring unbounded buffering. Sadly, existing stream fusion transformations cannot handle our example. As observed by \citet{kay2009you}, both pull-based and push-based fusion have fundamental limitations. Pull-based systems such as short cut stream fusion~\cite{coutts2007stream} cannot handle cases where a particular stream or intermediate result is used by multiple consumers. We refer to this situation as a \mbox{\emph{split} --- in the} dataflow network the flow from input stream @sIn1@ is split into both the @group@ and @merge@ consumers. 

% Leave this to related work. We've already mentioned a canonical pull-based system.
% Recent work on stream fusion by \citet{kiselyov2016stream} uses staged computation to ensure all combinators are inlined, but for splits this causes excessive inlining which duplicates work, due to values of the source arrays being read multiple times.

Push-based systems such as foldr/build fusion~\cite{gill1993short} also cannot fuse our example because they do not support operators with multiple inputs. We refer to such a situation as a \emph{join} --- in our example the @merge@ operator expresses a join in the data-flow graph. Some systems support both pull and push: data flow inspired array fusion~\cite{lippmeier2013data} allows both splits and joins but only for a limited, predefined set of operators. More recent work on polarized data flow fusion~\cite{lippmeier2016polarized} \emph{is} able to fuse our example, but requires the program to be rewritten to use explicitly polarized stream types. 

% The mechanism that combines the implementations of both operators, to yield efficient imperative code also depends on the general purpose compiler optimisations implemented by GHC, and it can be difficult to tell if these have ``worked'' without inspecting the intermediate representations of the compiler.

Synchronous dataflow languages such as Lucy-n~\cite{mandel2010lucy} reject value-dependent operators such as @merge@, while general dataflow languages fall back on less performant dynamic scheduling for these cases \cite{bouakaz2013real}. The polyhedral array fusion model~\cite{feautrier2011polyhedron} is used for loop transformations in imperative programs, but operates at a much lower level. The polyhedral model is based around affine loops, which makes it difficult to support filter-like operators such as @group@ and @merge@.

In our new system we still view the program as a concurrent process network. Each operator is a separate process, and the stream data flows through communication channels between the processes. Each operator is expressed as a restricted, sequential imperative program with commands that include both @pull@ for reading from an input stream and @push@ for writing to an output stream. The fusion transform takes the concurrent process network and \emph{sequentializes} it into a single process by choosing a particular evaluation order that requires no unbounded intermediate buffers. When the fusion transformation succeeds we know it has worked. There is no need to inspect intermediate representations of the compiler to debug poor performance, which is a common problem in systems based on general purpose program transformations \cite{lippmeier2012:guiding}.

In summary, we make the following contributions:
\begin{itemize}
\item a process calculus for encoding infinite streaming programs (\S\ref{s:Processes});
\item an algorithm for fusing these processes, the first to support arbitrary splits and joins (\S\ref{s:Fusion});
\item numerical results that demonstrate that the algorithm is well behaved when the number of fused processes is large. The size of the fused result program is not excessive. \TODO{Ref}
\item a formalization and proof of soundness for the core fusion algorithm in Coq (\S\ref{s:Proofs});
\end{itemize}

Our fusion transformation for infinite stream programs could also serve as the basis for an \emph{array} fusion system, using a natural extension to finite streams. We discuss this extension in \S\ref{s:Finite}.

% TODO: We can't make the appendix a contribution because the reviewers are not required to read appendices.
% \item and show our processes are general enough for many combinators, including segmented operations (\S\ref{s:Combinators}).

% \ben{Add a few more sentences on related work. Explain how this work extends the old flow fusion paper. It is not short-cut fusion like Oleg's recent work. We are not in the same space as Fortran style array fusion transformations like polyhedral}

% BL: describe this later.
% Furthermore, the data-flow fusion system of~\cite{lippmeier2013data} only deals with a fixed set of baked-in combinators. 

% BL: Shift the detailed description into a later section.
% The example above has three combinators, so the process network has three processes.
% The two @writeFile@s outputs are treated as sinks that values can be pushed to at any time, and are not converted to processes.
% During code generation, any output values from the @uniques@ and @union@ streams are sent to the corresponding @writeFile@ sink, but we do not address code generation in this paper.

% The process for @uniques@ is defined by the @group@ combinator, and can be thought of as an imperative loop: first it reads from its input stream @file1@ and stores that in a local variable.
% It also keeps track of the last pulled value, and compares that against the newly read value.
% If they are different, it pushes the new value to its output stream @uniques@.
% In either case, it updates the last pulled value and loops back to the start to pull from @file1@ again.

% The process for @merged@ is defined by the @merge@ combinator, which starts by reading from both @file1@ and @file2@ and storing these in local variables.
% It then compares its two values to see which is the smaller.
% If the value from @file1@ is smaller, it pushes that value and pulls a new value from @file1@, otherwise it pushes the value from @file2@ and pulls from @file2@.
% This is performed in a loop.

% We fuse these two processes by interleaving the two such that the shared input @file1@ is only pulled from when both processes agree.
% The new process pulls from @file1@, which is copied to the variables for both processes.
% The @uniques@ process now has all it needs to execute, so it checks the value against the last pulled value, pushes if necessary, and goes back to try to pull from @file1@ again.
% At this stage the @merged@ process still has a value from @file1@ that it has pulled but not used, so @uniques@ cannot pull from @file1@ again.
% We now let @merged@ run, pulling from @file2@ and checking which is smaller.
% If the value from @file1@ is smaller, the value is emitted and @merged@ wishes to pull a new value from @file1@.
% Both processes now agree on pulling from @file1@ again, so the new value is pulled and @uniques@ can run again.
% Otherwise if the value from @file1@ is not smaller, the value from @file2@ is emitted and @merged@ pulls from @file2@ with no coordination required.

% If we wish to ensure that each value is only read from the file once, we must coordinate between the two use sites: when @uniques@ requires a new value it must ensure that @union@ is ready to receive a new value, and vice versa. Note that we cannot just execute @uniques@ while storing the read values in a buffer, as this may require more memory than is available.
% In order to fuse this example, we require both pull \emph{and} push streams.
% The input streams must be pull streams since the order values are required is determined by the @merge@ combinator.
% For the same reason, the outputs sent to each @writeFile@ must be push streams.

% Fusion for array programs is important for removing intermediate arrays, reducing memory traffic and reducing allocations.
% However, when dealing with data too large to fit in memory such as tables on disk, removing intermediate arrays becomes essential rather than just desirable.
% Attempting to create an intermediate array of such amounts of data would lead to thrashing and swapping to disk, or perhaps even running out of swap.
% For these situations, some sort of assurance of total fusion is required: either the program can be fused with no intermediate arrays or unbounded buffers, or it will not compile at all.


% Fusion eliminates intermediate array buffers and converts pipelines of array combinators into low-level iteration based loop code. Different fusion systems can handle 

% When comparing fusion systems, three important criteria to consider are: whether the system supports splits, where a stream is used multiple times; whether it supports joins, where a combinator has multiple inputs; and whether arbitrary combinators such as @merge@ and segmented appends can be encoded. Existing fusion systems support one or two of these, but not three. We present a fusion system based on process calculus that supports all three: splits, joins and arbitrary combinators.

% Our system has been formalised in Coq where we have proved soundness of the fusion algorithm. It is expressive enough to encode a wide range of combinators including operations on segmented arrays.

% \amos{``Arbitrary combinators'' is not quite true. How can we distinguish combinators we support from 2013 data flow fusion paper? Perhaps by mentioning value-dependent input / access patterns.}

% Leave this to the description of the algorithm, not the abstract.
% We encode each combinator as a separate process with any number of input and output channels. Each process is sequential but multiple processes can be executed concurrently. We give a concurrent execution semantics for multiple processes, but these are used only as a specification for how the fused program must behave. The fused program itself is sequential and can easily be converted to simple imperative code.

% Our fusion algorithm takes two concurrently executable processes and creates a sequential interleaving of the two such that they execute with no unbounded buffers.
% If fusion would require unbounded buffers (or the fusion algorithm wrongly infers that it would) then fusion fails.
% If fusion fails, the user can be presented with an error message telling them which combinators could not be fused.
% For scenarios where fusion is required, this is a great advantage over fragile shortcut fusion systems.

% BL: leave the apologies to the conclusion.
% The version presented here deals with infinite streams, and we informally describe the extensions required to support finite streams.

% BL: leave this to the main intro.
% optimising high-level array and streaming computations, as it reduces memory traffic and intermediate arrays. The benefits of removing intermediate arrays are even more important as data sizes approach the size of memory.

%!TEX root = ../Main.tex
\section{Processes}
\label{s:Processes}

Each combinator defines a stream process which is expressed in terms of a simple imperative program with a local heap. The process pulls data from an aribrary number if input streams and pushes data to at least one output stream. \ben{Mention how we're going to convert arrays to streams, but we don't need to give the details yet}. 


% -----------------------------------------------------------------------------
\subsection{Grouping}
\begin{figure}
\begin{alltt}
 group 
   = \(\lambda\) (s1: Stream a) (s2: Stream a). 
     \(\nu\) (first: Bool)  (last: a) (value: a) (L0..L3: Label).
\end{alltt}
\begin{code}
     process
     { ins:    { s1 }
     , outs:   { s2 }
     , heap:   { first = True, last = 0, value = 0 }
     , label:  L0
     , instrs: { L0 = pull s1 value                    L1 {}
               , L1 = case (first || (last /= value))  L2 {}  L3 {}
               , L2 = push s2 value                    L3 { last = value, first = False }
               , L3 = drop s1                          L0 {} } }
\end{code}
\caption{The group combinator}
\label{fig:Process:Group}
\end{figure}


The definition of the @group@ combinator which removes consecutive elements from its input stream is given in Figure~\ref{fig:Process:Group}. The @group@ combinator has two parameters, @s1@ and @s2@ which bind the input and output streams respectively. The \emph{nu-binders} like \mbox{$\nu$ @(first: Bool)@...} indicate that each time we instantiate the @group@ combinator we should create fresh names for @first@, @last@ and so on that do not conflict with other instantiations. 

The body of the combinator is a record that defines the process. The @ins@ field of the record defines the set of input streams and the @outs@ field the set of output streams. The @heap@ field gives the starting state of each of the local variables. The @instrs@ field contains a set of labeled instructions that define the program, while the @label@ field gives the label of the first instruction. 

The instruction @(pull s1 value L1 {})@ pulls the next element from stream @s1@, writes it into the heap variable @value@, then continues with the instruction at label @L1@. The set @{}@ at the end of the instruction can be used to update values in the heap, but as we do not need to perform heap updates in this instance we leave it empty. 

The instruction @(case (first || (last /= value)) L2 {} L3 {})@ checks whether the predicate @(first || last /= value)@ is true and if so continues with the instruction at label @L2@, otherwise continues with the instruction at label @L3@. 

The instruction @(push s2 value L3 { last = value, first = False })@ pushes the value @value@ to the stream @s2@ and continues with the instruction at label @L3@, once the heap has been updated to set variable @last@ to @value@ and @first@ to @False@. 

Finally, the instruction @(drop s1 L0 {})@ signals that the current element that we pulled from stream @s1@ is no longer required, and contines with the instruction at @L0@. This @drop@ instruction is a synchronisation primitive that is used in the fusion algorithm but has no direct operational meaning. We discuss @drop@ further in \TODO{ref}. 

Overall, the @first@ variable tracks whether we are dealing with the first value from the stream, @last@ holds the last value pulled from the stream (or 0 if none have been read yet), and @value@ holds the current value pulled from the stream. The process emits the first value pulled from the stream and every value that is different from the last one that was pulled. For example, when executed on the input stream $[1, 2, 2, 3]$, this process will produce the output $[1, 2, 3]$.

Note that we refer to @group@ itself as a \emph{combinator} as it is parameterised by the names of the streams it deals with (@s1@ and @s2@). Once the combinator has been instantiated to work on particular streams the result is a concrete \emph{process}. 


% Recall that @group@ removes consecutive duplicates from its input stream.
% It has one input, @file1@, and one output, @uniques@.

% There are three variables in the heap: @first@ is initialised to @True@ as the first pulled value has no last value to compare against; @value@ stores the most recently pulled value, and @last@ stores the last pulled value.

% The @last@ and @value@ variables are initialised to $0$, but could be initialised to any value: these initial values will not be used.
% The initial label is @L0@, which pulls from the input @file1@. This blocks waiting for an input value, and when one is received, it is stored in @value@ and the process moves to @L1@.
% Instruction @L1@ performs a case analysis on a boolean: if it is the first read value, or the last value is not equal to the most recent value, it jumps to @L2@; otherwise it jumps to @L3@.
% Instruction @L2@ pushes the most recent value to the output and jumps to @L3@, updating the last value to the most recent value, and setting first to @False@.
% Finally, the instruction for @L3@ `drops' the recently pulled value from @file1@ and jumps back to @L0@.
% This dropping is required to coordinate between multiple processes that read from the same input: now that the read value from @file1@ has been dropped, another process is free to pull the next value from @file1@ if it so wishes.



% -----------------------------------------------------------------------------
\subsection{Merging}
\begin{figure}
\begin{alltt}
 merge
   = \(\lambda\) (s1: Stream Nat) (s2: Stream Nat) (s3: Stream Nat). 
     \(\nu\) (x1: Nat)  (x2: Nat) (L0..L3: Label) (FIXME).
\end{alltt}
\begin{code}
     process
     { ins:    { s1, s2 }
     , outs:   { s3 }
     , heap:   { x1 = 0, x2 = 0 }
     , label:  L0
     , instrs: { L0 = pull s1 x1     L1 {}
               , L1 = pull s2 x2     C0 {}
               , C0 = case (x1 < x2) A0 {}  B0 {}
               , A0 = push s3 x1     A1 {}
               , A1 = drop s1        A2 {}
               , A2 = pull s1 x1     C0 {}
               , B0 = push s3 x2     B1 {}
               , B1 = drop s2        B2 {}
               , B2 = pull s2  x2    C0 {} } }
\end{code}
\caption{The merge combinator}
\label{fig:Process:Merge}
\end{figure}

The definition of the @merge@ combinator which merges two input streams is given in Figure~\ref{fig:Process:Merge}. The combinator binds the two input streams to @s1@ and @s2@, while the output stream is @s3@. The two heap variables @x1@ and @x2@ are used to store the currently read values from each input. The forms of the instructions used to define this combinator are the same as the @group@ combinator in the previous section. We only need the four basic @pull@, @case@, @push@ and @drop@ instructions.

As this process merges infinite streams, if we execute it using a finite prefix then the final state will be an intermediate one that may not yet have pushed all available output. For example, if we execute the process with the input streams $[1, 4]$ and $[2, 3, 100]$ then the values $[1, 2, 3, 4]$ will be pushed to the output. The machine will arrive at instruction @L?@ which blocks waiting for the next value to be pulled from @arr1@. We discuss how to handle finite streams later in ~\S\ref{s:Finite}


% The instructions at @L0@ and @L1@ initialise the process by reading the first values from each stream, then move to @C0@.
% Instruction @C0@ checks which value is smaller: if the value @a@ read from @file1@ is smaller, it moves to @A0@; otherwise it moves to @B0@.
% Instructions @A0@, @A1@ and @A2@ push @file1@'s value to the output, drop it, and pull another one before moving back to @C0@ to compare again.
% Instructions @B0@, @B1@ and @B2@ are equivalent, except performing on @file2@ instead.


% -----------------------------------------------------------------------------
\subsection{Fusion}
\includegraphics[scale=0.8]{figures/state-group.pdf}


We are now ready to take our @group@ and @merge@ processes and fuse them together. 
We create a new process with inputs @file1@ and @file2@ and outputs @uniques@ and @merged@.
In order to make the fused process easier to understand, we have performed some minor optimisations on the output of the fusion algorithm described in~\S\ref{s:Fusion}, such as removing @jump@s and merging the two variables @value@ and @a@ into a single variable @value_a@.
Rather than explaining the process in detail, we simply note that the fused process is an interleaved scheduling of the two, starting with executing @group@, then executing @merge@, and going back to @group@ whenever pulling from @file1@ again.
The labels @L0@-@L2@ and @A1@-@A4@ are much the same as those in @group@, while @C0@, @A0@-@A2@ and @B0@-@B2@ are similar to those in @merge@.

\begin{code}
process (group/merge)
     ins: file1 file2
    outs: uniques merged
    heap: {first = True, last, value_a, b}
   label: L0
  blocks: L0 = pull file1   value_a               L1
          L1 = case (or first (last /= value_a))  L2    L3
          L2 = push uniques value_a               L3{last = value_a, first = False}
          L3 = pull file2 b                       C0

          C0 = case (value_a < b)                 A0    B0

          A0 = push merged  value_a               A1
          A1 = drop file1                         A2
          A2 = pull file1   value_a               A3
          A3 = case (or first (last /= value_a))  A4    C0
          A4 = push uniques value_a               C0{last = value_a, first=False}

          B0 = push merged b                      B1
          B1 = drop file2                         B2
          B2 = pull file2  b                      C0
\end{code}


\clearpage{}
\subsection{Process definitions}

%% This figure is referenced below, in 'Process definitions', but putting it up here before the big fused process stops the fused one from splitting across multiple pages.
%!TEX root = ../Main.tex

\begin{figure}
\begin{minipage}[t]{0.4\textwidth}
\begin{tabbing}
\Instr \TABDEF @MMMM@  \TABSKIP $\Exp$ \TABSKIP $\Exp$ \TABSKIP $\Exp$ \kill

\Exp,~$e$ \> $\to$ \> $x~|~v~|~e~e $ \\
  \> $\enskip|~$ \> $ (e~@||@~e) ~|~ e+e ~|~ e~@/=@~e ~|~ e < e$ \\
\Value,~$v$ \> $\to$ \> $\mathbb{N}~|~\mathbb{B}~|~(\lambda{}x.~e)$ \\
$\Sigma$ \> $\to$ \> $\cdot~|~\Sigma,~x~=~v$ \\
\\

\Proc \>:=\> @process@ \\
M \= M \= \kill
\> \> @ins:   @  $\MapType{\Chan}{\InputState}$ \\
\> \> @outs:  @  $\sgl{\Chan}$ \\
\> \> @heap:  @  $\Sigma$ \\
\> \> @label: @  \Label \\
\> \> @instrs:@  $\MapType{Label}{\Instr}$ \\
\\
\Instr \TABDEF \kill
\InputState \> := \> @pending@~\Value~$|$~@have@~$|$~@none@

\end{tabbing}
\end{minipage}
\begin{minipage}[t]{0.05\textwidth}
\quad
\end{minipage}
\begin{minipage}[t]{0.4\textwidth}
\begin{tabbing}
\Instr \TABDEF @MMMM@  \TABSKIP $\Chan$ \TABSKIP $\Chan$ \TABSKIP $\Exp$ \kill

\Var,~$x$ \> $\to$ \> (value variable) \\
\Chan,~$c$ \> $\to$ \> (channel/stream name) \\
\Label,~$l$ \> $\to$ \> (label name) \\
\\
\\

\Instr
    \> :=\> @pull@  \> \Chan  \> \Var  \> \Next \\
    \TABALT @drop@  \> \Chan  \>       \> \Next \\
    \TABALT @push@  \> \Chan  \> \Exp  \> \Next \\
    \TABALT @case@  \> \Exp   \> \Next \> \Next \\
    \TABALT @jump@  \>        \>       \> \Next \\
\\
\\
\Next \> := \> $\Label~\times~\MapType{\Var}{\Exp}$ \\
\end{tabbing}
\end{minipage}

\caption{Grammar and types for defining processes. Each process is a sequential stream computation, networks of which are evaluated concurrently with a single-element bounded buffer for each stream.}
\label{fig:Process:Def}
\end{figure}




The grammar for processes is shown in figure~\ref{fig:Process:Def}.
Channels, labels and variables are specified by some external, globally unique set of names.
For values and expressions we use a simple untyped lambda calculus with a few primitives chosen to facilitate the examples.

\amos{Settle on a terminology for channels / streams. Are both necessary?}
Streams are the abstract data flowing through while channels are particular endpoints, so a stream can have multiple channels.
A stream can have at most one output channel, and any number of input channels.

A process defines a stream computation, taking any number of input streams and producing at least one output stream.
Processes have an optional name in parentheses used only for descriptive purposes, and does not need to be unique.
The input streams are paired with an input state which is used for coordinating multiple processes during evaluation; in process definitions before any evaluation has occurred this should be @none@.
The input states are explained in detail later in \S\ref{s:Process:Eval}.

The output streams are in some sense ``owned'' by the process that produces them.
While a stream may be consumed by any number of processes, each stream can only appear as the output for one process.
This ensures a sort of determinism in the scheduling of multiple processes; if different processes could push to the same stream, the order of values would depend on the scheduled order.
A process may, however, produce multiple output streams.

The heap is used for evaluation of expressions.
Each process has its own private heap, meaning that the only communication between processes occurs by streams.

The label is the current block, and each block is the instruction to be executed in the current state.
Instructions can pull from a stream, drop an already pulled value, push a value, perform an if/case analysis on a boolean, or perform an internal jump.

As usual with Kahn processes~\cite{kahn1976coroutines}, pulling from a channel is blocking.
Unlike normal Kahn processes however, pushing to a channel can also block: each consumer has a single-element buffer and pushing only succeeds when all buffers are empty.
After values have been pulled, they must be disposed of with @drop@: this empties the value from the buffer and allows the producer process to push to that stream.

All instructions take as an argument the next label state to jump to, which includes any variable updates that should be performed on the private heap at the same time.
Combining variable update with stream instructions simplifies the fusion process in~\S\ref{s:Fusion}, as some parts of fusion need to perform both at once.


A process network is a set of multiple processes that can be evaluated concurrently.
The intersection of all process outputs should be empty - there should be no overlap.
Any inputs that are not mentioned as outputs of processes are assumed to be external inputs - their values will be provided by the environment.
Processes form the essence of stream computation, and a single process can be given a straightforward sequential semantics by mapping to an imperative language.
By fusing multiple processes into a single one, we are effectively giving a sequential interpretation for concurrent processes.


% \subsection{Map/map}
% \label{s:Process:MapMap}
% 
% One of the simplest combinators is @map@.
% This might need to go elsewhere.
% As well as needing a better example than map/map.
% Let's start with the process definition for @map@.
% The inputs here is actually a map with @as=none@, but we leave the value off when it is @none@.
% The next labels for the instructions are also shortened as @map0@ instead of writing the empty heap update afterwards.
% Mention that the initial heap has the names but no values, but they could be initialised to whatever.
% It doesn't matter since they'll be written before anything is read.
% 
% \begin{code}
% map f = process (map f)
%      ins: as
%     outs: bs
%     heap: {a = 0}
%    label: map0
%   blocks: map0 = pull as    a  map1
%           map1 = push bs (f a) map2
%           map2 = drop as       map0
% \end{code}
% 
% As well as a ``combinator network'', a function comprised exclusively of process combinators.
% The input streams are supplied as arguments, and output streams as return values.
% \begin{code}
% mapMap f g xs
%  = let ys = map f xs
%        zs = map g ys
%    in  zs
% \end{code}
% 
% It is not hard to assume that, given the process definitions and a combinator network, we can produce a process network.
% This is simple enough for a paragraph prose description.
% It's just a bit of inlining and renaming everything to be unique.
% 
% \begin{code}
% process (map f)
%      ins: xs
%     outs: ys
%     heap: {x}
%    label: p0
%   blocks: p0 = pull xs    x  p1
%           p1 = push ys (f x) p2
%           p2 = drop xs       p0
% process (map g)
%      ins: ys
%     outs: zs
%     heap: {y}
%    label: q0
%   blocks: q0 = pull ys    y  q1
%           q1 = push zs (g y) q2
%           q2 = drop ys       q0
% \end{code}
% 
% Now we can perform some kind of fusion on this network, resulting in one process that computes, as output, both @ys@ and @zs@.
% Later, when producing imperative code for this, the output pushes to @ys@ can be ignored and changed to jumps, as the original combinator network did not return them.
% 
% \begin{code}
% process (map f / map g)
%      ins: xs
%     outs: ys zs
%     heap: {x, y, _ys}
%    label: p0q0
%   blocks: p0q0            = pull xs    x  p1q0
%           p1q0            = push ys (f x) p2q0-pending-ys { _ys = f x }
%           p2q0-pending-ys = drop xs       p0q0-pending-ys
%           p0q0-pending-ys = jump          p0q1-have-ys    { y = _ys }
%           p0q1-have-ys    = push zs (g y) p0q2-have-ys
%           p0q2-have-ys    = jump          p0q0
% \end{code}

\subsection{Evaluation}
\label{s:Process:Eval}

%!TEX root = ../Main.tex

\begin{figure}

$$
\arrLR{
  \boxed{\ProcInject{\Proc}{\Chan}{\Value}{\Proc}}
}{
  \boxed{\ProcsInject{\sgl{\Proc}}{\Chan}{\Value}{\sgl{\Proc}}}
}
$$

$$
\ruleIN{
  c=@none@ \in @ins@~p
}{
  \ProcInject{p}{c}{v}{p ~@ins@~ \{ c = @pending@~v\}}
}{InjectValue}
\ruleIN{
  c \not\in @ins@~p
}{
  \ProcInject{p}{c}{v}{p}
}{InjectIgnore}
$$

$$
\ruleIN{
  \forall i.~ \ProcInject{p_i}{c}{v}{p'_i}
}{
  \ProcsInject{\sgl{p_i}}{c}{v}{\sgl{p'_i}}
}{ProcessesInject}
$$

\caption{Process evaluation: inject}
\label{fig:Process:Eval:Inject}
\end{figure}


\begin{figure}

$$
\alpha~@:=@~ \Push~\Chan~\Value ~|~ \tau
$$

$$
  \boxed{
    \ProcBlockShake
      {\Instr}{\MapType{\Chan}{\InputState}}{\Sigma}
      {\alpha}
      {\Label}{\MapType{\Chan}{\InputState}}{\phi}
  }
$$


$$
\ruleIN{
  c=@pending@~v \in i
}{
  \ProcBlockShake{@pull@~c~x~l[\phi]}{i}{\Sigma}{\tau}{l}{i[c=@have@]}{\phi,~x = v}
}{Pull}
\ruleIN{
  c=@have@ \in i
}{
  \ProcBlockShake{@drop@~c~l[\phi]}{i}{\Sigma}{\tau}{l}{i[c=@none@]}{\phi}
}{Drop}
$$

$$
\ruleIN{
  \ExpEval{\Sigma}{e}{v}
}{
  \ProcBlockShake{@push@~c~e~l[\phi]}{i}{\Sigma}{\Push~c~v}{l}{i}{\phi}
}{Push}
\ruleIN{
}{
  \ProcBlockShake{@jump@~l[\phi]}{i}{\Sigma}{\tau}{l}{i}{\phi}
}{Jump}
$$

$$
\ruleIN{
  \ExpEval{\Sigma}{e}{@true@}
}{
  \ProcBlockShake{@case@~e~l_t[\phi_t]~l_f[\phi_f]}{i}{\Sigma}{\tau}{l_t}{i}{\phi_t}
}{CaseT}
\ruleIN{
  \ExpEval{\Sigma}{e}{@false@}
}{
  \ProcBlockShake{@case@~e~l_t[\phi_t]~l_f[\phi_f]}{i}{\Sigma}{\tau}{l_f}{i}{\phi_f}
}{CaseF}
$$

$$
  \boxed{\ProcShake{\Proc}{\alpha}{\Proc}}
  \quad
  \boxed{\ProcsShake{\sgl{\Proc}}{\alpha}{\sgl{\Proc}}}
$$

$$
@let@~@block@~p~=~@blocks@~p~(@label@~p)
$$
$$
\ruleIN{
  \ProcBlockShake
    {@block@~p} {@ins@~p}{@heap@~p}
    {\alpha}
    {l}{i}{\phi}
  \quad
  \forall u~|~x_u=e_u \in \phi.~
    \ExpEval{@heap@~p}{e_u}{v_u}
}{
  \ProcShake{p}{\alpha}{p~@label@~=~l,~@heap@~=~@heap@[\sgl{x_u=v_u}],~@ins@~=~i}
}{Shake}
$$




$$
\ruleIN{
  \ProcShake{p_i}{\tau}{p'_i}
}{
  \ProcsShake{
    \sgl{p_0 \ldots p_i \ldots p_n}
  }{\tau}{
    \sgl{p_0 \ldots p'_i \ldots p_n}
  }
}{ProcessesInternal}
$$

$$
\ruleIN{
  \ProcShake{p_i}{\Push~c~v}{p'_i}
  \quad
  \forall j~|~j \neq i.~
  \ProcInject{p_j}{c}{v}{p'_j}
}{
  \ProcsShake{
    \sgl{p_0 \ldots p_i \ldots p_n}
  }{\Push~c~v}{
    \sgl{p'_0 \ldots p'_i \ldots p'_n}
  }
}{ProcessesPush}
$$


\caption{Process evaluation: shake}
\label{fig:Process:Eval:Shake}
\end{figure}


%!TEX root = ../Main.tex

\begin{figure}

% ---------------------------------------------------------
$$
  \boxed{\ProcsShake{\sgl{\Proc}}{\Action}{\sgl{\Proc}}}
$$

$$
\ruleIN{
  \ProcShake{p_i}{\cdot}{p'_i}
}{
  \ProcsShake{
    \sgl{p_0 \ldots p_i \ldots p_n}
  }{\cdot}{
    \sgl{p_0 \ldots p'_i \ldots p_n}
  }
}{ProcessesInternal}
$$

$$
\ruleIN{
  \ProcShake{p_i}{\Push~c~v}{p'_i}
  \quad
  \forall j~|~j \neq i.~
  \ProcInject{p_j}{c}{v}{p'_j}
}{
  \ProcsShake{
    \sgl{p_0 \ldots p_i \ldots p_n}
  }{\Push~c~v}{
    \sgl{p'_0 \ldots p'_i \ldots p'_n}
  }
}{ProcessesPush}
$$


% ---------------------------------------------------------
\vspace{1em}

\newcommand\vs {\ti{vs}}
\newcommand\accs {\ti{accs}}
\newcommand\network {\ti{ps}}

$$
  \boxed{
    \ProcsFeed
      {(\Chan \mapsto \overline{Value})~}
      {\sgl{\Proc}}
      {(\Chan \mapsto \overline{Value})~}
      {\sgl{\Proc}}
  }
$$
$$
\ruleIN{
  \ProcsShake
    {ps}
    {\cdot}
    {ps'}
}{
  \ProcsFeed
    {cvs}
    {ps}
    {cvs}
    {ps'}
}{FeedInternal}
$$


% $$
% \ruleIN{
%   \forall c \in \accs.~
%   \accs~c~=~[]
% }{
%   \ProcsFeed
%     {\accs}
%     {\network}
%     {\accs}
%     {\network}
% }{FeedStart}
% $$

% $$
% \ruleIN{
%   \ProcsFeed
%     {\accs}
%     {\network}
%     {\accs'}
%     {\network'}
% \quad
%   \ProcsShake
%     {\network'}
%     {\tau}
%     {\network''}
% }{
%   \ProcsFeed
%     {\accs}
%     {\network}
%     {\accs'}
%     {\network''}
% }{FeedInternal}
% $$



% $$
% \ruleIN{
%   \ProcsFeed
%     {\accs}
%     {\network}
%     {\accs'}
%     {\network'}
% \quad
%   \ProcsShake
%     {\network'}
%     {\Push~c~v}
%     {\network''}
% }{
%   \ProcsFeed
%     {\accs}
%     {\network}
%     {c=\accs'~c \listappend [v], \accs'}
%     {\network''}
% }{FeedPush}
% $$


$$
\ruleIN{
  \ProcsShake
    {ps}
    {\Push~c~v}
    {ps'}
}{
  \ProcsFeed
    {cvs}
    {ps}
    {cvs[c \mapsto (cvs[c] \listappend v)]}
    {ps'}
}{FeedPush}
$$


% $$
% \ruleIN{
%   (\forall p \in \network.~c \not\in @outs@~p)
% \quad
%   \ProcsFeed
%     {c=\vs, \accs}
%     {\network}
%     {\accs'}
%     {\network'}
% \quad
%   \ProcsInject
%     {\network'}
%     {c}{v}
%     {\network''}
% }{
%   \ProcsFeed
%     {c=\vs \listappend [v], \accs}
%     {\network}
%     {c=\vs \listappend [v], \accs'}
%     {\network''}
% }{FeedExternal}
% $$


$$
\ruleIN{
  (\forall p \in \network.~c \not\in p[@outs@])
\quad
  \ProcsInject
    {ps}
    {c}
    {v}
    {ps'}
}{
  \ProcsFeed
    {cvs[c \mapsto ([v] \listappend vs)]}
    {ps}
    {cvs[c \mapsto vs~]}
    {ps'}
}{FeedExternal}
$$



\caption{Feeding Process Networks}
\label{fig:Process:Eval:Feed}
\end{figure}



We now describe evaluation of processes and process networks.
We split evaluation into three main parts:
\begin{itemize}
\item Injection, where values are inserted into a process' input buffer.
Injection is only possible when the process input buffer is empty.
\item Shaking, where a process takes a step from one label to another.
Shaking a process results in a new process as well as an output message.
If a process pushes to a stream, the push value must be able to be injected to other processes.
\item Feeding, where an environment of input values are fed to the processes, and output values are collected.
This is the `top level' of evaluation that uses both injection and shaking.
\end{itemize}

Evaluation of a process network is non-deterministic, in that at any point there are many possible processes that can take a step.
However, because each process itself is deterministic and has blocking reads, overall evaluation is deterministic as per Kahn process networks.
That is: the order in which values are pushed to different output streams is not deterministic, but the order and values for a particular output stream \emph{are} deterministic.

Note that while process network evaluation is non-deterministic and concurrent, evaluating a single process is sequential and deterministic: code generation for fused processes only needs to deal with the sequential case.

Rules in figure~\ref{fig:Process:Eval:Inject} are about injecting values into a process; these are the values used when the process performs a @pull@.
The injected values may be pushed values from other processes for internal streams, or may come from an external source for the overall network's input streams.
Injection is just about orchestrating values between processes, and no actual computation happens here; it just makes values available to be pulled.

Injection can only happen when a process is ready to receive more input.
A process has a single element buffer for each input, stored in its input state.
This can be either @none@ meaning an empty buffer, @pending@ meaning a single value has been added to the buffer but has not been read yet, or @have@ meaning the value was added and in the process of being used.

(InjectValue) allows a value to be injected only when the input state is @none@, meaning the buffer is empty.
An attempt to inject a value while the buffer is @pending@ or @have@ would require an unbounded (or at least multiple element) buffer.
Injecting the value puts the value as @pending@ in the buffer.

(InjectIgnore) allows processes that do not use a particular input stream to ignore an injected input.

(ProcessesInject) performs injection over a process network.
Every process in the network must have the value injected into it.
This means if multiple processes read from that stream, all input buffers for that stream must be empty.

The rules in figure~\ref{fig:Process:Eval:Shake} are the `shake' part of evaluation, where actual computation occurs. 
As usual, $\alpha$ denotes the message type, with $\tau$ being an internal message. The \Push~ message is a single value being output on a channel.


The judgment form for shaking a single instruction $\ProcBlockShake{b}{i}{\Sigma}{\alpha}{l'}{i'}{u'}$
executes an instruction $b$ with the input states $i$ and the heap $\Sigma$.
The output message $\alpha$ can be an internal state change or an emitted value.
The result also has the new label, the new input state buffers, and the substitution to apply to the heap.

The two judgment forms for shaking processes are $\ProcShake{p}{\alpha}{p'}$ and $\ProcsShake{\sgl{p}}{\alpha}{\sgl{p}}$.
The process shaking just shakes a single instruction and updates the process.
Shaking a process network chooses a single process to shake, then if the result is an emitted value, that value is injected into all the other processes in the network.

(Pull) takes an already injected value from the input buffer, which changes its state from @pending@ to @have@.
The result substitution sets the variable to the pulled value, as well as any substitutions in the \Next~ of the instruction.

(Drop) changes the input buffer state from @have@ to @none@. A drop can only be executed after pull.

(Push) evaluates the push expression $e$ under the heap, and sends the value as the message.

(Jump) simply returns the new label and substitution.

(CaseT) and (CaseF) evaluate the case expression $e$ and jump to the true or false label depending on the value.

(Shake) unwraps a single process and evaluates the instruction.
The instruction updates are evaluated and updated in the process heap.
It updates the process with the new label, input state and heap.

(ProcessesInternal) chooses one process from the network and evaluates it.
When the process evaluates with an internal message ($\tau$), the entire network evaluates by replacing that process.

(ProcessesPush) chooses one process from the network and evaluates it, where the process evaluates with a push message.
The emitted push message is then injected into all other processes in the network, which means they must either ignore the channel or be ready to add it to their buffer.
If the process tries to emit a push message but it cannot be injected into all other processes, the push fails and another process will be tried.
The entire network then emits the same message.

The rules in figure~\ref{fig:Process:Eval:Feed} are the `feed' part of evaluation, where external input values are fed into a process network and output values are accumulated.
The judgment form for feeding is $\ProcsFeed{\ti{inputs}}{\ti{network}}{\ti{streams}}{\ti{network}'}$.
The input map $\ti{inputs}$ contains values for the network inputs: network outputs are not allowed, but ignored channels can have values.
The result $\ti{streams}$ contains the original inputs as well as accumulated output values.
Feeding evaluates the process network until all input values have been injected.

% Note that the result stream and network are not canonical, as an infinite @push@ loop has an infinite number of evaluations.
% The feed form does not ensure that the processes themselves have finished evaluating, only that all input values have been injected.

(FeedStart) is the axiom form where all input values have been injected and there are no input values left.
This is the start of evaluation.

(FeedInternal) first recursively feeds its input accumulator and process network, then allows the resulting network to take an internal step.
The internal step does not affect the accumulators.
% The recursive closure is performed \emph{before} the internal step rather than after for proof engineering reasons: it allows an extra step to be added to the end of a feed evaluation relatively easily.

(FeedPush) works similarly to (FeedInternal) except that the process network emits a push message.
The pushed value is added to the end of the accumulator for that channel.

(FeedExternal) allows inputs to be injected into the process network.
For any channel $c$ which is not an output of one of the processes, we take the last value off its list.
The recursive feed is evaluated with the last value removed from the accumulators.
The last value is then injected into the network, and added back to the result accumulators.



% -- cuts ---------------------------------------------------------------------
% BL: I don't think describing iota works at this point. This combinator is not used in the motivating example, so skipping to it seems disjointed.

% Before describing the @group@ process, we start by looking at one of the simplest combinators, @iota@, which produces a stream of increasing numbers.
% It takes no inputs, and produces one output stream @xs@.
% Each process has its own local heap where the values are stored, and in this case we initialise the local variable @i@ to @0@.
% This variable will be incremented and pushed.
% Each process also has a current label, which denotes the instruction to perform next.
% The initial label for @iota@ is @L0@.
% Each process has a mapping from labels to instructions.
% In this case we have two instructions, @L0@ and @L1@.
% The instruction for @L0@ pushes the current value of variable @i@ onto the output stream, then proceeds to move %to label @L1@. Instruction @L1@ moves back to @L0@, while also incrementing the variable @i@.

% \begin{code}
% process (iota)
%      ins: 
%     outs: xs
%     heap: {i = 0}
%    label: L0
%   blocks: L0 = push xs i  L1
%           L1 = jump       L0{i = i + 1}
% \end{code}

% When executed, this program produces an infinite stream of increasing numbers: $0, 1, 2\ldots$ while the label alternates between @L0@ and @L1@.

% The @uniques = group file1@ is a more interesting example.


%!TEX root = ../Main.tex
\section{Fusion}
\label{s:Fusion}


%!TEX root = ../Main.tex

\begin{figure}

\begin{tabbing}
@MMMMMMMMMMMM@   \TABDEF \kill

$\InputState_1$ \> := \> @pending@ $~|~$ @have@ $~|~$ @none@
\\
$\Label_1$ \> := \> $@label@_1~\Label~\MapType{\Chan}{\InputState_1}$ \\
$\Label$   \> := \> $\ldots ~|~@label@'~\Label_1~\Label_1 ~|~ \ldots$ \\
\\

$\ChanType_2$   \> := \> $@in2@~|~@in1@~|~@in1out1@~|~@out1@$ \\
\\
\ti{fuseNest} \> $:$ \> $\sgl{\Proc} \to  \Maybe~\Proc$ \\
\ti{fusePair} \> $:$ \> $\Proc \to \Proc \to  \Maybe~\Proc$ \\
\ti{tryStepPair} \> $:$ \> $\MapType{\Chan}{\ChanType_2} \to \Instr \to \Label_1 \to \Instr \to \Label_1 \to  \Maybe~\Instr$ \\
\ti{tryStep} \> $:$ \> $\MapType{\Chan}{\ChanType_2} \to \Instr \to \Label_1 \to \Label_1 \to  \Maybe~\Instr$ \\
\end{tabbing}

\caption{Fusion types}
\label{fig:Fusion:Types}
\end{figure}

\begin{figure}

\begin{tabbing}
$\ChanType_2$   \TABDEF \kill

\TODO{deal with maybes} \\
\\
\ti{fuseNest} \> $~:$ \> $\sgl{\Proc} \to  \Maybe~\Proc$ \\
\ti{fuseNest}~$\sgl{p}$ \> $=$ \> $p$ \\
\ti{fuseNest}~$\ti{ps}$
    \> $~|$      \> $p \in \ti{ps} ~\wedge~ q \in \ti{ps} ~\wedge~ p \not\eq q ~\wedge~ \ti{connected}~p~q$ \\
    \> $=$ \> $\ti{fuseNest}~(\sgl{\ti{fusePair}~p~q}~\cup~\ti{ps} \setminus \sgl{p,q})$ \\
    \> $~|$      \> @otherwise@ \\
    \> $=$ \> $\bot$ \\
\\
\ti{fusePair} \> $~:$ \> $\Proc \to \Proc \to  \Maybe~\Proc$ \\
$\ti{fusePair}~p~q$ \> $=$ \\
@    process@ \\
@        ins: @ $\sgl{c~|~c=t \in \ti{cs},~t \in \sgl{@in1@,@in2@}} $ \\
@       outs: @ $\sgl{c~|~c=t \in \ti{cs},~t \in \sgl{@in1out1@,@out1@}} $ \\
@       heap: @ $@heap@~p~\cup~@heap@~q$ \\
@      label: @ $l_0$ \\
@     blocks: @ $\ti{go}~\sgl{}~l_0$ \\
@ where@ \\
MM\=MM\=~=~\=\kill
 \> \ti{cs} \> $=$ \> $\ti{channels}~p~q$ \\
 \> $l_0$   \> $=$ \> $@label@'~l_p~l_q$ \\
 \> $l_p$   \> $=$ \> $@label@_1~(@label@~p)~\sgl{c=none~|~c~\in~cs}$ \\
 \> $l_q$   \> $=$ \> $@label@_1~(@label@~q)~\sgl{c=none~|~c~\in~cs}$ \\
 \\
 \> $\ti{go}~\ti{bs}~l$ \\
 \> \> $~|$ \> $l~\in~\ti{bs}$ \\
 \> \> $=$  \> $\ti{bs}$ \\
 \> \> $~|$ \> $l~\in~\ti{bs}$ \\
 \> \> $=$ \> $@label@_1~(@label@~q)~\sgl{c=none~|~c~\in~cs}$ \\
\end{tabbing}
\caption{Fusion top-level definitions}
\label{fig:Fusion:Def:Top}
\end{figure}
\begin{figure}

\begin{tabbing}
$\ChanType_2$   \TABDEF \kill

\ti{tryStep} \> $:$ \> $\MapType{\Chan}{\ChanType_2} \to \Instr \to \Label_1 \to \Label_1 \to  \Maybe~\Instr$ \\
\end{tabbing}

\caption{Fusion step definitions}
\label{fig:Fusion:Def:Step}
\end{figure}


Figure~\ref{fig:Fusion:Types} shows the type definitions for fusion.
Fusion proceeds by taking two processes in a network, and creating a single process that computes both.
The new process uses labels made of the product of the two input processes, as well as the static part of the input state - that is, whether the input buffer is pending after injection, has been pulled, or is empty, but without the actual injected value.


$\InputState_1$ defines the statis input state. $\Label_1$ is a single process label with the set of input states. We add the constructor $@label@'$ to the $\Label$ type which is the pair of process labels with input states.

$\ChanType_2$ classifies the kind of channels and the communications between two processes.
Two processes can read from the same channel (@in2@), in which case pulling must be coordinated together.
When one process reads from a channel and the other ignores it (@in1@) no coordination is required.
When one process writes to a channel and the other reads (@in1out1@) the reading process must wait for the other to write.
Finally, when one process writes and the other ignores (@out1@) no coordination is necessary.
As explained in~\S\ref{s:Process:Eval}, it is not allowed for two processes to write to the same channel.

Before looking at the definitions, it is worth noting the types and purposes of important functions:
\begin{itemize}
\item
\ti{fuseNetwork} takes a process network and repeatedly fuses pairs together. 
It chooses arbitrary processes with shared channels and fuses them together.

\item
\ti{fusePair} fuses a pair of processes together.
The heap variables must be distinct (this can be ensured by renaming).
The two processes must have some shared channels; otherwise no coordination is required at all, and the fusion process will choose an arbitrary interleaving of the processes.
This restriction is not an issue in practice since there is no benefit to fusing unrelated processes.
\end{itemize}

Figure~\ref{fig:Fusion:Def:Top} contains the definitions of top-level fusion functions.

Figures~\ref{fig:Fusion:Def:Step} and~\ref{fig:Fusion:Def:StepPair} define the judgment rules for fusing a single instruction at a time.


\begin{figure}

\begin{tabbing}
$\ChanType_2$   \TABDEF \kill

\ti{channels} \> $:$ \> $\Proc \to \Proc \to \MapType{\Chan}{\ChanType_2}$ \\

  \> $=$    \> $\{ c=@MMMMMMM@~$\= \kill
$\ti{channels}~p~q$
  \> $=$    \> $\{ c=@in2@$
            \> $|~c~\in~(@ins@~p~\cap~@ins@~q) \}$ \\

  \> $\cup$ \> $\{ c=@in1@$
            \> $ |~c~\in~(@ins@~p~\cup~@ins@~q)~\wedge~c~\not\in(@outs@~p~\cup~@outs@~q) \}$ \\

  \> $\cup$ \> $\{ c=@in1out1@$
            \> $|~c~\in~(@ins@~p~\cup~@ins@~q)~\wedge~c~\in(@outs@~p~\cup~@outs@~q) \}$ \\

  \> $\cup$ \> $\{ c=@out1@$
            \> $ |~c~\not\in~(@ins@~p~\cup~@ins@~q)~\wedge~c~\in(@outs@~p~\cup~@outs@~q) \}$ \\
\\

\ti{connected} \> $:$ \> $\Proc \to \Proc \to  \mathbb{B}$ \\
$\ti{connected}~p~q$
  \> $=$ \> $(@ins@~p~\cup~@outs@~p)~\cap~(@ins@~q~\cup~@outs@~q)~\not\eq~\sgl{}$ \\
\end{tabbing}

\newcommand\funClauseDef[3]
{ $\ti{#1}~(#2)$ \> $=$ \> $#3$ \\
}
\newcommand\outlabelsDef[2]
{ \funClauseDef{outlabels}{#1}{\sgl{#2}} 
}

\begin{tabbing}
$\ti{outlabels}~(@case@~e~l_t[u_t]~l_t[u_f])$ \TABSKIP $=$ \TABSKIP \kill
\ti{outlabels} \> $~:$ \> $\Instr \to \sgl{\Label}$ \\
\outlabelsDef{@pull@~c~x~l[u]}{l}
\outlabelsDef{@drop@~c~l[u]}{l}
\outlabelsDef{@push@~c~e~l[u]}{l}
\outlabelsDef{@case@~e~l[u]~l'[u']}{l, l'}
\outlabelsDef{@jump@~l[u]}{l}
\end{tabbing}

\begin{tabbing}
$\ti{swaplabel}~(@case@~e~l_t[u_t]~l_t[u_f])$ \TABSKIP $=$ \TABSKIP \kill
\ti{swaplabel} \> $~:$ \> $\Label \to \Label$ \\
$\ti{swaplabel}~(@label@'~l_1~l_2)$ \> $=$ \> $@label@'~l_2~l_1$ \\
\end{tabbing}

\begin{tabbing}
$\ti{swaplabels}~(@case@~e~l_t[u_t]~l_t[u_f])$ \TABSKIP $=$ \TABSKIP \kill
\ti{swaplabels} \> $~:$ \> $\Instr \to \Instr$ \\
\funClauseDef{swaplabels}{@pull@~c~x~l[u]}{@pull@~c~x~(\ti{swaplabel}~l)[u]}
\funClauseDef{swaplabels}{@drop@~c~l[u]}{@drop@~c~(\ti{swaplabel}~l)[u]}
\funClauseDef{swaplabels}{@push@~c~e~l[u]}{@push@~c~e~(\ti{swaplabel}~l)[u]}
\funClauseDef{swaplabels}{@case@~e~l[u]~l'[u']}{@case@~e~(\ti{swaplabel}~l)[u]~(\ti{swaplabel}~l')[u']}
\funClauseDef{swaplabels}{@jump@~l[u]}{@jump@~(\ti{swaplabel}~l)[u]}
\end{tabbing}

\caption{Utility functions}
\label{fig:Fusion:Utils}
\end{figure}



Figure~\ref{fig:Fusion:Utils} contains definitions of some utility functions which are not specific to fusion.
\ti{channels} computes the $\ChanType_2$ map for a pair of processes.
\ti{connected} checks whether two processes have any shared inputs or outputs - basically whether it is worth fusing them.
\ti{outlabels} gets the set of output labels for an instruction - this is used when computing the fixpoint of the blocks map.
\ti{swaplabels} flips the order of the compound labels in an instruction.


%!TEX root = ../Main.tex

\section{Proofs}
\label{s:Proofs}

Our fusion system is formalised in Coq, where we have proved soundness of \ti{fusePair}: if the fused program evaluates to a particular output, then the two original programs also evaluate to that output.
It is interesting to note that the converse is not necessarily true: just because two programs can evaluate to a particular output does not mean the fused program will evaluate to that.
This is because, as explained in~\S\ref{s:EvaluationOrder}, evaluation of a process network is non-deterministic, and fusion commits to a particular evaluation order.

The system described here has some differences to our Coq formalisation.
First, the Coq formalisation has a separate @update@ instruction which modifies a variable in the local heap, rather than allowing heap updates in the output \Next~label of any instruction.
This causes the fusion definition to be slightly more complicated, as two output instructions must be emitted when performing a push or pull followed by an update.
This is a fairly minor difference, and we have made this change in the paper version for ease of exposition.
Ideally, a future version of the formalisation would include this change.
Secondly, our formalisation does not implement the concurrent evaluation semantics for processes, only sequential evaluation for a single process.
Instead we sequentially evaluate both processes separately with the same input values and outputs.

Despite these differences, we believe the Coq formalisation gives sufficient confidence in the correctness of the version presented here.


%!TEX root = ../Main.tex
\eject
\section{Related Work}
\label{s:Conclusion}

% This paper has introduced Icicle, a streaming query language. The streaming, single-pass nature of Icicle allows all queries over the same table to be fused into a single loop over the data.
% Icicle's modal type system allows incremental computation, ensuring that queries give the same value regardless of how the input is sliced up, and when the increments are performed.
% The modal types encode when computations are available, and disallows using the results of folds before they are finished.
% \ben{the above paragraph doesn't add any new information}.

In Icicle there is only one stream, sourced from the input table, which is implicit in the bodies of queries. This approach is intentionally simpler than existing synchronous data flow languages such as Lucy~\cite{mandel2010lucy}, as well as our prior work on flow fusion~\cite{lippmeier2013data}. Synchronous data flow languages implement Kahn networks~\cite{vrba2009kahn} that are restricted to use bounded buffering~\cite{johnston2004advances} by clock typing and causal analysis~\cite{stephens1997survey}. In such languages, stream combinators with multiple inputs, such as @zip@, are assigned types that require their stream arguments to have the same clock --- meaning that elements always arrive in lockstep and the combinators themselves do not need to perform their own buffering. In Icicle the fact that the input stream is implicit and distributed to all combinators means that we can forgo clock analysis. All queries in a program execute in lock-step on the same element at the same moment, which ensures that fusion is a simple matter of concatenating the components of the loop anatomy of each query.

Short-cut fusion techniques such as foldr/build~\cite{gill1993short} and stream fusion~\cite{coutts2007stream} rely on inlining to expose fusion opportunities. In Haskell compilers such as GHC, the decision of when to inline is made by internal compiler heuristics, which makes it difficult for the programmer to predict when fusion will occur. In this environment, array fusion is considered a ``bonus'' optimization rather than integral part of the compilation method. In contrast, for our feature generation application we really must ensure that multiple queries over the same table are fused, so we cannot rely on heuristics.

StreamIt~\cite{thies2002streamit} is an imperative streaming language which has been extended with dynamic scheduling~\cite{soule2013dynamic}. Dynamic scheduling handles data flow graphs where the transfer rate between different stream operators is not known at compile time. Dynamic scheduling is trade-off: it is required for stream operators such as grouping and filtering where the output data rate is not known statically, but using dynamic techniques for graphs with static transfer rates tends to have a performance cost. Icicle includes grouping and filtering operators where the output rates are statically unknown, however the associated language constructs require grouped and filtered data to be aggregated rather than passed as as the input to another stream operator. This allows Icicle to retain fully static scheduling, so the compiled queries consist of straight line code with no buffering.

Icicle is closely related to work in continuous and shared queries. A continuous query is one that processes input data which may have new records added or removed from it at any time. The result of the continuous query must be updated as soon as the input data changes. Shared queries are ones in which the same sub expressions occur in several individual queries over the same data, and we wish to share the results of these sub expressions among all individuals that use them. For example, in Munagala \emph{et al}~\cite{munagala2007optimization}, input records are filtered by a conjunction of predicates, and the predicates occur in multiple queries. Madden \emph{et al}~\cite{madden2002continuously} uses a predicate index to avoid recomputing them. Andrade \emph{et al} describes a compiler for queries over geospacial imagery~\cite{andrade2003efficient} that shares the results of several pre-defined aggregation functions between queries. Continuous Query Language (CQL)~\cite{arasu2002abstract,stream2003stream} again allows aggregates in its queries, but they must be builtin aggregate functions. Icicle addresses a computationally similar problem, except that our input data sets can only have new records added rather than deleted, which allows us to support general aggregations rather than just filter predicates. It is not obvious how arbitrary aggregate functions could be supported while also allowing deletion of records from the input data --- other than by recomputing the entire aggregation after each deletion.



% How to architect a query compiler~\cite{shaikhha2016architect}.
% Scheduling dynamic dataflow~\cite{buck1993scheduling}.
% Co-iterative characterization~\cite{caspi1998co}.

%!TEX root = ../Main.tex
\section{Extensions and future work}
\label{s:FutureWork}

We now discuss some of the shortcomings of the system, and extensions and future work to ameliorate this.

\subsection{Finite streams}
\label{s:Finite}

The processes we have seen so far deal with infinite streams, but in practice most streams are finite.
Certain combinators such as @fold@ and @append@ only make sense on finite streams, and others like @take@ produce inherently finite output.
We have focussed on the infinite stream version because it is somewhat simpler to explain and prove, but the extensions required to support finite streams do not require substantial conceptual changes.

We now describe the extensions required to support finite streams.
We add a new @closed@ constructor to the \InputState~ to encode the end of the stream.
Once an input stream is in the closed state, it can never change to another state: it remains closed thereafter.

We modify the @pull@ instruction so that it has two output labels (like @case@).
The first label, the read branch, is executed as before when the pull succeeds and a value is read from the stream.
The second label, the close branch, is executed when the stream is closed, and no more values will ever be available.
After a pull takes the close branch, any subsequent pulls from that stream will also take the close branch.

We add two new instructions for closing output streams and disconnecting from input streams.
Closing an output stream $(@close@~\Chan~\Goto)$ is similar to pushing an end-of-file marker to all readers.
As with @push@, the evaluation semantics of @close@ can only proceed if all readers are in a position to accept the end-of-file, but instead of setting the new \InputState~ to @pending@ with a value, the \InputState~ is set to @closed@.
After a stream has been closed, no further values can be pushed.

Disconnecting from input streams $(@disconnect@~\Chan~\Goto)$ signals that a process is no longer interested in the values of a stream.
This can be used when a process requires the first values of a stream, but does not require the whole stream.
If a process read the first values of a stream and then stopped pulling, its \InputState~ buffer would fill up and never be cleared, so no other process would be able to continue pulling from that stream.
Disconnecting the stream allows other processes to use the stream without the disconnected process getting in the way of computation.
The evaluation semantics for @disconnect@ remove the channel from the inputs of the process.
After removing the channel from the inputs, when a writing process tries to inject values, this process will just be ignored rather than inserting into the \InputState~ buffer and potentially causing writing to block.
After a process disconnects from an input channel, it can no longer pull from that channel.

We also add an instruction for terminating the process (@done@).
After all input streams have been read to completion or disconnected and output streams closed, the process may execute @done@ to signal that processing is complete.

The fusion definition must be extended to deal with these new instructions.
The static input state has a @closed@ constructor added and disconnection is encoded by removal from the input state, and the \ti{tryStep} changes more or less follow the evaluation changes.
Shared and connected pulls now deal with two more possibilities in the input state: the input may be closed in which case the close branch of the pull is taken; or the other process may have disconnected in which case the pull is executed as in the non-shared non-connected case.
Connected pushes must also deal with when the other process has disconnected in which case the push is executed as if it were non-connected.
For @in1@ and @out1@ channels, the new @close@ and @disconnect@ instructions are used as normal with no coordination required.
For connected @close@, as with @push@, the receiving process must have @none@ and the next step performs the @close@ and sets the input state to @closed@.
For shared @disconnect@, the @disconnect@ is only performed after both processes have disconnected; otherwise the entry is just removed from the input state.
For connected @disconnect@, the @disconnect@ is not performed and the entry is removed from the input state.

Finally, \ti{tryStepPair} is modified so that @done@ is performed when both machines are @done@.

These modifications allow our system to fuse finite streams as well as infinite.
We have implemented an initial prototype that supports finite streams, but future work is required to prove them correct.

\subsection{Fully abstract case interpretation}

This cannot be fused because it requires an unbounded buffer.
\begin{code}
zipltgts :: [a] -> [a*a]
zipltgts as =
  let as1 = filter (<0) as
      as2 = filter (>0) as
      aas = zip as1 as2
  in  aas
\end{code}

You might think the following can be fused.
It cannot because we treat @case@ conditions as fully abstract and make no attempt to filter out impossible combinations.
So while it cannot be true that the first filter reaches @>0 = true@ case and the second filter reaches @>0 = false@, we try both combinations and treat it as unfusable.
\begin{code}
zipgts :: [a] -> [a*a]
zipgts as =
  let as1 = filter (>0) as
      as2 = filter (>0) as
      aas = zip as1 as2
  in  aas
\end{code}


%!TEX root = ../Main.tex
\appendix
\section{Combinators}
\label{s:Combinators}

Some simple combinator definitions.
Map 

\begin{code}
map f = stream_1_1 \is os.
  letrec
    p1   = pull is p2
    p2 i = push os (f i) p3
    p3   = drop is p1
  in p1
\end{code}

\begin{code}
filter f = stream_1_1 \is os.
  letrec
    p1   = pull is p2
    p2 i = case (f x) (p3 i) p4
    p3 i = push os i p4
    p4   = drop is p1
  in p1
\end{code}

\begin{code}
partition f = stream_1_2 \is ots ofs.
  letrec
    p1   = pull is p2
    p2 i = case (f x)
            (push ts i p3)
            (push fs i p3)
    p3   = drop is p1
  in p1
\end{code}

\begin{code}
zip = stream_2_1 \xs ys xys.
  letrec
    p1     = pull xs        p2
    p2 x   = pull ys        p3
    p3 x y = push xys (x,y) p4
    p4     = drop xs        p5
    p5     = drop ys        p1
  in p1
\end{code}


\begin{code}
merge = stream_2_1 \xs ys xys.
  letrec
    go x y = case (x < y)
             (pX x y)
             (pY x y)
    pX x y = push xys x
            (drop xs
            (pull xs (\x'. go x' y)))
    pY x y = push xys y
            (drop ys
            (pull ys (\y'. go x y')))
  in pull xs (\x. pull ys (\y. go x y))
\end{code}






\bibliography{Main}

\end{document}


