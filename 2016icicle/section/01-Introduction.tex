%!TEX root = ../Main.tex
\section{Introduction}
\label{s:Introduction}

For queries over a large amount of data, care must be taken to reduce the number of iterations over the input data.
When performing multiple queries, it is important to limit the amount of duplicate work and iterations.

For example, suppose we have a table called @stocks@ with a price for each company, for each day.
Computing the sum and mean over all companies and all time is quite simple in SQL:
\begin{code}
SELECT SUM(price)
FROM stocks;

SELECT SUM(price) / COUNT(price)
FROM stocks;
\end{code}

It is also easy to express a query which computes both at once:
\begin{code}
SELECT SUM(price)
     , SUM(price) / COUNT(price)
FROM stocks;
\end{code}

Suppose we also wish to compute the mean of each company:
\begin{code}
SELECT company
     , SUM(price) / COUNT(price)
FROM stocks
GROUP BY company;
\end{code}

Now, it becomes less obvious how to express both in a single query.
One can use a nested query which may imply another iteration, or windowing functions and partition, but at this stage the resemblance to the original queries starts to diminish.
\begin{code}
SELECT company
     , SUM(price) / COUNT(price)
     , (SELECT SUM(price) / COUNT(price) FROM stocks)
FROM stocks
GROUP BY company;

SELECT company
     , SUM(price) / COUNT(price) OVER (PARTITION BY company)
     , SUM(price) / COUNT(price) OVER (PARTITION BY 1)
FROM stocks;
\end{code}

Even innocent-seeming features such as WHERE clauses (filters) require non-trivial rewriting to express as a single query.

\begin{code}
SELECT SUM(price) / COUNT(price)
FROM stocks
WHERE price > 0
GROUP BY company;
\end{code}


We now introduce Icicle, a streaming query language for fusing multiple queries (horizontal fusion).
By carefully limiting Icicle queries and supporting no sorting, subqueries or joins, we are able to merge all queries on the same input data, and reduce duplicate computation.


