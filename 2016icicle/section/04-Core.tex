%!TEX root = ../Main.tex
\section{Core language}
\label{s:Core}

This section presents the Core language, which we use as an intermediate language before converting to imperative code.
\TODO{Do we need formal semantics and type system?}


\subsection{Grammar}

%!TEX root = ../Main.tex

\begin{figure}

\begin{tabbing}
MMMM \= MM \= MM \= MM \= \kill
$\mi{e_c}$
\GrammarDef $x$ \> $|$ \> $v$
\GrammarAlt $e_c~e_c$ \> $|$ \> $\lambda{}x~:~\tau.~e_c$
\GrammarAlt $p$
\\
\\
$c$
\GrammarDef $x$ \> $|$ \> $\mi{Always}$
\\
\\
$\mi{FoldGroup}$
\GrammarDef $\mi{Op}*$
\\
\\
$\mi{Op}$
\GrammarDef $@Let@[c]~x~=~e_c$
\GrammarAlt $@Fold@~x~=~e_c,~(@When@~c~e_c)$
\\
\\
$\mi{Program}$
    \> $=$  \> @Inputs@ \\
    \>      \> \> $(x~:~@Stream@~c~\tau)*$ \\
    \>      \> @Operators@ \\
    \>      \> \> $\mi{FoldGroup}*$ \\
    \>      \> @OutStreams@ \\
    \>      \> \> $(x,~c)*$ \\
    \>      \> @OutFolds@ \\
    \>      \> \> $(x)*$ \\
\end{tabbing}


\caption{Grammar for Icicle Core}
\label{fig:core:grammar}
\end{figure}



\TODO{Think it might actually be better to put @Streams@ and @Folds@ together, and allow @Stream@ worker functions to refer to previous folds but only as scans. Would make implementation of scans easier?}

In figure~\ref{fig:core:grammar} we introduce the intermediate language.
The top-level of this language is $\mit{Program}$ with four stages of computation: pure computations, followed by stream transformers, then folds, and finally extracting the result of the folds.
Each stage of computation can only refer to the stages before it.

The expression type for Core is much simpler than Source, $\mit{CoreExp}$ having just names, primitive application and let bindings.
More primitives are required for the Core language, as some filters in the Source language are converted to @If@s, and groups are converted to use map primitives such as lookup and insert.
The @scan@ primitive is no longer used in the Core language as it is just converted to a variable reference.

The first section of the program is the @Befores@ section, which is just binding names to pure expressions.
These expressions cannot refer to any part of the stream, any computation, nor the results of any folds.

Next are @Streams@: streams and stream transformers.
The first stream type is @Source@, which is simply the unmodified input stream.
The two stream transformers take a function and the name of the input stream.
These are @Map@ which applies a function to each element, and @Filter@ which filters out according to a predicate.
Neither of the worker functions can refer to any streams, only the bindings in the @Befores@ section.
All of these stream transformers can be computed \emph{on-line} and by looking at a single element at a time, without any sort of accumulator.

In the third section are the @Folds@, which perform some reduction over an input stream.
Each fold has a constructor function which is successively applied to the latest accumulator and each stream value; an initialiser or zero value for the accumulator; and the name of the input stream.
The initialiser is pure, and so can only refer to variables defined in @Befores@.
The constructor function can refer to @Befores@, previous @Folds@, but not directly to the @Streams@.
Referring to previous @Folds@ in the constructor body gives the latest value of the accumulator, this is used to implement the @scan@ primitive.

Then the @Extracts@ are bound which can refer to the results of @Folds@ and any @Befores@.
These are used to perform and post-computation on the folds which cannot be done at each step, for example dividing the sum and the count to get the mean.

Finally, the @Outputs@ are defined, which define which variables are to be returned as the result of the query.
These can refer to @Befores@, @Folds@ and @Extracts@.

\subsection{Fusion}

When fusing together two queries, we can simply give each unique names, then append each respective section together; the @Befores@ appended to the @Befores@, @Streams@ to @Streams@ and so on.
After appending them together, we remove duplicate bindings by iterating through each binding set and comparing it to the previous bindings in the set.
If a duplicate is found, we remove the later binding and change all references in the rest of the program to the earlier binding.

The program definition is more complicated than necessary, as all programs of this form can actually be expressed as a single fold and an extract.
However, having smaller components makes finding duplicate computations after fusion easier.
For example, if mean is expressed as two folds (sum and count) then removing the duplicate after fusing with another count is rather simple.
If mean were expressed as a single fold containing both sum and count, removing the duplicate count would require somehow splitting apart the sum and count or removing duplicates some other way.


\subsection{Conversion from Source}

Yes yes convert from Source fix that

%!TEX root = ../Main.tex

\begin{figure*}

\newcommand \acc {\tau_\mathit{acc}}
\newcommand \xtr {\tau_\mathit{extract}}
\newcommand \lam[1] {\lambda{} #1.~}

$$
\boxed{\ToCoreX{x}{M~\tau}{@CoreExp@}}
$$

$$
\ruleI
{
}
{ 
    \ToCoreX{x}{\Agg{\tau}}{@bottom@~:~\Agg{\tau}}
}
\textrm{(CoreXAgg)}
$$
\TODO{This judgment is only used for pure or element computations. Any aggregates inside here must be dead code, so we need not convert it.}

$$
\ruleI
{
}
{ 
    \ToCoreX{n}{M~\tau}{n}
}
\textrm{(CoreVar)}
\quad
\ruleI
{
}
{ 
    \ToCoreX{\_}{v}{M~\tau}{v}
}
\textrm{(CoreValue)}
$$

$$
\ruleI
{
    \ToCoreX{x}{M_x~\tau_x}{x'}
    \quad
    \ToCoreX{y}{M_y~\tau_y}{y'}
}
{ 
    \ToCoreX{p~x~y}{M~\tau}{p~x'~y'}
}
\textrm{(CorePrim)}
$$

$$
\ruleI
{
    \ToCoreX{x}{M_x~\tau_x}{x'}
    \quad
    \ToCoreX{y}{M_y~\tau_y}{y'}
}
{ 
    \ToCoreX{@let@~n~=~x~@in@~y}{M~\tau}{@let@~n~=~x'~@in@~y'}
}
\textrm{(CoreLet)}
$$


$$
\boxed{\ToCore{n}{x}{M~\tau}{@Program@}}
$$
\TODO{Actually this returns a slightly different kind of program with only one unnamed output (call it return)}


$$
\ruleI
{
    n_f~\in~\mi{fresh}
    \quad
    \ToCoreX{x}{M~\tau}{x'}
}
{ 
    \ToCore{\_}{x}{M~\tau}{@Befores@~n_f~=~x';~@Return@~n_f}
}
\textrm{(CorePre)}
$$

$$
\ruleI
{
}
{ 
    \ToCore{\_}{p}{M~\tau}{@Return@~n}
}
\textrm{(CoreVar)}
\quad
\ruleI
{
    n_f~\in~\mi{fresh}
}
{ 
    \ToCore{\_}{v}{M~\tau}{@Befores@~n_f~=~v;~@Return@~n_f}
}
\textrm{(CoreValue)}
$$



\caption{Conversion to Core}
\label{fig:core:compile}
\end{figure*}




