%!TEX root = ../Main.tex
\section{Core language}
\label{s:Core}

In figure~\ref{fig:core:grammar} we introduce the intermediate language.
The top-level of this language is $\mit{Program}$ with four stages of computation: pure computations, followed by stream transformers, then folds, and finally extracting the result of the folds.

The program definition may seem more complicated than necessary, as all programs could be expressed as a single fold and an extract, but separating these out turns out to be useful for efficiency.
This will be expanded upon later.

The expression type for Core is much simpler than Source, $\mit{CoreExp}$ having just names, primitive application and let bindings.
More primitives are required for the Core language, as some filters in the Source language are converted to @If@s, and groups are converted to use map primitives such as lookup and insert.
The @scan@ primitive is no longer used in the Core language as it is just converted to a variable reference.





%!TEX root = ../Main.tex

\begin{figure}

\begin{tabbing}
MMMM \= M \= M \= \kill
$\mit{CoreExp}$
    \> $=$  \> $n$          \\
    \> $~|$ \> $v$ \\
    \> $~|$ \> $\mit{CorePrim}~(n*)$ \\
\\
$\mit{CorePrim}$
    \> $=$  \> $@+@~|~@>@~|~@if@$ \\
\\
$\mit{FoldInit}$
    \> $=$  \> $\mit{CoreExp}$ \\
$\mit{FoldAcc}$
    \> $=$  \> $(n~=~\mit{CoreExp})*~\mit{CoreExp}$ \\
\\
$\mit{Fold}$
    \> $=$  \> $@Fold@~\mit{FoldInit}~\mit{FoldAcc}$ \\
\\
\\
$\mit{Program}$
    \> $=$  \> @Befores@ \\
    \>      \> \> $(n~=~\mit{CoreExp})*$ \\
    \>      \> @Streams@ \\
    \>      \> \> $(n~=~\mit{CoreExp})*$ \\
    \>      \> @Folds@ \\
    \>      \> \> $(n~=~\mit{Fold})*$ \\
    \>      \> @Extracts@ \\
    \>      \> \> $(n~=~\mit{CoreExp})*$ \\
    \>      \> @Outputs@ \\
    \>      \> \> $(n~=~n)*$ \\
\end{tabbing}


\caption{Grammar for Icicle Core}
\label{fig:core:grammar}
\end{figure}





%!TEX root = ../Main.tex

\begin{figure*}

$$
\boxed{\CoreComp{e}{[Bind]}{[Op]}{x}}
$$

$$
\ruleAx
{
    \CoreComp{v}{\emptyset}{@let@~x'~=~v}{x'}
}{CVal}
\ruleAx
{
    \CoreComp{x}{\emptyset}{\emptyset}{x}
}{CVar}
\ruleIN
{
    \CoreComp{e_1}{b_1}{o_1}{x_1}
    \quad
    \CoreComp{e_2}{b_2}{o_2}{x_2}
}{
    \CoreComp{e_1~e_2}{b_1;b_2}{o_1;o_2;~@let@~x'~=~x_1~x_2}{x'}
}{CApp}
$$


\caption{Conversion to Core}
\label{fig:core:compile}
\end{figure*}




