%!TEX root = ../Main.tex
\section{Icicle Source}
\label{s:Source}

We present a cut down version of our implementation; our implementation has many more primitives, case expressions, as well as primitive contexts for windowing based on time, folding over groups, and so on.

The first example defines $\mi{sum}$ as a fold over the input data.
The initial value for the fold is $0$, followed by the previous value $s$ added to the current value $v$.

\begin{tabbing}
MM \= MM \= MMMMMMMMMMM \= \kill
$\mi{sum}~v$    \\
\> $=$  \> $@fold@~s~=~0~@then@~s~+~v$ \\
\> @in@ \> $s$ \\
\end{tabbing}

Next, we can define $\mi{count}$ and $\mi{mean}$ in terms of this sum.
\begin{tabbing}
MM \= MM \= MMMMMMMMMMM \= \kill
$\mi{count}$                                        \\
 \> $=$  \> $\mi{sum}~1$                            \\
                                                    \\
$\mi{mean}~v$                                       \\
 \> $=$  \> $\mi{sum}~v~/~\mi{count}$               \\
\end{tabbing}

We can use @filter@ to count only the days where the close price is higher than the open.
\begin{tabbing}
MM \= MM \= MMMMMMMMMMM \= \kill
$\mi{gap}$                                          \\
 \> $=$  \> $@filter@~\mi{close}~>~\mi{open}$       \\
 \> @in@ \> $\mi{count}$                            \\
                                                    \\
$\mi{proportion}$                                   \\
 \> $=$  \> $\mi{gap}~/~\mi{count}$
\end{tabbing}

Finally, we can use @group@ to put into buckets based on the percentage of growth.
Here the key for the map is the percentage of growth, and the value is the number of days with that amount of growth, divided by the total number of days.
\begin{tabbing}
MM \= MM \= MMMMMMMMMMM \= \kill
$\mi{growthPercent}$                                        \\
 \> $=$  \> $((\mi{close}-\mi{open}) \times 100 ) / \mi{open}$  \\
                                                            \\
$\mi{growthBy}$                                             \\
 \> $=$  \> $@let@~\mi{total} = \mi{count}$                 \\
 \> @in@ \> $@group@~\mi{growthPercent}$                    \\
 \> @in@ \> $\mi{count} / \mi{total}$                    \\
\end{tabbing}

\TODO{Mention resumables and bubblegum?}

\TODO{Talk about scan}

This language is not referentially transparent, which may seem odd, but this actually allows us to restrict the language and uphold our performance guarantees.
The input stream is implicitly passed as an argument to all parts of the program, but cannot be referred to explicitly.
Additionally, operations like @filter@ create a new context where the input stream is modified.
This means that a @let@ binding outside of a @filter@ can have different semantics when moved inside the @filter@, despite having the same body.
However, by passing this argument implicitly we remove a class of programs that would be hard to guarantee performance for.
By prohibiting the programmer from accessing the underlying stream, they cannot perform certain operations like zipping two streams with different filters together.

We formally define the grammar of the language in figure~\ref{fig:source:grammar}.
Note that general application is not allowed: arguments can only be applied to primitives and named functions.
Similarly, functions and primitives must be fully applied.
This, combined with the lack of lambda construction, means that higher order functions are outlawed.
While a first-order language (one lacking higher order functions) offers somewhat less abstraction, it simplifies the typing rules and makes it easier to support our performance guarantees.

While the examples shown allow user defined functions, there is no recursion allowed.
This means that all function definitions can be inlined into the user query with guaranteed termination.
When converting to the intermediate language, we assume that this inlining has occurred.

\TODO{Should we go through the evaluation rules here, or after typechecking?}
If we can write the evaluation rules to be ``agnostic'' of the type, it would be ideal to put them here.

%!TEX root = ../Main.tex

\begin{figure}

\begin{tabbing}
MMMM \= M \= MMMMMMMMMMM \= \kill
@Exp@
    \> $=$  \> $n$ \\
    \> $~|$ \> $@Prim@$ \\
    \> $~|$ \> $@Exp@~@Exp@$ \\
\\
    \> $~|$ \> $@let@~n~@=@~@Exp@$
            \> $\flowsinto~@Exp@$ \\
    \> $~|$ \> $@fold@~n~@=@~@Exp;@~@Exp@$
            \> $\flowsinto~@Exp@$ \\
\\
    \> $~|$ \> $@filter@~@Exp@$
            \> $\flowsinto~@Exp@$ \\
    \> $~|$ \> $@group@~@Exp@$
            \> $\flowsinto~@Exp@$ \\
\\
@Prim@
    \> $=$  \> $@+@~|~@>@~|~\NN~|~@scan@$ \\
\end{tabbing}


\caption{Grammar for Icicle Source}
\label{fig:source:grammar}
\end{figure}



%!TEX root = ../Main.tex

\begin{figure*}

\begin{tabbing}
MMM \= MM \= MMM \= MM \= MMMMMM \= \kill
$\mi{V}$
\GrammarDef
  $\NN~|~\BB~|~@Map@~(V \Rightarrow V_\bot)~|~(V~\times~\cdots~\times~V)$
\\
$\mi{V'}$
\GrammarDef
  $\VValue{V}~|~\VStream{(\Sigma \to V)}~|~\VFold{V}{(\Sigma \to V \to V)}{(V \to V)}$
\\
$\mi{E'}$
\GrammarDef
  $\mi{Exp}[V~:=~V']$
\\
\end{tabbing}

$$
\boxed{\SourceStepX{E'}{V'}}
$$

$$
\ruleIN
{
    v~\in~V'
}
{
    \SourceStepX{v}{v}
}{EVal}
\ruleIN
{
    \SourceStepX{e}{\VValue{v}}
}
{
    \SourceStepX{e}{\VStream{(\lam{\sigma} v)}}
}{EBoxStream}
\ruleIN
{
    \SourceStepX{e}{\VValue{v}}
}
{
    \SourceStepX{e}{\VFold{()}{(\lam{\sigma~()} ())}{(\lam{()} v)}}
}{EBoxFold}
$$

$$
\ruleIN
{
  \{ \SourceStepX{e_i}{\VValue{v_i}} \}_i
}
{
  \SourceStepX
    {p~\{ e_i \}_i }
    {\VValue{(p~\{v_i\}_i)}}
}{EPrimValue}
\ruleIN
{
  \{ \SourceStepX{e_i}{\VStream{v_i}} \}_i
}
{
  \SourceStepX
    {p~\{ e_i \}_i }
    {\VStream{(\lam{\sigma} p~\{v_i~\sigma\}_i)}}
}{EPrimStream}
$$

$$
\ruleIN
{
  \{ \SourceStepX{e_i}{\VFold{z_i}{k_i}{x_i}} \}_i
}
{
  \SourceStepX
    {p~\{ e_i \}_i }
    {\VFold
      {(\times_i~z_i)}
      {(\lam{\sigma~(\times_i~v_i)}
        \times_i (k_i~\sigma~v_i))}
      {(\lam{(\times_i~v_i)}
        p~\{x_i~v_i\}_i)}}
}{EPrimFold}
$$

$$
\ruleIN
{
  \SourceStepX{e}{v}
  \quad
  \SourceStepX{e'[x:=v]}{v'}
}
{
  \SourceStepX
    {@let@~x~=~e~@in@~e'}
    {v'}
}{ELet}
\ruleIN
{
  \SourceStepX{z}{\VValue{z'}}
  \quad
  \SourceStepX{k[x:=\VStream{(\lam{\sigma} \sigma~x)}]}{\VStream{k'}}
}
{
  \SourceStepX
    {@fold@~x~=~z~@then@~k}
    {\VFold
      {z'}
      {(\lam{\sigma~v} k'~(x:=v,~\sigma))}
      {(\lam{v} v)}}
}{EFold}
$$

$$
\ruleIN
{
  \SourceStepX{p}{\VStream{p'}}
  \quad
  \SourceStepX{e}{\VFold{z}{k}{x}}
}
{
  \SourceStepX
    {@filter@~p~@of@~e}
    {\VFold
      {z}
      {(\lam{\sigma~v}
         @if@~p'~\sigma~@then@~k~\sigma~v~@else@~v)}
      {x}}
}{EFilter}
$$

$$
\ruleIN
{
  \SourceStepX{p}{\VStream{p'}}
  \quad
  \SourceStepX{e}{\VFold{z}{k}{x}}
}
{
  \SourceStepX
    {@group@~p~@of@~e}
    {\VFold
      {\{\_~\Rightarrow~\bot\}}
      {(\lam{\sigma~m}
        @let@~k~=~p'~\sigma@,@~
              v~=~m~k~\vee~z
        ~@in@~
          m[k~\Rightarrow~k~\sigma~v])}
      {\{k_i~\Rightarrow~x~(m~k_i)\}_i}}
}{EGroup}
$$


\caption{Evaluation rules}
\label{fig:source:eval}
\end{figure*}



