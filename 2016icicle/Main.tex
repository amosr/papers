\documentclass[preprint]{sigplanconf}
\usepackage{amssymb}
\usepackage{amsthm}
\usepackage{graphicx}
\usepackage{amsmath}
\usepackage{mathptmx}
\usepackage{mathtools}
\usepackage{stmaryrd}
\usepackage{hyperref}
\usepackage{alltt}
\usepackage{url}
\usepackage{float}
\usepackage{style/code}
\usepackage{style/proof}
\usepackage{style/utils}
\usepackage{style/judgements}

% -----------------------------------------------------------------------------
\begin{document}

% \exclusivelicense
% \conferenceinfo{}{}
% \copyrightyear{2015}
% \copyrightdata{}
\doi{}
% \pagenumbering{gobble} 

\title{Icicle: a single-pass query language}

\authorinfo{
  Amos Robinson$^\dagger$$^\ddagger$
  \and Ben Lippmeier$^\dagger$
}{
  \vspace{5pt}
  \shortstack{
    $^\dagger$Computer Science and Engineering \\
    University of New South Wales, Australia \\[2pt]
    \textsf{amosr,benl@cse.unsw.edu.au}
  }
  \shortstack{
    $^\ddagger$Ambiata      \\
    Big data and shit       \\[2pt]
    \textsf{amos.robinson@ambiata.com}
  }
}

\maketitle
\makeatactive

\begin{abstract}
When streaming a large amount of data, simply iterating over the data may take hours.
In this case, the difference between a query that takes one iteration and a query that takes multiple iterations must be very explicit.
For many purposes, limiting queries to a single pass is sufficient.
We introduce a type system based on modality types to enforce this single pass restriction.
By using an appropriate intermediate language we guarantee fusion between queries on the same table, before extracting efficient single-loop C code.
\end{abstract}


\category
	{D.3.4}
	{Programming Languages}
	{Processors---Compilers; Optimization}

\terms
	Languages, Performance

\keywords
	Arrays; Fusion

%!TEX root = ../Main.tex
\section{Introduction}

Data flow fusion~\cite{lippmeier2013flow} is a technique to compile a specific class of data flow programs into single, efficient imperative loops. This process of ``compilation'' is equivalent to performing array fusion on a combinator based functional array program, as per related work on stream fusion~\cite{coutts2007streamfusion} and delayed arrays~\cite{keller2010repa}. The key benefits of data flow fusion over this prior work are: 1) it fuses programs that use branching data flows where a produced array is consumed by several consumers, and 2) complete fusion into a single loop is guaranteed for all programs that operate on the same size input data, and contain no fusion-preventing dependencies between operators.

Fusion-preventing dependencies express the fact that some operators simply must wait for others to complete before they can produce their own output. For example, in the following:
\begin{code}
  normalize :: Array Int -> Array Int
  normalize xs = let sum = fold (+) 0 xs
                 in  map (/ sum) xs
\end{code}

If we wish to divide every element of an array by the sum of all elements, then it seems we are forever destined to compute the result using at least two loops: one to determine the sum, and one to divide the elements. The evaluation of @fold@ demands all elements of its source array, and we cannot produce any elements of the result array until we know the value of @sum@. 

However, many programs \emph{do} contain opportunities for fusion, if we only knew which opportunities to take. The following example offers \emph{several} unique, but mutually exclusive approaches to fusion. Figure~\ref{f:normalize2-cluterings} on the next page shows some of the possibilities.
\begin{code}
 normalize2 :: Array Int -> Array Int
 normalize2 xs
  = let sum1 = fold   (+)  0   xs
        gts  = filter (> 0)    xs
        sum2 = fold   (+)  0   gts
        ys1  = map    (/ sum1) xs
        ys2  = map    (/ sum2) xs
    in (ys1, ys2)
\end{code}

In Figure~\ref{f:normalize2-cluterings}, the dotted lines show possible clusterings of operators. Stream fusion implicitly choses the solution on the left as its compilation process cannot fuse a produced array into multiple consumers. The best existing ILP approach will chose the solution on the right as it cannot cluster operators that consume arrays of different lengths. Our system choses the solution in the middle, which is also optimal for this example. 

% NOTE: This set of bullets needs to fit on the first page, without spilling to the second.
Our contributions are as follows:
\begin{itemize}
\item   
We extend prior work by Megiddo~\cite{megiddo1998optimal} and Darte~\cite{darte2002contraction}, with support for length changing operators. Length changing operators can be clustered with the operators that generate their source arrays, and compiled naturally with data-flow fusion (\S\ref{s:ILP}).

\item
We present a simplification to constraint generation that is also applicable to some existing integer linear programming formulations such as Megiddo's,
where constraints between two nodes need not be generated if there exists a fusion-preventing path between the two (\S\ref{s:OptimisedConstraints}).

\item
Our constraint system also encodes a total ordering on the cost of clusterings, expressed using weights on the integer linear program. For example, we encode that memory traffic is more expensive than loop overheads, so given a choice between the two, the memory traffic will be reduced (\S\ref{s:ObjectiveFunction}).

\item
We present benchmarks of our algorithm applied to several common programming patterns, and to several pathological examples.
Our algorithm is complete and yields good results in practice, though if array sizes are unknown, an optimal solution is uncomputable in general. \TODO{ref}
\end{itemize}

The reduction of the clustering problem to integer linear programming was previously described by~\cite{megiddo1998optimal}, though they do not consider length changing operators.


% We must also decide which clustering is the `best' or most optimal. One obvious criterion for this is the minimum number of loops, but there may even be multiple clusterings with the minimum number of loops. In this case, the number of required manifest arrays must also be taken into account. 

% As real programs contain tens or hundreds of individual operators, performing an exhaustive search for an optimal clustering is not feasible, and greedy algorithms tend to produce poor solutions. 


%!TEX root = ../Main.tex
\section{Icicle}
\label{s:IcicleSource}

%!TEX root = ../Main.tex

\begin{figure}

\begin{tabbing}
MMMM \= M \= MMMMMMMMMMM \= \kill
@Exp@
    \> $=$  \> $n$ \\
    \> $~|$ \> $@Prim@$ \\
    \> $~|$ \> $@Exp@~@Exp@$ \\
\\
    \> $~|$ \> $@let@~n~@=@~@Exp@$
            \> $\flowsinto~@Exp@$ \\
    \> $~|$ \> $@fold@~n~@=@~@Exp;@~@Exp@$
            \> $\flowsinto~@Exp@$ \\
\\
    \> $~|$ \> $@filter@~@Exp@$
            \> $\flowsinto~@Exp@$ \\
    \> $~|$ \> $@group@~@Exp@$
            \> $\flowsinto~@Exp@$ \\
\\
@Prim@
    \> $=$  \> $@+@~|~@>@~|~\NN~|~@scan@$ \\
\end{tabbing}


\caption{Grammar for Icicle Source}
\label{fig:source:grammar}
\end{figure}


%!TEX root = ../Main.tex

\begin{figure*}

\begin{tabbing}
MM \= MM \= \kill
$\mi{Kind}$
\GrammarDef $@Data@~|~@Clock@~|~@Flow@$ \\

T
\GrammarDef $x~|~\NN~|~\BB~|~@()@
    ~|~ @List@~T
    ~|~ @Stream@~T~T
    ~|~ @Fold@~T
    ~|~ T~\to~T$ \\
\end{tabbing}

$$
\boxed{\TypeWf{\Delta}{\tau}{\mi{Kind}}}
$$

$$
\ruleIN
{
    x~:_k~k~\in~\Delta
}
{
    \TypeWf{\Delta}{x}{k}
}{TVar}
\ruleAx
{
    \TypeWf{\Delta}{\NN}{@Data@}
}{TNat}
\ruleAx
{
    \TypeWf{\Delta}{\BB}{@Data@}
}{TBool}
\ruleAx
{
    \TypeWf{\Delta}{@()@}{@Data@}
}{TUnit}
\ruleIN
{
    \TypeWf{\Delta}{\tau}{@Data@}
}
{
    \TypeWf{\Delta}{@List@~\tau}{@Data@}
}{TList}
$$

$$
\ruleIN
{
    \TypeWf{\Delta}{c}{@Clock@}
    \quad
    \TypeWf{\Delta}{\tau}{@Data@}
}
{
    \TypeWf{\Delta}{@Stream@~c~\tau}{@Flow@}
}{TStream}
\ruleIN
{
    \TypeWf{\Delta}{\tau}{@Data@}
}
{
    \TypeWf{\Delta}{@Fold@~\tau}{@Flow@}
}{TFold}
$$

$$
\ruleIN
{
    \TypeWf{\Delta}{\tau_1}{@Data@}
    \quad
    \TypeWf{\Delta}{\tau_2}{@Data@}
}
{
    \TypeWf{\Delta}{\tau_1~\to~\tau_2}{@Data@}
}{TFunData}
\ruleIN
{
    \TypeWf{\Delta}{\tau_1}{@Flow@}
    \quad
    \TypeWf{\Delta}{\tau_2}{@Data@}
}
{
    \TypeWf{\Delta}{\tau_1~\to~\tau_2}{@Flow@}
}{TFunFlow1}
$$
$$
\ruleIN
{
    \TypeWf{\Delta}{\tau_1}{@Data@}
    \quad
    \TypeWf{\Delta}{\tau_2}{@Flow@}
}
{
    \TypeWf{\Delta}{\tau_1~\to~\tau_2}{@Flow@}
}{TFunFlow2}
\ruleIN
{
    \TypeWf{\Delta}{\tau_1}{@Flow@}
    \quad
    \TypeWf{\Delta}{\tau_2}{@Flow@}
}
{
    \TypeWf{\Delta}{\tau_1~\to~\tau_2}{@Flow@}
}{TFunFlow3}
$$



\caption{Types and their kinds}
\label{fig:source:type:kinds}
\end{figure*}


\begin{figure*}

$$
\boxed{\Typecheck{\Delta}{\Gamma}{e}{T}}
$$


$$
\ruleIN
{
    (x~:~\tau)~\in~\Gamma
}
{ 
    \Typecheck{\Delta}{\Gamma}{x}{\tau}
}{TcVar}
\ruleIN
{
    v~:~\tau
}
{ 
    \Typecheck{\Delta}{\Gamma}{v}{\tau}
}{TcValue}
\ruleIN
{
    \Typecheck{\Delta}{\Gamma}{e_1}{\tau_1~\to~\tau_2}
    \quad
    \Typecheck{\Delta}{\Gamma}{e_2}{\tau_1}
}
{ 
    \Typecheck{\Delta}{\Gamma}{e_1~e_2}{\tau_2}
}{TcApp}
$$

$$
\ruleIN
{
    \Typecheck{\Delta}{\Gamma,~x~:~\tau}{e}{\tau'}
}
{
    \Typecheck{\Delta}{\Gamma}{\lambda{}x~:~\tau.~e}{\tau~\to~\tau'}
}{TcLam}
\ruleIN
{
    \Typecheck{\Delta}{\Gamma,~x~:~\tau}{e}{\tau}
    \quad
    \TypeWf{\Delta}{\tau}{@Data@}
}
{
    \Typecheck{\Delta}{\Gamma}{@fix@~(x~:~\tau)~e}{\tau}
}{TcFix}
$$

$$
\ruleIN
{
    \Typecheck{\Delta}{\Gamma}{e_1}{@Stream@~c~\tau}
    \quad
    \Typecheck{\Delta}{\Gamma}{e_2}{@Fold@~\tau}
}
{
    \Typecheck{\Delta}{\Gamma}{@when@~e_1~e_2}{@Fold@~\tau}
}{TcWhen}
\ruleIN
{
    \Typecheck{\Delta}{\Gamma}{e_1}{@Stream@~c~@()@}
    \quad
    \Typecheck{\Delta}{\Gamma}{e_2}{@Fold@~\tau}
}
{
    \Typecheck{\Delta}{\Gamma}{@sample@~e_1~e_2}{@Stream@~c~\tau}
}{TcSample}
$$

$$
\ruleIN
{
    \Typecheck{\Delta}{\Gamma}{e_1}{\tau_1~\to~\tau_2}
    \quad
    \Typecheck{\Delta}{\Gamma}{e_2}{@Stream@~c~\tau_1}
}
{
    \Typecheck{\Delta}{\Gamma}{@mapS@~e_1~e_2}{@Stream@~c~\tau_2}
}{TcMapS}
\ruleIN
{
    \Typecheck{\Delta}{\Gamma}{e_1}{@Stream@~c~\tau_1}
    \quad
    \Typecheck{\Delta}{\Gamma}{e_2}{@Stream@~c~\tau_2}
}
{
    \Typecheck{\Delta}{\Gamma}{@zipS@~e_1~e_2}{@Stream@~c~(\tau_1,\tau_2)}
}{TcZipS}
$$

$$
\ruleIN
{
    \Typecheck{\Delta}{\Gamma}{e_1}{\tau_1~\to~\tau_2}
    \quad
    \Typecheck{\Delta}{\Gamma}{e_2}{@Fold@~\tau_1}
}
{
    \Typecheck{\Delta}{\Gamma}{@mapF@~e_1~e_2}{@Fold@~\tau_2}
}{TcMapF}
\ruleIN
{
    \Typecheck{\Delta}{\Gamma}{e_1}{@Fold@~\tau_1}
    \quad
    \Typecheck{\Delta}{\Gamma}{e_2}{@Fold@~\tau_2}
}
{
    \Typecheck{\Delta}{\Gamma}{@zipF@~e_1~e_2}{@Fold@~(\tau_1,\tau_2)}
}{TcZipF}
$$

$$
\ruleIN
{
    \Typecheck{\Delta}{\Gamma}{e_1}{\tau_1}
    \quad
    \Typecheck{\Delta}{\Gamma,~x~:~\tau_1}{e_2}{\tau_2}
}
{
    \Typecheck{\Delta}{\Gamma}{@let@~x~=~e_1~@in@~e_2}{\tau_2}
}{TcLet}
$$

$$
\ruleIN
{
    \Gamma'~=~\Gamma,~\{~x_i~:~@Fold@~\tau_i~\}_i
    \quad
    \{
    \Typecheck{\Delta}{\Gamma}{e_{z_i}}{\tau_i}
    \}_i
    \quad
    \{
    \Typecheck{\Delta}{\Gamma'}{e_{k_i}}{@Fold@~\tau_i}
    \}_i
    \quad
    \Typecheck{\Delta}{\Gamma'}{e}{\tau}
}
{
    \Typecheck{\Delta}{\Gamma}
        {@let@~@folds@~\{~x_i~=~e_{z_i}~@then@~e_{k_i}~\}_i~@in@~e}
        {\tau}
}{TcLetFolds}
$$

$$
\ruleIN
{
    \Typecheck{\Delta}{\Gamma}{e_1}{@Fold@~\tau_1}
    \quad
    x_\tau~\not\in~\Delta
    \quad
    \Typecheck
        {\Delta,~x_\tau :_k @Data@}
        {\Gamma,~x_z : x_\tau,~x_k : @Fold@~x_\tau \to @Fold@~x_\tau,~x_r : @Fold@~x_\tau \to @Fold@~\tau_1}
        {e_2}{\tau_2}
}
{
    \Typecheck{\Delta}{\Gamma}
        {@let@~@unpack@~(x_\tau,x_z,x_k,x_r)~=~e_1~@in@~e_2}
        {\tau_2}
}{TcLetUnpack}
$$

$$
\ruleIN
{
    \Typecheck{\Delta}{\Gamma}{e_1}{@Stream@~c~\BB}
    \quad
    x_c~\not\in~\Delta
    \quad
    \Typecheck
        {\Delta,~x_c :_k @Clock@}
        {\Gamma,~x_s : @Stream@~x_c~\BB}
        {e_2}
        {\tau_2}
}
{
    \Typecheck{\Delta}{\Gamma}
        {@let@~@subrate@~(x_c,x_s)~=~e_1~@in@~e_2}
        {\tau_2}
}{TcLetSubrate}
$$


\caption{Types of expressions}
\label{fig:source:type:exp}
\end{figure*}


%!TEX root = ../Main.tex

\begin{figure*}

\begin{tabbing}
MMM \= MM \= MMM \= MM \= MMMMMM \= \kill
$\mi{V}$
\GrammarDef
  $\NN~|~\BB~|~@Map@~(V \Rightarrow V_\bot)~|~(V~\times~\cdots~\times~V)$
\\
$\mi{V'}$
\GrammarDef
  $\VValue{V}~|~\VStream{(\Sigma \to V)}~|~\VFold{V}{(\Sigma \to V \to V)}{(V \to V)}$
\\
$\mi{E'}$
\GrammarDef
  $\mi{Exp}[V~:=~V']$
\\
\end{tabbing}

$$
\boxed{\SourceStepX{E'}{V'}}
$$

$$
\ruleIN
{
    v~\in~V'
}
{
    \SourceStepX{v}{v}
}{EVal}
\ruleIN
{
    \SourceStepX{e}{\VValue{v}}
}
{
    \SourceStepX{e}{\VStream{(\lam{\sigma} v)}}
}{EBoxStream}
\ruleIN
{
    \SourceStepX{e}{\VValue{v}}
}
{
    \SourceStepX{e}{\VFold{()}{(\lam{\sigma~()} ())}{(\lam{()} v)}}
}{EBoxFold}
$$

$$
\ruleIN
{
  \{ \SourceStepX{e_i}{\VValue{v_i}} \}_i
}
{
  \SourceStepX
    {p~\{ e_i \}_i }
    {\VValue{(p~\{v_i\}_i)}}
}{EPrimValue}
\ruleIN
{
  \{ \SourceStepX{e_i}{\VStream{v_i}} \}_i
}
{
  \SourceStepX
    {p~\{ e_i \}_i }
    {\VStream{(\lam{\sigma} p~\{v_i~\sigma\}_i)}}
}{EPrimStream}
$$

$$
\ruleIN
{
  \{ \SourceStepX{e_i}{\VFold{z_i}{k_i}{x_i}} \}_i
}
{
  \SourceStepX
    {p~\{ e_i \}_i }
    {\VFold
      {(\times_i~z_i)}
      {(\lam{\sigma~(\times_i~v_i)}
        \times_i (k_i~\sigma~v_i))}
      {(\lam{(\times_i~v_i)}
        p~\{x_i~v_i\}_i)}}
}{EPrimFold}
$$

$$
\ruleIN
{
  \SourceStepX{e}{v}
  \quad
  \SourceStepX{e'[x:=v]}{v'}
}
{
  \SourceStepX
    {@let@~x~=~e~@in@~e'}
    {v'}
}{ELet}
\ruleIN
{
  \SourceStepX{z}{\VValue{z'}}
  \quad
  \SourceStepX{k[x:=\VStream{(\lam{\sigma} \sigma~x)}]}{\VStream{k'}}
}
{
  \SourceStepX
    {@fold@~x~=~z~@then@~k}
    {\VFold
      {z'}
      {(\lam{\sigma~v} k'~(x:=v,~\sigma))}
      {(\lam{v} v)}}
}{EFold}
$$

$$
\ruleIN
{
  \SourceStepX{p}{\VStream{p'}}
  \quad
  \SourceStepX{e}{\VFold{z}{k}{x}}
}
{
  \SourceStepX
    {@filter@~p~@of@~e}
    {\VFold
      {z}
      {(\lam{\sigma~v}
         @if@~p'~\sigma~@then@~k~\sigma~v~@else@~v)}
      {x}}
}{EFilter}
$$

$$
\ruleIN
{
  \SourceStepX{p}{\VStream{p'}}
  \quad
  \SourceStepX{e}{\VFold{z}{k}{x}}
}
{
  \SourceStepX
    {@group@~p~@of@~e}
    {\VFold
      {\{\_~\Rightarrow~\bot\}}
      {(\lam{\sigma~m}
        @let@~k~=~p'~\sigma@,@~
              v~=~m~k~\vee~z
        ~@in@~
          m[k~\Rightarrow~k~\sigma~v])}
      {\{k_i~\Rightarrow~x~(m~k_i)\}_i}}
}{EGroup}
$$


\caption{Evaluation rules}
\label{fig:source:eval}
\end{figure*}




The grammar for Icicle is given in figure~\ref{fig:source:grammar}.
Value types are given in $T$ and can be numbers, booleans or maps.
Modal types, $T_M$ can be a pure value type, or a modality associated with a value type.
Function types, $T_\to$, can be either a non-function modal type, or a function from modal type to function type.
As Icicle is a first-order language, function types are not value types.

The table definition in $\mi{Table}$ gives a table name and the names and types of columns.
$\mi{Exp}$ defines the expressions used as query and function bodies.
$\mi{Prim}$ defines the primitive types.
Function definitions are given in $\mi{Fun}$.
Query definitions are given in $\mi{Query}$.
$\mi{Top}$ is the top-level program, which specifies a table, the set of function bindings, and the set of queries.

\TODO{update this to fit new grammar}
Expressions can be variable names, primitive operators such as addition and division, function application, or nested queries.
Function application @x Exp*@ is the name of a top-level function applied to any number of arguments.
Primitive application @Prim Exp*@ is written prefix, but in the following we use infix-operator short-hand for convenience; for example, @(>) 0 1@ can be written as @0 > 1@.

Queries can define folds over the input, let-bindings, filtering according to some predicate, or grouping by some key.
The fold form takes the name of the fold binding, and two expressions: the initial value and the update value for each element.
Count can be expressed by the following fold:
\begin{code}
fold count = 0 then count + 1
\end{code}

Let-bindings allow a name to be used in place of the expression, in the rest of some query.
\begin{code}
let diff = open - close
in  mean diff
\end{code}

Filters take a predicate, which can be thought of as a stream of booleans, and a query to perform for the satisfying values.
The following query will count the number of entries where the open price exceeds the close price.
\begin{code}
filter open > close in count
\end{code}

Filters correspond to @WHERE@ clauses in SQL, except that the predicate does not apply to the entire top-level query, only to the subquery.
This makes it slightly easier to define queries like the proportion of entities satisfying some predicate to all entities: 
\begin{code}
(filter open > close in count) / count
\end{code}

Group takes a key to group by, and a query to perform on the values of each grouping.
This query groups by the company code, and counts the number of records in each company.
\begin{code}
group company in count
\end{code}

@Fun@ is used for function definitions.
The @count@ function takes no arguments and returns a fold counting the number of elements.
\begin{code}
function count
 = fold c = 0 then c + 1
\end{code}

Sum can be defined taking a single argument, @value@, which is the elements to compute the sum of.
It returns a fold that starts at zero and for every element, increases the old sum by @value@.
\begin{code}
function sum (value : Element Int)
 = fold s = 0 then s + value
\end{code}

The definition of mean is just the sum divided by the count.
\begin{code}
function mean (value : Element Int)
 = sum value / count
\end{code}




\subsection{Single-pass restriction}

All Icicle queries must execute in a single pass over the data, as reading the data multiple times is expensive. 
Ensuring that queries only require a single pass over the data also allows us to resume queries from where they left off, when new data arrives.

In order to ensure that queries can be executed as a single pass, we use a modal type system inspired by staged computation\cite{davies2001modal}.
We introduce two modalities, called @Element@ and @Aggregate@, which denote when computations must occur.

The @Element@ modality means that a computation is defined for each record in the table, or element in the stream.
Each column in the table is @Element@; for example the @open@ column has type @Element Int@, meaning that for each record in the column there is an @Int@.
This can be thought of as being represented by a stream of values.

The @Aggregate@ modality means that a computation is available only after all records in the table have been seen, or after the end of the stream.
These are used for the results of folds.
When computing count, the final value is not known until all the records have been seen, so @count@ has type @Aggregate Int@.

The modalities are automatically boxed and unboxed, so that if a function expecting a pure value is passed an @Element@ computation, the result of the function becomes an @Element@ computation too.
For example, if @open : Element Int@, then @open == 1@ has type @Element Bool@.
@Element@ and @Aggregate@ are mutually exclusive: a computation cannot be both an element and an aggregate.

Some examples with their types:
\begin{code}
1             :         Int
open          : Element Int
open > 1      : Element Int
sum           : Element Int -> Aggregate Int
sum open      : Aggregate Int
sum open > 1  : Aggregate Bool
\end{code}

Folds have a name and two expressions.
The first expression, the initialiser, is pure.
The second expression, the update, has the fold's name bound to an @Element@ of the previous value.
The type of the expression must be @Element@.
The final return of the fold is an @Aggregate@.
\begin{code}
fold x = T then Element T : Aggregate T
\end{code}

Let-bindings are independent of the modality.

For filters, the predicate is an @Element@, which could be thought of as a stream of booleans.
The rest of the query must be an @Aggregate@; that is, it must return a scalar rather than a stream, to enforce the bounded buffer restriction explained in \sref{s:IcicleSource:bounded}.
\begin{code}
filter (Element Bool) in (Aggregate T)
\end{code}

Grouping is similar to filters; the key must be an @Element@, while the group value must be an @Aggregate@.
\begin{code}
group (Element Key) in (Aggregate Value)
\end{code}

The following program attemps to find the number of entries where the open price is above the mean of the close price.
This requires two passes over the data, as we do not know what the mean of all the data is until we have seen all the data.
This program violates the single-pass restriction, and is outlawed by the typesystem.
The left side of the @>@ has type @Element Int@, while the right side has type @Aggregate Int@, and these two are mutually exclusive.
\begin{code}
filter open > mean close
in count
(Error: (>) must be Element and Aggregate)
\end{code}



Here is an example to highlight one benefit of tracking these modalities.
Suppose we have a table with two fields: key and value.
We wish to find the mean of values whose key matches the most recent key.
Someone new to stream programming might write the following query, forgetting that they can only read the data once: first find the most recent key, then go back and filter all the data to only thosewith the right key.

\begin{code}
   let k    = newest key
in let avg  = filter (key == k) in mean value
in avg
\end{code}

This will not typecheck in Icicle, as @newest key@ has the modality @Aggregate@, while the filter condition @key == k@ attempts to check if an @Element@ is equal to an @Aggregate@.

This query could be translated to another synchronous streaming language such as {\sc Lustre}\cite{halbwachs1991synchronous}, and would typecheck.
However, the semantics of the translated query will not be what was originally desired.
Rather than checking each key against the most recent key as-of the end of the stream,
it will check each key against the current most recent key, which is the key itself.
This means the filter predicate is always be true, and thus returns the mean of the entire stream.
These semantics may be surprising to the novice stream programmer.

Hopefully our programmer is not too discouraged by the @Element /= Aggregate@ type error, and pushes on.
They now know that their query, as formulated, requires multiple passes over the data: the problem now is to reformulate it as a single pass.
With a little ingenuity, we can rewrite it as such: we can group by the key and perform the mean for each group.
After the group, we can perform a lookup by the most recent key.
This requires storing and computing the means for all keys despite only needing one at the end, so we are assuming that the number of keys are bounded in some way.

\begin{code}
   let k    = newest key
in let avgs = group  key in mean value
in mapLookup avgs k
\end{code}

\subsection{Bounded buffer restriction}
\label{s:IcicleSource:bounded}
As Icicle is designed to be a streaming language, the amount of data to be streamed may not necessarily fit in memory.
Any operation which requires a buffer must be bounded in size, as an unbounded buffer could potentially grow too large to fit in memory.

Here is an example stream program that requires unbounded buffering:
it takes an input stream @xs@, and filters it into those above zero, and those below zero.
These two filtered streams are then joined together pairwise, so the first positive element is paired with the first negative element, and so on.
The program, written in Haskell:

\begin{code}
zip (filter (>0) xs) (filter (<0) xs)
\end{code}

This program requires an unbounded buffer: if the input stream contains ten positive values followed by one negative value, all ten of the positive values must be held onto until the negative value is seen.
Similarly, if the entire stream is positive, all of the elements must be retained until the very end, just in case a negative value shows up.

In Icicle, we outlaw this kind of program by implicitly threading the input stream through operations.
The streams themselves are not materialised in the program: stream operations like @fold@ and @filter@ do not take parameters of the streams, but instead operate on the context stream.
By removing the explicit stream parameter, stream elements can only be joined from the same context, when both elements are available.

For example, in the query @filter p in mean value@, the @mean value@ is only applied to stream values which satisfy the predicate @p@.

This is a different approach than existing synchronous streaming languages\cite{mandel2010lucy} and flow fusion\cite{lippmeier2013data}, which perform clock analysis or `rate inference'.
Here, streams are given clock types denoting when the streams have elements available, and only streams with the same clock can be zipped together.
Filters produce a stream with a different output clock to its input, so different filters cannot be zipped together.
Our system requires no clock analysis, but is less expressive as streams are not `first class'.


%!TEX root = ../Main.tex
\section{Types}
\label{s:Types}

We will now formally describe the typesystem of Icicle, and show how modality types are used to enforce the single-pass restriction.

%!TEX root = ../Main.tex

\begin{figure*}

\begin{tabbing}
MM \= MM \= \kill
$\mi{Kind}$
\GrammarDef $@Data@~|~@Clock@~|~@Flow@$ \\

T
\GrammarDef $x~|~\NN~|~\BB~|~@()@
    ~|~ @List@~T
    ~|~ @Stream@~T~T
    ~|~ @Fold@~T
    ~|~ T~\to~T$ \\
\end{tabbing}

$$
\boxed{\TypeWf{\Delta}{\tau}{\mi{Kind}}}
$$

$$
\ruleIN
{
    x~:_k~k~\in~\Delta
}
{
    \TypeWf{\Delta}{x}{k}
}{TVar}
\ruleAx
{
    \TypeWf{\Delta}{\NN}{@Data@}
}{TNat}
\ruleAx
{
    \TypeWf{\Delta}{\BB}{@Data@}
}{TBool}
\ruleAx
{
    \TypeWf{\Delta}{@()@}{@Data@}
}{TUnit}
\ruleIN
{
    \TypeWf{\Delta}{\tau}{@Data@}
}
{
    \TypeWf{\Delta}{@List@~\tau}{@Data@}
}{TList}
$$

$$
\ruleIN
{
    \TypeWf{\Delta}{c}{@Clock@}
    \quad
    \TypeWf{\Delta}{\tau}{@Data@}
}
{
    \TypeWf{\Delta}{@Stream@~c~\tau}{@Flow@}
}{TStream}
\ruleIN
{
    \TypeWf{\Delta}{\tau}{@Data@}
}
{
    \TypeWf{\Delta}{@Fold@~\tau}{@Flow@}
}{TFold}
$$

$$
\ruleIN
{
    \TypeWf{\Delta}{\tau_1}{@Data@}
    \quad
    \TypeWf{\Delta}{\tau_2}{@Data@}
}
{
    \TypeWf{\Delta}{\tau_1~\to~\tau_2}{@Data@}
}{TFunData}
\ruleIN
{
    \TypeWf{\Delta}{\tau_1}{@Flow@}
    \quad
    \TypeWf{\Delta}{\tau_2}{@Data@}
}
{
    \TypeWf{\Delta}{\tau_1~\to~\tau_2}{@Flow@}
}{TFunFlow1}
$$
$$
\ruleIN
{
    \TypeWf{\Delta}{\tau_1}{@Data@}
    \quad
    \TypeWf{\Delta}{\tau_2}{@Flow@}
}
{
    \TypeWf{\Delta}{\tau_1~\to~\tau_2}{@Flow@}
}{TFunFlow2}
\ruleIN
{
    \TypeWf{\Delta}{\tau_1}{@Flow@}
    \quad
    \TypeWf{\Delta}{\tau_2}{@Flow@}
}
{
    \TypeWf{\Delta}{\tau_1~\to~\tau_2}{@Flow@}
}{TFunFlow3}
$$



\caption{Types and their kinds}
\label{fig:source:type:kinds}
\end{figure*}


\begin{figure*}

$$
\boxed{\Typecheck{\Delta}{\Gamma}{e}{T}}
$$


$$
\ruleIN
{
    (x~:~\tau)~\in~\Gamma
}
{ 
    \Typecheck{\Delta}{\Gamma}{x}{\tau}
}{TcVar}
\ruleIN
{
    v~:~\tau
}
{ 
    \Typecheck{\Delta}{\Gamma}{v}{\tau}
}{TcValue}
\ruleIN
{
    \Typecheck{\Delta}{\Gamma}{e_1}{\tau_1~\to~\tau_2}
    \quad
    \Typecheck{\Delta}{\Gamma}{e_2}{\tau_1}
}
{ 
    \Typecheck{\Delta}{\Gamma}{e_1~e_2}{\tau_2}
}{TcApp}
$$

$$
\ruleIN
{
    \Typecheck{\Delta}{\Gamma,~x~:~\tau}{e}{\tau'}
}
{
    \Typecheck{\Delta}{\Gamma}{\lambda{}x~:~\tau.~e}{\tau~\to~\tau'}
}{TcLam}
\ruleIN
{
    \Typecheck{\Delta}{\Gamma,~x~:~\tau}{e}{\tau}
    \quad
    \TypeWf{\Delta}{\tau}{@Data@}
}
{
    \Typecheck{\Delta}{\Gamma}{@fix@~(x~:~\tau)~e}{\tau}
}{TcFix}
$$

$$
\ruleIN
{
    \Typecheck{\Delta}{\Gamma}{e_1}{@Stream@~c~\tau}
    \quad
    \Typecheck{\Delta}{\Gamma}{e_2}{@Fold@~\tau}
}
{
    \Typecheck{\Delta}{\Gamma}{@when@~e_1~e_2}{@Fold@~\tau}
}{TcWhen}
\ruleIN
{
    \Typecheck{\Delta}{\Gamma}{e_1}{@Stream@~c~@()@}
    \quad
    \Typecheck{\Delta}{\Gamma}{e_2}{@Fold@~\tau}
}
{
    \Typecheck{\Delta}{\Gamma}{@sample@~e_1~e_2}{@Stream@~c~\tau}
}{TcSample}
$$

$$
\ruleIN
{
    \Typecheck{\Delta}{\Gamma}{e_1}{\tau_1~\to~\tau_2}
    \quad
    \Typecheck{\Delta}{\Gamma}{e_2}{@Stream@~c~\tau_1}
}
{
    \Typecheck{\Delta}{\Gamma}{@mapS@~e_1~e_2}{@Stream@~c~\tau_2}
}{TcMapS}
\ruleIN
{
    \Typecheck{\Delta}{\Gamma}{e_1}{@Stream@~c~\tau_1}
    \quad
    \Typecheck{\Delta}{\Gamma}{e_2}{@Stream@~c~\tau_2}
}
{
    \Typecheck{\Delta}{\Gamma}{@zipS@~e_1~e_2}{@Stream@~c~(\tau_1,\tau_2)}
}{TcZipS}
$$

$$
\ruleIN
{
    \Typecheck{\Delta}{\Gamma}{e_1}{\tau_1~\to~\tau_2}
    \quad
    \Typecheck{\Delta}{\Gamma}{e_2}{@Fold@~\tau_1}
}
{
    \Typecheck{\Delta}{\Gamma}{@mapF@~e_1~e_2}{@Fold@~\tau_2}
}{TcMapF}
\ruleIN
{
    \Typecheck{\Delta}{\Gamma}{e_1}{@Fold@~\tau_1}
    \quad
    \Typecheck{\Delta}{\Gamma}{e_2}{@Fold@~\tau_2}
}
{
    \Typecheck{\Delta}{\Gamma}{@zipF@~e_1~e_2}{@Fold@~(\tau_1,\tau_2)}
}{TcZipF}
$$

$$
\ruleIN
{
    \Typecheck{\Delta}{\Gamma}{e_1}{\tau_1}
    \quad
    \Typecheck{\Delta}{\Gamma,~x~:~\tau_1}{e_2}{\tau_2}
}
{
    \Typecheck{\Delta}{\Gamma}{@let@~x~=~e_1~@in@~e_2}{\tau_2}
}{TcLet}
$$

$$
\ruleIN
{
    \Gamma'~=~\Gamma,~\{~x_i~:~@Fold@~\tau_i~\}_i
    \quad
    \{
    \Typecheck{\Delta}{\Gamma}{e_{z_i}}{\tau_i}
    \}_i
    \quad
    \{
    \Typecheck{\Delta}{\Gamma'}{e_{k_i}}{@Fold@~\tau_i}
    \}_i
    \quad
    \Typecheck{\Delta}{\Gamma'}{e}{\tau}
}
{
    \Typecheck{\Delta}{\Gamma}
        {@let@~@folds@~\{~x_i~=~e_{z_i}~@then@~e_{k_i}~\}_i~@in@~e}
        {\tau}
}{TcLetFolds}
$$

$$
\ruleIN
{
    \Typecheck{\Delta}{\Gamma}{e_1}{@Fold@~\tau_1}
    \quad
    x_\tau~\not\in~\Delta
    \quad
    \Typecheck
        {\Delta,~x_\tau :_k @Data@}
        {\Gamma,~x_z : x_\tau,~x_k : @Fold@~x_\tau \to @Fold@~x_\tau,~x_r : @Fold@~x_\tau \to @Fold@~\tau_1}
        {e_2}{\tau_2}
}
{
    \Typecheck{\Delta}{\Gamma}
        {@let@~@unpack@~(x_\tau,x_z,x_k,x_r)~=~e_1~@in@~e_2}
        {\tau_2}
}{TcLetUnpack}
$$

$$
\ruleIN
{
    \Typecheck{\Delta}{\Gamma}{e_1}{@Stream@~c~\BB}
    \quad
    x_c~\not\in~\Delta
    \quad
    \Typecheck
        {\Delta,~x_c :_k @Clock@}
        {\Gamma,~x_s : @Stream@~x_c~\BB}
        {e_2}
        {\tau_2}
}
{
    \Typecheck{\Delta}{\Gamma}
        {@let@~@subrate@~(x_c,x_s)~=~e_1~@in@~e_2}
        {\tau_2}
}{TcLetSubrate}
$$


\caption{Types of expressions}
\label{fig:source:type:exp}
\end{figure*}



\subsection{Types}
In figure~\ref{fig:source:type:defs} we define the types.
We use $T$ for base types, of which we only have natural numbers and booleans.

Modalities are $M$, with @Aggregate@ meaning the result of some fold, @Element@ being an element of the input stream, and @Pure@ being a pure computation.
The modalities are defined so that @Aggregate@ and @Element@ computations cannot be mixed.
There is no way to turn an aggregate computation into an element computation, as aggregate computations are only available once the entire stream has been read.
Element computations must be explicitly folded over or aggregated in some way, in order to produce an aggregate computation based off it.
\TODO{Staged computation similarities}

The @Pure@ modality is actually the absence of a mode; this is simply made explicit for clarity.

Finally we define function types as $T_\to$, which can be a simple return type with a modality, or a function taking a modality argument and returning a function.
For simplicity of compilation, we have strictly disallowed higher order functions by requiring the argument to be a base type rather than a function type.

At various points in the following rules we will implicitly convert function types $T_\to$ to modal base types $M~T$; at these points function arrows are not allowed.



\subsection{Function application with unboxing}
In figure~\ref{fig:source:type:wrap} there are two judgments defined, which are both used for the type of function application.
In contrast to traditional modality type systems, we infer the boxing and unboxing of various modalities.
\TODO{Citation and find something similar.}
\TODO{Probably talk about comonads or something}

Informally, if a function expects an @Element@ and is passed a @Pure@ argument, we should box the argument into @Element@.
If a function expects a @Pure@ computation and is passed an @Element@, we must unbox the argument, apply the function, then rebox the result.
However, this reboxing can only occur if the result is @Pure@ or @Element@ - an @Aggregate@ result cannot be boxed into an @Element@.

These function application judgments only deal with the unboxing case; boxing is dealt with later as the single rule (TcBox).

The judgment $\WrapMode{\tau}{m}{\tau'}$, corresponds to unboxing the a function argument of type $m$ and reboxing the function return type $\tau$ into modality $m$, with $\tau'$ as the boxed return type.
The rule (WrapModeArrow) simply recurses down through the function arguments, applying the judgment to the result.
The rule (WrapElement) applies when reboxing an @Element@ result as an @Element@, and similarly for (WrapAggregate).
For (WrapPure) the result type is a @Pure@, so it can be boxed to any mode.
If these rules do not apply, for example when an @Aggregate@ result is attempted to be reboxed as an @Element@, the function application is an error.

The judgment $\WrapApp{\tau}{\phi\ldots}{\tau'}$ attempts to apply the function type $\tau$ to the argument list $\phi\ldots$, performing unboxing as necessary.
When the argument list is empty and the function type is a nullary function or simple value type, (WrapAppFinished) simply returns the result type.

In the next rules, we use the syntax $a;~b$ for a list with $a$ as its head and $b$ as its tail.

When the actual argument and expected function argument modalities and types are equal, (WrapAppEqual) applies the function as normal, and continues along with any remaining arguments.

When the function expects a @Pure@ argument and is passed an @Element@ or an @Aggregate@, we perform reboxing as in (WrapAppPure).
This attempts to rebox the function result to the argument modality, and then continues by wrapping the remaining arguments.



\subsection{Expression}
With function application already defined, typechecking expressions in figure~\ref{fig:source:type:ctx} becomes quite standard.
The judgment $\Typecheck{\Gamma}{x}{\tau}$ means that under a context $\Gamma$, the expression $x$ has type and modality $\tau$.
Being a first-order language, the typing rule returns a first-order type rather than a function type.

Rules (TcVar) and (TcPrim) simply look up the variable or primitive in the environment.
While the environment contains function types $T_\to$, these rules only apply if the result is a nullary function with no arguments.

The rule (TcBox) allows any @Pure@ computation to be implicitly cast to another mode, such as @Element@ or @Aggregate@.
This is equivalent to boxing the expression.

There are two different application rules (TcAppVar) and (TcAppPrim), which simply look up the function type in different environments before using the boxed function application judgement from figure~\ref{fig:source:type:wrap}.
In these ruler we use ``$\ldots$'' as shorthand for finding the types of all arguments.

The rule (TcLet) typechecks its definition and adds it to the environment when typechecking the body.

For (TcFilter), the predicate must be an @Element@, as it is executed for each element of the input.
The rest of the computation must return an @Aggregate@ over the filtered input.

For (TcFold), the form of fold is $@fold@~v~@=@~x~@then@~x'$ where $v$ is the name of the fold binding, $x$ is its initial value, and $x'$ computes the successor value, based on the previous value of $v$.
The initial expression $x$ must be @Pure@ of some type $\tau$, and the successor expression $x'$ is typechecked with the fold binding set to $@Element@~\tau$ in the environment.
The result of the successor expression must also have type $@Element@~\tau$.
This is because the current value of the fold is available to the successor expression, as well as the current elements, but the results of other folds are not available.
In the remaining context, the fold value is only available as an @Aggregate@.

Finally, (TcGroup) requires its key or group ``by'' to be an @Element@, and the value to be an @Aggregate@.
It builds up a map from key to value, repeatedly applying the aggregate as new values are found.

\subsection{Single-pass restriction}

We now describe how this typesystem encodes the single-pass restriction: that is, that if a given query type-checks, it only needs to look at each element in the stream once.

First, look at an example that would require two passes over the data: maybe @filter open > mean open in count@.
In this case, $\mit{open}$ has type $\Elm{\NN}$ while $\mit{mean~open}$ has type $\Agg{\NN}$.
The primitive $>$ has type $\NN \to \NN \to \NN$ and can be boxed to either @Element@ or @Pure@, but not both.
This means that $\mit{open} > \mit{mean open}$ is a type error.

In Icicle, folding and boxing pure values are the only ways to produce @Aggregate@ types.
The other typing rules that refer to @Aggregate@s require an existing @Aggregate@ and do not produce an aggregate on their own.
This means that all @Aggregate@s must either be the result of folds, which in general require all their input before they can be computed, or pure computations.

There is also no way to take an @Aggregate@ and convert it to a @Pure@ or an @Element@.

The only way to require multiple passes over the data, therefore, is to have a @Pure@ or @Element@ computation that depends on the result of an @Aggregate@ computation.
And so on.



%!TEX root = ../Main.tex
\section{Core language}
\label{s:Core}

This section presents the Core language, which we use as an intermediate language before converting to imperative code.
\TODO{Do we need formal semantics and type system?}


\subsection{Grammar}

%!TEX root = ../Main.tex

\begin{figure}

\begin{tabbing}
MMMM \= M \= M \= \kill
$\mit{CoreExp}$
    \> $=$  \> $n$          \\
    \> $~|$ \> $v$ \\
    \> $~|$ \> $\mit{CorePrim}~(n*)$ \\
\\
$\mit{CorePrim}$
    \> $=$  \> $@+@~|~@>@~|~@if@$ \\
\\
$\mit{FoldInit}$
    \> $=$  \> $\mit{CoreExp}$ \\
$\mit{FoldAcc}$
    \> $=$  \> $(n~=~\mit{CoreExp})*~\mit{CoreExp}$ \\
\\
$\mit{Fold}$
    \> $=$  \> $@Fold@~\mit{FoldInit}~\mit{FoldAcc}$ \\
\\
\\
$\mit{Program}$
    \> $=$  \> @Befores@ \\
    \>      \> \> $(n~=~\mit{CoreExp})*$ \\
    \>      \> @Streams@ \\
    \>      \> \> $(n~=~\mit{CoreExp})*$ \\
    \>      \> @Folds@ \\
    \>      \> \> $(n~=~\mit{Fold})*$ \\
    \>      \> @Extracts@ \\
    \>      \> \> $(n~=~\mit{CoreExp})*$ \\
    \>      \> @Outputs@ \\
    \>      \> \> $(n~=~n)*$ \\
\end{tabbing}


\caption{Grammar for Icicle Core}
\label{fig:core:grammar}
\end{figure}



\TODO{Think it might actually be better to put @Streams@ and @Folds@ together, and allow @Stream@ worker functions to refer to previous folds but only as scans. Would make implementation of scans easier?}

In figure~\ref{fig:core:grammar} we introduce the intermediate language.
The top-level of this language is $\mit{Program}$ with four stages of computation: pure computations, followed by stream transformers, then folds, and finally extracting the result of the folds.
Each stage of computation can only refer to the stages before it.

The expression type for Core is much simpler than Source, $\mit{CoreExp}$ having just names, primitive application and let bindings.
More primitives are required for the Core language, as some filters in the Source language are converted to @If@s, and groups are converted to use map primitives such as lookup and insert.
The @scan@ primitive is no longer used in the Core language as it is just converted to a variable reference.

The first section of the program is the @Befores@ section, which is just binding names to pure expressions.
These expressions cannot refer to any part of the stream, any computation, nor the results of any folds.

Next are @Streams@: streams and stream transformers.
The first stream type is @Source@, which is simply the unmodified input stream.
The two stream transformers take a function and the name of the input stream.
These are @Map@ which applies a function to each element, and @Filter@ which filters out according to a predicate.
Neither of the worker functions can refer to any streams, only the bindings in the @Befores@ section.
All of these stream transformers can be computed \emph{on-line} and by looking at a single element at a time, without any sort of accumulator.

In the third section are the @Folds@, which perform some reduction over an input stream.
Each fold has a constructor function which is successively applied to the latest accumulator and each stream value; an initialiser or zero value for the accumulator; and the name of the input stream.
The initialiser is pure, and so can only refer to variables defined in @Befores@.
The constructor function can refer to @Befores@, previous @Folds@, but not directly to the @Streams@.
Referring to previous @Folds@ in the constructor body gives the latest value of the accumulator, this is used to implement the @scan@ primitive.

Then the @Extracts@ are bound which can refer to the results of @Folds@ and any @Befores@.
These are used to perform and post-computation on the folds which cannot be done at each step, for example dividing the sum and the count to get the mean.

Finally, the @Outputs@ are defined, which define which variables are to be returned as the result of the query.
These can refer to @Befores@, @Folds@ and @Extracts@.

\subsection{Fusion}

When fusing together two queries, we can simply give each unique names, then append each respective section together; the @Befores@ appended to the @Befores@, @Streams@ to @Streams@ and so on.
After appending them together, we remove duplicate bindings by iterating through each binding set and comparing it to the previous bindings in the set.
If a duplicate is found, we remove the later binding and change all references in the rest of the program to the earlier binding.

The program definition is more complicated than necessary, as all programs of this form can actually be expressed as a single fold and an extract.
However, having smaller components makes finding duplicate computations after fusion easier.
For example, if mean is expressed as two folds (sum and count) then removing the duplicate after fusing with another count is rather simple.
If mean were expressed as a single fold containing both sum and count, removing the duplicate count would require somehow splitting apart the sum and count or removing duplicates some other way.


\subsection{Conversion from Source}

Yes yes convert from Source fix that

% %!TEX root = ../Main.tex

\begin{figure*}

$$
\boxed{\CoreComp{e}{[Bind]}{[Op]}{x}}
$$

$$
\ruleAx
{
    \CoreComp{v}{\emptyset}{@let@~x'~=~v}{x'}
}{CVal}
\ruleAx
{
    \CoreComp{x}{\emptyset}{\emptyset}{x}
}{CVar}
\ruleIN
{
    \CoreComp{e_1}{b_1}{o_1}{x_1}
    \quad
    \CoreComp{e_2}{b_2}{o_2}{x_2}
}{
    \CoreComp{e_1~e_2}{b_1;b_2}{o_1;o_2;~@let@~x'~=~x_1~x_2}{x'}
}{CApp}
$$


\caption{Conversion to Core}
\label{fig:core:compile}
\end{figure*}





%!TEX root = ../Main.tex
\section{Code generation}
\label{s:Generation}

Code generation should be pretty easy I guess cite flow fusion paper.

Because we don't have any zips or other that takes multiple streams, we don't have to deal with hard problems.

While not all streams have the same rate, there is no way to express an operation that takes two different rates.


%!TEX root = ../Main.tex
\section{Conclusion}
\label{s:Conclusion}

In this paper we have described a fusion algorithm that handles combinators with value-dependent access patterns, such as @merge@.
This is achieved by converting combinators to deterministic finite automata, then combining machines in a way that is similar to parallel execution of the machines.

As our algorithm only takes two machines, an entire combinator program must be fused by repeatedly fusing pairs of machines.
However, the order in which machines are fused can affect whether or not fusion is possible.
For example, given two inputs, maps of the inputs, and zipping the results:

\begin{code}
zipf xs ys
 = let xs' = map (+1) xs
       ys' = map (+1) ys
       zs  = zip xs' ys'
   in  zs
\end{code}

In this example, if the machines for @xs'@ and @ys'@ are fused together first, the fusion will prematurely decide on an ordering of both machines: perhaps all @xs'@ will be computed followed by all @ys'@, perhaps the other way around, or maybe they will be interspersed as we wish.
Then, when this arbitrarily-ordered machine is fused with the @zip@, it will fail as the @zip@ would need to read all the @xs'@ before finding a @ys'@.
It is important to note, however, that the ordering of fusion does not affect the semantics of the result, just whether a result is found.

To ensure a fusion result in this case, we must first fuse @zs@ with one of its inputs, then with the other input.
It is possible to attempt all possible permutations of the order and take the first that succeeds.
We are confident that there exists an algorithm to find a valid ordering, but this is left to future work.


Most dataflow language optimisations tend to focus on fully static networks where the exact access pattern is known at compile time\cite{thies2002streamit}, but disallowing combinators like merge join, append and filter.
At the other side of the spectrum, some dataflow languages such as Lucid\cite{stephens1997survey} focus on expressivity, forgoing any kind of static analyses for optimisations.

Merge joins, appends, and filters are not typically necessary for the bulk kind of operations such as audio transforms, video compression and so on, that regular dataflow has found its applications in\cite{johnston2004advances}.

For querying large data sets merge-like combinators are essential.
The ``MapReduce'' framework has been touted as a solution to easy distribution of workloads, but has been found to be lacking in flexibility\cite{vrba2009kahn}, in favour of Kahn process networks.
Kahn process networks alone are too flexible: many interesting properties we would like to assure, such as the absence of deadlocks, and ability to run in bounded memory, are undecidable.
Regular dataflow languages correspond to a subset of Kahn process networks, restricted to the point of keeping these desired properties\cite{thies2009language}.

In functional languages, fusion systems such as stream fusion\cite{coutts2007stream} and fold/build fusion\cite{jones2001playing} rely on the inliner to move producers into their consumers, after which rewrite rules can remove intermediate allocations.
This short-cut fusion works well for vertical fusion when producers have only single consumers, but when a producer is used by multiple consumers it cannot be inlined without duplicating work, so fusion will not occur.
Our earlier work on flow fusion\cite{lippmeier2013data} is able to fuse these multiple consumer cases, but only for a small set of combinators; neither @append@ nor @merge@ are handled.

Our goal is to extend the regular dataflow languages enough to allow this subset of dynamic combinators, while keeping the desired properties, for the purpose of compilation and optimisation.





%!TEX root = ../Main.tex
\section{Related work}
\label{s:Related}
\subsection{Data flow languages}

The closest related work are synchronous data flow languages such as {\sc Lustre}, Icicle programs are restricted to those that can be computed in bounded memory.
{\sc Lustre}\cite{halbwachs1991synchronous} achieves this in three main ways:

\begin{enumerate}
\item They allow only restricted set of primitives such as @when@ for filtering, @pre@ for a single element buffer, and so on.
While the primitives used are quite different, our primitives must also be carefully chosen to ensure bounded memory.
\item Cycles in the graph must contain at least one @pre@, to break the dependency loop.
In comparison we have only non-recursive definitions and recursion is handled by the @fold@ primitive.
\item Operators such as addition can only be applied to input streams with the same clock or rate; an expression like @(X when C)@ @+@ @(Y when (not C))@ would require an unbounded buffer\CITE{Lustre clock stuff}.
In contrast, because our only clock changing operation (@filter@) must return an aggregate there is no possibility of applying a primitive to different clocks.
\end{enumerate}

Existing data flow languages tend to focus on stream transformers running over potentially infinite data, while our purpose is specifically for performing aggregates that are only available at the end of the stream.
While this means these languages are more expressive, being able to output streams as well as aggregates, it can obscure the desired meaning.
By making the separation between @Element@ and @Aggregate@ computations explicit, a programmer cannot accidentally mistake an intermediate value of an aggregate for  the final value.

For example, checking an element of the stream against the mean of the entire stream is impossible in a single pass, and our typesystem rules this program out:
\begin{code}
   filter value > mean value
-> count
\end{code}

However, a similar program written in {\sc Lustre} would compare the running mean against the current value.
\begin{code}
count when (value > mean(value))
\end{code}

While this may be the desired behaviour in some cases, we believe this decision must be made explicit, by using the @running@ primitive.
\begin{code}
   filter value > running (mean value)
-> count
\end{code}

In summary, while our language is quite similar to existing synchronous data flow languages, the extra restrictions we impose allow us to use a simpler type system.


% %!TEX root = ../Main.tex

\newcommand\JudgeK[2]
{       #1 :: #2
}

\newcommand\JudgeT[3]
{       #1 \vdash #2 :: #3
}

\newcommand\JudgeTS[5]
{       #1 \vdash #2 :: #3 ~;~ #4 ~;~ #5
}

\newcommand\kbox        {\textrm{\textbf{box}}}
\newcommand\krun        {\textrm{\textbf{run}}}
\newcommand\kthen       {\textrm{\textbf{then}}}
\newcommand\kpure       {\textrm{\textbf{pure}}}

\newcommand\rData       {\textrm{Data}}
\newcommand\rClock      {\textrm{Clock}}
\newcommand\rDefn       {\textrm{Defn}}
\newcommand\rComp       {\textrm{Comp}}

\newcommand\rStream     {\textrm{Stream}}
\newcommand\rArray      {\textrm{Array}}
\newcommand\rNat        {\textrm{Nat}}
\newcommand\rBool       {\textrm{Bool}}

\newcommand\rdrain      {\textrm{drain}}
\newcommand\rstream     {\textrm{stream}}
\newcommand\rsum        {\textrm{sum}}
\newcommand\rsmap       {\textrm{smap}}
\newcommand\rsfold      {\textrm{sfold}}
\newcommand\rsscan      {\textrm{sscan}}
\newcommand\rsfilter    {\textrm{sfilter}}


\begin{figure*}
$$
\boxed{\JudgeT{\Gamma}{e}{\tau}}
$$

$$
\ruleI
{       x : \tau \in \Gamma
}
{       \JudgeT{\Gamma}{x}{\tau}
}
\textrm{(TyVar)}
\quad\quad
\ruleI
{       \JudgeT{\Gamma}{e_1}{\tau_1 \to \tau_2}
        \quad
        \JudgeT{\Gamma}{e_2}{\tau_1}
}
{       \JudgeT{\Gamma}{e_1~e_2}{\tau_2}       
}
\textrm{(TyApp)}
\quad\quad
\ruleI
{       \JudgeT{\Gamma,x:\tau_1}{e_2}{\tau_2}
}
{       \JudgeT{\Gamma}{\lambda x : \tau_1}{\tau_1 \to \tau_2}
}
\textrm{(TyAbs)}
$$



% -- Effectful --------------------------------------------
$$
\boxed{\JudgeTS{\Gamma}{e}{\tau}{n}{k}}
$$

$$
\ruleI
{       \JudgeTS{\Gamma}{e}{\tau}{n}{k}
}
{       \JudgeT{\Gamma}{\kbox~ e}{\Box~ n~ k~ \tau}
}
\textrm{(TyBox)}
\quad\quad
\ruleI
{       \JudgeT{\Gamma}{e}{\tau}
}
{       \JudgeTS{\Gamma}{\kpure~e}{\tau}{n}{k}
}
\textrm{(TyPure)}
$$

$$
\ruleI
{       \{ \JudgeT
                {\Gamma}{e_i}{\Box~ n_i~ k~ \tau_i} \}^i 
        \quad
        \JudgeTS
                {\Gamma,~ \{x_i : \tau_i \}^i}
                {e'}{\tau'}{n'}{k'}
}
{       \JudgeTS
                {\Gamma}
                {\krun~ \{ x_i : \tau_i = e_i \}^i ~\kthen~ e'}
                {\tau'}
                {max~ \{ n_i \}^i + n'}
                {k'}
}
\textrm{(TyRun)}
$$

% -- Prims ------------------------------------------------
\vspace{2em}
$$
\begin{array}{ll}
\rNat,\rBool      & :: \rData
\\[1ex]
%
\rArray           & :: \rClock \to \rData \to \rData
\\[1ex]
%
\rStream          & :: \rClock \to \rData \to \rDefn
\\[1ex]
%
\Box              & :: \rNat   \to \rClock \to \rData \to \rComp
\\[1ex]
%
\rdrain_{k,\tau}  & :: \rStream~k~\tau \to \Box^1_k~ (\rArray~ k~ \tau)
\\[1ex]
%
\rstream_{k,\tau} & :: \rArray~k~\tau  \to \rStream~k~\tau
\\[1ex]
%
\rsum_k           & :: \rStream~k~\rNat \to \Box^1_k~ \rNat
\\[1ex]
%
\rsmap_{k,a,b}    & :: (a \to b) \to \rStream~k~a \to \rStream~k~b
\\[1ex]
%
\rsfold_{k,a,b}   & :: (a \to b \to b) \to b \to \rStream~k~a \to \Box^1_k~ b
\\[1ex]
%
\rsscan_{k,a,b}   & :: (a \to b \to b) \to b \to \rStream~k~a \to \rStream~k~b
\end{array}
$$

\caption{Typing}
\label{fig:source:type:modal}
\end{figure*}


\section*{Acknowledgements}

\bibliographystyle{plain}
\bibliography{Main}

\end{document}


