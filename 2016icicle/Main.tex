\documentclass[preprint]{sigplanconf}
\usepackage{amssymb}
\usepackage{amsthm}
\usepackage{graphicx}
\usepackage{amsmath}
\usepackage{mathptmx}
\usepackage{mathtools}
\usepackage{stmaryrd}
\usepackage{hyperref}
\usepackage{alltt}
\usepackage{url}
\usepackage{float}
\usepackage{style/utils}
\usepackage{style/code}
\usepackage{style/proof}
\usepackage{style/judgements}

% -----------------------------------------------------------------------------
\begin{document}

% \exclusivelicense
% \conferenceinfo{}{}
% \copyrightyear{2015}
% \copyrightdata{}
\doi{}
% \pagenumbering{gobble} 

\title{Icicle: a single-pass query language}

\authorinfo{
  Amos Robinson$^\dagger$$^\ddagger$
  \and Huw Campbell$^\ddagger$
  \and Mark Hibberd$^\ddagger$
  \and Tran Ma$^\ddagger$
  \and Jacob Stanley$^\ddagger$
}{
  \vspace{5pt}
  \shortstack{
    $^\dagger$Computer Science and Engineering \\
    University of New South Wales, Australia \\[2pt]
    \textsf{amosr@cse.unsw.edu.au}
  }
  \shortstack{
    $^\ddagger$Ambiata      \\
    Big data and shit       \\[2pt]
    \textsf{\{first.last\}@ambiata.com}
  }
}

\maketitle
\makeatactive

\begin{abstract}
When streaming a large amount of data, simply iterating over the data may take hours.
In this case, the difference between a query that takes one iteration and a query that takes multiple iterations must be very explicit.
For many purposes, limiting queries to a single pass is sufficient.
We introduce a type system based on modality types to enforce this single pass restriction.
By using an appropriate intermediate language we can fuse multiple independent queries together, before extracting efficient single-loop C code.
\end{abstract}


\category
	{D.3.4}
	{Programming Languages}
	{Processors---Compilers; Optimization}

\terms
	Languages, Performance

\keywords
	Arrays; Fusion

%!TEX root = ../Main.tex
\section{Introduction}
\label{s:Introduction}

Suppose we have two input streams of numeric identifiers, and wish to perform some analysis on these identifiers. The identifiers from both streams arrive sorted, but may include duplicates. We wish to produce an output stream of unique identifiers from the first input stream, as well as produce the unique union of identifiers from both streams. Can we perform both of these tasks at once, without needing to read through the stream data multiple times, and without needing unbounded buffering? Here is how we might write the source code, where @S@ is for @S@-tream.
\begin{code}
  uniquesUnion : S Nat -> S Nat -> (S Nat, S Nat)
  uniquesUnion sIn1 sIn2
   = let  sUnique = group sIn1
          sMerged = merge sIn1 sIn2
          sUnion  = group sMerged
     in   (sUnique, sUnion)
\end{code}

In this implementation the @group@ operator filters out consecutive duplicates, while @merge@ combines two sorted streams so that the output remains sorted. This example has a few interesting properties. Firstly, the data-access pattern of @merge@ is \emph{value-dependent}, meaning that the order in which this operator pulls values from @sIn1@ and @sIn2@ depends on the values themselves. If all the values from @sIn1@ are smaller than the values in @sIn2@, then @merge@ will pull all values from @sIn1@ before pulling the rest from @sIn2@, and vice versa. Secondly, although @sIn1@ occurs twice in the program, at runtime we only want to handle the elements of each stream once. To achieve this, the compiled program must coordinate between the two uses of @sIn1@, so that values are only read when both the @group@ and @merge@ operators are ready to receive a new value. Finally, as the stream length is assumed to be unbounded, we cannot buffer an arbitrary number of elements read from either stream, or risk running out of local storage space.

For an implementation which does \emph{not} use stream fusion, we might implement each of the operators as a separate concurrent process, and send each identifier value using an intra-process communication mechanism. Developing such an implementation could be easy or hard, depending on what language features are available for concurrency. However, worrying about the \emph{performance tuning} of such a system, such as whether we need back-pressure, or how to chunk the stream data to reduce the amount of communication overhead, is invariably a headache. 

We might instead define some sort of uniform interface for data sources, with a single `pull' function that provides the next value in each stream. Each operator could be given this interface, so that the next value from each result stream is computed on demand. This is approach is commonly taken with implementations of physical operators in data base systems. However, this `pull only' model does not support operators with multiple outputs, such as our derived @uniquesUnion@ operator, at least not without unbounded buffering. Suppose a consumer pulls many elements from the result @sUnique@ stream. The implementation needs to pull the corresponding source elements from @sIn1@ \emph{as well} as buffering an arbitrary number of matching elements from @sIn2@. It needs to buffer an aribrary number of elements from @sIn2@ because there is no guarantee of when a consumer will also pull from the @sUnion@ result stream. Once that happens the elements from @sIn2@ no longer need to be retained, but not before.

Instead, for a single threaded program, we want to perform \emph{stream fusion}, which takes the dataflow network and produces a simple sequential loop that gets the job done without requiring extra process-control abstractions and without requiring unbounded buffering. Sadly, existing stream fusion transformations cannot handle our example. As observed by \citet{kay2009you}, both pull-based and push-based fusion have fundamental limitations. Pull-based systems such as short cut stream fusion~\cite{coutts2007stream} cannot handle cases where a particular stream or intermediate result is used by multiple consumers. We refer to this situation as a \mbox{\emph{split} --- in the} dataflow network the flow from input stream @sIn1@ is split into both the @group@ and @merge@ consumers. 

% Leave this to related work. We've already mentioned a canonical pull-based system.
% Recent work on stream fusion by \citet{kiselyov2016stream} uses staged computation to ensure all combinators are inlined, but for splits this causes excessive inlining which duplicates work, due to values of the source arrays being read multiple times.

Push-based systems such as foldr/build fusion~\cite{gill1993short} also cannot fuse our example because they do not support operators with multiple inputs. We refer to such a situation as a \emph{join} --- in our example the @merge@ operator expresses a join in the data-flow graph. Some systems support both pull and push: data flow inspired array fusion~\cite{lippmeier2013data} allows both splits and joins but only for a limited, predefined set of operators. More recent work on polarized data flow fusion~\cite{lippmeier2016polarized} \emph{is} able to fuse our example, but requires the program to be rewritten to use explicitly polarized stream types. 

% The mechanism that combines the implementations of both operators, to yield efficient imperative code also depends on the general purpose compiler optimisations implemented by GHC, and it can be difficult to tell if these have ``worked'' without inspecting the intermediate representations of the compiler.

Synchronous dataflow languages such as Lucy-n~\cite{mandel2010lucy} reject value-dependent operators such as @merge@, while general dataflow languages fall back on less performant dynamic scheduling for these cases \cite{bouakaz2013real}. The polyhedral array fusion model~\cite{feautrier2011polyhedron} is used for loop transformations in imperative programs, but operates at a much lower level. The polyhedral model is based around affine loops, which makes it difficult to support filter-like operators such as @group@ and @merge@.

In our new system we still view the program as a concurrent process network. Each operator is a separate process, and the stream data flows through communication channels between the processes. Each operator is expressed as a restricted, sequential imperative program with commands that include both @pull@ for reading from an input stream and @push@ for writing to an output stream. The fusion transform takes the concurrent process network and \emph{sequentializes} it into a single process by choosing a particular evaluation order that requires no unbounded intermediate buffers. When the fusion transformation succeeds we know it has worked. There is no need to inspect intermediate representations of the compiler to debug poor performance, which is a common problem in systems based on general purpose program transformations \cite{lippmeier2012:guiding}.

In summary, we make the following contributions:
\begin{itemize}
\item a process calculus for encoding infinite streaming programs (\S\ref{s:Processes});
\item an algorithm for fusing these processes, the first to support arbitrary splits and joins (\S\ref{s:Fusion});
\item numerical results that demonstrate that the algorithm is well behaved when the number of fused processes is large. The size of the fused result program is not excessive. \TODO{Ref}
\item a formalization and proof of soundness for the core fusion algorithm in Coq (\S\ref{s:Proofs});
\end{itemize}

Our fusion transformation for infinite stream programs could also serve as the basis for an \emph{array} fusion system, using a natural extension to finite streams. We discuss this extension in \S\ref{s:Finite}.

% TODO: We can't make the appendix a contribution because the reviewers are not required to read appendices.
% \item and show our processes are general enough for many combinators, including segmented operations (\S\ref{s:Combinators}).

% \ben{Add a few more sentences on related work. Explain how this work extends the old flow fusion paper. It is not short-cut fusion like Oleg's recent work. We are not in the same space as Fortran style array fusion transformations like polyhedral}

% BL: describe this later.
% Furthermore, the data-flow fusion system of~\cite{lippmeier2013data} only deals with a fixed set of baked-in combinators. 

% BL: Shift the detailed description into a later section.
% The example above has three combinators, so the process network has three processes.
% The two @writeFile@s outputs are treated as sinks that values can be pushed to at any time, and are not converted to processes.
% During code generation, any output values from the @uniques@ and @union@ streams are sent to the corresponding @writeFile@ sink, but we do not address code generation in this paper.

% The process for @uniques@ is defined by the @group@ combinator, and can be thought of as an imperative loop: first it reads from its input stream @file1@ and stores that in a local variable.
% It also keeps track of the last pulled value, and compares that against the newly read value.
% If they are different, it pushes the new value to its output stream @uniques@.
% In either case, it updates the last pulled value and loops back to the start to pull from @file1@ again.

% The process for @merged@ is defined by the @merge@ combinator, which starts by reading from both @file1@ and @file2@ and storing these in local variables.
% It then compares its two values to see which is the smaller.
% If the value from @file1@ is smaller, it pushes that value and pulls a new value from @file1@, otherwise it pushes the value from @file2@ and pulls from @file2@.
% This is performed in a loop.

% We fuse these two processes by interleaving the two such that the shared input @file1@ is only pulled from when both processes agree.
% The new process pulls from @file1@, which is copied to the variables for both processes.
% The @uniques@ process now has all it needs to execute, so it checks the value against the last pulled value, pushes if necessary, and goes back to try to pull from @file1@ again.
% At this stage the @merged@ process still has a value from @file1@ that it has pulled but not used, so @uniques@ cannot pull from @file1@ again.
% We now let @merged@ run, pulling from @file2@ and checking which is smaller.
% If the value from @file1@ is smaller, the value is emitted and @merged@ wishes to pull a new value from @file1@.
% Both processes now agree on pulling from @file1@ again, so the new value is pulled and @uniques@ can run again.
% Otherwise if the value from @file1@ is not smaller, the value from @file2@ is emitted and @merged@ pulls from @file2@ with no coordination required.

% If we wish to ensure that each value is only read from the file once, we must coordinate between the two use sites: when @uniques@ requires a new value it must ensure that @union@ is ready to receive a new value, and vice versa. Note that we cannot just execute @uniques@ while storing the read values in a buffer, as this may require more memory than is available.
% In order to fuse this example, we require both pull \emph{and} push streams.
% The input streams must be pull streams since the order values are required is determined by the @merge@ combinator.
% For the same reason, the outputs sent to each @writeFile@ must be push streams.

% Fusion for array programs is important for removing intermediate arrays, reducing memory traffic and reducing allocations.
% However, when dealing with data too large to fit in memory such as tables on disk, removing intermediate arrays becomes essential rather than just desirable.
% Attempting to create an intermediate array of such amounts of data would lead to thrashing and swapping to disk, or perhaps even running out of swap.
% For these situations, some sort of assurance of total fusion is required: either the program can be fused with no intermediate arrays or unbounded buffers, or it will not compile at all.


% Fusion eliminates intermediate array buffers and converts pipelines of array combinators into low-level iteration based loop code. Different fusion systems can handle 

% When comparing fusion systems, three important criteria to consider are: whether the system supports splits, where a stream is used multiple times; whether it supports joins, where a combinator has multiple inputs; and whether arbitrary combinators such as @merge@ and segmented appends can be encoded. Existing fusion systems support one or two of these, but not three. We present a fusion system based on process calculus that supports all three: splits, joins and arbitrary combinators.

% Our system has been formalised in Coq where we have proved soundness of the fusion algorithm. It is expressive enough to encode a wide range of combinators including operations on segmented arrays.

% \amos{``Arbitrary combinators'' is not quite true. How can we distinguish combinators we support from 2013 data flow fusion paper? Perhaps by mentioning value-dependent input / access patterns.}

% Leave this to the description of the algorithm, not the abstract.
% We encode each combinator as a separate process with any number of input and output channels. Each process is sequential but multiple processes can be executed concurrently. We give a concurrent execution semantics for multiple processes, but these are used only as a specification for how the fused program must behave. The fused program itself is sequential and can easily be converted to simple imperative code.

% Our fusion algorithm takes two concurrently executable processes and creates a sequential interleaving of the two such that they execute with no unbounded buffers.
% If fusion would require unbounded buffers (or the fusion algorithm wrongly infers that it would) then fusion fails.
% If fusion fails, the user can be presented with an error message telling them which combinators could not be fused.
% For scenarios where fusion is required, this is a great advantage over fragile shortcut fusion systems.

% BL: leave the apologies to the conclusion.
% The version presented here deals with infinite streams, and we informally describe the extensions required to support finite streams.

% BL: leave this to the main intro.
% optimising high-level array and streaming computations, as it reduces memory traffic and intermediate arrays. The benefits of removing intermediate arrays are even more important as data sizes approach the size of memory.

%!TEX root = ../Main.tex
\section{Icicle Source}
\label{s:Source}

In this section we describe the grammar and evaluation semantics of the Icicle language.

Icicle queries execute over a single table, which must at least contain an entity identifier and a date.
Each query implicitly groups the data by entity and sorts it by date, before iterating through the data and computing some aggregate.
We define the stocks table as follows, leaving out the company code (entity identifier) and date.

\begin{tabbing}
MM \= MM \= MMMMMMMMMMM \= \kill
$@table@~\mi{stocks}$    \\
$\{$ \> $\mi{open}$ \> $:~\NN$ \\
$~,$ \> $\mi{close}$ \> $:~\NN$ \\
$\}$
\end{tabbing}

This defines the columns as natural numbers, and allows them to be referred to by name.
In the definitions and queries below, any reference to these names means the column itself, with the entire data for a given entity.
This is annotated with the mode @Element@, meaning that it exists for every element of the input.

We now define some useful functions and other definitions that can be used in the queries themselves.

First we define a function to compute the sum of some input data.
While traditionally sum would take just an input stream of numbers, we separate this into two distinct parameters.
The first parameter is the clock for the input stream, which defines the parts of the input stream to sample.
This can be thought of as a stream of boolean flags, where true denotes that the input stream should be sampled at this time.
The second parameter is the actual value to sample at the clock.
In Icicle this has type $\Elm{\NN}$, meaning it is available as a number at any part of the stream.

By separating the clock from the element values, we guarantee that any references to elements denote the same point in the stream, and can always be used together.
For example, computing the difference between the two fields above ($\mi{open} - \mi{close}$) refers to the open and close fields at the same point.
The clock is used when performing a fold, and determine at which points to sample the elements and update the accumulator.
This is in constrast to traditional streaming dataflow languages, where the element data and the clock are intertwined.
This requires clock analysis\CITE to tell whether two element streams can be used together with a bounded buffer.

We denote clocks in brackets as they are a different universe to normal expressions.
\begin{tabbing}
MM \= MM \= MMMMMMMMMMM \= \kill
@function@ 
$\mi{sum}[c]~v$    \\
\> $=$  \> $@fold@[c]~s~=~0~@then@~s~+~v$ \\
\> @in@ \> $s$. \\
\end{tabbing}

The result of $\mi{sum}$ above has type $\Agg{\NN}$.
Aggregate types denote that their value is only available after the entire input stream has been seen.
By restricting the bodies of folds to be element types, and their output to be aggregate types, it means no reference to the result of folds can be made in the body of a fold.

We can define a query that computes the sum of each company's open prices.
Query definitions are similar to function definitions, except they only take one argument: the clock describing all available input.
They also must return aggregates, as an element query could be unbounded in size.
\begin{tabbing}
MM \= MM \= MMMMMMMMMMM \= \kill
@query@ 
$\mi{opensum}[c]$    \\
\> $=$  \> $\mi{sum}[c]~\mi{open}$. \\
\end{tabbing}

We can run $\mi{opensum}$ over some example data.
This example data has the company code and date for each entry, as well as the stock's open and close prices.
Note that the sum only becomes available at the end of each entity, and only applies to a particular entity's elements.

\begin{code}
CODE, OPEN, CLOSE, DATE,        OPENSUM
ABC,    10,    12, 2015-01-01,       10
IAG,    11,    12, 2015-01-01
IAG,    12,    10, 2015-01-02,       23
\end{code}


Next, we can define functions $\mi{count}$ and $\mi{mean}$ in terms of $\mi{sum}$.
Count passes the argument $1$ to sum, and this is interpreted as an element containing all ones.
The input clocks are passed along as-is.
\begin{tabbing}
MM \= MM \= MMMMMMMMMMM \= \kill
@function@ 
$\mi{count}[c]$                                     \\
 \> $=$  \> $\mi{sum}[c]~1$.                           \\
                                                    \\
@function@ 
$\mi{mean}[c]~v$                                       \\
 \> $=$  \> $\mi{sum}[c]~v~/~\mi{count}[c]$.              \\
\end{tabbing}

Clocks can be reduced to filter out certain elements using @where@.
In this query, we use @where@ to count the number of days where the close price is higher than the open price.
\begin{tabbing}
MM \= MM \= MMMMMMMMMMM \= \kill
@query@ 
$\mi{gap}[c]$                                        \\
 \> $=$  \> $\mi{count}[c~@where@~\mi{close}~>~\mi{open}]$.       \\
\end{tabbing}

In the example below we have shown the result of the element expression $\mi{close}~>~\mi{open}$ for clarity only; this is not included in the output of the $\mi{gap}$ query.
\begin{code}
CODE, OPEN, CLOSE, CLOSE > OPEN, GAP
ABC,    10,    12,            T,   1
IAG,    11,    12,            T
IAG,    12,    10,            F,   1
\end{code}

\TODO{Mention resumables and bubblegum?}

\TODO{Talk about scan}

We formally define the grammar of the language in figure~\ref{fig:source:grammar}.
Note that general application is not allowed: arguments can only be applied to primitives and named functions.
Similarly, functions and primitives must be fully applied.
This, combined with the lack of lambda construction, means that higher order functions are outlawed.
While a first-order language (one lacking higher order functions) offers somewhat less abstraction, it simplifies the typing rules and makes it easier to support our performance guarantees.

While the examples shown allow user defined functions, there is no recursion allowed.
This means that all function definitions can be inlined into the user query with guaranteed termination.
When converting to the intermediate language, we assume that this inlining has occurred.

\subsection{Evaluation}

%!TEX root = ../Main.tex

\begin{figure}

\begin{tabbing}
MMM \= MM \= MMM \= MM \= MMMMMM \= \kill
$\mi{T}$
\GrammarDef
  $@Int@~|~@Bool@~|~@Map@~T~T$
\\
$\mi{\TauMode}$
\GrammarDef
  $T~|~@Element@~T~|~@Aggregate@~T$
\\
$\mi{\TauFun}$
\GrammarDef
  $\ov{(\TauMode)}~\to~\TauMode$
\\
\\

$\mi{Table}$
\GrammarDef
  $@table@~x~@{@~\ov{(x~:~T;)}~@}@$
\\
\\

$\mi{Exp}$
\GrammarDef
  $x~|~\NN~|~\mi{Prim}~\ov{\mi{Exp}}~|~x~\ov{\mi{Exp}}$
\GrammarAlt
  $@let@$   \> $x~=$ \> $\mi{Exp}~@  in@~\mi{Exp}$
\GrammarAlt
  $@fold@$  \> $x~=$ \> $\mi{Exp}~@then@~\mi{Exp}$
\GrammarAlt
  $@filter@$\> \> $\mi{Exp}~@  of@~\mi{Exp}$
\GrammarAlt
  $@group@$ \> \> $\mi{Exp}~@  of@~\mi{Exp}$
\\
\\

$\mi{Prim}$
\GrammarDef
  $(@+@)~|~(@-@)~|~(@*@)~|~(@/@)~|~(@==@)~|~(@/=@)~|~(@<@)~|~(@>@)$
\GrammarAlt
  $@lookup@$
\\
\\


$\mi{Def}$
\GrammarDef
  $@function@~f~\ov{(x~:~\TauMode)}~=~\mi{Exp}$
\GrammarAlt
  $@query@~x~=~\mi{Exp}$
\\
\\
$\mi{Top}$
\GrammarDef
  $\mi{Table};~\ov{\mi{Def};}$
\end{tabbing}

\caption{Icicle Grammar}
\label{fig:source:grammar}
\end{figure}



In figure~\ref{fig:source:eval} we describe the bigstep evaluation semantics of the language.
We use the shorthand $\ov{\Gamma}$ to mean a list of $\Gamma$, and this same overline syntax is used as a variable containing a list.
The judgment $\BigstepG{x}{\ov{v}}{v}$ uses a list $\ov{\Gamma}$, with one environment for each element of the input stream.
The environment $\Gamma$ is used for scalar variables that are not associated with any particular input element.
Next is the expression $x$ that is to be evaluated.
The last two are outputs $\ov{v}$ and $v$.
The output $\ov{v}$ is the intermediate result of the computation for each part of the input stream.
That is, each element of $\ov{v}$ is the result of evaluating $x$ on the input stream up to that element.
The final output, $v$, is the result at the end of the stream.
In most cases it is the same as the last element of $\ov{v}$, unless $\ov{v}$ is empty or for a variable lookup.

The reason for returning the result list $\ov{v}$ is to make implementing the @scan@ primitive easier.
This allows @scan@ itself to be evalauted effectively as a no-op, but does mean that each other rule has to implement its own @scan@ as well.


\TODO{It seems like we should be able to get rid of the single element return $v$, but then would it be strange with the return value being of length $max~1~|\ov{v}|$?}

When evaluating on an input such as our @stocks@ table, we would create an environment for each row, with open and close as the variables.
The rule (EvTop) simply uses the empty environment for the scalar environment, and ignores the stream output.

For variables, the rules (EvVar) and (EvVarInput) look up the variable in the contexts.
In the case of direct inputs as introduced by (EvTop), there is no corresponding entry in the scalar environment, so (EvVarInput) is used which returns an error as the scalar output.
In this way, if the input stream is used as an aggregate it will be an evaluation error, but if it is used as an element value and folded upon, the error value will be thrown away.
For normal variables, there should be a binding in all environments.

Let bindings are handled by rule (EvLet), which evaluates its definition, adds part of the stream result to each stream environments, and adds the scalar result to the scalar environment.

For the rule (EvFold), first the zero or initial value of the fold is computed, using only its scalar value.
The extra rules (ScanNil) and (ScanCons) are then used to repeatedly apply the fold computation to each stream input, starting from the initial value.
The list of fold values are added to the stream environments, and the last of the fold values, or the initial value if none, is added to the scalar environment.

In (EvFilter) the predicate is evaluated to a list of results. 
The stream environments are filtered according to this predicate list using $\mit{pack}$, and the rest of the expression is computed.
However, the result list of the rest of the expression is only the scan of the filtered stream inputs.
We use $\mit{extend}$ to add back parts of the stream that have been filtered out, but must first find the initial value of the scan to start from.

The primitive operator rules (EvAdd) and (EvGt) apply their operator to each element of the inputs.
The primitive (EvNat) returns a list of constant numbers, one for each stream element.
Finally, (EvScan) is a no-op since the hard work of computing the @scan@s is actually implemented in the rest of the rules.


%!TEX root = ../Main.tex

\begin{figure*}

$$
\boxed{\SourceStepX{\RawMode}{\Sigma}{e}{V'}}
$$
$$
\ruleAx
{
    \SourceStepX{n}{\Sigma}{V'}{V'}
}{EVal}
\ruleIN
{
    x~=~V'~\in~\Sigma
}
{
    \SourceStepX{n}{\Sigma}{x}{V'}
}{EVar}
\ruleIN
{
  \SourceStepX{n'}{\Sigma}{e}{v}
  \quad
  \SourceStepX{n}{\Sigma,~x=v}{e'}{v'}
}
{
  \SourceStepX
    {n}
    {\Sigma}
    {@let@~(x~:~n'~\tau')~=~e~@in@~e'}
    {v'}
}{ELet}
$$

$$
\ruleIN
{
    \SourceStepXP{\Sigma}{e}{\VValue{v}}
}
{
    \SourceStepXE{\Sigma}{e}{\VStream{(\lam{\store} v)}}
}{EBoxStream}
\ruleIN
{
    \SourceStepXP{\Sigma}{e}{\VValue{v}}
}
{
    \SourceStepXA{\Sigma}{e}{\VFold{()}{(\lam{\store~()} ())}{(\lam{()} v)}}
}{EBoxFold}
$$

$$
\ruleIN
{
  \{ \SourceStepXP{\Sigma}{e_i}{\VValue{v_i}} \}
}
{
  \SourceStepXP
    {\Sigma}
    {p~\{ e_i \} }
    {\VValue{(p~\{v_i\})}}
}{EPrimValue}
\ruleIN
{
  \{ \SourceStepXE{\Sigma}{e_i}{\VStream{v_i}} \}
}
{
  \SourceStepXE
    {\Sigma}
    {p~\{ e_i \} }
    {\VStream{(\lam{\store} p~\{v_i~\store\})}}
}{EPrimStream}
$$

$$
\ruleIN
{
  \{ \SourceStepXA{\Sigma}{e_i}{\VFold{z_i}{k_i}{j_i}} \}
}
{
  \SourceStepXA{\Sigma}
    {p~\{ e_i \} }
    {\VFold
      {(z_0 \times \cdots \times z_i)}
      {(\lam{\store~(v_0 \times \cdots \times v_i)}
        k_0~\store~v_0 \times \cdots \times k_i~\store~v_i)}
      {(\lam{(v_0 \times \cdots \times v_i)}
        p~\{j_i~v_i\})}}
}{EPrimFold}
$$

$$
\ruleIN
{
  \SourceStepXE{\Sigma}{e}{\VStream{f}}
  \quad
  \SourceStepXA{\Sigma}{e'}{\VFold{z}{k}{j}}
}
{
  \SourceStepXA{\Sigma}
    {@filter@~e~@of@~e'}
    {\VFold
      {z}
      {(\lam{\store~v}
         @if@~f~\store~@then@~k~\store~v~@else@~v)}
      {j}}
}{EFilter}
$$

$$
\ruleIN
{
  \SourceStepXE{\Sigma}{e}{\VStream{f}}
  \quad
  \SourceStepXA{\Sigma}{e'}{\VFold{z}{k}{j}}
}
{
  \SourceStepXA{\Sigma}
    {@group@~e~@of@~e'}
    {\VFold
      {\{\_~\Rightarrow~z\}}
      {(\lam{\store~m}
% \{ k_i~\Rightarrow~k~\store~v_i ~|~ k_i \Rightarrow v_i~\in~m~\wedge~k_i~=~f~\store \} \cup m)}
        m[f~\store~\Rightarrow~k~\store~(m[f~\store])])}
      {(\lam{m}
        \{k_i~\Rightarrow~j~v_i~|~k_i~\Rightarrow~v_i~\in~m\})}}
}{EGroup}
$$

$$
\ruleIN
{
  \SourceStepXP{\Sigma}{z}{\VValue{z'}}
  \quad
  \SourceStepXE{\{x_i~=~\VStream{(f_i \cdot @snd@)}~|~x_i=\VStream{f_i}~\in~\Sigma\},~x~=~\VStream{@fst@},~\Sigma}{k}{\VStream{k'}}
}
{
  \SourceStepXA{\Sigma}
    {@fold@~x~=~z~@then@~k}
    {\VFold
      {z'}
      {(\lam{\store~v} k'~(v,~\store))}
      {(\lam{v} v)}}
}{EFold}
$$

$$
\boxed{\{x~\Rightarrow~\ov{V}\}~|~e~\Downarrow~V}
$$
$$
\ruleIN
{
  \SourceStepX
    {@Aggregate@}
    %% Review #1
    %% Figure 3, rule ETable: Should “vs_i” read “v_i”?
    {\{x_i~=~\VStream{(@fst@~\cdot~@snd@^i)}~|~x_i~\Rightarrow~\mi{v}_i~\in~t\}}
    {e}
    {\VFold{z}{k}{j}}
  \quad
}
{
  t~|~e~\Downarrow~
  j~(\mi{fold}~k~z~\{v_0 \times \cdots \times v_i \times ()~|~x_i~\Rightarrow~v_i~\in~t\})
}{ETable}
$$

$$
V'~     ::=~\VValue{V}
        ~|~\VStream {(V \stackrel{\bullet}{\to} V)}
        ~|~\VFold{V}{(V \stackrel{\bullet}{\to} V \stackrel{\bullet}{\to} V)}
                    {(V \stackrel{\bullet}{\to} V)}
$$
$$
\begin{array}{ll}

\RawMode~::=~@Pure@~|~@Element@~|~@Aggregate@

&

\Sigma~::=~\cdot~|~\Sigma,~x~=~V'

\end{array}
$$


\caption{Evaluation rules}
\label{fig:source:eval}
\end{figure*}




%!TEX root = ../Main.tex
\section{Types}
\label{s:Types}

%!TEX root = ../Main.tex

\begin{figure}

\begin{tabbing}
MM \= M \= AggregateM \= M \= ElementM \= M \= \kill
$T$
    \> $=$  \> $\NN~|~\BB~|~@Map@~T~T~|~@Array@~T$ \\
\\
$M$
    \> $=$  \> $@Aggregate@$ \> $~|~$ \> $@Element@$ \> $~|~$ \> $@Pure@$         \\
\\
$T_{\to}$
    \> $=$  
            \> $M~T~\to~T_{\to}$
            \> $~|~$
            \> $M~T$ \\
\\
$\Gamma$
    \> $=$  
            \> $\cdot$
            \> $~|~$
            \> $\Gamma,~n~:~T_\to$ \\
\end{tabbing}

\caption{Type definitions}
\label{fig:source:type:defs}
\end{figure}


\begin{figure*}

$$
\boxed{\TypecheckPrim{p}{T_\to}}
$$


$$
\ruleI
{
}
{ 
    \TypecheckPrim{@+@}{\NN~\to~\NN~\to~\NN}
}
\textrm{(TcPrimAdd)}
\quad
\ruleI
{
}
{
    \TypecheckPrim{@>@}{\NN~\to~\NN~\to~\BB}
}
\textrm{(TcPrimGt)}
\quad
\ruleI
{
}
{ 
    \TypecheckPrim{\NN}{\NN}
}
\textrm{(TcPrimNat)}
$$

$$
\ruleI
{
}
{ 
    \TypecheckPrim{@scan@}{\Agg{a}~\to~\Elm{a}}
}
\textrm{(TcPrimScan)}
$$

\caption{Typing primitives}
\label{fig:source:type:prim}
\end{figure*}


\begin{figure*}

$$
\boxed{\WrapMode{T_{\to}}{M}{T_{\to}}}
$$

$$
\ruleI
{
    \WrapMode{\tau}{m}{\tau'}
}
{
    \WrapMode{(\phi~\to~\tau)}{m}{(\phi~\to~\tau')}
}
\textrm{(WrapModeArrow)}
$$

$$
\ruleI
{ }
{
    \WrapMode{@Element@~\tau}{@Element@}{@Element@~\tau}
}
\textrm{(WrapElement)}
\quad
\ruleI
{ }
{
    \WrapMode{@Aggregate@~\tau}{@Aggregate@}{@Aggregate@~\tau}
}
\textrm{(WrapAggregate)}
\quad
\ruleI
{ }
{
    \WrapMode{@Pure@~\tau}{m}{m~\tau}
}
\textrm{(WrapPure)}
$$


$$
\boxed{\WrapApp{T_{\to}}{[T]}{M~T}}
$$

$$
\ruleI
{
}
{
    \WrapApp{(m~\tau)}{[]}{m~\tau}
}
\textrm{(WrapAppFinished)}
\quad
\ruleI
{
    \WrapApp{\tau_f}{\tau_x}{\tau}
}
{
    \WrapApp{(m~\phi~\to~\tau_f)}{(m~\phi;~\tau_x)}{\tau}
}
\textrm{(WrapAppEqual)}
\quad
\ruleI
{
    \WrapMode{\tau_f}{m}{\tau_f'}
    \quad
    \WrapApp{\tau_f'}{\tau_x}{\tau}
}
{
    \WrapApp{(@Pure@~\phi~\to~\tau_f)}{(m~\phi;~\tau_x)}{\tau}
}
\textrm{(WrapAppPure)}
$$

\caption{Function application with unboxing}
\label{fig:source:type:wrap}
\end{figure*}

\begin{figure*}

$$
\boxed{\Typecheck{\Gamma}{x}{M~T}}
$$


$$
\ruleI
{
    (n~:~\mu~\tau)~\in~\Gamma
}
{ 
    \Typecheck{\Gamma}{n}{\mu~\tau}
}
\textrm{(TcVar)}
\quad
\ruleI
{
    \TypecheckPrim{p}{\mu~\tau}
}
{
    \Typecheck{\Gamma}{p}{\mu~\tau}
}
\textrm{(TcPrim)}
\quad
\ruleI
{
    \Typecheck{\Gamma}{x}{@Pure@~\tau}
}
{
    \Typecheck{\Gamma}{x}{\mu~\tau}
}
\textrm{(TcBox)}
$$

$$
\ruleI
{
    (n~:~\tau_f)~\in~\Gamma
    \quad
    \Typecheck{\Gamma}{x\ldots}{\tau_x\ldots}
    \quad
    \WrapApp{\tau_f}{\tau_x\ldots}{\tau'}
}
{
    \Typecheck{\Gamma}{n~x\ldots}{\tau'}
}
\textrm{(TcAppVar)}
\quad
\ruleI
{
    \TypecheckPrim{p}{\tau_f}
    \quad
    \Typecheck{\Gamma}{x\ldots}{\tau_x\ldots}
    \quad
    \WrapApp{\tau_f}{\tau_x\ldots}{\tau'}
}
{
    \Typecheck{\Gamma}{p~x\ldots}{\tau'}
}
\textrm{(TcAppPrim)}
$$


$$
\ruleI
{
    \Typecheck{\Gamma}{x}{\tau}
    \quad
    \Typecheck{\Gamma,~n~:~\tau}{c}{\tau'}
}
{
    \Typecheck{\Gamma}{@let@~n~@=@~x~\flowsinto~c}{\tau'}
}
\textrm{(TcLet)}
$$

$$
\ruleI
{
    \Typecheck{\Gamma}{x}{@Pure@~\tau}
    \quad
    \Typecheck{\Gamma,~n~:~@Element@~\tau}{x'}{@Element@~\tau}
    \quad
    \Typecheck{\Gamma,~n~:~@Aggregate@~\tau}{c}{\tau'}
}
{
    \Typecheck{\Gamma}{@fold@~n~@=@~x~@then@~x'~\flowsinto~c}{\tau'}
}
\textrm{(TcFold)}
$$

$$
\ruleI
{
    \Typecheck{\Gamma}{x}{@Element@~\BB}
    \quad
    \Typecheck{\Gamma}{c}{@Aggregate@~\tau'}
}
{
    \Typecheck{\Gamma}{@filter@~x~\flowsinto~c}{@Aggregate@~\tau'}
}
\textrm{(TcFilter)}
\quad
\ruleI
{
    \Typecheck{\Gamma}{x}{@Element@~\tau}
    \quad
    \Typecheck{\Gamma}{c}{@Aggregate@~\tau'}
}
{
    \Typecheck{\Gamma}{@group@~x~\flowsinto~c}{@Aggregate@~(@Map@~\tau~\tau')}
}
\textrm{(TcGroup)}
$$



\caption{Typing contexts}
\label{fig:source:type:ctx}
\end{figure*}



%%% Types
In figure~\ref{fig:source:type:defs} we define the types.
We use $T$ for base types, of which we only have natural numbers and booleans.

Modalities are $M$, with @Aggregate@ meaning the result of some fold, @Element@ being an element of the input stream, and @Pure@ being a pure computation.
The modalities are defined so that @Aggregate@ and @Element@ computations cannot be mixed.
There is no way to turn an aggregate computation into an element computation, as aggregate computations are only available once the entire stream has been read.
Element computations must be explicitly folded over or aggregated in some way, in order to produce an aggregate computation based off it.
\TODO{Staged computation similarities}

The @Pure@ modality is actually the absence of a mode; this is simply made explicit for clarity.

Finally we define function types as $T_\to$, which can be a simple return type with a modality, or a function taking a modality argument and returning a function.
For simplicity of compilation, we have strictly disallowed higher order functions by requiring the argument to be a base type rather than a function type.

At various points in the following rules we will implicitly convert function types $T_\to$ to modal base types $M~T$; at these points function arrows are not allowed.



%%% Function application
In figure~\ref{fig:source:type:wrap} there are two judgments defined, which are both used for the type of function application.
In contrast to traditional modality type systems, we infer the boxing and unboxing of various modalities.

Informally, if a function expects an @Element@ and is passed a @Pure@ argument, we should box the argument into @Element@.
If a function expects a @Pure@ computation and is passed an @Element@, we must unbox the argument, apply the function, then rebox the result.
However, this reboxing can only occur if the result is @Pure@ or @Element@ - an @Aggregate@ result cannot be boxed into an @Element@.

The judgment $\WrapMode{\tau}{m}{\tau'}$, corresponds to reboxing the function return type $\tau$ into modality $m$, returning $\tau'$ as the boxed return type.
The rule (WrapModeArrow) simply recurses down through the function arguments, applying the judgment to the result.
The rule (WrapElement) applies when reboxing an @Element@ result as an @Element@, and similarly for (WrapAggregate).
For (WrapPure) the result type is a @Pure@, so it can be boxed to any mode.

The judgment $\WrapApp{\tau}{\phi}{\tau'}$ attempts to apply the function type $\tau$ to argument $\phi$, performing boxing and unboxing if necessary.
If both modalities and types are equal, (WrapApp) applies the function as normal.
When the function expects a @Pure@ argument and is passed an @Element@ or an @Aggregate@, we perform reboxing as in (WrapAppElement) and (WrapAppAggregate).
Finally, no matter what the function expects, if it is passed a @Pure@ it is simply boxed and applied as in (WrapAppPure), with no need to rebox the result.

\TODO{Or should we just use an explicitly box/unbox version?}



%%% Expressions
With function application already defined, typechecking expressions in figure~\ref{fig:source:type:exp} is quite simple.
The judgment $\Typecheck{\Gamma}{x}{\tau}$ means that under a context $\Gamma$, the expression $x$ has type $\tau$.
Rules (TcVar) and (TcPrim) simply look up the variable or primitive in the environment.
For (TcApp) we simply use the boxed function application judgment from figure~\ref{fig:source:type:wrap}.

\TODO{I think we need a box/cast rule for going from @Pure@ to @Element@ etc}


%%% Contexts
Figure~\ref{fig:source:type:ctx} shows the typing rules for contexts.
The judgment $\Typecheck{\Gamma}{c}{\tau}$ means that under a context $\Gamma$, the context $c$ has type $\tau$.
For simplicity, we reuse the same judgment for expressions and contexts.

The rule (TcLet) typechecks its definition and adds it to the environment when typechecking the body.

For (TcFilter), the predicate must be an @Element@, as it is executed for each element of the input.
The rest of the computation must return an @Aggregate@ over the filtered input.

For (TcFold), the form of fold is $@fold@~v~@=@~x~@then@~x'$ where $v$ is the name of the fold binding, $x$ is its initial value, and $x'$ computes the successor value, based on the previous value of $v$.
The initial expression $x$ must be @Pure@ of some type $\tau$, and the successor expression $x'$ is typechecked with the fold binding set to $@Element@~\tau$ in the environment.
The result of the successor expression must also have type $@Element@~\tau$.
This is because the current value of the fold is available to the successor expression, as well as the current elements, but the results of other folds are not available.
In the remaining context, the fold value is only available as an @Aggregate@.

Finally, (TcGroup) requires its key or group ``by'' to be an @Element@, and the value to be an @Aggregate@.
It builds up a map from key to value, repeatedly applying the aggregate as new values are found.



%!TEX root = ../Main.tex

\begin{figure*}

\newcommand \acc {\tau_\mathit{acc}}
\newcommand \xtr {\tau_\mathit{extract}}
\newcommand \lam[1] {\lambda{} #1.~}

$$
\boxed{\ToCore{x}{\acc}{\acc~\to~\acc}{\acc~\to~\xtr}}
$$


$$
\ruleI
{
}
{ 
    \ToCore{n}{n}{\lam{\_} n}{\lam{\_} n}
}
\textrm{(CoreVar)}
$$

$$
\ruleI
{
    \ToCore{p}{z_p}{k_p}{x_p}
    \quad
    \ToCore{p}{z_q}{k_q}{x_q}
}
{ 
    \ToCore{f~p~q}{(z_p, z_q)}{\lam{a} (k_p~(@fst@~a), k_q~(@snd@~a))~}{\lam{a} f (@fst@~a) (@snd@~a)}
}
\textrm{(CoreApp2)}
$$

$$
\ruleI
{
    \ToCore{b}{z}{k}{x}
    \quad
    \ToCore{c}{z'}{k'}{x'}
}
{ 
    \ToCore{@let@~n~@=@~b~\flowsinto~c}{@let@~n~@=@~z~@in@~z'}{@let@~n~@=@~k~@in@~k'}{@let@~n~@=@~x~@in@~x'}
}
\textrm{(CoreVarLet)}
$$

$$
\ruleI
{
    \ToCore{f_z}{z}{\_}{\_}
    \quad
    \ToCore{f_k}{\_}{k}{\_}
    \quad
    \ToCore{c}{z'}{k'}{x'}
}
{ 
    \ToCore{@fold@~n~@=@~f_z@;@~f_k~\flowsinto~c}{@let@~n~@=@~z~@in@~(z,~z')}{@let@~n~@=@~k~@in@~(k,~k')}{@let@~n~@=@~@fst@~k~@in@~x'~(@snd@~k)}
}
\textrm{(CoreVarFold)}
$$

$$
\ruleI
{
    \ToCore{p}{\_}{p'}{\_}
    \ToCore{c}{z}{k}{x}
}
{ 
    \ToCore{@filter@~p~\flowsinto~c}{z}{@if@~p'~@then@~k~a~@else@~a}{x}
}
\textrm{(CoreVarFilter)}
$$



\caption{Conversion to Core}
\label{fig:source:core}
\end{figure*}







\section*{Acknowledgements}
I would like to thank all of Ambiata for allowing me to work on such an interesting and useful problem, and on top of that being the most sensible group of people I have ever had the pleasure of working with.

\bibliographystyle{plain}
\bibliography{Main}

\end{document}


