\documentclass[a4paper,UKenglish,cleveref, autoref, thm-restate]{lipics-v2021}
% To anonymise:
% \documentclass[a4paper,UKenglish,cleveref, autoref, thm-restate,anonymous]{lipics-v2021}

% \pdfoutput=1 %uncomment to ensure pdflatex processing (mandatatory e.g. to submit to arXiv)
\hideLIPIcs  %uncomment to remove references to LIPIcs series (logo, DOI, ...), e.g. when preparing a pre-final version to be uploaded to arXiv or another public repository

\bibliographystyle{plainurl}

% is this package necessary?
\usepackage{mathtools}

\usepackage{style/utils}
\usepackage{style/code}
\usepackage{style/proof}
\usepackage{style/keywords}
\usepackage{style/judgements}


\title{Pipit on the Post: Proving Pre- and Post-conditions of Reactive Systems}


\author{Amos Robinson}{Australian National University, Canberra, Australia}{amos.robinson@anu.edu.au}{https://orcid.org/0009-0004-4837-4981}{}

\author{Alex Potanin}{Australian National University, Canberra, Australia}{alex.potanin@anu.edu.au}{https://orcid.org/0000-0002-4242-2725}{}

\authorrunning{A. Robinson and A. Potanin}
\Copyright{Amos Robinson and Alex Potanin}
% \ccsdesc[300]{Computer systems organization~Embedded software}
\ccsdesc[500]{Computer systems organization~Real-time languages}
\ccsdesc[500]{Theory of computation~Program verification}
% \ccsdesc[100]{Theory of computation~Modal and temporal logics}
\ccsdesc[500]{Software and its engineering~Specialized application languages}
\keywords{Lustre, streaming, reactive, verification}

% \category{}

\begin{document}

\maketitle

\begin{abstract}
  Reactive languages such as Lustre and Scade are used to implement safety-critical control systems; proving such programs correct and having the proved properties apply to the compiled code is therefore equally critical.
  We introduce Pipit, a small reactive language embedded in \fstar{}, designed for verifying control systems and executing them in real-time.
  Pipit includes a verified translation to transition systems; by reusing \fstar{}'s existing proof automation, certain safety properties can be automatically proved by k-induction on the transition system.
  Pipit can also generate imperative code in a subset of \fstar{} which is suitable for compilation and real-time execution on embedded devices.
  This translation to imperative code preserves types by construction; the proof that the imperative code preserves semantics is ongoing.
\end{abstract}


% enable @verbatim@ syntax
\makeatactive

%!TEX root = ../Main.tex
\section{Introduction}
\label{s:Introduction}

Suppose we have two input streams of numeric identifiers, and wish to perform some analysis on these identifiers. The identifiers from both streams arrive sorted, but may include duplicates. We wish to produce an output stream of unique identifiers from the first input stream, as well as produce the unique union of identifiers from both streams. Can we perform both of these tasks at once, without needing to read through the stream data multiple times, and without needing unbounded buffering? Here is how we might write the source code, where @S@ is for @S@-tream.
\begin{code}
  uniquesUnion : S Nat -> S Nat -> (S Nat, S Nat)
  uniquesUnion sIn1 sIn2
   = let  sUnique = group sIn1
          sMerged = merge sIn1 sIn2
          sUnion  = group sMerged
     in   (sUnique, sUnion)
\end{code}

In this implementation the @group@ operator filters out consecutive duplicates, while @merge@ combines two sorted streams so that the output remains sorted. This example has a few interesting properties. Firstly, the data-access pattern of @merge@ is \emph{value-dependent}, meaning that the order in which this operator pulls values from @sIn1@ and @sIn2@ depends on the values themselves. If all the values from @sIn1@ are smaller than the values in @sIn2@, then @merge@ will pull all values from @sIn1@ before pulling the rest from @sIn2@, and vice versa. Secondly, although @sIn1@ occurs twice in the program, at runtime we only want to handle the elements of each stream once. To achieve this, the compiled program must coordinate between the two uses of @sIn1@, so that values are only read when both the @group@ and @merge@ operators are ready to receive a new value. Finally, as the stream length is assumed to be unbounded, we cannot buffer an arbitrary number of elements read from either stream, or risk running out of local storage space.

For an implementation which does \emph{not} use stream fusion, we might implement each of the operators as a separate concurrent process, and send each identifier value using an intra-process communication mechanism. Developing such an implementation could be easy or hard, depending on what language features are available for concurrency. However, worrying about the \emph{performance tuning} of such a system, such as whether we need back-pressure, or how to chunk the stream data to reduce the amount of communication overhead, is invariably a headache. 

We might instead define some sort of uniform interface for data sources, with a single `pull' function that provides the next value in each stream. Each operator could be given this interface, so that the next value from each result stream is computed on demand. This is approach is commonly taken with implementations of physical operators in data base systems. However, this `pull only' model does not support operators with multiple outputs, such as our derived @uniquesUnion@ operator, at least not without unbounded buffering. Suppose a consumer pulls many elements from the result @sUnique@ stream. The implementation needs to pull the corresponding source elements from @sIn1@ \emph{as well} as buffering an arbitrary number of matching elements from @sIn2@. It needs to buffer an aribrary number of elements from @sIn2@ because there is no guarantee of when a consumer will also pull from the @sUnion@ result stream. Once that happens the elements from @sIn2@ no longer need to be retained, but not before.

Instead, for a single threaded program, we want to perform \emph{stream fusion}, which takes the dataflow network and produces a simple sequential loop that gets the job done without requiring extra process-control abstractions and without requiring unbounded buffering. Sadly, existing stream fusion transformations cannot handle our example. As observed by \citet{kay2009you}, both pull-based and push-based fusion have fundamental limitations. Pull-based systems such as short cut stream fusion~\cite{coutts2007stream} cannot handle cases where a particular stream or intermediate result is used by multiple consumers. We refer to this situation as a \mbox{\emph{split} --- in the} dataflow network the flow from input stream @sIn1@ is split into both the @group@ and @merge@ consumers. 

% Leave this to related work. We've already mentioned a canonical pull-based system.
% Recent work on stream fusion by \citet{kiselyov2016stream} uses staged computation to ensure all combinators are inlined, but for splits this causes excessive inlining which duplicates work, due to values of the source arrays being read multiple times.

Push-based systems such as foldr/build fusion~\cite{gill1993short} also cannot fuse our example because they do not support operators with multiple inputs. We refer to such a situation as a \emph{join} --- in our example the @merge@ operator expresses a join in the data-flow graph. Some systems support both pull and push: data flow inspired array fusion~\cite{lippmeier2013data} allows both splits and joins but only for a limited, predefined set of operators. More recent work on polarized data flow fusion~\cite{lippmeier2016polarized} \emph{is} able to fuse our example, but requires the program to be rewritten to use explicitly polarized stream types. 

% The mechanism that combines the implementations of both operators, to yield efficient imperative code also depends on the general purpose compiler optimisations implemented by GHC, and it can be difficult to tell if these have ``worked'' without inspecting the intermediate representations of the compiler.

Synchronous dataflow languages such as Lucy-n~\cite{mandel2010lucy} reject value-dependent operators such as @merge@, while general dataflow languages fall back on less performant dynamic scheduling for these cases \cite{bouakaz2013real}. The polyhedral array fusion model~\cite{feautrier2011polyhedron} is used for loop transformations in imperative programs, but operates at a much lower level. The polyhedral model is based around affine loops, which makes it difficult to support filter-like operators such as @group@ and @merge@.

In our new system we still view the program as a concurrent process network. Each operator is a separate process, and the stream data flows through communication channels between the processes. Each operator is expressed as a restricted, sequential imperative program with commands that include both @pull@ for reading from an input stream and @push@ for writing to an output stream. The fusion transform takes the concurrent process network and \emph{sequentializes} it into a single process by choosing a particular evaluation order that requires no unbounded intermediate buffers. When the fusion transformation succeeds we know it has worked. There is no need to inspect intermediate representations of the compiler to debug poor performance, which is a common problem in systems based on general purpose program transformations \cite{lippmeier2012:guiding}.

In summary, we make the following contributions:
\begin{itemize}
\item a process calculus for encoding infinite streaming programs (\S\ref{s:Processes});
\item an algorithm for fusing these processes, the first to support arbitrary splits and joins (\S\ref{s:Fusion});
\item numerical results that demonstrate that the algorithm is well behaved when the number of fused processes is large. The size of the fused result program is not excessive. \TODO{Ref}
\item a formalization and proof of soundness for the core fusion algorithm in Coq (\S\ref{s:Proofs});
\end{itemize}

Our fusion transformation for infinite stream programs could also serve as the basis for an \emph{array} fusion system, using a natural extension to finite streams. We discuss this extension in \S\ref{s:Finite}.

% TODO: We can't make the appendix a contribution because the reviewers are not required to read appendices.
% \item and show our processes are general enough for many combinators, including segmented operations (\S\ref{s:Combinators}).

% \ben{Add a few more sentences on related work. Explain how this work extends the old flow fusion paper. It is not short-cut fusion like Oleg's recent work. We are not in the same space as Fortran style array fusion transformations like polyhedral}

% BL: describe this later.
% Furthermore, the data-flow fusion system of~\cite{lippmeier2013data} only deals with a fixed set of baked-in combinators. 

% BL: Shift the detailed description into a later section.
% The example above has three combinators, so the process network has three processes.
% The two @writeFile@s outputs are treated as sinks that values can be pushed to at any time, and are not converted to processes.
% During code generation, any output values from the @uniques@ and @union@ streams are sent to the corresponding @writeFile@ sink, but we do not address code generation in this paper.

% The process for @uniques@ is defined by the @group@ combinator, and can be thought of as an imperative loop: first it reads from its input stream @file1@ and stores that in a local variable.
% It also keeps track of the last pulled value, and compares that against the newly read value.
% If they are different, it pushes the new value to its output stream @uniques@.
% In either case, it updates the last pulled value and loops back to the start to pull from @file1@ again.

% The process for @merged@ is defined by the @merge@ combinator, which starts by reading from both @file1@ and @file2@ and storing these in local variables.
% It then compares its two values to see which is the smaller.
% If the value from @file1@ is smaller, it pushes that value and pulls a new value from @file1@, otherwise it pushes the value from @file2@ and pulls from @file2@.
% This is performed in a loop.

% We fuse these two processes by interleaving the two such that the shared input @file1@ is only pulled from when both processes agree.
% The new process pulls from @file1@, which is copied to the variables for both processes.
% The @uniques@ process now has all it needs to execute, so it checks the value against the last pulled value, pushes if necessary, and goes back to try to pull from @file1@ again.
% At this stage the @merged@ process still has a value from @file1@ that it has pulled but not used, so @uniques@ cannot pull from @file1@ again.
% We now let @merged@ run, pulling from @file2@ and checking which is smaller.
% If the value from @file1@ is smaller, the value is emitted and @merged@ wishes to pull a new value from @file1@.
% Both processes now agree on pulling from @file1@ again, so the new value is pulled and @uniques@ can run again.
% Otherwise if the value from @file1@ is not smaller, the value from @file2@ is emitted and @merged@ pulls from @file2@ with no coordination required.

% If we wish to ensure that each value is only read from the file once, we must coordinate between the two use sites: when @uniques@ requires a new value it must ensure that @union@ is ready to receive a new value, and vice versa. Note that we cannot just execute @uniques@ while storing the read values in a buffer, as this may require more memory than is available.
% In order to fuse this example, we require both pull \emph{and} push streams.
% The input streams must be pull streams since the order values are required is determined by the @merge@ combinator.
% For the same reason, the outputs sent to each @writeFile@ must be push streams.

% Fusion for array programs is important for removing intermediate arrays, reducing memory traffic and reducing allocations.
% However, when dealing with data too large to fit in memory such as tables on disk, removing intermediate arrays becomes essential rather than just desirable.
% Attempting to create an intermediate array of such amounts of data would lead to thrashing and swapping to disk, or perhaps even running out of swap.
% For these situations, some sort of assurance of total fusion is required: either the program can be fused with no intermediate arrays or unbounded buffers, or it will not compile at all.


% Fusion eliminates intermediate array buffers and converts pipelines of array combinators into low-level iteration based loop code. Different fusion systems can handle 

% When comparing fusion systems, three important criteria to consider are: whether the system supports splits, where a stream is used multiple times; whether it supports joins, where a combinator has multiple inputs; and whether arbitrary combinators such as @merge@ and segmented appends can be encoded. Existing fusion systems support one or two of these, but not three. We present a fusion system based on process calculus that supports all three: splits, joins and arbitrary combinators.

% Our system has been formalised in Coq where we have proved soundness of the fusion algorithm. It is expressive enough to encode a wide range of combinators including operations on segmented arrays.

% \amos{``Arbitrary combinators'' is not quite true. How can we distinguish combinators we support from 2013 data flow fusion paper? Perhaps by mentioning value-dependent input / access patterns.}

% Leave this to the description of the algorithm, not the abstract.
% We encode each combinator as a separate process with any number of input and output channels. Each process is sequential but multiple processes can be executed concurrently. We give a concurrent execution semantics for multiple processes, but these are used only as a specification for how the fused program must behave. The fused program itself is sequential and can easily be converted to simple imperative code.

% Our fusion algorithm takes two concurrently executable processes and creates a sequential interleaving of the two such that they execute with no unbounded buffers.
% If fusion would require unbounded buffers (or the fusion algorithm wrongly infers that it would) then fusion fails.
% If fusion fails, the user can be presented with an error message telling them which combinators could not be fused.
% For scenarios where fusion is required, this is a great advantage over fragile shortcut fusion systems.

% BL: leave the apologies to the conclusion.
% The version presented here deals with infinite streams, and we informally describe the extensions required to support finite streams.

% BL: leave this to the main intro.
% optimising high-level array and streaming computations, as it reduces memory traffic and intermediate arrays. The benefits of removing intermediate arrays are even more important as data sizes approach the size of memory.



% \section{Pipit in flight}
% \pagebreak
\section{Programming and verifying in Pipit}
\label{s:tut}

A common requirement in controllers is to filter an input signal, perhaps using a \emph{finite impulse response} (FIR) filter, which is equivalent to a weighted moving average.
An FIR filter takes a vector of coefficients and an input signal; at every point in time, it computes the dot product of the coefficients and the most recent values of the signal.
We can implement an FIR filter in Pipit as follows:

\newcommand\signal{\textit{signal}}
\begin{tabbing}
  @MM@\= @MMMMMM@ \= \kill
  @let@ fir ($\textit{coefficients}$: list $\RR$) ($\signal$: stream $\RR$): stream $\RR$ = \\
    \> @match@ $\textit{coefficients}$ @with@ \\
    \> @|@ $[]$ \> $\to 0$ \\
    \> @|@ $c :: \cs $ \> $\to (\signal \cdot c) + (\xfby{0}{(\mbox{fir~} \cs~\signal)})$
\end{tabbing}

% Our FIR filter takes two arguments, one describing the coefficients and one describing the signal to filter.
The coefficient vector is represented by a list of reals, while the signal is a \emph{stream} of reals, and the result is the filtered stream.
% \footnote{The actual implementation uses integers with fixed-point arithmetic, but we use reals here to reduce clutter.}
The implementation starts by looking at the list of coefficients and returns zero if the list is empty.
If the list is not empty, then we multiply the most recent value of the stream by the coefficient ($\signal \cdot c$); we also take the result of applying the remaining coefficients to the signal stream ($\mbox{fir~} \cs ~ \signal$) and delay it ($\xfby{0}{\ldots}$) before summing the two parts.
At the start of execution the delay is initially zero.

Pipit is an \emph{embedded} language, like Bedrock~\cite{chlipala2013bedrock}: in this example, the stream type denotes an actual Pipit expression, while the list type and its associated pattern match are part of the \fstar{} meta-language.
To get the real Pipit program, we need to apply the filter to a concrete list of coefficients:

\newcommand\bibo{\mbox{bibo}}
\newcommand\fir{\mbox{fir}}
\newcommand\ii{\textit{input}}
\newcommand\oo{\textit{output}}
\newcommand\ok{\textit{ok}}
\begin{tabbing}
  @MM@\= @MMMMMM@ \= \kill
  @let@ $\mbox{fir}_2$ ($\ii$: stream $\RR$): stream $\RR$ = \\
    \> $\mbox{fir}~[0.7; 0.3] \ii$
\end{tabbing}

Normalising this definition evaluates away all of the lists.
The result fits in the core language defined in \autoref{s:core}, for which we can generate real-time imperative code:

\begin{tabbing}
    @MM@\= @MMMMMM@ \= \kill
      @let@ $\mbox{fir}_2$ ($\ii$: stream $\RR$): stream $\RR$ = \\
    \> $(0.7 \cdot \ii) + (\xfby{0}{((0.3 \cdot \ii) + (\xfby{0}{0}))})$
\end{tabbing}

The properties that we want to state about reactive programs usually involve some temporal aspect.
Rather than defining a separate specification language, we implement computable variants of temporal operators from \emph{past-time} linear temporal logic \cite{halbwachs1993executable,lichtenstein1985glory}.
We name the past-globally operator \emph{sofar}, as in \emph{the predicate has been true so far}:

\begin{tabbing}
  @MM@\= @MMMMMM@ \= \kill
  @let@ sofar ($p$: stream $\BB$): stream $\BB$ = \\
    \> $\xrec{p'}{p \wedge (\xfby{@true@}{p'})}$
\end{tabbing}

This definition takes a stream of predicates $p$ and introduces a recursive stream $p'$.
At each step, the recursive stream $p'$ checks that the current predicate is true ($p$), and also checks that \emph{sofar} was previously true ($\xfby{@true@}{p'}$).
If there is no previous value, it defaults to true.

\subsection{Bounded input, bounded output}
\label{ss:tut:bibo}
We can now state a \emph{bounded-input-bounded-output} (BIBO) property, which says that if the inputs have always been within some particular range, then the outputs are also within the range:
% Properties like this can occur as an internal invariant in a larger system:

\begin{tabbing}
  @MM@\= @LET OUTPUT@\= @EXAMPLE1 INPUT@ \= \kill
  @let@ $\mbox{bibo}_2$ (n: $\RRgez$) ($\ii$: stream $\RR$): stream $\BB$ = \\
  \> @check@ $(\mbox{sofar}(|\ii| \le n) \implies |\mbox{fir}_2~\ii| \le n)$
\end{tabbing}

This property states that if the input has always been in the range $[-n, n]$, then the output is also within the range $[-n, n]$.
Note that the upper bound $n$ is a nonnegative real rather than a stream of reals, which means that $n$ stays constant across the whole stream.
To prove that this property holds, we translate to a transition system and show that the stream is always true.
In this case, induction over the transition relation is sufficient to prove the property.
There is some boilerplate required to perform the induction, but both base and step cases are automatically proved by \fstar{}:

\begin{tabbing}
  @MM@\= @MMMMMMMMMMMMMMMMMMMMMMM@ \= \kill
  @let@ $\mbox{proof}_2$ (n: $\RRgez$): Lemma (induct ($\bibo_2$ n)) = \\
  \> @assert@ (base\_case ($\bibo_2 ~n$)) \> @by@ (pipit\_simplify ()); \\
  \> @assert@ (step\_case ($\bibo_2 ~n$)) \> @by@ (pipit\_simplify ())
\end{tabbing}

This definition uses \fstar{}'s \emph{lemma} syntax to state that the BIBO property holds inductively for any $n$.
The two assertions prove the inductive cases separately, using our \emph{simplify} tactic to ensure that the translation to transition system is normalised away, and any translation artefacts are removed.

If we wish to prove a similar BIBO property for a filter with more coefficients, standard induction over the transition system is not sufficient: the relationship between the stacked delays in the filter and sofar is not clear from a single step of the transition system.
One simple automated way to strengthen invariants is via \emph{k-induction}~\cite{hagen2008scaling}, which adds more context by assuming that the property holds for $k$ previous steps of the transition relation.
We can define analogous functions $\fir_3$ and $\bibo_3$ which operate on the coefficients $[0.7; 0.2; 0.1]$, and use k-induction for $k = 2$ as follows:

\begin{tabbing}
  @MM@\= @MMMMMMMMMMMMMMMMMMMMMMMMMMM@ \= \kill
  @let@ $\mbox{proof}_3$ (n: $\RRgez$): Lemma (induct\_k 2 ($\bibo_3$ n)) = \\
  \> @assert@ (base\_case\_k 2 ($\bibo_3 ~n$)) \> @by@ (pipit\_simplify ()); \\
  \> @assert@ (step\_case\_k 2 ($\bibo_3 ~n$)) \> @by@ (pipit\_simplify ())
\end{tabbing}

Although the properties here boil down to simple properties about linear arithmetic, we believe that this example demonstrates a promising way to use \fstar{}'s existing proof automation to verify reactive systems.

% \pagebreak
\section{Core language}
\label{s:core}

%!TEX root = ../Main.tex

\begin{figure}
  \[
  \begin{array}{lrlr}
    e, e' & := & v ~|~ x ~|~ p(\ov{e}) & \mbox{(values, variables and operations)} \\
          & | & \xfby{v}{e} ~|~ \xrec{x}{e[x]} & \mbox{(followed-by (delay) and recursive streams)} \\ % & | & \xthen{e}{e'} \\
          & | & \xlet{x}{e}{e'[x]} & \mbox{(let-expressions)}\\
          & | & \xcheckP{\PStatus}{e_{\text{prop}}} & \mbox{(checked properties)} \\
          & | & \xcontractP{\PStatus}{\erely}{\eimpl}{x\!.~\eguar} & \mbox{(rely-guarantee contracts)} \\
          % & | & \xcheck{e} \\
    \\
    v & := & n \in \mathbb{N} ~|~ b \in \mathbb{B} ~|~ r \in \mathbb{R} ~|~ \hdots  & \mbox{(values)} \\
    p & := & (+) ~|~ (-) ~|~ (\times) ~|~ @if-then-else@ ~|~ \hdots & \mbox{(primitives)} \\
    \\
    \PStatus & := & \PSValid ~|~ \PSUnknown & \mbox{(property statuses: valid or unknown)}\\
    \\
    \sigma & := & \sgl{\ov{x \mapsto v}} & \mbox{(heaps)} \\
    \Sigma & := & \sigma ~|~ \Sigma;\sigma & \mbox{(streaming history environments)} \\
    %   \\
    \tau, \tau' & := & \mathbb{N} ~|~ \mathbb{B} ~|~ \tau \times \tau ~|~ \hdots & \mbox{(value types)} \\
    \Gamma & := & \cdot ~|~ x : \tau, \Gamma & \mbox{(type environments)}  \\
    \end{array}
  \]
  \caption{Pipit core language grammar, which contains expressions $e$, values $v$, primitive operations $p$, and property statuses $\PStatus$.}
  \label{f:core-grammar}
\end{figure}
\begin{figure}
  \[
    \boxed{\typing{\Gamma}{e}{\tau}}
    \quad
    \boxed{\mtyping{}{v}{\tau}}
  \]

  \[
    \ruleIN{
      \mtyping{}{v}{\tau}
    }{
      \typing{\Gamma}{v}{\tau}
    }{TValue}
    \quad
    \ruleAx{\typing{\Gamma}{x}{\Gamma(v)}}{TVar}
  \]

  \[
    \ruleIN{
      \typing{\Gamma}{e}{\tau \to \tau'}
      \qquad
      \typing{\Gamma}{e'}{\tau}
    }{\typing{\Gamma}{e~e'}{\tau'}}{TApp}
  \]

  \[
    \ruleIN{
      \typing{\Gamma}{v}{\tau}
      \qquad
      \typing{\Gamma}{e'}{\tau}
    }{
      \typing{\Gamma}{\xfby{v}{e'}}{\tau}
    }{TFby}
  \]

  \[
    \ruleIN{
      \typing{\Gamma}{e}{\tau}
      \qquad
      \typing{\Gamma}{e'}{\tau}
    }{\typing{\Gamma}{\xthen{e}{e'}}{\tau}}{TThen}
  \]

  \[
    \ruleIN{
      \typing{x : \tau, \Gamma}{e}{\tau}
    }{
      \typing{\Gamma}{\xrec{x}{e[x]}}{\tau}
    }{TRec}
  \]

  \[
    \ruleIN{
      \typing{\Gamma}{e}{\tau}
      \qquad
      \typing{x : \tau, \Gamma}{e'}{\tau'}
    }{
      \typing{\Gamma}{\xlet{x}{e}{e'[x]}}{\tau'}
    }{TLet}
  \]

  \[
    \ruleIN{
      \typing{\Gamma}{e}{\mathbb{B}}
    }{
      \typing{\Gamma}{\xcheck{e}}{\tt{unit}}
    }{TCheck}
  \]

  \caption{Typing rules for Pipit defined in terms of two judgment forms: $\typing{\Gamma}{e}{\tau}$ denotes that expression $e$ describes a \emph{stream} of values of type $\tau$; and $\mtyping{}{v}{\tau}$, which denotes that closed meta-value $v$ has type $\tau$ and is not defined here.}\label{f:core-typing}
\end{figure}
%!TEX root = ../Main.tex

\begin{figure}[t]
  \begin{mathpar}
    \boxed{\bigstep{\Sigma}{e}{v}}
  \end{mathpar}
  \begin{mathpar}
    \ruleAx{\bigstep{\Sigma; \sigma}{x}{\sigma(x)}}{Var}
    \quad
    \ruleAx{\bigstep{\Sigma}{v}{v}}{Value}

    \ruleIN{
      \bigstep{\Sigma}{e'[x := e]}{v}
    }{
      \bigstep{\Sigma}{\xlet{x}{e}{e'[x]}}{v}
    }{Let}

    \ruleIN{
      \bigstep{\Sigma}{e_1}{v_1} \quad \hdots \quad
      \bigstep{\Sigma}{e_n}{v_n}
    }{\bigstep{\Sigma}{p(\ov{e})}{\text{prim-sem}(p, \ov{v})}}{Prim}


    % \[
  %   \ruleIN{\bigstep{\Sigma}{e}{v}}{\bigstep{\Sigma; \sigma}{\xpre{e}}{v}}{Pre}
  % \]

  \ruleAx{\bigstep{\sigma}{\xfby{v}{e'}}{v}}{$\mbox{Fby}_1$}
    \quad
    \ruleIN{\text{length}(\Sigma) > 0 \and \bigstep{\Sigma}{e'}{v'}}{\bigstep{\Sigma; \sigma}{\xfby{v}{e'}}{v'}}{$\mbox{Fby}_S$}

    % \ruleIN{\bigstep{\sigma}{e}{v}}{\bigstep{\sigma}{\xthen{e}{e'}}{v}}{$\xthenarrow_1$}
    % \quad
    % \ruleIN{\bigstep{\Sigma}{e'}{v'}}{\bigstep{\Sigma; \sigma}{\xthen{e}{e'}}{v'}}{$\xthenarrow_S$}

    \ruleIN{
      \bigstep{\Sigma}{e[x := \xrec{x}{e}]}{v}
    }{
      \bigstep{\Sigma}{\xrec{x}{e[x]}}{v}
    }{Rec}

    \ruleIN{
      % \bigstep{\Sigma}{e}{\top}
    }{
      \bigstep{\Sigma}{\xcheckP{\PStatus}{e}}{()}
    }{Check}

    \ruleIN{
      \bigstep{\Sigma}{\ebody}{v}
    }{
      \bigstep{\Sigma}{\xcontractP{\PStatus}{\erely}{\ebody}{\rawbind{x}{\eguar[x]}}}{v}
    }{Contract}
  \end{mathpar}

  \begin{mathpar}
    \boxed{\bigsteps{\Sigma}{e}{V}}

    \boxed{\bigstepalways{\Sigma}{e}}
  \end{mathpar}
  \begin{mathpar}
    \ruleAx{\bigsteps{\cdot}{e}{\cdot}}{$\mbox{Steps}_0$}
    % \ruleIN{\bigstep{\sigma}{e}{v}}{\bigsteps{\sigma}{e}{v}}{$\mbox{Steps}_1$}

    \ruleIN{
      \bigstep{\Sigma}{e}{V}
      \and
      \bigstep{\Sigma; \sigma}{e}{v}
    }{\bigstep{\Sigma; \sigma}{e}{V; v}}{$\mbox{Steps}_S$}

    \ruleIN{\bigsteps{\Sigma}{e}{\true; \hdots}}{\bigstepalways{\Sigma}{e}}{Always}
  \end{mathpar}


  \caption{Dynamic semantics for Pipit; the judgment form $\bigstep{\Sigma}{e}{v}$ denotes that evaluating expression $e$ under streaming history $\Sigma$ results in value $v$.}\label{f:core-bigstep}
\end{figure}


The grammar of Pipit is defined in \autoref{f:core-grammar}.
The expression form $e$ includes standard syntax for values ($v$), variables ($x$) and applications ($e~e'$); however, it does not include any form for defining functions except reusing closed functions from the \fstar{} meta-language $(\mlamX{})$.
Most of the expression forms were introduced informally in \autoref{s:tut} and correspond to the clock-free primitives in Lustre \cite{caspi1995functional}.
% The expression syntax for delayed streams ($\xfby{v}{e}$) denotes the previous value of the stream $e$, with an initial value of $v$ when there is no previous value.
Streams can also be composed together using the \emph{then} notation ($\xthen{e}{e'}$) which denotes that the value of stream $e$ is used for the first step, followed by the values from stream $e'$ for subsequent steps.

Recursive streams, which can refer to previous values of the stream itself, are defined using the fixpoint operator ($\xrec{x}{e[x]}$); the syntax $e[x]$ means that the variable $x$ can occur in $e$.
% To ensure that streams are productive,
As in Lustre, recursive streams can only refer to their previous values and must be \emph{guarded} by a delay: the stream $(\xrec{x}{\xfby{0}{(x + 1)}})$ is well-defined, but stream $(\xrec{x}{x + 1})$ is invalid and has no computational interpretation.
This form of recursion differs slightly from standard Lustre, which uses a set of mutually-recursive bindings.
We use this form to define a substitution-based operational semantics that is syntax-directed, as opposed to the mutually-recursive form in \cite{caspi1995functional} which is not syntax-directed.
Our semantics has a simpler proof of determinism; we believe it has simplified other necessary proofs too and will perform further evaluation.
Although we cannot express mutually-recursive bindings in the core syntax here, we can express them as a notation on the surface syntax at the expense of potentially duplicating expressions.


\section{Extraction}
\label{s:extraction}

Pipit can generate executable code which is suitable for real-time execution on embedded devices.
The code extraction is implemented in a currently-unverified transform that takes a deeply embedded representation of a Pipit expression and generates a shallow imperative representation of the program.
\fstar{} can generate C code from a subset of the language called \lowstar{}~\cite{protzenko2017verified}; the result of our translation to imperative code fits in this subset.
During code extraction, we use \fstar{}'s tactic support~\cite{martinez2019meta} to fully normalise the translation to imperative code, conceptually similar to staged compilation.

% For the simple examples in \autoref{ss:tut:bibo}, we probably could have proved the properties directly on the generated code.
% In the future, however, we intend to support the abstraction of streams through \emph{contracts}, which require a relational semantics rather than a functional one.

% noextract
% let expr = Sugar.run2 Pump.controller'

% noextract
% let system: Pipit.Exec.Exp.xexec expr =
%   XX.exec_of_exp expr

% [@@(Tac.postprocess_with XL.tac_extract)]
% let reset = XL.mk_reset system

% [@@(Tac.postprocess_with XL.tac_extract)]
% let step (inp: input) = XL.mk_step system (inp.estop, (inp.level_low, ()))
% This allows us to generate C code which executes in bounded space on embedded devices.

\section{Evaluation}
\label{s:coffee-machine}

To demonstrate the feasibility of Pipit, we have implemented and verified a simple controller.
This system controls a water-flow solenoid to fill the reservoir of a coffee machine and includes multiple safeguards to reduce the risk of flooding.
The controller has two boolean inputs: the \emph{stop} switch and the \emph{low level} indicator; it returns a boolean indicating whether to engage the solenoid.
The stop switch indicates whether the reservoir's lid is open or closed; the system should never operate while the lid is open as water could spill out.
The controller should not allow water to flow for more than a minute as this may indicate a leak; if so, the controller enters a terminal error state.
Finally, to avoid switching the solenoid too often, the controller waits for ten seconds of low water level before trying to engage:

\newcommand\estop{\textit{stop}}
\newcommand\low{\textit{low}}
\newcommand\soltry{\textit{try}}
\newcommand\error{\textit{error}}
\newcommand\solen{\textit{engage}}
\begin{tabbing}
  @MM@\= @let@ engage \= @MMMM@ \= \kill
  @let@ $\mbox{reservoir}$ (\estop{} \low{}: stream $\BB$): stream $\BB$ = \\
  \> @let@ \soltry{} \> = true\_for \textit{TEN\_SECONDS} (not \estop{} $\wedge$ \low{}) @in@ \\
  \> @let@ \error{} \> = any (true\_for \textit{ONE\_MINUTE} \soltry{}) @in@ \\
  \> @let@ \solen{} \> = \soltry{} $\wedge$ not \error{} @in@ \\
  \> @check@ (\solen{} $\implies$ not \estop{}); \\
  \> @check@ (\solen{} $\implies$ \low{}); \\
  \> \solen{}
\end{tabbing}
\pagebreak

Predicate \emph{true\_for t} is true if a signal has been true for time $t$; \emph{any} is true if a signal has ever been true.
% The \emph{true\_for t} is time-bounded past-globally and \emph{any} is past-finally.
As with the previous examples, the two properties can be automatically verified.
% by \fstar{}'s proof automation.
Pipit generates real-time C code for this example\footnote{For a video of the controller in action, see \url{https://youtu.be/6IybbQFPOl8}}.

\section{Related work}
\label{s:related-work}
% TODO arrays

Although the FIR filter from \autoref{s:tut} is quite simple, verifying it \emph{and} executing it with existing Lustre tools is nontrivial.
Lustre itself does not support lists, as dynamically-allocated data structures are not well-suited to real-time execution.
To write this filter in Lustre we would either need to unroll the lists ourselves or reformulate the program to use arrays.
However, Vélus does not support arrays~\cite{bourke2017formally}; Kind2 uses a custom syntax for arrays with no compiler support~\cite{champion2016kind2}; and the Lustre V6 compiler does support arrays~\cite{jahier2016lustre}, but its model-checker Lesar cannot reason about integers or reals~\cite{raymond2008synchronous}.

In terms of model-checking reactive systems,
% earlier model-checkers used explicit state-space exploration~\cite{raymond2008synchronous};
recent work uses SMT solvers to check inductive proofs~\cite{hagen2008scaling,champion2016kind2} or to check refinement types~\cite{chen2022synchronous}.
These model-checkers have definite advantages over the general-purpose-prover approach offered here: they can often generate concrete counterexamples and implement counterexample-based invariant-generation techniques such as ICE~\cite{garg2014ice} and PDR~\cite{bradley2011sat,een2011efficient}.
However, these model-checkers do not provide much assurance that the semantics they use for proofs matches the compiled code.
We believe that once Pipit's imperative code generation is verified, Pipit will have a stronger assurance case.

The embedded language Copilot generates real-time C code for runtime monitoring and supports model-checking properties \cite{laurent2015assuring}, but suffers from the same semantic gap.

Early work embedding a denotational semantics of Lucid Synchrone in an interactive theorem prover focussed on the semantics itself, rather than proving programs~\cite{boulme2001clocked}.
There is ongoing work to construct a denotational semantics of Vélus for program verification~\cite{bourke2022towards}.
We believe that the hybrid SMT approach of \fstar{} will allow for a better mixture of automated proofs with manual proofs;
however, the trusted computing base of Pipit is much larger than Vélus, as we depend on all of \fstar{}, \lowstar{}'s C code extraction, the SMT solver, as well as our currently-unverified imperative code generator.

% Refinement types have also been used to verify reactive systems~\cite{chen2022synchronous}, but this work does not address the issue of correct compilation.

% cite also:
% hardware language in Agda \cite{harrison2021mechanized}
% Refinement types for Zélus \cite{chen2022synchronous}
% Invariant generation: ICE \cite{garg2014ice}; template-based \cite{kahsai2011instantiation}; PDR explanation \cite{een2011efficient}; IC3 first introduced \cite{bradley2011sat}

% Other compilers for Lustre-style languages, such as Vélus \cite{bourke2017formally} and Heptagon \cite{gerard2012modular} ensure that the generated code can be executed in bounded memory and that each step takes a bounded time.
% With a carefully chosen set of primitives, the core streaming operations shown here would ensure bounded memory and time.
% For practicality, however, we currently allow embedding arbitrary total functions from the \fstar{} meta-language, which are not necessarily bounded in time.
% (Is this the same as Lucid Synchrone and Zélus? \CITE)

\section{Conclusion}

Our preliminary results show that \fstar{}'s proof automation and code extraction are suitable for verifying reactive systems and executing them in real-time; these results still require further work.
Next, we intend to verify the imperative code generation.
To verify large programs, we also need some way to separately prove smaller pieces which can then be composed together, such as contracts~\cite{champion2016kind2}.
Finally, we need to evaluate Pipit on larger control systems before extending the language to support more features, such as Lustre's clocks for describing partially-defined streams~\cite{caspi1995functional}.

% TODO
% Future work: verify imperative code generation; verify CSE; case studies: antilock braking; clocks \cite{caspi1995functional}; contracts \cite{champion2016kind2}.

\bibliography{Main}


% \appendix

\end{document}
\endinput
