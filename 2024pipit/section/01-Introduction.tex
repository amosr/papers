%!TEX root = ../Main.tex
\section{Introduction}
Safety-critical control systems, such as the anti-lock braking systems that are present in most cars today, need to be correct and execute in real-time.
One approach, favoured by parts of the aerospace industry, is to implement the controllers in a high-level language such as Lustre~\cite{caspi1995functional} or Scade~\cite{colaco2017scade}, and verify that the implementations satisfy the high-level specification using a model-checker, such as Kind2~\cite{champion2016kind2}.
These model-checkers can prove many interesting safety properties automatically, but do not provide many options for manual proofs when the automated proof techniques fail.
Additionally, the semantics used by the model-checker may not match the semantics of the compiled code, in which case properties proved do not necessarily hold on the real system.
This mismatch may occur even when the compiler has been verified to be correct, as in the case of Vélus~\cite{bourke2017formally}.
For example, in Vélus, integer division rounds towards zero, matching the semantics of C; however, integer division in Kind2 rounds to negative infinity, matching SMT-lib~\cite{BarFT2016SMTLIB,kind2023intdiv}.

To be confident that our proofs hold on the real system, we need a single semantics that is shared between the compiler and the model-checker or prover.
In this paper we introduce Pipit\footnote{Implementation available at \GITHUBURL}, an embedded domain-specific language for implementing and verifying controllers in \fstar{}.
Pipit aims to provide a high-level language based on Lustre, while reusing \fstar{}'s proof automation and manual proofs for verifying controllers~\cite{martinez2019meta}, and using \lowstar{}'s C-code generation for real-time execution~\cite{protzenko2017verified}.
% By designing Pipit as an embedded language, we can reuse many of the \fstar{} meta-language's features while retaining a small core language.
To verify programs, Pipit translates its expression language to a transition system for k-inductive proofs, which is verified to be an abstraction of the original semantics.
To execute programs, Pipit can generate executable code, which is total and semantics-preserving.



In this paper, we make the following contributions:

\begin{itemize}
  \item we motivate the need to combine manual and automated proofs of reactive systems with a strong specification language (\autoref{s:motivation});
  \item we introduce Pipit, a minimal reactive language that supports rely-guarantee contracts and properties; crucially, proof obligations are annotated with a status --- \emph{valid} or \emph{deferred} --- allowing proofs to be delayed until more is known of the program context (\autoref{s:core});
  \item we describe a \emph{checked semantics} for Pipit, which is parameterised by the property status; after checking deferred properties, programs can be \emph{blessed}, and their properties lifted to valid status (\autoref{s:core:checked});
  \item we describe an encoding of transition systems that can express under-specified rely-guarantee contracts as functions rather than relations; composing functions results in simpler transition systems (\autoref{s:transition});
  \item we identify the invariants and lemmas required to prove that the abstract transition system is an abstraction of the original semantics (\autoref{s:core:causality}, \autoref{s:transition:proof});
  \item similarly, we offer a mechanised proof that the executable transition system preserves the original semantics (\autoref{s:extraction});
  \item finally, we evaluate Pipit by implementing the high-level logic of a time-triggered Controller Area Network (CAN) bus driver and verifying an abstract model of a key component (\autoref{s:evaluation}).
\end{itemize}
