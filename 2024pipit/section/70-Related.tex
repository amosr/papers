%!TEX root = ../Main.tex

\section{Related work}
\label{s:related-work}

% SMT-based model-checkers~\cite{hagen2008scaling}

Using existing Lustre tools to verify \emph{and} execute the time-triggered CAN driver from \autoref{s:motivation} is nontrivial.
Compiling the triggers array with an unverified compiler such as Lustre~V6~\cite{jahier2016lustre} or Heptagon~\cite{gerard2012modular} is straightforward; however, the verified Lustre compiler Vélus~\cite{bourke2023verified} does not support arrays or a foreign-function interface.
Recent work on translation validation for LustreC~\cite{brun2023equation} also does not yet support arrays.

Verifying the time-triggered CAN driver is trickier, as the restrictions placed on the triggers array --- that triggers are sorted by time-mark, there must be an adequate time-gap between a trigger and its next-enabled, and a trigger's time-mark must be greater-than-or-equal-to its index --- naturally require quantifiers.
As described in \autoref{s:evaluation}, Kind2 does include experimental array and quantifier support, but in our experiments was unable to verify arrays up to the 64 triggers supported by hardware implementations.
Additionally, due to the limitations on top-level array definitions, compiling the program with Lustre~V6 would result in the entire triggers array being copied to the stack each iteration.

Other model-checkers for Lustre such as Lesar~\cite{raymond2008synchronous}, JKind~\cite{gacek2018jkind} and the original Kind~\cite{hagen2008scaling} do not support quantifiers.
It may be possible to encode the quantifiers as fixed-size loops, but ensuring that these loops do not affect the execution or runtime complexity of the generated code does not appear to be straightforward.

These model-checkers have definite usability advantages over the general-purpose-prover approach offered here: they can often generate concrete counterexamples and implement counterexample-based invariant-generation techniques such as ICE~\cite{garg2014ice} and PDR~\cite{bradley2011sat,een2011efficient}.
However, even when the problem can be expressed, these model-checkers do not provide much assurance that the semantics they use for proofs matches the compiled code.
In the future, we would like to investigate integrating Pipit with a model-checker via an unverified extraction: such an extraction may allow some of the usability benefits such as counterexamples and invariant generation.
If this integration were used solely for debugging and suggesting candidate invariants, then such a change would not expand the trusted computing base.

Recent work has also introduced a form of refinement types for Lustre~\cite{chen2022synchronous}.
Rather than using transition systems, this work generates self-contained verification conditions based on the types of streams.
Such a type-based approach promises to allow abstraction of the implementation details.
However, for general-purpose functions such as \emph{count_when} from \autoref{s:motivation}, it is not clear how to give it a specification that actually \emph{abstracts} the implementation: a simple specification that the result is within some range would hide too much and be insufficient for verifying the rest of the system.
For such functions, the best specification is likely to include a re-statement of the implementation itself.

The embedded language Copilot generates real-time C code for runtime monitoring~\cite{laurent2015assuring}.
Recent work has used translation validation to show that the generated C code matches the high-level semantics~\cite{scott2023trustworthy}.
Copilot supports model-checking via Kind2; however, the model-checking has a limited specification language and does not support contracts.

Early work embedding a denotational semantics of Lucid Synchrone in an interactive theorem prover focussed on the semantics itself, rather than proving programs~\cite{boulme2001clocked}.
There is ongoing work to construct a denotational semantics of Vélus for program verification~\cite{bourke2022towards}.
We believe that the hybrid SMT approach of \fstar{} will allow for a better mixture of automated proofs with manual proofs.
Compared to Vélus alone, the trusted computing base of Pipit is larger: we depend on all of \fstar{}, \lowstar{}'s unverified C code extraction and the Z3 SMT solver; in comparison, Vélus' C code generation is verified and does not depend on any SMT solver.

% Refinement types have also been used to verify reactive systems~\cite{chen2022synchronous}, but this work does not address the issue of correct compilation.

% cite also:
% hardware language in Agda \cite{harrison2021mechanized}
% template-based invariant generation \cite{kahsai2011instantiation}

% Other compilers for Lustre-style languages, such as Vélus \cite{bourke2017formally} and Heptagon \cite{gerard2012modular} ensure that the generated code can be executed in bounded memory and that each step takes a bounded time.
% With a carefully chosen set of primitives, the core streaming operations shown here would ensure bounded memory and time.
% For practicality, however, we currently allow embedding arbitrary total functions from the \fstar{} meta-language, which are not necessarily bounded in time.
% (Is this the same as Lucid Synchrone and Zélus? \CITE)

The deferred aspect of our proofs is similar to the deferred proofs of verification conditions for imperative programs, such as \cite{oconnor2019deferring}.
However, such verification conditions are \emph{syntactically} deferred so that the verification condition can be proved later; in our case, the verification conditions are \emph{semantically} deferred, so that more knowledge of the enclosing program can be exploited in the proof.
In imperative programs, this sort of extra knowledge is generally provided explicitly as loop invariants, and non-looping statements have their weakest precondition computed automatically.
In Lustre-style reactive languages such as ours, programs tend to be composed of many nested recursive streams, which perform a similar function to loops.
Explicitly specifying an invariant for each recursive stream would be cumbersome; deferring the proof allows such invariants to be implicit.


% \subsection{Comparison to a combinator-based verification approach}
% \TODO{compare to refinement types, verification conditions, etc here}

% Our stream type \emph{stream $\tau$} denotes a (finite prefix of a) stream of elements of simple type $\tau$, but says nothing about how the stream evolves over time.
% Although Pipit is embedded in \fstar{}, which itself supports full dependent types and refinement types, we have opted to for a comparatively weakly-typed language.

% We could consider an alternative approach that was closer to an imperative program logic with preconditions and postconditions.
% For example, we could consider a richer type $\stream \tau~ \{ \rawbind{x}{\phi} \}$ which then the result, of type $\tau$, always satisfies stream postcondition $\phi$.
% We could imagine specifying count_when with such a type:

% \begin{tabbing}
%   @val@ count_when ($\textit{max}$: $\NN$) ($\textit{inc}$: stream $\BB$): stream $\{ \top \} ~\NN~ \{ \rawbind{c}{0 \le c \le max} \}$\\
% \end{tabbing}

% An alternative would be to use refinement types \cite{chen2022synchronous} or some kind of program logic with pre- and post-conditions.

% We believe that such type-based approaches are useful for composing programs together, but 
%  the nested-recursive nature of Lustre programs makes such approaches cumbersome.
% In our system, the recursive combinator has type $@rec@: (\stream \alpha \to \stream \alpha) \to \stream \alpha$.
% Consider instead, if we had a refined stream $\stream \alpha \{ \psi \}$

