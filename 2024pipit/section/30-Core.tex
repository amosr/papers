%!TEX root = ../Main.tex

% \pagebreak
\section{Core language}
\label{s:core}

%!TEX root = ../Main.tex

\begin{figure}
  \[
  \begin{array}{lrlr}
    e, e' & := & v ~|~ x ~|~ p(\ov{e}) & \mbox{(values, variables and operations)} \\
          & | & \xfby{v}{e} ~|~ \xrec{x}{e[x]} & \mbox{(followed-by (delay) and recursive streams)} \\ % & | & \xthen{e}{e'} \\
          & | & \xlet{x}{e}{e'[x]} & \mbox{(let-expressions)}\\
          & | & \xcheckP{\PStatus}{e_{\text{prop}}} & \mbox{(checked properties)} \\
          & | & \xcontractP{\PStatus}{\erely}{\eimpl}{x\!.~\eguar} & \mbox{(rely-guarantee contracts)} \\
          % & | & \xcheck{e} \\
    \\
    v & := & n \in \mathbb{N} ~|~ b \in \mathbb{B} ~|~ r \in \mathbb{R} ~|~ \hdots  & \mbox{(values)} \\
    p & := & (+) ~|~ (-) ~|~ (\times) ~|~ @if-then-else@ ~|~ \hdots & \mbox{(primitives)} \\
    \\
    \PStatus & := & \PSValid ~|~ \PSUnknown & \mbox{(property statuses: valid or unknown)}\\
    \\
    \sigma & := & \sgl{\ov{x \mapsto v}} & \mbox{(heaps)} \\
    \Sigma & := & \sigma ~|~ \Sigma;\sigma & \mbox{(streaming history environments)} \\
    %   \\
    \tau, \tau' & := & \mathbb{N} ~|~ \mathbb{B} ~|~ \tau \times \tau ~|~ \hdots & \mbox{(value types)} \\
    \Gamma & := & \cdot ~|~ x : \tau, \Gamma & \mbox{(type environments)}  \\
    \end{array}
  \]
  \caption{Pipit core language grammar, which contains expressions $e$, values $v$, primitive operations $p$, and property statuses $\PStatus$.}
  \label{f:core-grammar}
\end{figure}

We now introduce the core Pipit language.
Note that this form differs slightly from the surface syntax presented earlier in \autoref{s:motivation}, which used the syntax of the metalanguage \fstar{}, as well as including proofs in \fstar{} itself.

The grammar of Pipit is defined in \autoref{f:core-grammar}.
The expression form $e$ includes standard syntax for values ($v$), variables ($x$) and primitive applications ($p(\ov{e})$).
Most of the expression forms were introduced informally in \autoref{s:motivation} and correspond to the clock-free expressions of Lustre \cite{caspi1995functional}.

Values, variables and primitives are standard.
The expression syntax for delayed streams ($\xfby{v}{e}$) denotes the previous value of the stream $e$, with an initial value of $v$ when there is no previous value.
% Streams can also be composed together using the \emph{then} notation ($\xthen{e}{e'}$) which denotes that the value of stream $e$ is used for the first step, followed by the values from stream $e'$ for subsequent steps.

Recursive streams, which can refer to previous values of the stream itself, are defined using the fixpoint operator ($\xrec{x}{e[x]}$); the syntax $e[x]$ means that the variable $x$ can occur in $e$.
% To ensure that streams are productive,
As in Lustre, recursive streams can only refer to their previous values and must be \emph{guarded} by a delay: the stream $(\xrec{x}{\xfby{0}{(x + 1)}})$ is well-defined, but stream $(\xrec{x}{x + 1})$ is invalid and has no computational interpretation.
This form of recursion differs slightly from standard Lustre, which uses a set of mutually-recursive bindings.
We use this form to define a substitution-based operational semantics that is syntax-directed, as opposed to the mutually-recursive form in \cite{caspi1995functional} which is not syntax-directed.
Our semantics has a simpler proof of determinism; we believe it has simplified other necessary proofs too and will perform further evaluation.
Although we cannot express mutually-recursive bindings in the core syntax here, we can express them as a notation on the surface syntax by combining the bindings together into a single tuple.

Checked properties and contracts are annotated with their property status.

We support a fixed set of values and primitives.

Streams $V$ are represented as a non-empty sequence of values; streaming history environments $\Sigma$ are streams of heaps.
Types $\tau$ and type environments $\Gamma$ are standard.

\begin{figure}
  \[
    \boxed{\typing{\Gamma}{e}{\tau}}
    \quad
    \boxed{\mtyping{}{v}{\tau}}
  \]

  \[
    \ruleIN{
      \mtyping{}{v}{\tau}
    }{
      \typing{\Gamma}{v}{\tau}
    }{TValue}
    \quad
    \ruleAx{\typing{\Gamma}{x}{\Gamma(v)}}{TVar}
  \]

  \[
    \ruleIN{
      \typing{\Gamma}{e}{\tau \to \tau'}
      \qquad
      \typing{\Gamma}{e'}{\tau}
    }{\typing{\Gamma}{e~e'}{\tau'}}{TApp}
  \]

  \[
    \ruleIN{
      \typing{\Gamma}{v}{\tau}
      \qquad
      \typing{\Gamma}{e'}{\tau}
    }{
      \typing{\Gamma}{\xfby{v}{e'}}{\tau}
    }{TFby}
  \]

  \[
    \ruleIN{
      \typing{\Gamma}{e}{\tau}
      \qquad
      \typing{\Gamma}{e'}{\tau}
    }{\typing{\Gamma}{\xthen{e}{e'}}{\tau}}{TThen}
  \]

  \[
    \ruleIN{
      \typing{x : \tau, \Gamma}{e}{\tau}
    }{
      \typing{\Gamma}{\xrec{x}{e[x]}}{\tau}
    }{TRec}
  \]

  \[
    \ruleIN{
      \typing{\Gamma}{e}{\tau}
      \qquad
      \typing{x : \tau, \Gamma}{e'}{\tau'}
    }{
      \typing{\Gamma}{\xlet{x}{e}{e'[x]}}{\tau'}
    }{TLet}
  \]

  \[
    \ruleIN{
      \typing{\Gamma}{e}{\mathbb{B}}
    }{
      \typing{\Gamma}{\xcheck{e}}{\tt{unit}}
    }{TCheck}
  \]

  \caption{Typing rules for Pipit defined in terms of two judgment forms: $\typing{\Gamma}{e}{\tau}$ denotes that expression $e$ describes a \emph{stream} of values of type $\tau$; and $\mtyping{}{v}{\tau}$, which denotes that closed meta-value $v$ has type $\tau$ and is not defined here.}\label{f:core-typing}
\end{figure}

We define the typing judgments for Pipit in \autoref{f:core-typing}.
Most of the typing rules are fairly standard for an unclocked Lustre.
The typing judgment $\typing{\Gamma}{e}{\tau}$ denotes that, in an environment of streams $\Gamma$, expression $e$ denotes a stream of type $\tau$.
This core typing judgment differs from the surface syntax used in \autoref{s:motivation}, which used an explicit stream type.
For the core language, we assume that everything is a stream.

For values, we use an auxiliary judgment form $\mtypingval{v}{\tau}$ to denote that value $v$ has type $\tau$.
Likewise, for primitives we use the auxiliary judgment form $\mtypingprim{p}{(\tau_1 \times \cdots \hdots \times \tau_n) \to \tau'}$ to denote that primitive $p$ takes arguments of type $\tau_i$ and returns a result of type $\tau'$.
Primitives are pure, non-streaming functions.

Rules \textsc{TValue}, \textsc{TVar}, \textsc{TPrim} and \textsc{TLet} are standard.

Rule \textsc{TFby} states that expression $\xfby{v}{e}$ requires both $v$ and $e$ to have equal types; the result is the same type.

Rule \textsc{TRec} states that a recursive stream $\xrec{x}{e}$ has the recursive stream bound inside $e$.
The recursion must also be guarded, in that any recursive references to $x$ are delayed, but this requirement is performed as a separate syntactic check described in \REF{}.

Rule \textsc{TCheck} states that checking a property $\xcheckP{\PStatus}{e}$ requires a boolean property $e$ and returns unit.

Finally, rule \textsc{TContract} applies for a contract $\xcontractP{\PStatus}{\erely}{\eimpl}{\eguar}$ with a body expression of some type $\tau$.
The overall expression has result type $\tau$.
Both rely and guarantee clauses must be boolean expressions.
Additionally, the guarantee clause can refer to the result value by $x$.

\subsection{Dynamic semantics}
%!TEX root = ../Main.tex

\begin{figure}[t]
  \begin{mathpar}
    \boxed{\bigstep{\Sigma}{e}{v}}
  \end{mathpar}
  \begin{mathpar}
    \ruleAx{\bigstep{\Sigma; \sigma}{x}{\sigma(x)}}{Var}
    \quad
    \ruleAx{\bigstep{\Sigma}{v}{v}}{Value}

    \ruleIN{
      \bigstep{\Sigma}{e'[x := e]}{v}
    }{
      \bigstep{\Sigma}{\xlet{x}{e}{e'[x]}}{v}
    }{Let}

    \ruleIN{
      \bigstep{\Sigma}{e_1}{v_1} \quad \hdots \quad
      \bigstep{\Sigma}{e_n}{v_n}
    }{\bigstep{\Sigma}{p(\ov{e})}{\text{prim-sem}(p, \ov{v})}}{Prim}


    % \[
  %   \ruleIN{\bigstep{\Sigma}{e}{v}}{\bigstep{\Sigma; \sigma}{\xpre{e}}{v}}{Pre}
  % \]

  \ruleAx{\bigstep{\sigma}{\xfby{v}{e'}}{v}}{$\mbox{Fby}_1$}
    \quad
    \ruleIN{\text{length}(\Sigma) > 0 \and \bigstep{\Sigma}{e'}{v'}}{\bigstep{\Sigma; \sigma}{\xfby{v}{e'}}{v'}}{$\mbox{Fby}_S$}

    % \ruleIN{\bigstep{\sigma}{e}{v}}{\bigstep{\sigma}{\xthen{e}{e'}}{v}}{$\xthenarrow_1$}
    % \quad
    % \ruleIN{\bigstep{\Sigma}{e'}{v'}}{\bigstep{\Sigma; \sigma}{\xthen{e}{e'}}{v'}}{$\xthenarrow_S$}

    \ruleIN{
      \bigstep{\Sigma}{e[x := \xrec{x}{e}]}{v}
    }{
      \bigstep{\Sigma}{\xrec{x}{e[x]}}{v}
    }{Rec}

    \ruleIN{
      % \bigstep{\Sigma}{e}{\top}
    }{
      \bigstep{\Sigma}{\xcheckP{\PStatus}{e}}{()}
    }{Check}

    \ruleIN{
      \bigstep{\Sigma}{\ebody}{v}
    }{
      \bigstep{\Sigma}{\xcontractP{\PStatus}{\erely}{\ebody}{\rawbind{x}{\eguar[x]}}}{v}
    }{Contract}
  \end{mathpar}

  \begin{mathpar}
    \boxed{\bigsteps{\Sigma}{e}{V}}

    \boxed{\bigstepalways{\Sigma}{e}}
  \end{mathpar}
  \begin{mathpar}
    \ruleAx{\bigsteps{\cdot}{e}{\cdot}}{$\mbox{Steps}_0$}
    % \ruleIN{\bigstep{\sigma}{e}{v}}{\bigsteps{\sigma}{e}{v}}{$\mbox{Steps}_1$}

    \ruleIN{
      \bigstep{\Sigma}{e}{V}
      \and
      \bigstep{\Sigma; \sigma}{e}{v}
    }{\bigstep{\Sigma; \sigma}{e}{V; v}}{$\mbox{Steps}_S$}

    \ruleIN{\bigsteps{\Sigma}{e}{\true; \hdots}}{\bigstepalways{\Sigma}{e}}{Always}
  \end{mathpar}


  \caption{Dynamic semantics for Pipit; the judgment form $\bigstep{\Sigma}{e}{v}$ denotes that evaluating expression $e$ under streaming history $\Sigma$ results in value $v$.}\label{f:core-bigstep}
\end{figure}


The dynamic semantics of Pipit are defined in \autoref{f:core-bigstep}.
We present our semantics in a big-step form.
This differs somewhat from traditional \emph{reactive} semantics of Lustre presented in \cite{caspi1995functional}.
We believe that the big-step semantics is better for performing equational reasoning, as it is substitution-based, while the reactive semantics better emphasises the finite-state execution of the system.
For reasoning about the finite-state execution, we instead use transition systems (\autoref{s:transition}).


The judgment form $\bigstep{\Sigma}{e}{v}$ denotes that expression $e$ evaluates to value $v$ under streaming history $\Sigma$.
The streaming history is a stream of heaps.



\subsection{Checked semantics}
%!TEX root = ../Main.tex

\begin{figure}
  \begin{mathpar}
    \boxed{\semcheck{\Sigma}{\PStatus}{e}}
  \end{mathpar}

  \begin{mathpar}
    \ruleAx{\semcheck{\Sigma}{\PStatus}{v}}{ChkValue}
    \quad
    \ruleAx{\semcheck{\Sigma}{\PStatus}{x}}{ChkVar}

    \ruleIN{
      \semcheck{\Sigma}{\PStatus}{e_1} \quad \hdots \quad
      \semcheck{\Sigma}{\PStatus}{e_n}
    }{\semcheck{\Sigma}{\PStatus}{p(\ov{e})}}{ChkPrim}


  \ruleAx{\semcheck{\sigma}{\PStatus}{\xfby{v}{e'}}}{$\mbox{ChkFby}_1$}

    \ruleIN{\text{length}(\Sigma) > 0 \and \semcheck{\Sigma}{\PStatus}{e'}}{\semcheck{\Sigma; \sigma}{\PStatus}{\xfby{v}{e'}}}{$\mbox{ChkFby}_S$}

    \ruleIN{
      \semcheck{\Sigma}{\PStatus}{e[x := \xrec{x}{e}]}
    }{
      \semcheck{\Sigma}{\PStatus}{\xrec{x}{e[x]}}
    }{ChkRec}

    \ruleIN{
      \semcheck{\Sigma}{\PStatus}{e'[x := e]}
    }{
      \semcheck{\Sigma}{\PStatus}{\xlet{x}{e}{e'[x]}}
    }{ChkLet}

    \ruleIN{
      (\PStatus = \PStatus' \implies \bigstepalways{\Sigma}{e})
      \and
      \semcheck{\Sigma}{\PStatus}{e}
    }{
      \semcheck{\Sigma}{\PStatus}{\xcheckP{\PStatus'}{e}}
    }{ChkCheck}

    % \ruleIN{
    %   \PStatus \neq \PStatus'
    % }{
    %   \semcheck{\Sigma}{\PStatus}{\xcheckP{\PStatus'}{e}}
    % }{ChkNoCheck}

    \inferrule{
      (\PStatus = \PStatus' \implies \bigstepalways{\Sigma}{\erely})
      \\\\
      (\PStatus = \PSValid \implies \bigstepalways{\Sigma}{\erely} \implies \bigstepalways{\Sigma}{\eguar[x := \ebody]})
      \\\\
      \semcheck{\Sigma}{\PStatus}{\erely}
      \\\\
      (\bigstepalways{\Sigma}{\erely} \implies \semcheck{\Sigma}{\PStatus}{\ebody} ~\wedge~ \semcheck{\Sigma}{\PStatus}{\eguar[x := \ebody]})
    }{
      \semcheck{\Sigma}{\PStatus}{\xcontractP{\PStatus'}{\erely}{\ebody}{\rawbind{x}{\eguar[x]}}}
    }(\textsc{ChkContract})

    % \ruleIN{
    %   \PStatus \neq \PStatus'
    %   \and
    %   (\bigstepalways{\Sigma}{\erely}
    %    \implies \bigstepalways{\Sigma}{\eguar[x := \ebody]})
    % }{
    %   \semcheck{\Sigma}{\PStatus}{\xcontractP{\PStatus'}{\erely}{\ebody}{\rawbind{x}{\eguar[x]}}}
    % }{ChkNoContract}
  \end{mathpar}

  \caption{Checked semantics for Pipit; the judgment form $\semcheck{\Sigma}{\PStatus}{e}$ denotes that evaluating expression $e$ under streaming history $\Sigma$ satisfies the checks and rely-guarantee contract requirements that are labelled with property status $\PStatus$.}\label{f:core-check}
\end{figure}


We define the checked semantics of Pipit in \autoref{f:core-check}.
\TODO{}

\TODO{Define bless function}
