%!TEX root = ../Main.tex

% \pagebreak
\section{Core language}
\label{s:core}

%!TEX root = ../Main.tex

\begin{figure}
  \[
  \begin{array}{lrlr}
    e, e' & := & v ~|~ x ~|~ p(\ov{e}) & \mbox{(values, variables and operations)} \\
          & | & \xfby{v}{e} ~|~ \xrec{x}{e[x]} & \mbox{(followed-by (delay) and recursive streams)} \\ % & | & \xthen{e}{e'} \\
          & | & \xlet{x}{e}{e'[x]} & \mbox{(let-expressions)}\\
          & | & \xcheckP{\PStatus}{e_{\text{prop}}} & \mbox{(checked properties)} \\
          & | & \xcontractP{\PStatus}{\erely}{\eimpl}{x\!.~\eguar} & \mbox{(rely-guarantee contracts)} \\
          % & | & \xcheck{e} \\
    \\
    v & := & n \in \mathbb{N} ~|~ b \in \mathbb{B} ~|~ r \in \mathbb{R} ~|~ \hdots  & \mbox{(values)} \\
    p & := & (+) ~|~ (-) ~|~ (\times) ~|~ @if-then-else@ ~|~ \hdots & \mbox{(primitives)} \\
    \\
    \PStatus & := & \PSValid ~|~ \PSUnknown & \mbox{(property statuses: valid or unknown)}\\
    \\
    \sigma & := & \sgl{\ov{x \mapsto v}} & \mbox{(heaps)} \\
    \Sigma & := & \sigma ~|~ \Sigma;\sigma & \mbox{(streaming history environments)} \\
    %   \\
    \tau, \tau' & := & \mathbb{N} ~|~ \mathbb{B} ~|~ \tau \times \tau ~|~ \hdots & \mbox{(value types)} \\
    \Gamma & := & \cdot ~|~ x : \tau, \Gamma & \mbox{(type environments)}  \\
    \end{array}
  \]
  \caption{Pipit core language grammar, which contains expressions $e$, values $v$, primitive operations $p$, and property statuses $\PStatus$.}
  \label{f:core-grammar}
\end{figure}
\begin{figure}
  \[
    \boxed{\typing{\Gamma}{e}{\tau}}
    \quad
    \boxed{\mtyping{}{v}{\tau}}
  \]

  \[
    \ruleIN{
      \mtyping{}{v}{\tau}
    }{
      \typing{\Gamma}{v}{\tau}
    }{TValue}
    \quad
    \ruleAx{\typing{\Gamma}{x}{\Gamma(v)}}{TVar}
  \]

  \[
    \ruleIN{
      \typing{\Gamma}{e}{\tau \to \tau'}
      \qquad
      \typing{\Gamma}{e'}{\tau}
    }{\typing{\Gamma}{e~e'}{\tau'}}{TApp}
  \]

  \[
    \ruleIN{
      \typing{\Gamma}{v}{\tau}
      \qquad
      \typing{\Gamma}{e'}{\tau}
    }{
      \typing{\Gamma}{\xfby{v}{e'}}{\tau}
    }{TFby}
  \]

  \[
    \ruleIN{
      \typing{\Gamma}{e}{\tau}
      \qquad
      \typing{\Gamma}{e'}{\tau}
    }{\typing{\Gamma}{\xthen{e}{e'}}{\tau}}{TThen}
  \]

  \[
    \ruleIN{
      \typing{x : \tau, \Gamma}{e}{\tau}
    }{
      \typing{\Gamma}{\xrec{x}{e[x]}}{\tau}
    }{TRec}
  \]

  \[
    \ruleIN{
      \typing{\Gamma}{e}{\tau}
      \qquad
      \typing{x : \tau, \Gamma}{e'}{\tau'}
    }{
      \typing{\Gamma}{\xlet{x}{e}{e'[x]}}{\tau'}
    }{TLet}
  \]

  \[
    \ruleIN{
      \typing{\Gamma}{e}{\mathbb{B}}
    }{
      \typing{\Gamma}{\xcheck{e}}{\tt{unit}}
    }{TCheck}
  \]

  \caption{Typing rules for Pipit defined in terms of two judgment forms: $\typing{\Gamma}{e}{\tau}$ denotes that expression $e$ describes a \emph{stream} of values of type $\tau$; and $\mtyping{}{v}{\tau}$, which denotes that closed meta-value $v$ has type $\tau$ and is not defined here.}\label{f:core-typing}
\end{figure}
\begin{figure}
  \[ \boxed{\bigstep{\Sigma}{e}{v}} \]

  \[
    \ruleAx{\bigstep{\Sigma}{v}{v}}{Value}
    \quad
    \ruleAx{\bigstep{\Sigma_\bot; \sigma}{x}{\sigma(v)}}{Var}
  \]

  \[
    \ruleIN{
      \bigstep{\Sigma}{e}{(\mlamX)}
      \qquad
      \bigstep{\Sigma}{e'}{v'}
    }{\bigstep{\Sigma}{e~e'}{(\mlamX)~v'}}{App-Meta}
  \]

  % \[
  %   \ruleIN{\bigstep{\Sigma}{e}{v}}{\bigstep{\Sigma; \sigma}{\xpre{e}}{v}}{Pre}
  % \]

  \[
    \ruleAx{\bigstep{\sigma}{\xfby{v}{e'}}{v}}{$\mbox{Fby}_1$}
    \quad
    \ruleIN{\bigstep{\Sigma}{e'}{v'}}{\bigstep{\Sigma; \sigma}{\xfby{v}{e'}}{v'}}{$\mbox{Fby}_S$}
  \]

  \[
    \ruleIN{\bigstep{\sigma}{e}{v}}{\bigstep{\sigma}{\xthen{e}{e'}}{v}}{$\xthenarrow_1$}
    \quad
    \ruleIN{\bigstep{\Sigma}{e'}{v'}}{\bigstep{\Sigma; \sigma}{\xthen{e}{e'}}{v'}}{$\xthenarrow_S$}
  \]

  \[
    \ruleIN{
      \bigstep{\Sigma}{e[x := \xrec{x}{e}]}{v}
    }{
      \bigstep{\Sigma}{\xrec{x}{e[x]}}{v}
    }{Rec}
  \]

  \[
    \ruleIN{
      \bigstep{\Sigma}{e'[x := e]}{v}
    }{
      \bigstep{\Sigma}{\xlet{x}{e}{e'[x]}}{v}
    }{Let}
  \]

  \[
    \ruleIN{
      \bigstep{\Sigma}{e}{\top}
    }{
      \bigstep{\Sigma}{\xcheck{e}}{()}
    }{Check}
  \]

  \caption{Bigstep operational semantics for Pipit defined in terms of the judgment form $\bigstep{\Sigma}{e}{v}$; this judgment form denotes that evaluating expression $e$ under streaming history $\Sigma$ results in value $v$.}\label{f:core-bigstep}
\end{figure}
%!TEX root = ../Main.tex

\begin{figure}
  \begin{mathpar}
    \boxed{\semcheck{\Sigma}{\PStatus}{e}}
  \end{mathpar}

  \begin{mathpar}
    \ruleAx{\semcheck{\Sigma}{\PStatus}{v}}{ChkValue}
    \quad
    \ruleAx{\semcheck{\Sigma}{\PStatus}{x}}{ChkVar}

    \ruleIN{
      \semcheck{\Sigma}{\PStatus}{e_1} \quad \hdots \quad
      \semcheck{\Sigma}{\PStatus}{e_n}
    }{\semcheck{\Sigma}{\PStatus}{p(\ov{e})}}{ChkPrim}


    \ruleIN{\semcheck{\Sigma}{\PStatus}{e'}}{\semcheck{\Sigma}{\PStatus}{\xfby{v}{e'}}}{ChkFby}

    \ruleIN{
      \bigsteps{\Sigma}{\xrec{x}{e}}{V}
      \and
      \semcheck{\Sigma[x \mapsto V]}{\PStatus}{e}
    }{
      \semcheck{\Sigma}{\PStatus}{\xrec{x}{e[x]}}
    }{ChkRec}

    \ruleIN{
      \semcheck{\Sigma}{\PStatus}{e}
      \and
      \bigsteps{\Sigma}{e}{V}
      \and
      \semcheck{\Sigma[x \mapsto V]}{\PStatus}{e'}
    }{
      \semcheck{\Sigma}{\PStatus}{\xlet{x}{e}{e'[x]}}
    }{ChkLet}

    \ruleIN{
      (\PStatus = \PStatus' \implies \bigstepalways{\Sigma}{e})
      \and
      \semcheck{\Sigma}{\PStatus}{e}
    }{
      \semcheck{\Sigma}{\PStatus}{\xcheckP{\PStatus'}{e}}
    }{ChkCheck}

    % \ruleIN{
    %   \PStatus \neq \PStatus'
    % }{
    %   \semcheck{\Sigma}{\PStatus}{\xcheckP{\PStatus'}{e}}
    % }{ChkNoCheck}

    \inferrule{
      \bigsteps{\Sigma}{\ebody}{V}
      \\\\
      (\PStatus = \PStatus' \implies \bigstepalways{\Sigma}{\erely})
      \\\\
      (\PStatus = \PSValid \implies \bigstepalways{\Sigma}{\erely} \implies \bigstepalways{\Sigma[x \mapsto V]}{\eguar})
      \\\\
      \semcheck{\Sigma}{\PStatus}{\erely}
      \\\\
      (\bigstepalways{\Sigma}{\erely} \implies \semcheck{\Sigma}{\PStatus}{\ebody} ~\wedge~ \semcheck{\Sigma[x \mapsto V]}{\PStatus}{\eguar})
    }{
      \semcheck{\Sigma}{\PStatus}{\xcontractP{\PStatus'}{\erely}{\ebody}{\rawbind{x}{\eguar[x]}}}
    }(\textsc{ChkContract})

    % \ruleIN{
    %   \PStatus \neq \PStatus'
    %   \and
    %   (\bigstepalways{\Sigma}{\erely}
    %    \implies \bigstepalways{\Sigma}{\eguar[x := \ebody]})
    % }{
    %   \semcheck{\Sigma}{\PStatus}{\xcontractP{\PStatus'}{\erely}{\ebody}{\rawbind{x}{\eguar[x]}}}
    % }{ChkNoContract}
  \end{mathpar}

  \caption{Checked semantics for Pipit; the judgment form $\semcheck{\Sigma}{\PStatus}{e}$ denotes that evaluating expression $e$ under streaming history $\Sigma$ satisfies the checks and rely-guarantee contract requirements that are labelled with property status $\PStatus$.}\label{f:core-check}
\end{figure}


The grammar of Pipit is defined in \autoref{f:core-grammar}.
The expression form $e$ includes standard syntax for values ($v$), variables ($x$) and applications ($e~e'$); however, it does not include any form for defining functions except reusing closed functions from the \fstar{} meta-language $(x)$.
Most of the expression forms were introduced informally in \autoref{s:tut} and correspond to the clock-free primitives in Lustre \cite{caspi1995functional}.
% The expression syntax for delayed streams ($\xfby{v}{e}$) denotes the previous value of the stream $e$, with an initial value of $v$ when there is no previous value.
Streams can also be composed together using the \emph{then} notation ($\xthen{e}{e'}$) which denotes that the value of stream $e$ is used for the first step, followed by the values from stream $e'$ for subsequent steps.

Recursive streams, which can refer to previous values of the stream itself, are defined using the fixpoint operator ($\xrec{x}{e[x]}$); the syntax $e[x]$ means that the variable $x$ can occur in $e$.
% To ensure that streams are productive,
As in Lustre, recursive streams can only refer to their previous values and must be \emph{guarded} by a delay: the stream $(\xrec{x}{\xfby{0}{(x + 1)}})$ is well-defined, but stream $(\xrec{x}{x + 1})$ is invalid and has no computational interpretation.
This form of recursion differs slightly from standard Lustre, which uses a set of mutually-recursive bindings.
We use this form to define a substitution-based operational semantics that is syntax-directed, as opposed to the mutually-recursive form in \cite{caspi1995functional} which is not syntax-directed.
Our semantics has a simpler proof of determinism; we believe it has simplified other necessary proofs too and will perform further evaluation.
Although we cannot express mutually-recursive bindings in the core syntax here, we can express them as a notation on the surface syntax at the expense of potentially duplicating expressions.
