%!TEX root = ../Main.tex

\section{Conclusion}

We have presented Pipit, a verified compiler and proof system for reactive systems.
Our implementation of the TTCAN driver logic shows that, by embedding pure \fstar{} functions for array operations, Pipit can express programs which are currently unsupported by other verified Lustre compilers.
Pipit can also verify high-level program properties which are difficult to express and prove in existing Lustre model-checkers.
Our development includes verified translations to both abstract and executable transition systems; both are shown to preserve the dynamic semantics.
We also introduced a checked semantics, which describes the semantics of checked properties and contracts; proof obligations generated by translation to abstract transition system are verified to correspond to these semantics.

In the future, we intend to verify the remainder of the TTCAN driver logic.
We also intend to increase the expressivity of Pipit by adding \emph{clocks}, which are used to describe partially-defined streams~\cite{caspi1995functional}.
Clocks are important for composing complex systems together and avoiding unnecessary computation; they may be useful if it becomes necessary to optimise the runtime of the TTCAN driver.

We are interested in further pursuing the intersection of model-checking with interactive theorem proving.
A smart-contract called Djed \cite{zahnentferner2023djed} currently uses a mixture of Kind2 \cite{champion2016kind2} and manual Isabelle/HOL proofs to show that the contract is well-behaved.
In future work, we would like to further investigate whether Pipit's integration of streaming proofs with \fstar{}'s automated proof system would be able to provide similar proofs, without introducing any semantic gap between the two systems.
