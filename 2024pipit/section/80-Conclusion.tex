%!TEX root = ../Main.tex

\section{Conclusion}

\TODO{requires rewrite}
\TODO{future work: clocks}

Our preliminary results show that \fstar{}'s proof automation and code extraction are suitable for verifying reactive systems and executing them in real-time; these results still require further work.
Next, we intend to verify the imperative code generation.
%  To verify large programs, we also need some way to separately prove smaller pieces which can then be composed together, such as contracts~\cite{champion2016kind2}.
Finally, we need to evaluate Pipit on larger control systems before extending the language to support more features, such as Lustre's clocks for describing partially-defined streams~\cite{caspi1995functional}.

% TODO
%  Future work: verify imperative code generation; verify CSE; case studies: antilock braking; clocks \cite{caspi1995functional}; contracts \cite{champion2016kind2}.

We are interested in further pursuing the intersection of model-checking with interactive theorem proving.
A smart contract called Djed \cite{zahnentferner2023djed} currently uses a mixture of Kind2 \cite{champion2016kind2} and manual Isabelle/HOL proofs to show that the contract is well-behaved.
In future work, we would like to further investigate whether Pipit's integration of streaming proofs with \fstar{}'s automated proof system would be able to provide similar proofs, without introducing any semantic gap between the two systems.

Our current array support is limited: constant arrays provide a pure index function, while we only support fixed-size mutable arrays by wrapping bit-vectors.
Array support is an obvious direction for future work; integrating with a verified array-fusion system such as \cite{robinson2017machine} would be an interesting and useful extension.
