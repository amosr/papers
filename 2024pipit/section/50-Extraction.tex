%!TEX root = ../Main.tex


\section{Extraction}
\label{s:extraction}

Pipit can generate executable code which is suitable for real-time execution on embedded devices.
The code extraction is implemented in a currently-unverified transform that takes a deeply embedded representation of a Pipit expression and generates a shallow imperative representation of the program.
\fstar{} can generate C code from a subset of the language called \lowstar{}~\cite{protzenko2017verified}; the result of our translation to imperative code fits in this subset.
During code extraction, we use \fstar{}'s tactic support~\cite{martinez2019meta} to fully normalise the translation to imperative code, conceptually similar to staged compilation.

% For the simple examples in \autoref{ss:tut:bibo}, we probably could have proved the properties directly on the generated code.
% In the future, however, we intend to support the abstraction of streams through \emph{contracts}, which require a relational semantics rather than a functional one.

% noextract
% let expr = Sugar.run2 Pump.controller'

% noextract
% let system: Pipit.Exec.Exp.xexec expr =
%   XX.exec_of_exp expr

% [@@(Tac.postprocess_with XL.tac_extract)]
% let reset = XL.mk_reset system

% [@@(Tac.postprocess_with XL.tac_extract)]
% let step (inp: input) = XL.mk_step system (inp.estop, (inp.level_low, ()))
% This allows us to generate C code which executes in bounded space on embedded devices.
