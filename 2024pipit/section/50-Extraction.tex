%!TEX root = ../Main.tex


\section{Extraction}
\label{s:extraction}

Pipit can generate executable code which is suitable for real-time execution on embedded devices.
The code extraction uses a variation of the abstract transition system described in \autoref{s:transition}, with two main differences to ensure that the result is executable without relying on the environment to provide values for the free context.
Contracts are straightforward to execute by using the body of the contract rather than abstracting over the implementation.

To execute recursive expressions $\xrec{x}{e} : \tau$, we require an arbitrary value of type $\tau$ to seed the fixpoint, as described in \autoref{s:core:causality}.
We first call the step function to evaluate $e$ with $x$ bound to $\bot_\tau$.
This step call returns the correct value, but the updated state is invalid, as it may refer to the bottom value.
To get the correct state, we call the step function again, this time with $e$ bound to $v$.

For example, for the \emph{sum} contract with body $\sumrecX{\textit{ints}}$, we generate an executable system that takes an input context containing integer variable \textit{ints}, with a single state variable for the followed-by, and returning an integer:

  \begin{tabbing}
  MM \= update: \= = \= @let (sum,y) @ \= \kill
  @let@ sum_def: system (ints: $\ZZ$) $\ZZ$ = \{ \\
  \> state   \> = (sum_fby: $\ZZ$); \\
  \> init  \> = \{ sum_fby = 0; \}; \\
  \> step  \> = $\lambda{} i~s.$ \\
  \> \> \> @let@ $(\textit{fby}_0, s_0)$ \> $= (s.\text{sum_fby}, s~\{ \text{sum_fby} = \bot_\ZZ \})$ @in@ \\
  \> \> \> @let@ $(\textit{sum}_0, s_0)$ \> $= (\textit{fby}_0 + i.\text{ints}, s_0)$ @in@ \\
  \> \> \> @let@ $(\textit{fby}_1, s_1)$ \> $= (s.\text{sum_fby}, s~\{ \text{sum_fby} = \textit{sum}_0 \})$ @in@ \\
  \> \> \> @let@ $(\textit{sum}_1, s_1)$ \> $= (\textit{fby}_1 + i.\text{ints}, s_1)$ @in@ \\
  \> \> \> $(\textit{sum}_0, s_1)$ \}
  \end{tabbing}

Here, the step function takes records describing instances of the input and state contexts, and returns a pair of the result value and the updated state.
The first two bindings correspond to the seeded evaluation with the recursive value for the sum set to $\bot_\ZZ$; as such, the resulting state $s_0$ is invalid.
The last two bindings recompute the state, this time with the correct recursive value.
This duplication of work is often removed by partial evaluation and dead-code-elimination, which are performed during code extraction.

This translation to transition systems is verified to preserve the original semantics.
The invariant is very similar to that of \autoref{s:transition:proof}, except that the invariant descends into the implementations of contracts.
For the abstract systems we only showed abstraction; to prove that executable systems are equivalent to the original semantics, we rely on the fact that the original semantics and transition systems are both deterministic and total (\autoref{s:core:causality}).

\begin{theorem}[execution-equivalence]
  For a well-typed causal expression $e$ and streaming history $\Sigma$, $e$ evaluates to stream $V$ $(\bigsteps{\Sigma}{e}{V})$ if-and-only-if repeated application of the transition system's step on $\Sigma$ also results in $V$.
\end{theorem}

To extract the program, we use a \emph{hybrid embedding} as described in \cite{ho2022noise}, which is similar to staged-compilation.
The hybrid embedding involves a deep embedding of the Pipit core language, while the translation to executable transition systems produces a shallow embedding.
We use the \fstar{} host language's normalisation-by-evaluation and tactic support~\cite{martinez2019meta} to specialise the application of the translation to a particular input program.
This specialisation results in a concrete transition system that fits in the \lowstar{} subset of \fstar{}, which can then be extracted to statically-allocated C code~\cite{protzenko2017verified}.

The generated C code for \emph{sum}\footnote{This interface is for a variant of the sum contract with 32-bit integers instead of unbounded integers.} includes a struct type to hold the state information, as well as reset and step functions; this interface is standard for Lustre compilers~\cite{bourke2017formally,gerard2012modular}:
  \begin{tabbing}
    @struct sum_state { uint32_t sum_fby; }@ \\
    @void   sum_reset(struct sum_state* state);@ \\
    @int    sum_step(struct sum_state* state, uint32_t ints);@
  \end{tabbing}

The reset function takes the pointer to the state struct and sets it to its initial values.
The step function takes the pointer to the state struct and the inputs, and returns the result integer.
The state struct is updated in-place.
The implementations of these functions avoid dynamic (heap) allocation and are suitable for embedded systems.


Unfortunately, our current approach is unsuitable for generating imperative array code, as our shallowly-embedded pure transition system requires pure arrays.
In the future, we intend to address array computations and the above work duplication by introducing an intermediate imperative language such as Obc~\cite{biernacki2008clock}, a static object-based language suitable for synchronous systems.
Even with an added intermediate language, we believe that a variant of our current translation and proof-of-correctness will remain useful as an intermediate semantics.
% We currently generate a monolithic step function for each Pipit program; we hope that an embedded foreign-function interface similar to that used by Accelerate~\cite{clifton2014embedding} may also enable separate-compilation of programs.
