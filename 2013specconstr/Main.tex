\documentclass{tmr}
\usepackage{style/utils}
\usepackage{style/code}

\title{Constructor specialisation in GHC}
\author{Amos Robinson\email{amosr@cse.unsw.edu.au}}

\begin{document}
\makeatactive

\begin{introduction}
In Haskell, loops are expressed as recursive functions. 
When loop variables are updated, each iteration ends up allocating new objects in memory, only to be deconstructed immediately in the next iteration.
For high-performance code, these spurious allocations turn out to have serious penalties.

This problem has been solved specifically for boxed objects such as @Int@, @Float@ and others in the worker/wrapper transform~\CITE.
However, in the case of code produced by stream fusion~\CITE, other types such as @Either@ and tuples are used as loop variables.
Constructor specialisation, or SpecConstr, is a generalisation of worker/wrapper for arbitrary types.
This is able to remove such allocations in stream-fusion code, and in some circumstances, produce object code that competes with hand-written C.
\end{introduction}

\section{Background}
\subsection{Core}
Core is GHC's intermediate language, an explicitly typed System-F style language.
After the source program has been parsed, its types checked and inferred, it is converted to core.
The majority of optimisations are then performed on the core code,
before eventually being converted to object code.

To an experienced Haskell programmer, learning to read core should take little effort, and is a valuable skill for writing high-performance code.
Where clauses are turned into @let@s, or @letrec@s for recursive expressions.
Pattern matching and guards are converted to explicit @case@ expressions.

To illustrate this, there follows an example implementation of @foldl@, and its core equivalent.
\begin{code}
foldl :: (a -> b -> a) -> a -> [b] -> a
foldl k z xs = go xs z
 where
  go []     acc = acc
  go (x:xs) acc = go xs (k acc x)
\end{code}

The core is quite a bit noisier,
but it is easy to see that the idea is the same.

\begin{code}
foldl = \@a @b k z xs ->
 letrec {
     go = \xs' acc ->
       case xs' of {
         []     -> acc;
         : x xs -> go xs (k acc x)
       }
 } in go xs z
\end{code}



\subsection{Simplifier}
The simplifier is the workhorse of GHC's optimisations.
It includes many different local transforms such as inlining, case-of-constructor, etc.


\subsection{Rewrite rules}

\subsection{Unboxed types}
In GHC, the @Int@ type is defined as a wrapper around an unboxed int,
and unboxed int literals and operators have @#@ appended to their name.

\begin{code}
data Int = I# Int#

(+) :: Int -> Int -> Int
(+) (I# x) (I# y) = I# (x +# y)
\end{code}

With sufficient inlining and case-of-constructor simplification,
an arithmetic expression can be reduced to as few allocations as possible.

In the following sections, ints will be boxed and unboxed explicitly to make reductions in allocations more obvious.

\section{Motivation}
Consider a simple example of summing a list of @Int@s.
\begin{code}
sum :: [Int] -> Int
sum = foldl (+) 0
\end{code}

After inlining @foldl@, we get a recursive worker function.
\begin{code}
sum xs = go xs (I# 0#)
 where
  go []          (I# acc) = I# acc
  go ((I# x):xs) (I# acc) = go xs (I# (acc +# x))
\end{code}



\begin{code}
dotp :: Vector Double -> Vector Double -> Double
dotp as bs = go (Nothing, 0) 0
 where
  go (_, i) acc
   | i > V.length as
   = acc
  go (Nothing, i) acc
   = go (Just (as!i), i) acc
  go (Just a, i) acc
   = go (Nothing, i + 1) (acc + (a * bs!i))
\end{code}

\section{Constructor specialisation}

SpecConstr

\subsection{Code blowup}
\subsection{Termination}
\subsection{ForceSpecConstr}

ForceSpecConstr, termination

And not specialising ``too many times'' on recursive types

Code blowup

Seeding specialisation of non-exported functions


\end{document}
