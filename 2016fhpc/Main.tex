\documentclass[preprint]{sigplanconf}
\usepackage{amssymb}
\usepackage{amsthm}
\usepackage{graphicx}
\usepackage{amsmath}
\usepackage{mathptmx}
\usepackage{mathtools}
\usepackage{stmaryrd}
\usepackage{hyperref}
\usepackage{alltt}
\usepackage{url}
\usepackage{float}
\usepackage{style/code}
\usepackage{style/proof}
\usepackage{style/utils}
\usepackage{style/judgements}

% -----------------------------------------------------------------------------
\begin{document}

% \exclusivelicense
% \conferenceinfo{}{}
% \copyrightyear{2015}
% \copyrightdata{}
\doi{}
% \pagenumbering{gobble} 

\title{Icicle: fuse your queries}

\authorinfo{
  Amos Robinson$^\dagger$$^\ddagger$
  \and Ben Lippmeier$^\dagger$
}{
  \vspace{5pt}
  \shortstack{
    $^\dagger$Computer Science and Engineering \\
    University of New South Wales, Australia \\[2pt]
    \textsf{amosr,benl@cse.unsw.edu.au}
  }
  \shortstack{
    $^\ddagger$Ambiata      \\
    Grow with data          \\[2pt]
    \textsf{amos.robinson@ambiata.com}
  }
}

\maketitle
\makeatactive

\begin{abstract}
When streaming a large amount of data, simply iterating over the data may take hours.
If multiple queries are to be performed, it is important that work is not duplicated.
Queries that can be performed together must be performed in the same iteration.
We introduce a simple streaming language for computing queries in a single-pass over the data.
By using an appropriate intermediate language we guarantee fusion between queries on the same input streams, before extracting efficient C code.
\end{abstract}


\category
	{D.3.4}
	{Programming Languages}
	{Processors---Compilers; Optimization}

\terms
	Languages, Performance

\keywords
	Arrays; Fusion

%!TEX root = ../Main.tex
\section{Introduction}

Data flow fusion~\cite{lippmeier2013flow} is a technique to compile a specific class of data flow programs into single, efficient imperative loops. This process of ``compilation'' is equivalent to performing array fusion on a combinator based functional array program, as per related work on stream fusion~\cite{coutts2007streamfusion} and delayed arrays~\cite{keller2010repa}. The key benefits of data flow fusion over this prior work are: 1) it fuses programs that use branching data flows where a produced array is consumed by several consumers, and 2) complete fusion into a single loop is guaranteed for all programs that operate on the same size input data, and contain no fusion-preventing dependencies between operators.

Fusion-preventing dependencies express the fact that some operators simply must wait for others to complete before they can produce their own output. For example, in the following:
\begin{code}
  normalize :: Array Int -> Array Int
  normalize xs = let sum = fold (+) 0 xs
                 in  map (/ sum) xs
\end{code}

If we wish to divide every element of an array by the sum of all elements, then it seems we are forever destined to compute the result using at least two loops: one to determine the sum, and one to divide the elements. The evaluation of @fold@ demands all elements of its source array, and we cannot produce any elements of the result array until we know the value of @sum@. 

However, many programs \emph{do} contain opportunities for fusion, if we only knew which opportunities to take. The following example offers \emph{several} unique, but mutually exclusive approaches to fusion. Figure~\ref{f:normalize2-cluterings} on the next page shows some of the possibilities.
\begin{code}
 normalize2 :: Array Int -> Array Int
 normalize2 xs
  = let sum1 = fold   (+)  0   xs
        gts  = filter (> 0)    xs
        sum2 = fold   (+)  0   gts
        ys1  = map    (/ sum1) xs
        ys2  = map    (/ sum2) xs
    in (ys1, ys2)
\end{code}

In Figure~\ref{f:normalize2-cluterings}, the dotted lines show possible clusterings of operators. Stream fusion implicitly choses the solution on the left as its compilation process cannot fuse a produced array into multiple consumers. The best existing ILP approach will chose the solution on the right as it cannot cluster operators that consume arrays of different lengths. Our system choses the solution in the middle, which is also optimal for this example. 

% NOTE: This set of bullets needs to fit on the first page, without spilling to the second.
Our contributions are as follows:
\begin{itemize}
\item   
We extend prior work by Megiddo~\cite{megiddo1998optimal} and Darte~\cite{darte2002contraction}, with support for length changing operators. Length changing operators can be clustered with the operators that generate their source arrays, and compiled naturally with data-flow fusion (\S\ref{s:ILP}).

\item
We present a simplification to constraint generation that is also applicable to some existing integer linear programming formulations such as Megiddo's,
where constraints between two nodes need not be generated if there exists a fusion-preventing path between the two (\S\ref{s:OptimisedConstraints}).

\item
Our constraint system also encodes a total ordering on the cost of clusterings, expressed using weights on the integer linear program. For example, we encode that memory traffic is more expensive than loop overheads, so given a choice between the two, the memory traffic will be reduced (\S\ref{s:ObjectiveFunction}).

\item
We present benchmarks of our algorithm applied to several common programming patterns, and to several pathological examples.
Our algorithm is complete and yields good results in practice, though if array sizes are unknown, an optimal solution is uncomputable in general. \TODO{ref}
\end{itemize}

The reduction of the clustering problem to integer linear programming was previously described by~\cite{megiddo1998optimal}, though they do not consider length changing operators.


% We must also decide which clustering is the `best' or most optimal. One obvious criterion for this is the minimum number of loops, but there may even be multiple clusterings with the minimum number of loops. In this case, the number of required manifest arrays must also be taken into account. 

% As real programs contain tens or hundreds of individual operators, performing an exhaustive search for an optimal clustering is not feasible, and greedy algorithms tend to produce poor solutions. 


%!TEX root = ../Main.tex
\section{Icicle Core}
\label{s:IcicleCore}

Horizontal fusion for imperative loops is relatively easily, when the loops have the same number of iterations.
The problem with imperative loops occurs when trying to remove duplicate computations.
Firstly, the `definition' of a computation is split across multiple places: because the initial value of an accumulator and its update expression are in separate parts of the program, some analysis must be done to recover these.
For example, in the following loop, @a@ and @b@ have equivalent `update' expressions, but could denote very different computations depending on the implementation of @function@.
\begin{code}
a = 0;
b = 1;
for (...) {
  a = function(a);
  b = function(b);
}
\end{code}

Secondly, the order of statements in a loop affects the meaning: when accumulators are mutually dependent on each other, one can reorder the statements to produce a different, but still valid program.
These two loops are not equivalent:
\begin{code}
a = 0;
b = 1;
for (...) {
  a = a + b;
  b = a - b;
}
for (...) {
  b = a - b;
  a = a + b;
}
\end{code}

Neither of these problems are insurmountable, but the analysis required certainly complicates our goal of removing duplicate computation after fusion.

With this in mind, we will introduce our intermediate language based on what we call ``fold nests'', as opposed to loop nests.
Firstly, each accumulator should be defined in one place: this means that if two accumulators are alpha equivalent, they are equivalent.
Secondly, reordering statements (or accumulators) should not affect the meaning: two programs are equivalent if they contain equivalent sets of bindings, regardless of ordering.

We now convert the first example to folds:
\begin{code}
a = 0;
b = 1;
for (...) {
  a = function(a);
  b = function(b);
}

==>
loop ... {
  stage {
    fold a = 0
        then function(a);
  }
  stage {
    fold b = 1
        then function(b);
  }
}
\end{code}

Each fold binding must be grouped inside a @stage@ block, which allows mutual recursion across folds.
In this case, there is no mutual recursion, so the stages only contain one binding each.
The first fold is given the name @a@ with @fold a@.
Its initial value comes straight after the equals sign, so @a@ starts with a value of @0@.
The update or `kons' expression comes after @then@, and in the update expression any reference to @a@ refers to the current value of the fold.

It now becomes easier to see that the two folds are not equivalent, as the entire definitions are in one place.

The next example requires mutual recursion.
\begin{code}
a = 0;
b = 1;
for (...) {
  a = a + b;
  b = a - b;
}

==>
loop ... {
  stage {
    fold a = 0
        then a + b;
    fold b = 1
        then new a - b;
  }
}
\end{code}

Here, the update expression for @a@ refers to a later fold, @b@.
Update expressions can refer to later fold bindings in the same stage, as well as all earlier fold bindings.
The value referred to is the value of the binding at the start of the iteration, before any updates have occurred.
The new value of earlier folds, after being updated, can be referred to by suffixing a prime to the name: in the binding for @b@, the updated @a@ is referred to by @new a@.
This restriction of only accessing the new value of earlier folds is akin to the causality restriction in dataflow languages\cite{mandel2010lucy}.

Another way to think of the new value restriction above is disallowing cycles in the references to new values:
when @b@ references @new a@, the new value of @a@, it means that @new a@ must be computed before @new b@ can be.
If there were a cycle where @new a@ also depended upon @new b@, there would be no order to compute them in.
Disallowing cycles ensures an execution order.

By making the distinction between `old value' and `new value' explicit, the ordering in the program no longer has any effect on the semantics.
After converting the reordered loop with different semantics, the two programs are no longer alpha-equivalent:
\begin{code}
a = 0;
b = 1;
for (...) {
  b = a - b;
  a = a + b;
}

==>
loop ... {
  stage {
    fold b = 1
        then a - b;
    fold a = 0
        then a + new b;
  }
}
\end{code}

\begin{figure}

\begin{code}
N     := a,b,...
FoldN := a,b,...

Exp   := N
       | FoldN  | new FoldN
       | Exp Exp
       | \N. Exp

Bind  := fold FoldN : T = Exp then Exp;
       | let  N     : T = Exp;

Stage := stage { Bind...  }
Query := loop N : T { Stage... }

\end{code}


\caption{Grammar}
\label{fig:grammar}
\end{figure}


\newcommand\TypecheckX[3]
{                   #1
        ~\vdash~    #2~:~#3
}

\newcommand\TypecheckS[3]
{                   #1
        ~\vdash~    #2~\dashv~#3
}


\begin{figure*}

$$
\boxed{\TypecheckS{\Gamma}{\mi{Query}}{\Gamma'}}
$$

$$
\ruleAx
{
  \TypecheckS{\Gamma}{@loop@~x~:~\tau~\{\}}{\emptyset}
}{QueryNil}
\ruleIN
{
  \TypecheckS{\Gamma,~x~:~\tau}{s}{\Gamma'}
  \quad
  \TypecheckS{\Gamma,~\Gamma'}{@loop@~x~:~\tau~\{s'\}}{\Gamma''}
}{
  \TypecheckS{\Gamma}{@loop@~x~:~\tau~\{s;~s'\}}{\Gamma' \cup \Gamma''}
}{QueryCons}
$$



$$
\boxed{\TypecheckS{\Gamma}{\mi{Stage}}{\Gamma'}}
$$

$$
\ruleAx
{
  \TypecheckS{\Gamma}{@stage@~\{\}}{\emptyset}
}{StageNil}
\ruleIN
{
  \TypecheckS{\Gamma,~x~:~\tau}{@stage@~\{s'\}}{\Gamma'}
  \quad
  \TypecheckX{\Gamma}{x}{\tau}
}
{
  \TypecheckS{\Gamma}{@stage@~\{@let@~x~:~\tau~=~e;~s'\}}{\Gamma',~x~:~\tau}
}{StageLet}
$$

$$
\begin{array}{l}

\ruleIN
{
  \TypecheckS{\Gamma,~x~:~\tau,~@new@~x~:~\tau}{@stage@~\{s'\}}{\Gamma'}
  
  \quad
  \TypecheckX{\Gamma}{z}{\tau}
  \quad
  \TypecheckX{\Gamma~\cup~\mi{folds}}{k}{\tau}
}
{
  \TypecheckS{\Gamma}{@stage@~\{@fold@~x~:~\tau~=~z~@then@~k;~s'\}}{\Gamma',~x~:~\tau,~@new@~x~:~\tau}
}{StageFold}

  \\
  \mbox{where} \\
  \quad
  \mi{folds}~=~\{v~:~\tau'~|~@new@~v~:~\tau'~\in~\Gamma'\}
  \end{array}
$$


\caption{Typing rules}
\label{fig:typing}
\end{figure*}


The full grammar is in figure~\ref{fig:grammar}, however primitives and types have been omitted.
It is assumed that there are some primitives such as @1@, @+@ and types such as @Int@, but the elements do not actually matter.

The typing rules are in figure~\ref{fig:typing}.
The rules shown only cover the staging and folds.
As with the grammar, we assume there is a typing rule $\TypecheckX{\Gamma}{\mi{Exp}}{\tau}$ which checks an expression under the given environment.

The first judgment, $\TypecheckS{\Gamma}{\mi{Query}}{\Gamma'}$ 



\section*{Acknowledgements}

\bibliographystyle{plain}
\bibliography{Main}

\end{document}


