%!TEX root = ../Main.tex
\section{Intermediate language}
\label{s:IcicleCore}

\begin{figure}

\begin{code}
Exp   := x | Prim Exp*

Plan  := plan   x { (x : T;)* }
       before { x : T = Exp;* }
       folds  { x : T = Exp then Exp;* }
       after  { x : T = Exp;* }
       return { x : T = x;* }
\end{code}


\caption{Grammar}
\label{fig:core:grammar}
\end{figure}

The intermediate language can be thought of as similar to a query plan in other databases.
Queries in the source language are converted to query plans, and then query plans on the same table are fused together.
Common subexpression elimination, partial evaluation and other optimisations are performed at the query plan level.

The grammar for the intermediate language is given in figure~\ref{fig:core:grammar}.
Expressions can be variables or primitive applications: function applications are not allowed in expressions here, as their definitions are inlined before converting to query plans.
The @Plan@ itself is split into stages of computation, which correspond to the modal stages in the Source query.
The first stage, @before@, is used for pure computations which do not depend on any stream values at all.
The next stage, @folds@, defines the @Element@ computations and how they are converted into @Aggregate@ computations.
Next are @after@ which correspond to @Aggregate@ computations, occurring after the entire stream has been seen.
Finally, the @return@ specifies the output values of the query; a single query will have only one output value, but the result of fusion can have many outputs.

One of the simplest queries to convert to a query plan is a count.
The query plan header specifies the table and the type of each column in the table.
This defines a single fold called @count@ which starts at zero, and is incremented for each element.
At the end of the query, the final count is assigned to the output @count_query@, which is the query's name.
\begin{code}
table stocks { open : Int; close : Int; }
query count_query = count

==>

plan stocks { open : Int; close : Int; }
before { }
folds  { count : Int = 0 then count + 1 }
after  { }
return { count_query : Int = count }
\end{code}

Computing the sum is only slightly more complicated.
Here there is again a single fold, @sum@, starting at zero and being incremented by the current column's @open@.
\begin{code}
query sum_open = sum open


==>

plan stocks { open : Int; close : Int; }
before { }
folds  { sum : Int = 0 then sum + open }
after  { }
return { sum_open : Int = sum }
\end{code}

In order to compute the mean, we compute the sum and the count, and after all computation has finished, divide the two.
Here we use two folds, one for the sum, and one for the count.
While we could have used a single fold containing a pair of the two, keeping the folds separate exposes more opportunities for common subexpression elimination.
\begin{code}
query mean_open = sum open / count

==>

plan stocks { open : Int; close : Int; }
before { }
folds  { sum       : Int = 0 then sum + open 
         count     : Int = 0 then count + 1 }
after  { mean      : Int = sum / count }
return { mean_open : Int = mean }
\end{code}

If we now wish to fuse these query plans together, it is as simple as giving the bindings unique names, and appending each part of the plan together.
The names of the returns and the column names stay the same.
Query plans can only be fused if they act on the same table.
\begin{code}
plan stocks { open : Int; close : Int; }
before { }
folds  { c1 : Int = 0 then c1 + 1
         c2 : Int = 0 then c2 + 1
         s1 : Int = 0 then s1 + open 
         s2 : Int = 0 then s2 + open }
after  { m1 : Int = s2 / c2 }
return { count_query : Int = c1
         sum_open    : Int = s1
         mean_open   : Int = m1 }
\end{code}

The above query plan computes all three queries at once, but has some duplicate work: the counts and the sums are performed twice.
We can perform fairly standard common subexpression elimination to remove these.
\begin{code}
plan stocks { open : Int; close : Int; }
before { }
folds  { c1 : Int = 0 then c1 + 1
         s1 : Int = 0 then s1 + open }
after  { m1 : Int = s1 / c1 }
return { count_query : Int = c1
         sum_open    : Int = s1
         mean_open   : Int = m1 }
\end{code}

The interesting thing here is just how simple fusion is.
Thanks to the single-pass restriction, we know there are no dependencies between any two query plans: if there were, the source query would have required multiple passes.
The single-pass restriction means that any source query can be expressed as a single fold over the data.
By keeping the folds separate we have exposed more opportunities for common subexpression elimination.

An efficient imperative loop can be generated from this query plan rather easily, in a similar way as previous work on flow fusion\cite{lippmeier2013data}.

