%!TEX root = ../Main.tex
\section{Introduction}
\label{s:Introduction}


For queries over a large amount of data, care must be taken to reduce the number of iterations over the input data.
When performing multiple queries, it is important to limit the amount of duplicate work and iterations.

For example, suppose we have a table called @stocks@ with open and close prices for each company, for each day.
We may wish to find the number of days where the open price exceeds the close price, and vice versa:
\begin{code}
SELECT COUNT(*)
FROM   stocks
WHERE  open > close;

SELECT COUNT(*)
FROM   stocks
WHERE  open < close;
\end{code}

In order to fuse these so that both are computed in a single query, some ingenuity is required.
This is because the @WHERE@ clause filters over the entire query, but when putting both in one query we need two different filters.
\begin{code}
SELECT COUNT(IF(open > close, 1, 0))
     , COUNT(IF(open < close, 1, 0))
FROM stocks;
\end{code}

If we also wish to compute the mean of open price on those days, we can do this like so:
\begin{code}
SELECT SUM(open) / COUNT(*)
FROM   stocks
WHERE  open > close;
\end{code}

Now we can perform a similar trick to fuse all three together.
\begin{code}
SELECT COUNT(IF(open > close, 1, 0))
     , COUNT(IF(open < close, 1, 0))
     , SUM  (IF(open > close, open, 0))
     / COUNT(IF(open > close, 1, 0))
FROM stocks;
\end{code}

However, there are some problems with this approach.
This fusion on SQL source queries is manual and error-prone.
More importantly, we have no guarantees about which queries can be fused.
We would like a guarantee that \emph{any} two queries can be fused this way.

Another problem is that we have no guarantee that duplicate computations introduced by fusion will be eliminated.
Above we are computing the count where open exceeds close twice: once in the count, and again for the mean.

The purpose of Icicle is two-fold: we wish to ensure that all queries over the same table can be fused together, and guarantee that such duplicate computations will be removed.
This is achieved by having a very restricted query language with no sorting, and allowing only a single-pass over the data.

In Icicle, the above queries are expressed as:
\begin{code}
table stocks { open : Int; close : Int; }
query more = filter open > close in count;
query less = filter open < close in count;
query mean = filter open > close in sum open / count;
\end{code}

These queries are then fused together, common subexpression elimination is performed, and efficient C code is generated.

The contributions of this paper are:
\begin{itemize}
\item
we show how Icicle guarantees any two queries on the same table can be fused together, as well as allowing queries to be resumed as new data is received\sref{s:IcicleSource};

\item
we show an intermediate fold-based language, which makes fusion a simple matter of appending two programs\sref{s:IcicleCore};

\item
and we present benchmarks of our generated code, which compares favourably against standard unix utilities and outperforms our existing system by orders of magnitude\sref{s:Benchmarks}.
\end{itemize}
