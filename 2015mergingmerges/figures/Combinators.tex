%!TEX root = ../Main.tex


\begin{figure}
\centering
\Large
\begin{dot2tex}[scale=0.5]
digraph {
    rankdir=LR;
    mindist=0.1
    nodesep=0.1;
    ranksep=0.1;
    node [shape="circle", margin=0];
    10 [label="Pull xs"];
    20 [label="Out o (f x)"];
    30 [label="Release x"];

    90 [label="OutDone o"];
    91 [label="Done"];

    10 -> 20 [label="Some x"];
    10 -> 90 [label="None"];

    90 -> 91;

    20 -> 30;
    30 -> 10;
}
\end{dot2tex}
\caption{$o =$~map~$f$~$xs$}
\label{fig:com:map}
\end{figure}

\begin{figure}
\centering
\Large
\begin{dot2tex}[scale=0.5]
digraph {
    rankdir=LR;
    mindist=0.1
    nodesep=0.1;
    ranksep=0.1;
    node [shape="circle", margin=0];
    10 [label="Pull xs"];
    15 [label="If (p x)"];
    20 [label="Out o x"];
    30 [label="Release x"];

    90 [label="OutDone o"];
    91 [label="Done"];

    10 -> 15 [label="Some x"];
    15 -> 20 [label="True"];
    15 -> 30 [label="False"];
    10 -> 90 [label="None"];

    90 -> 91;

    20 -> 30;
    30 -> 10;
}
\end{dot2tex}
\caption{$o =$~filter~$p$~$xs$}
\label{fig:com:filter}
\end{figure}

\begin{figure}
\centering
\Large
\begin{dot2tex}[scale=0.5]
digraph {
    rankdir=LR;
    mindist=0.1
    nodesep=0.1;
    ranksep=0.1;
    node [shape="circle", margin=0];
    10 [label="Pull xs"];
    11 [label="Pull ys"];
    20 [label="Out o (x,y)"];
    30 [label="Release x"];
    31 [label="Release y"];

    80 [label="Close ys"];
    88 [label="Close xs"];
    89 [label="Release x"];

    90 [label="OutDone o"];
    91 [label="Done"];

    10 -> 11 [label="Some x"];
    11 -> 20 [label="Some y"];
    10 -> 80 [label="None"];
    80 -> 90;
    11 -> 88 [label="None"];
    88 -> 89;

    89 -> 90;

    90 -> 91;

    20 -> 30;
    30 -> 31;
    31 -> 10;
}
\end{dot2tex}
\caption{$o =$~zip~$xs$~$ys$}
\label{fig:com:zip}
\end{figure}

\begin{figure*}
\centering
\Large
\begin{dot2tex}[scale=0.45]
digraph {
    rankdir=LR;
    mindist=0.0
    nodesep=0.0;
    ranksep=0.4;
    node [shape="circle", margin=0];

    10 [label="Pull xs"];
    20 [label="Pull ys"];
    30 [label="If (x le y)"];

    40 [label="Out o x"];
    41 [label="Release x"];
    42 [label="Pull xs"];

    50 [label="Out o y"];
    51 [label="Release y"];
    52 [label="Pull ys"];

    100 [label="Pull ys"];
    101 [label="Out o y"];
    102 [label="Release y"];

    200 [label="Pull xs"];
    201 [label="Out o x"];
    202 [label="Release x"];

    900 [label="OutDone o"];
    901 [label="Done"];

    10 -> 20 [label="Some x"];
    10 -> 100 [label="None"];
    20 -> 30 [label="Some y"];
    20 -> 201 [label="None"];

    30 -> 40 [label="True"];
    30 -> 50 [label="False"];

    40 -> 41;
    41 -> 42;

    42 -> 30 [label="Some x"];
    42 -> 101 [label="None"];

    50 -> 51;
    51 -> 52;
    52 -> 30 [label="Some y"];
    52 -> 201 [label="None"];

    100 -> 101 [label="Some y"];
    100 -> 900 [label="None"];
    101 -> 102;
    102 -> 100;

    200 -> 201 [label="Some x"];
    200 -> 900 [label="None"];

    201 -> 202;
    202 -> 200;

    900 -> 901;
}
\end{dot2tex}
\caption{$o =$~merge~$xs$~$ys$}
\label{fig:com:merge}
\end{figure*}


\begin{figure}
{ \centering
\Large
\begin{dot2tex}[scale=0.35]
digraph {
    rankdir=LR;
    mindist=0.1
    nodesep=0.1;
    ranksep=0.1;
    node [shape="circle", margin=0];
    0 [label="Pull xs (init)"];
    1 [label="Update o_s = xs"];
    2 [label="Release xs"];
    3 [label="Pull xs"];

    4 [label="If eqfst"];

    10 [label="Update o_s = update"];

    20 [label="Out o_s"];

    30 [label="Out o_s"];

    40 [label="OutDone o"];

    50 [label="Done"];

    0 -> 1 [label="Some"];
    0 -> 40 [label="None"];
    2 -> 3;
    3 -> 4 [label="Some"];
    3 -> 30 [label="None"];

    4 -> 10 [label="True"];
    4 -> 20 [label="False"];

    10 -> 2;

    20 -> 1;

    30 -> 40;
    40 -> 50;
}
\end{dot2tex}

}

Note that @o_s@ denotes the state variable for output channel @o@, while the following functions check for equality of keys, and update their associated values respectively.

\begin{code}
eqfst  =  fst o_s == fst xs
update = (fst o_s, f (snd o_s) (snd xs))
\end{code}
\caption{$o =$~groupby~$f~xs$}
\label{fig:com:groupby}
\end{figure}

