%!TEX root = ../Main.tex
\section{Merging machines}
\label{s:Merging}

This section describes how machines can be merged together to form a single machine that computes both.

The idea is to create a new machine with the product of the two states; at each state, one or the other of the machines may be able to take a step.
If so, we update that machine's state and leave the other one as-is.
For shared inputs or outputs, both machines must take the steps at the same time: if they both pull from the same input, they must pull at the same time.
Similarly, if one machine produces an input and the other reads it, the first machine can only produce its output if the second is pulling on it.
If neither machine can make a move, the program is disallowed.
For example, if the first machine is trying to read from a shared input while the second tries to read from the first machine's input, executing it may require buffering the input until the second machine catches up.

However, the above explanation is too strict and outlaws programs such as @zip xs ys@ merged with @zip ys xs@, as neither machine can progress until the other does, leading to deadlock.
To resolve this, we add to the resulting machine's state a set of pending events for each machine.
The first machine may read from a shared input as long as there is not already a ``read'' event pending for the other machine to deal with.
Reading from a shared input adds a pending ``read'' event to the other machine, and the other machine may then skip past a @Pull@ for that source, as it is known that the other machine has already pulled on it.

\subsection{General merging}
Talk about merging with
\begin{code}
data Event
 = Read n
 | Finished n

merge :: Machine l1 -> Machine l2
      -> Machine (l1, l2, Set Event)
\end{code}

\subsection{Producer-consumer}
The general case above is sufficient, but in some cases can generate larger machines than necessary.
Talk about special case for producer-consumers.

\begin{code}
data Which = LeftExec | RightExec

mergeV :: Machine l1 -> Machine l2
       -> Machine (l1, l2, Which)
\end{code}

