%!TEX root = ../Main.tex
\section{Access patterns}
\label{s:Access}

The key insight is to treat combinators' access patterns as regular expressions over the names of streams.
When an input stream is mentioned in a regular expression, it signifies that a blocking read must be made from that stream before the combinator can proceed.
When output, written @E@ for emit, is mentioned, a blocking write is made to the output.
We associate these regular expression access patterns with a combinator, process or network, such that for any given set of inputs, there exists an execution trace that satisfies the regular expression:
\[ \forall g : @Graph@.~ \forall is : @Inputs@.~ \exists t : @Trace@~g~is.~
    \flatten(t) \in \regex(g) \]

Where $\flatten : @Trace@~g~i \to [@Index@~n]$ converts an execution trace of a graph to the list of indices as they fired, and $\regex : @Graph@ \to @Regex@~(@Index@~n)$ retrieves the access pattern of a given graph.

For example, @map a@ reads from @a@ at every step, then emits a value @E@, and repeats until no more data remains in @a@.
We define the operator @:~@ as ``has access pattern''.
\begin{code}
map     a   :~ (aE)*
filter  a   :~ (aE?)*
zip     a b :~ (abE)*
append  a b :~ (aE)*(bE)*
merjoin a b :~ ((aE)*(bE)*)*
\end{code}
The access pattern of the merge join (@merjoin@) also illustrates its relationship with append and segmented append.
These access patterns be composed together to apply to whole graphs:

\begin{code}
append (map a)  (filter b) :~ (aE)*(bE?)*
append (zip a b) (zip a b) :~ (abE)*(abE)*
\end{code}
\TODO{but I don't have a good definition for this yet}

\subsection{Traces}
\TODO{Define what a trace is}.

We define filter functions over traces and regular expressions, which, given two graphs and the trace or acess pattern of one, filters it to the intersection of the names.
\[ \filterT(g', t) = \filter (\in flatten(g')) t \]
\[ \filterR(g', r) = \filter (\in flatten(g')) r \]
It also holds that
\[ \forall rx. \forall s. \forall w. w \in rx \implies \filter(\in s) w \in \filter(\in s) rx\]

We say that two traces are \emph{affable} if they agree on common parts:
\[ \affable(t : @Trace@~g~is, t' : @Trace@~g'~is) = \filterT(g', t) == \filterT(g, t') \]

For example, any traces of @map a@ and @zip a b@ would be affable, since after filtering out the @b@s both traces would be equal.

Two graphs are fusible if, for all possible inputs, there exist affable traces:
\begin{align*}
\mathit{fusible}(g, g')
        &=  \forall is : @Inputs@~(g \cap g').  \\
        &   \exists t  : @Trace@~g~is.          \\
        &   \exists t' : @Trace@~g'~is.         \\
        &   \affable(t, t')                     %
\end{align*}


Indeed, if we have two graphs, $\forall g~g' : @Graph@.$
\begin{eqnarray*}
 & \filterR(\regex(g)) = \filterR(\regex(g')) \\
 \implies & \forall w. w \in \filterR(\regex(g)) \iff w \in \filterR(\regex(g')) \\
 \implies & \forall is. \exists t : @Trace@~g~is. \exists t' : @Trace@~g'~is. \\
          & \filterT(\flatten(t)) \in \filterR(\regex(g)) \\
          & \wedge~\filterT(\flatten(t')) \in \filterR(\regex(g')) \\
\end{eqnarray*}

But it is not necessarily true that these two traces $t$ and $t'$ have the same filtered flattening.
I \emph{think} it is true, because they have the same inputs, but I need to think about how to prove it.
Something about determinism.
\begin{eqnarray*}
          & \wedge~\filterT t = \filterT t' \\
 \implies & \affable(t, t') \\
 \implies & \mathit{fusible}(g, g') \\
\end{eqnarray*}

We can split the combinators into two types: value dependent and independent.
Filter, merge join and indices are value dependent, whereas map, append, zip, and so on are value independent.
Value independent combinators have the property that for all inputs, the trace can be determined solely by the length of the inputs, regardless of the values.



\subsubsection{Examples}
Going back to the original @app2@ example, let us compute its access pattern.
I use a simplified, canonical form here and discard the worker functions.

\begin{code}
app2 :: [(Bool,a)] -> [(Bool,a)]
app2 X
 = let F = filter X     :~ (XF?)*
       G = filter X     :~ (XG?)*
       H = append F G   :~ (XF?)*(XG?)*
   in  H
\end{code}
If we try to fuse @F@ and @G@, we start by finding the common parts, which is just @X@. Filtering both access patterns to only occurrences of @X@ leaves @X*@ and @X*@, which are trivially equal.
Therefore, @F@ and @G@ can be fused.

However, if we attempt to fuse @F@ with @H@, we must again find the intersection: this being @X@ and @F@.
We find that @F@'s filtered access pattern remains the same, @(XF?)*@, while @H@'s filtered access pattern is @(XF?)*X*@. These two regular expressions are not equal, and thus @F@ and @H@ may not be fused.

