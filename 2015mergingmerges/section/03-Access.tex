%!TEX root = ../Main.tex
\section{Access patterns}
\label{s:Access}

The key insight is to treat combinators' access patterns as regular expressions over the names of streams.
When an input stream is mentioned in a regular expression, it signifies that a blocking read must be made from that stream before the combinator can proceed.
When output, written @E@ for emit, is mentioned, a blocking write is made to the output.
We associate these regular expression access patterns with a combinator, process or network, such that for any given set of inputs, there exists an execution trace that satisfies the regular expression:
\[ \forall g : @Graph@.~ \forall is : @Inputs@.~ \exists t : @Trace@~g~is.~
    \flatten(t) \in \regex(g) \]

Where $\flatten : @Trace@~g~i \to [@Index@~n]$ converts an execution trace of a graph to the list of indices as they fired, and $\regex : @Graph@ \to @Regex@~(@Index@~n)$ retrieves the access pattern of a given graph.

For example, @map a@ reads from @a@ at every step, then emits a value @E@, and repeats until no more data remains in @a@.
We define the operator @:~@ as ``has access pattern''.
\begin{code}
map     a   :~ (aE)*
filter  a   :~ (aE?)*
zip     a b :~ (abE)*
append  a b :~ (aE)*(bE)*
merjoin a b :~ ((aE)*(bE)*)*
\end{code}
The access pattern of the merge join (@merjoin@) also illustrates its relationship with append and segmented append.
These access patterns be composed together to apply to whole graphs:

\begin{code}
append (map a) (filter b) :~ (aE)*(bE?)*
append (zip a b) (zip a b) :~ (abE)*(abE)*
\end{code}
\TODO{but I don't have a good definition for this yet}

\subsection{Traces}
\TODO{Define what a trace is}.

We define filter functions over traces and regular expressions, which, given two graphs and the trace or acess pattern of one, filters it to the intersection of the names.
\[ \filterT(g', t) = \filter (\in flatten(g')) t \]
\[ \filterR(g', r) = \filter (\in flatten(g')) r \]
It also holds that
\[ \forall rx. \forall s. \forall w. w \in rx \implies \filter(\in s) w \in \filter(\in s) rx\]

We say that two traces are \emph{affable} if they agree on common parts:
\[ \affable(t : @Trace@~g~is, t' : @Trace@~g'~is) = \filterT(g', t) == \filterT(g, t') \]

For example, any traces of @map a@ and @zip a b@ would be affable, since after filtering out the @b@s both traces would be equal.

Two graphs are fusible if, for all possible inputs, there exist affable traces:
\begin{align*}
\mathit{fusible}(g, g')
        &=  \forall is : @Inputs@~(g \cap g').  \\
        &   \exists t  : @Trace@~g~is.          \\
        &   \exists t' : @Trace@~g'~is.         \\
        &   \affable(t, t')                     %
\end{align*}


Indeed, if we have two graphs, $\forall g~g' : @Graph@.$
\begin{eqnarray*}
 & \filterR(\regex(g)) = \filterR(\regex(g')) \\
 \implies & \forall w. w \in \filterR(\regex(g)) \iff w \in \filterR(\regex(g')) \\
 \implies & \forall is. \exists t : @Trace@~g~is. \exists t' : @Trace@~g'~is. \\
          & \filterT(\flatten(t)) \in \filterR(\regex(g)) \\
          & \wedge~\filterT(\flatten(t')) \in \filterR(\regex(g')) \\
\end{eqnarray*}
but it is not necessarily true that these two traces $t$ and $t'$ have the same filered flattening...
Not true:
\begin{eqnarray*}
          & \wedge~\filterT t = \filterT t' \\
 \implies & \affable(t, t') \\
 \implies & \mathit{fusible}(g, g') \\
\end{eqnarray*}

