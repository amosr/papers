%!TEX root = ../Main.tex
\section{Machines}
\label{s:Machines}

We use a particular type of deterministic finite automata (DFAs) where each state has a type that restricts the alphabet of output transitions.
We use state types such as @Pull a@ with output transitions @Some a@ and @None@, or type @Out b@ with single output transition @Unit@.

\subsection{Restricted DFA}

We describe a restricted automaton as a eight-tuple, extending the standard description by three elements: $(Q, \Sigma, \delta, q_0, F, \Gamma, \gamma, \sigma)$.

As usual, $Q$ is the set of states, $\Sigma$ the alphabet, but in addition $\Gamma$ is a set of state types.
We define the types of functions as follows:

\[ \gamma : Q \to \Gamma \]
Every state has a type.

\[ \sigma : \Gamma \to \{ \Sigma \} \]
Every type has some subset of the alphabet that transitions are required for.

\[ \delta : \forall q : Q. \sigma(\gamma(q)) \to Q \]
For each state, the transition function need only be defined for those letters of the state's type.

These restricted DFAs can be converted to normal automata by adding a rejecting sink state, and adding transitions for all missing letters of the alphabet to this rejecting state.

\subsection{State types}
We now describe a set of types and alphabet for describing combinator programs as a restricted DFA.
These are parameterised by both the names of sources, sinks and bindings $n$, and the type of worker functions $f$ (such as @Expression@ for code generation).

\begin{tabbing}
MM \= MM \= MMMMMM \= M \= M \kill
$\Gamma$ \> $=$ \> @Pull@     \>     \> $n$         \\
         \>     \> (Attempt to read from a named input) \\

         \> $~|$\> @Release@  \>     \> $n$         \\
         \>     \> (Read inputs must be released) \\

         \> $~|$\> @Out@      \> $f$ \> $n$         \\
         \>     \> (Emit the value computed by $f$ to output channel $n$) \\

         \> $~|$\> @OutDone@  \>     \> $n$         \\
         \>     \> (Signal that output channel $n$ is complete) \\

         \> $~|$\> @If@       \> $f$ \> $n$         \\
         \>     \> (If based on current state of output channel $n$) \\

         \> $~|$\> @Update@   \> $f$ \> $n$         \\
         \>     \> (Update output channel $n$'s current state) \\

         \> $~|$\> @Skip@     \>                    \\
         \>     \> (Simple goto, useful for code generation) \\

         \> $~|$\> @Done@     \>                    \\
         \>     \> (The program is finished) \\
\end{tabbing}

Each of these state types requires a corresponding set of output transitions to be defined.
\begin{tabbing}
MMMMMMMM \= M \= \kill
$\gamma(@Pull @n)$
                            \> $=$ \> \{@Some@ n, @None@\} \\
$\gamma(@Release @n)$
                            \> $=$ \> \{@Unit@\} \\
$\gamma(@Out @   f~n)$
                            \> $=$ \> \{@Unit@\} \\
$\gamma(@OutDone @ n)$
                            \> $=$ \> \{@Unit@\} \\
$\gamma(@If @    f~n)$
                            \> $=$ \> \{@True@, @False@\} \\
$\gamma(@Update @f~n)$
                            \> $=$ \> \{@Unit@\} \\
$\gamma(@Skip@      )$
                            \> $=$ \> \{@Unit@\} \\
$\gamma(@Done@      )$
                            \> $=$ \> \{\} \\
\end{tabbing}


\subsection{Conversion of combinators}
The combinators from section~\ref{s:Combinators} can be converted to these DFAs quite easily.

Well, do it.
I'm still not sure, maybe just showing the pictures would be better.
