%!TEX root = ../Main.tex
\section{Machines}
\label{s:Machines}

We use a particular type of deterministic finite automata (DFAs) where each state has a type that restricts the alphabet of output transitions.
We use state types such as @Pull a@ with output transitions @Some a@ and @None@, or type @Out b@ with single output transition @Unit@.

\subsection{Restricted DFA}

We describe a restricted automaton as a eight-tuple, extending the standard description by three elements: $(Q, \Sigma, \delta, q_0, F, \Gamma, \gamma, \sigma)$.

As usual, $Q$ is the set of states, $\Sigma$ the alphabet, but in addition $\Gamma$ is a set of state types.
We define the types of functions as follows:

\[ \gamma : Q \to \Gamma \]
Every state has a type.

\[ \sigma : \Gamma \to \{ \Sigma \} \]
Every type has some subset of the alphabet that transitions are required for.

\[ \delta : \forall q : Q. \sigma(\gamma(q)) \to Q \]
For each state, the transition function need only be defined for those letters of the state's type.

These restricted DFAs can be converted to normal automata by adding a rejecting sink state, and adding transitions for all missing letters of the alphabet to this rejecting state.

\subsection{State types}
We now describe a set of types and alphabet for describing combinator programs as a restricted DFA.
These are parameterised by both the names of sources, sinks and bindings $n$, and the type of worker functions $f$ (such as @Expression@ for code generation).

\begin{tabbing}
MM \= MM \= MMMMMM \= M \= M \kill
$\Gamma$ \> $=$ \> @Pull@     \>     \> $n$         \\
         \>     \> (Attempt to read from a named input) \\

         \> $~|$\> @Release@  \>     \> $n$         \\
         \>     \> (Values must be released before the input can be pulled again) \\

         \> $~|$\> @Close@    \>     \> $n$         \\
         \>     \> (A named input may still have data, but we will not read it) \\

         \> $~|$\> @Out@      \> $f$ \> $n$         \\
         \>     \> (Emit the value computed by $f$ to output channel $n$) \\

         \> $~|$\> @OutDone@  \>     \> $n$         \\
         \>     \> (Signal that output channel $n$ is complete) \\

         \> $~|$\> @If@       \> $f$ \> $n$         \\
         \>     \> (If based on current state of output channel $n$) \\

         \> $~|$\> @Update@   \> $f$ \> $n$         \\
         \>     \> (Update output channel $n$'s current state) \\

         \> $~|$\> @Skip@     \>                    \\
         \>     \> (Simple goto, simplifies merge algorithm) \\

         \> $~|$\> @Done@     \>                    \\
         \>     \> (The program is finished) \\
\end{tabbing}

Each of these state types requires a corresponding set of output transitions to be defined.
\begin{tabbing}
MMMMMMMM \= M \= \kill
$\gamma(@Pull @n)$
                            \> $=$ \> \{@Some@ n, @None@\} \\
$\gamma(@Release @n)$
                            \> $=$ \> \{@Unit@\} \\
$\gamma(@Close @n)$
                            \> $=$ \> \{@Unit@\} \\
$\gamma(@Out @   f~n)$
                            \> $=$ \> \{@Unit@\} \\
$\gamma(@OutDone @ n)$
                            \> $=$ \> \{@Unit@\} \\
$\gamma(@If @    f~n)$
                            \> $=$ \> \{@True@, @False@\} \\
$\gamma(@Update @f~n)$
                            \> $=$ \> \{@Unit@\} \\
$\gamma(@Skip@      )$
                            \> $=$ \> \{@Unit@\} \\
$\gamma(@Done@      )$
                            \> $=$ \> \{\} \\
\end{tabbing}

Note that @Out@, @OutDone@, @If@ and @Update@ are also annotated with the name of the output channel.
While a single combinator will only have one output, after merging multiple machines together there could be multiple outputs.
Each combinator has its own local mutable state, accessible by the functions given to @Update@, @Out@ and @If@.
These machine state types must be annotated with the name of the output channel to designate which local mutable state, if any.

In the proceeding parts, any reference to @Machine@ is a restricted DFA with these state types.

\subsection{Invariants}
Not all machines are valid or meaningful, and we wish to rule them out.
For example, a machine with an initial state of $@If@~(x>5)~n$ is meaningless because there is no local $x$ that has been read from an input. 
On the other hand, some programs could be meaningful, but requiring them to be in a particular form simplifies fusion, later.
An example of such is a program that does not release its input before pulling again; a sympathetic code extraction would be to simply release before every pull, but the explicit pulls makes synchronising two machines easier.

The invariants are:
\begin{itemize}
\item each function mentioned in @Out@, @If@ or @Update@ can only refer to previously read values;
\item each successful @Pull@ must be @Released@ before another @Pull@ can be made;
\item all inputs must either be finished by an empty @Pull@ or @Close@d at @Done@;
\item all outputs must be finished with @OutDone@ at @Done@;
\item inputs cannot be read after an empty @Pull@ or after they are @Close@d;
\item outputs cannot be written after they are finished with @OutDone@.
\end{itemize}

\begin{code}
check_machine m
 = check_state (_init m) (M.singleton (_init m) S.empty)
 where
  check_state l acc
   | Just t <- M.lookup l (_trans m)
   , Just s <- M.lookup l acc
   = case t of
      Pull n l1 l2
       | not $ Value    n `S.member` s
       , not $ Finished n `S.member` s
       ->  go l1 (S.insert (Value n) s) acc
       >>= go l2 (S.insert (Finished n) s)

      Release n l'
       | Value n `S.member` s
       -> go l' (S.delete (Value n) s)
          acc

      Close   n l'
       | not $ Value n `S.member` s
       -> go l' (S.insert (Finished n) s)
          acc

      Update f l'
       | all (`S.member` s) (fvs f)
       -> go l' s acc

      If f l1 l2
       | all (`S.member` s) (fvs f)
       -> go l1 s acc >>= go l2 s

      Out f l'
       | all (`S.member` s) (fvs f)
       -> go l' (S.insert (Value (_state f)) s) acc

      OutFinished n l'
       | not $ Finished n `S.member` s
       -> go l' (S.insert (Finished n) s) acc
       | otherwise

      Skip l'
       -> go l' s acc

      Done
       | all (\n -> Finished n `S.member` s) 
       $ S.toList
       $ S.union inputs outputs
       -> Right acc

  go l s m
   | Just s' <- M.lookup l m
   = let sI  = S.intersection s s'
         sD  = S.difference   s s'
     -- Allow 
     in  if         s == s'
         then       Right m
         else if    all is_output (S.toList sD)
         then       check_state l (M.insert l sI m)
         else       Error
   | otherwise
   = check_state l (M.insert l s m)
\end{code}

\subsection{Combinators}
Many interesting combinators can be described using these machines.
In figures 1-4, the machines for some important combinators are given.
To reduce clutter, these figures omit labels for @Unit@ transitions.

While writing combinators in this form is neither pleasant nor pretty, it only needs to be done once.
The implementation contains more combinators such as group by, append and indices of segment lengths.



%!TEX root = ../Main.tex


\begin{figure}
\centering
\Large
\begin{dot2tex}[scale=0.5]
digraph {
    rankdir=LR;
    mindist=0.1
    nodesep=0.1;
    ranksep=0.1;
    node [shape="circle", margin=0];
    10 [label="Pull xs"];
    20 [label="Out o (f x)"];
    30 [label="Release x"];

    90 [label="OutDone o"];
    91 [label="Done"];

    10 -> 20 [label="Some x"];
    10 -> 90 [label="None"];

    90 -> 91;

    20 -> 30;
    30 -> 10;
}
\end{dot2tex}
\caption{$o =$~map~$f$~$xs$}
\label{fig:com:map}
\end{figure}

\begin{figure}
\centering
\Large
\begin{dot2tex}[scale=0.5]
digraph {
    rankdir=LR;
    mindist=0.1
    nodesep=0.1;
    ranksep=0.1;
    node [shape="circle", margin=0];
    10 [label="Pull xs"];
    15 [label="If (p x)"];
    20 [label="Out o x"];
    30 [label="Release x"];

    90 [label="OutDone o"];
    91 [label="Done"];

    10 -> 15 [label="Some x"];
    15 -> 20 [label="True"];
    15 -> 30 [label="False"];
    10 -> 90 [label="None"];

    90 -> 91;

    20 -> 30;
    30 -> 10;
}
\end{dot2tex}
\caption{$o =$~filter~$p$~$xs$}
\label{fig:com:filter}
\end{figure}

\begin{figure}
\centering
\Large
\begin{dot2tex}[scale=0.5]
digraph {
    rankdir=LR;
    mindist=0.1
    nodesep=0.1;
    ranksep=0.1;
    node [shape="circle", margin=0];
    10 [label="Pull xs"];
    11 [label="Pull ys"];
    20 [label="Out o (x,y)"];
    30 [label="Release x"];
    31 [label="Release y"];

    80 [label="Close ys"];
    88 [label="Close xs"];
    89 [label="Release x"];

    90 [label="OutDone o"];
    91 [label="Done"];

    10 -> 11 [label="Some x"];
    11 -> 20 [label="Some y"];
    10 -> 80 [label="None"];
    80 -> 90;
    11 -> 88 [label="None"];
    88 -> 89;

    89 -> 90;

    90 -> 91;

    20 -> 30;
    30 -> 31;
    31 -> 10;
}
\end{dot2tex}
\caption{$o =$~zip~$xs$~$ys$}
\label{fig:com:zip}
\end{figure}

\begin{figure*}
\centering
\Large
\begin{dot2tex}[scale=0.45]
digraph {
    rankdir=LR;
    mindist=0.0
    nodesep=0.0;
    ranksep=0.4;
    node [shape="circle", margin=0];

    10 [label="Pull xs"];
    20 [label="Pull ys"];
    30 [label="If (x le y)"];

    40 [label="Out o x"];
    41 [label="Release x"];
    42 [label="Pull xs"];

    50 [label="Out o y"];
    51 [label="Release y"];
    52 [label="Pull ys"];

    100 [label="Pull ys"];
    101 [label="Out o y"];
    102 [label="Release y"];

    200 [label="Pull xs"];
    201 [label="Out o x"];
    202 [label="Release x"];

    900 [label="OutDone o"];
    901 [label="Done"];

    10 -> 20 [label="Some x"];
    10 -> 100 [label="None"];
    20 -> 30 [label="Some y"];
    20 -> 201 [label="None"];

    30 -> 40 [label="True"];
    30 -> 50 [label="False"];

    40 -> 41;
    41 -> 42;

    42 -> 30 [label="Some x"];
    42 -> 101 [label="None"];

    50 -> 51;
    51 -> 52;
    52 -> 30 [label="Some y"];
    52 -> 201 [label="None"];

    100 -> 101 [label="Some y"];
    100 -> 900 [label="None"];
    101 -> 102;
    102 -> 100;

    200 -> 201 [label="Some x"];
    200 -> 900 [label="None"];

    201 -> 202;
    202 -> 200;

    900 -> 901;
}
\end{dot2tex}
\caption{$o =$~merge~$xs$~$ys$}
\label{fig:com:merge}
\end{figure*}


\begin{figure}
{ \centering
\Large
\begin{dot2tex}[scale=0.35]
digraph {
    rankdir=LR;
    mindist=0.1
    nodesep=0.1;
    ranksep=0.1;
    node [shape="circle", margin=0];
    0 [label="Pull xs (init)"];
    1 [label="Update o_s = xs"];
    2 [label="Release xs"];
    3 [label="Pull xs"];

    4 [label="If eqfst"];

    10 [label="Update o_s = update"];

    20 [label="Out o_s"];

    30 [label="Out o_s"];

    40 [label="OutDone o"];

    50 [label="Done"];

    0 -> 1 [label="Some"];
    0 -> 40 [label="None"];
    2 -> 3;
    3 -> 4 [label="Some"];
    3 -> 30 [label="None"];

    4 -> 10 [label="True"];
    4 -> 20 [label="False"];

    10 -> 2;

    20 -> 1;

    30 -> 40;
    40 -> 50;
}
\end{dot2tex}

}

Note that @o_s@ denotes the state variable for output channel @o@, while the following functions check for equality of keys, and update their associated values respectively.

\begin{code}
eqfst  =  fst o_s == fst xs
update = (fst o_s, f (snd o_s) (snd xs))
\end{code}
\caption{$o =$~groupby~$f~xs$}
\label{fig:com:groupby}
\end{figure}



\subsection{Code extraction}

Extracting imperative code from these machines is quite straightforward.
Variables are created for each @Pull@ source, to store a single value read from the source, and each output channel has a variable to store its latest output and its current state.
Each state is converted to a basic block and given a unique identifier as its label.

For example, the @filter@ in Figure~\ref{fig:com:filter} is compiled to something like the following:
\begin{code}
run :: IO Int -> (Maybe Int -> IO ()) -> IO ()
run pull_xs when_o = l10
 where
  l10 = do  x  <- pull_xs
            case x of
             Nothing -> l90
             Just x' -> l15 x'

  l15 x | p x       = l20 x
        | otherwise = l30 x

  l20 x =   when_o (Just x)  >> l30 x

  l30 x =   l10

  l90   =   when_o Nothing  >> l91

  l91   =   return ()
\end{code}

