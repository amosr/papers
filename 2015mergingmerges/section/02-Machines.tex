%!TEX root = ../Main.tex
\section{Machines}
\label{s:Machines}

We use a particular type of deterministic finite automata (DFAs) where each state has a type that restricts the alphabet of output transitions.
We use state types such as @Pull a@ with output transitions @Some a@ and @None@, or type @Out b@ with single output transition @Unit@.

\subsection{Restricted DFA}

We describe a restricted automaton as an eight-tuple, extending the standard description by three elements: $(Q, \Sigma, \delta, q_0, F, \Gamma, \gamma, \sigma)$.

As usual, $Q$ is the set of states, $\Sigma$ the alphabet, but in addition $\Gamma$ is a set of state types.
We define the types of functions as follows:

\[ \gamma : Q \to \Gamma \]
Every state has a type.

\[ \sigma : \Gamma \to \{ \Sigma \} \]
Every type has some subset of the alphabet that transitions are required for.

\[ \delta ~:~\forall q : Q.\ \sigma(\gamma(q))~\to~Q \]
For each state, the transition function need only be defined for those letters of the state's type.

These restricted DFAs can be converted to normal automata by adding a rejecting sink state, and adding transitions for all missing letters of the alphabet to this rejecting state.

\subsection{State types}
We now describe the set of types and alphabet for describing combinator programs as a restricted DFA.
These are parameterised by both the names of sources, sinks and bindings $n$, and the type of worker functions $f$ (such as @Expression@ for code generation).

\begin{tabbing}
MM \= MM \= MMMMMM \= M \= M \kill
$\Gamma$ \> $=$ \> @Pull@     \>     \> $n$         \\
         \>     \> (Attempt to read from a named input) \\

         \> $~|$\> @Release@  \>     \> $n$         \\
         \>     \> (Values must be released before the input can be pulled again) \\

         \> $~|$\> @Close@    \>     \> $n$         \\
         \>     \> (A named input may still have data, but we will not read it) \\

         \> $~|$\> @Out@      \> $f$ \> $n$         \\
         \>     \> (Emit the value computed by $f$ to output channel $n$) \\

         \> $~|$\> @OutDone@  \>     \> $n$         \\
         \>     \> (Signal that output channel $n$ is complete) \\

         \> $~|$\> @If@       \> $f$ \> $n$         \\
         \>     \> (If based on current state of output channel $n$) \\

         \> $~|$\> @Update@   \> $f$ \> $n$         \\
         \>     \> (Update output channel $n$'s current state) \\

         \> $~|$\> @Skip@     \>                    \\
         \>     \> (Simple goto, simplifies merge algorithm) \\

         \> $~|$\> @Done@     \>                    \\
         \>     \> (The program is finished) \\
\end{tabbing}

Each of these state types requires a corresponding set of output transitions to be defined.
\begin{tabbing}
MMMMMMMM \= M \= \kill
$\gamma(@Pull @n)$
                            \> $=$ \> \{@Some@ n, @None@\} \\
$\gamma(@Release @n)$
                            \> $=$ \> \{@Unit@\} \\
$\gamma(@Close @n)$
                            \> $=$ \> \{@Unit@\} \\
$\gamma(@Out @   f~n)$
                            \> $=$ \> \{@Unit@\} \\
$\gamma(@OutDone @ n)$
                            \> $=$ \> \{@Unit@\} \\
$\gamma(@If @    f~n)$
                            \> $=$ \> \{@True@, @False@\} \\
$\gamma(@Update @f~n)$
                            \> $=$ \> \{@Unit@\} \\
$\gamma(@Skip@      )$
                            \> $=$ \> \{@Unit@\} \\
$\gamma(@Done@      )$
                            \> $=$ \> \{\} \\
\end{tabbing}

Note that @Out@, @OutDone@, @If@ and @Update@ are also annotated with the name of the output channel.
While a single combinator will only have one output, after merging multiple machines together there could be multiple outputs.
Each combinator has its own local mutable state, accessible by the functions given to @Update@, @Out@ and @If@.
These machine state types must be annotated with the name of the output channel to designate which local mutable state, if any.

In the proceeding parts, any reference to @Machine@ is a restricted DFA with these state types.

\subsection{Invariants}
\label{s:Machines:Invariants}
Not all machines are valid or meaningful, and we wish to rule them out.
For example, a machine with an initial state of $@If@~(x>5)~n$ is meaningless because there is no local $x$ that has been read from an input. 
On the other hand, some programs could be meaningful, but requiring them to be in a particular form simplifies fusion, later.
An example of such is a program that does not release its input before pulling again; a sympathetic code extraction would be to simply release before every pull, but having explicit releases makes synchronising two machines easier.

The invariants are:
\begin{itemize}
\item each function mentioned in @Out@, @If@ or @Update@ can only refer to previously read values;
\item each successful @Pull@ must be @Released@ before another @Pull@ can be made;
\item all inputs must either be finished by an empty @Pull@ or @Close@d at @Done@;
\item all outputs must be finished with @OutDone@ at @Done@;
\item inputs cannot be read after an empty @Pull@ or after they are @Close@d;
\item outputs cannot be written after they are finished with @OutDone@.
\end{itemize}

To perform this invariant checking, we annotate each state with a set of available information at that state.
The available information can be that there exists some value that can be used, or that a channel is finished or closed.
For invariant checking, we do not distinguish between finished and closed channels.

\begin{tabbing}
MM \= M \= \kill
$\Psi$ \> $=$  \> $@Value@~n$ \\
       \> $~|$ \> $@Finished@~n$ \\
       \> $~|$ \> $@Closed@~n$ \\
\end{tabbing}

To begin, we annotate the initial state with an empty set of information: at the start of the machine, we have no read values available, nor is it known (or likely!) that any channels are finished.
Then, for each output transition, we annotate the output state according to the current state type and the transition label: @Pull n@ adds a @Value n@ or @Finished n@ to its @Some n@ and @None@ transitions respectively.
If the output state has already been annotated, we check that the two annotations are the same, modulo values produced by @Out@.
The reason that annotations may be different in the values produced by @Out@ is that the @Out@ values are not explicitly released, and may be referred to by other functions, for example after fusing two combinators together.
These @Out@ values are then implicitly released.

For each state, we check that its annotation allows its execution: a @Pull n@ cannot run if $n$ is already finished; @Release n@ can only run if there is a @Value n@ to release, functions of @Out@ and @If@ require their values to be present, and so on.

%!TEX root = ../Main.tex

\begin{figure*}

$$
\ruleI
{
    \begin{array}{lr}
        \CheckOutGSA                    &
        \CheckStateType{s}{\Pull{n}} 
    \end{array}
}
{ 
    \CheckOutTrans
        { \Some{n} }
        { \cupsgl{a}{\Value{n}} }
}
\quad
\textrm{(Pull Some)}
\quad
\ruleI
{
    \begin{array}{lr}
        \CheckOutGSA                    &
        \CheckStateType{s}{\Pull{n}}
    \end{array}
}
{ 
    \CheckOutTrans
        { \None }
        { \cupsgl{a}{\Finished{n}} }
}
\quad
\textrm{(Pull None)}
$$

$$
\ruleI
{
    \begin{array}{lr}
        \CheckOutGSA                        &
        \CheckStateType{s}{\Release{n}}
    \end{array}
}
{ 
    \CheckOutTransU
        { a \setminus \sgl{\Value{n}} }
}
\quad
\textrm{(Release)}
\quad
\ruleI
{
    \begin{array}{lr}
        \CheckOutGSA                    &
        \CheckStateType{s}{\Close{n}}
    \end{array}
}
{ 
    \CheckOutTransU
        { \cupsgl{a}{\Closed{n}} }
}
\quad
\textrm{(Close)}
$$

$$
\ruleI
{
    \begin{array}{lr}
        \CheckOutGSA                        &
        \CheckStateType{s}{\Update{f}{n}}
    \end{array}
}
{ 
    \CheckOutTransU
        { a }
}
\quad
\textrm{(Update)}
\quad
\ruleI
{
    \begin{array}{lr}
        \CheckOutGSA                    &
        \CheckStateType{s}{\If{f}{n}}
    \end{array}
}
{ 
    \CheckOutTransU
        { a }
}
\quad
\textrm{(If)}
$$

$$
\ruleI
{
    \begin{array}{lr}
        \CheckOutGSA                      &
        \CheckStateType{s}{\Out{f}{n}}
    \end{array}
}
{ 
    \CheckOutTransU
        { \cupsgl{a}{\Value{n}} }
}
\quad
\textrm{(Out)}
\quad
\ruleI
{
    \begin{array}{lr}
        \CheckOutGSA                           &
        \CheckStateType{s}{\OutFinished{n}}
    \end{array}
}
{ 
    \CheckOutTransU
        { \cupsgl{a}{\Finished{n}} }
}
\quad
\textrm{(OutFinished)}
$$

$$
\ruleI
{
    \begin{array}{lr}
        \CheckOutGSA                 &
        \CheckStateType{s}{\Skip}
    \end{array}
}
{ 
    \CheckOutTransU
        { a }
}
\quad
\textrm{(Skip)}
$$


\caption{Generating available set for transition}
\label{fig:inv:generation}
\end{figure*}

\begin{figure*}

$$
\ruleI
{
    \begin{array}{lcccr}
        \CheckOutGSA                    &
        \CheckStateType{s}{\Pull{n}}    &
        \Value{n} \not\in a             &
        \Finished{n} \not\in a          &
        \Closed{n} \not\in a
    \end{array}
}
{ 
    \StateOK
}
\quad
\textrm{(Pull)}
$$

$$
\ruleI
{
    \begin{array}{lcr}
        \CheckOutGSA                        &
        \CheckStateType{s}{\Release{n}}    &
        \Value{n} \in a
    \end{array}
}
{ 
    \StateOK
}
\quad
\textrm{(Release)}
\quad
\ruleI
{
    \begin{array}{lccr}
        \CheckOutGSA                    &
        \CheckStateType{s}{\Close{n}}    &
        \Value{n} \not\in a             &
        \Finished{n} \not\in a
    \end{array}
}
{ 
    \StateOK
}
\quad
\textrm{(Close)}
$$

$$
\ruleI
{
    \begin{array}{lcr}
        \CheckOutGSA                        &
        \CheckStateType{s}{\Update{f}{n}}    &
        \fvs{f} \subset a
    \end{array}
}
{ 
    \StateOK
}
\quad
\textrm{(Update)}
\quad
\ruleI
{
    \begin{array}{lcr}
        \CheckOutGSA                        &
        \CheckStateType{s}{\If{f}{n}}    &
        \fvs{f} \subset a
    \end{array}
}
{ 
    \StateOK
}
\quad
\textrm{(If)}
$$

$$
\ruleI
{
    \begin{array}{lccr}
        \CheckOutGSA                      &
        \CheckStateType{s}{\Out{f}{n}}    &
        \fvs{f} \subset a                   &
        \Finished{n} \not\in a
    \end{array}
}
{ 
    \StateOK
}
\quad
\textrm{(Out)}
\quad
\ruleI
{
    \begin{array}{lcr}
        \CheckOutGSA                            &
        \CheckStateType{s}{\OutFinished{n}}     &
        \Finished{n}    \not\in a
    \end{array}
}
{ 
    \StateOK
}
\quad
\textrm{(OutFinished)}
$$

$$
\ruleI
{
    \begin{array}{lr}
        \CheckOutGSA                &
        \CheckStateType{s}{\Skip}
    \end{array}
}
{ 
    \StateOK
}
\quad
\textrm{(Skip)}
$$

$$
\ruleI
{
    \begin{array}{lcr}
        \CheckOutGSA                &
        \CheckStateType{s}{\Done}    &
        \forall c \in @inputs@~m \cup @outputs@~m.~(\Finished{c} \in a \vee \Closed{c} \in a)
    \end{array}
}
{ 
    \StateOK
}
\quad
\textrm{(Done)}
$$

\caption{Checking single state}
\label{fig:inv:checking}
\end{figure*}

\begin{figure*}

$$
\EnvGrow{\Gamma}{\emptyset}{\Gamma}
\quad
\textrm{(Finished)}
$$

$$
\ruleI
{
    \CheckOutGSA
    \quad
    s~:_\psi~b~\in~p
    \quad
    a~=~b
    \quad
    \EnvGrow{\Gamma}{p \setminus \{s\}}{\Gamma'}
}
{
    \EnvGrow{\Gamma}{p}{\Gamma'}
}
\quad
\textrm{(State already computed)}
$$

$$
\ruleI
{
    \CheckOutGSA
    \quad
    s~:_\psi~b~\in~p
    \quad
    \forall n \in (a \setminus b) \cup (b \setminus a).~ n \in @outputs@~m
    \quad
    \EnvGrow{\Gamma \setminus s}{p, s~:_\psi~(a~\cup~b)}{\Gamma'}
}
{
    \EnvGrow{\Gamma}{p}{\Gamma'}
}
\quad
\textrm{(Allow different output values)}
$$

$$
\ruleI
{
    s~:_\psi~a~\in~p
    \quad
    s~\not\in~\Gamma
    \quad
    p'~=~\{ \CheckOutTrans{l}{b} ~|~ s~\xRightarrow{l}~ t~\in~ m\}
    \quad
    \EnvGrow{\Gamma,s~:_\psi~a}{p~\setminus~s~\cup~ p' }{\Gamma'}
}
{
    \EnvGrow{\Gamma}{p}{\Gamma'}
}
\quad
\textrm{(Compute transitions)}
$$

\caption{Environment closure}
\label{fig:inv:closure}
\end{figure*}


\begin{figure*}

$$
\ruleI
{
    \EnvGrow{\emptyset}{@initial@~m~:_\psi~\emptyset}{\Gamma'}
    \quad
    \forall s \in m.\ \StateOK
}
{
    m~@ok@
}
\quad
\textrm{(Check entire machine)}
$$


\caption{Check entire machine}
\label{fig:inv:entire}
\end{figure*}


\subsubsection{Generation}
Figure~\ref{fig:inv:generation}.

Each transition modifies its input's annotation, depending on the state type and the transition label.
For example, the @Some@ output edge of a @Pull@ adds new information that a @Value@ is available.

$$ \CheckOut{\Gamma}{s}{a} $$
Under environment $\Gamma$ and machine $m$, the state $s$ has available information $a$.

$$ \CheckOutTrans{l}{a} $$
Under environment $\Gamma$ and machine $m$, the transition from state $s$ to $t$ under label $l$ has available information $a$.

$$ \CheckStateType{s}{T} $$
In machine $m$, state $s$ has state type $T$.


\subsubsection{Checking}
Figure~\ref{fig:inv:checking}.

We check each state type against its final annotation. For example, @Release@ requires that there exists a @Value@ to release.

$$ \StateOK $$
Under environment $\Gamma$ and machine $m$, the state $s$ has a valid annotation.

$$ \fvs{f} \subset a $$
Any free variables mentioned in a function must exist as @Value@s in the annotation, in order for the function to use these values.

\subsubsection{Environment closure}
Figure~\ref{fig:inv:closure}.

We must compute the transitive closure of each state's annotation.
Given two environments, the final environment and unvetted environment, states are pulled from the unvetted environment and checked against the final environment.
If the state already exists in the final environment and the annotations are equal, that state has been computed and can be removed from the unvetted environment.
If the state exists in both environments but they are different, either the only differences are output values, and we proceed with the intersection of the two, or if there are other differences, the machine is invalid.
Finally, if the state does not exist in the final environment, we add it to the final environment, compute all output transitions of the state, and add them to the unvetted environment.

\TODO{Perhaps this would be described better as a function than an inference rule.}
The problem, of course, is that state machines are not trees, so a derivation tree for typing does not make much sense.

$$ \EnvGrow{\Gamma}{p}{\Gamma'} $$
Under environment $\Gamma$ and machine $m$, with $p$ as an unvetted environment to be merged into $\Gamma$, the transitive closure of adding transitions is $\Gamma'$.

\subsubsection{Entire machine}
Figure~\ref{fig:inv:entire}.

First, the environment closure is computed, starting with the initial state of the machine having no available information.
Next, every state of the machine is checked to be valid against the environment closure.

$$ m~@ok@ $$
The entire machine is valid.


\subsection{Combinators}
Many interesting combinators can be described using these machines.
In figures~\ref{fig:com:map}-\ref{fig:com:merge}, the machines for some important combinators are given.
To reduce clutter, these figures omit labels for @Unit@ transitions.

While writing combinators in this form is neither pleasant nor pretty, it only needs to be done once.
The implementation contains more combinators such as group by, append and indices of segment lengths.



%!TEX root = ../Main.tex


\begin{figure}
\centering
\Large
\begin{dot2tex}[scale=0.5]
digraph {
    rankdir=LR;
    mindist=0.1
    nodesep=0.1;
    ranksep=0.1;
    node [shape="circle", margin=0];
    10 [label="Pull xs"];
    20 [label="Out o (f x)"];
    30 [label="Release x"];

    90 [label="OutDone o"];
    91 [label="Done"];

    10 -> 20 [label="Some x"];
    10 -> 90 [label="None"];

    90 -> 91;

    20 -> 30;
    30 -> 10;
}
\end{dot2tex}
\caption{$o =$~map~$f$~$xs$}
\label{fig:com:map}
\end{figure}

\begin{figure}
\centering
\Large
\begin{dot2tex}[scale=0.5]
digraph {
    rankdir=LR;
    mindist=0.1
    nodesep=0.1;
    ranksep=0.1;
    node [shape="circle", margin=0];
    10 [label="Pull xs"];
    15 [label="If (p x)"];
    20 [label="Out o x"];
    30 [label="Release x"];

    90 [label="OutDone o"];
    91 [label="Done"];

    10 -> 15 [label="Some x"];
    15 -> 20 [label="True"];
    15 -> 30 [label="False"];
    10 -> 90 [label="None"];

    90 -> 91;

    20 -> 30;
    30 -> 10;
}
\end{dot2tex}
\caption{$o =$~filter~$p$~$xs$}
\label{fig:com:filter}
\end{figure}

\begin{figure}
\centering
\Large
\begin{dot2tex}[scale=0.5]
digraph {
    rankdir=LR;
    mindist=0.1
    nodesep=0.1;
    ranksep=0.1;
    node [shape="circle", margin=0];
    10 [label="Pull xs"];
    11 [label="Pull ys"];
    20 [label="Out o (x,y)"];
    30 [label="Release x"];
    31 [label="Release y"];

    80 [label="Close ys"];
    88 [label="Close xs"];
    89 [label="Release x"];

    90 [label="OutDone o"];
    91 [label="Done"];

    10 -> 11 [label="Some x"];
    11 -> 20 [label="Some y"];
    10 -> 80 [label="None"];
    80 -> 90;
    11 -> 88 [label="None"];
    88 -> 89;

    89 -> 90;

    90 -> 91;

    20 -> 30;
    30 -> 31;
    31 -> 10;
}
\end{dot2tex}
\caption{$o =$~zip~$xs$~$ys$}
\label{fig:com:zip}
\end{figure}

\begin{figure*}
\centering
\Large
\begin{dot2tex}[scale=0.45]
digraph {
    rankdir=LR;
    mindist=0.0
    nodesep=0.0;
    ranksep=0.4;
    node [shape="circle", margin=0];

    10 [label="Pull xs"];
    20 [label="Pull ys"];
    30 [label="If (x le y)"];

    40 [label="Out o x"];
    41 [label="Release x"];
    42 [label="Pull xs"];

    50 [label="Out o y"];
    51 [label="Release y"];
    52 [label="Pull ys"];

    100 [label="Pull ys"];
    101 [label="Out o y"];
    102 [label="Release y"];

    200 [label="Pull xs"];
    201 [label="Out o x"];
    202 [label="Release x"];

    900 [label="OutDone o"];
    901 [label="Done"];

    10 -> 20 [label="Some x"];
    10 -> 100 [label="None"];
    20 -> 30 [label="Some y"];
    20 -> 201 [label="None"];

    30 -> 40 [label="True"];
    30 -> 50 [label="False"];

    40 -> 41;
    41 -> 42;

    42 -> 30 [label="Some x"];
    42 -> 101 [label="None"];

    50 -> 51;
    51 -> 52;
    52 -> 30 [label="Some y"];
    52 -> 201 [label="None"];

    100 -> 101 [label="Some y"];
    100 -> 900 [label="None"];
    101 -> 102;
    102 -> 100;

    200 -> 201 [label="Some x"];
    200 -> 900 [label="None"];

    201 -> 202;
    202 -> 200;

    900 -> 901;
}
\end{dot2tex}
\caption{$o =$~merge~$xs$~$ys$}
\label{fig:com:merge}
\end{figure*}


\begin{figure}
{ \centering
\Large
\begin{dot2tex}[scale=0.35]
digraph {
    rankdir=LR;
    mindist=0.1
    nodesep=0.1;
    ranksep=0.1;
    node [shape="circle", margin=0];
    0 [label="Pull xs (init)"];
    1 [label="Update o_s = xs"];
    2 [label="Release xs"];
    3 [label="Pull xs"];

    4 [label="If eqfst"];

    10 [label="Update o_s = update"];

    20 [label="Out o_s"];

    30 [label="Out o_s"];

    40 [label="OutDone o"];

    50 [label="Done"];

    0 -> 1 [label="Some"];
    0 -> 40 [label="None"];
    2 -> 3;
    3 -> 4 [label="Some"];
    3 -> 30 [label="None"];

    4 -> 10 [label="True"];
    4 -> 20 [label="False"];

    10 -> 2;

    20 -> 1;

    30 -> 40;
    40 -> 50;
}
\end{dot2tex}

}

Note that @o_s@ denotes the state variable for output channel @o@, while the following functions check for equality of keys, and update their associated values respectively.

\begin{code}
eqfst  =  fst o_s == fst xs
update = (fst o_s, f (snd o_s) (snd xs))
\end{code}
\caption{$o =$~groupby~$f~xs$}
\label{fig:com:groupby}
\end{figure}



\subsection{Code extraction}

Extracting imperative code from these machines is quite straightforward.
Variables are created for each @Pull@ source, to store a single value read from the source, and each output channel has a variable to store its latest output and its current state.
Each state is converted to a basic block and given a unique identifier as its label.

For example, the @filter@ in Figure~\ref{fig:com:filter} is compiled to something like the Figure~\ref{fig:extract:filter}.

\begin{figure}
\begin{code}
run :: IO Int -> (Maybe Int -> IO ()) -> IO ()
run pull_xs when_o = l10
 where
  l10 = do  x  <- pull_xs
            case x of
             Nothing -> l90
             Just x' -> l15 x'

  l15 x | p x       = l20 x
        | otherwise = l30 x

  l20 x =   when_o (Just x)  >> l30 x

  l30 x =   l10

  l90   =   when_o Nothing  >> l91

  l91   =   return ()
\end{code}
\caption{Extracted code for filter}
\label{fig:extract:filter}
\end{figure}

