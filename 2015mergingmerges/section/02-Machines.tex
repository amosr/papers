%!TEX root = ../Main.tex
\section{Machines}
\label{s:Machines}

A machine is a deterministic finite automata (DFA) that describes a streaming computation of a group of combinators.
We use a particular type of DFA where each state has a type that restricts the allowed output transitions, meaning that unnecessary edges are omitted.
A state type such as @Pull a@ with output transitions @Some a@ and @None@ denotes reading from an input source named @a@, and a state type of @Out (a+1) b@ with single output transition @Unit@ for writing @a+1@ to the output channel @b@.

\subsection{State types and transitions}
Each state has an associated state type.
These are parameterised by both the names of sources, sinks and bindings $n$, and the type of worker functions $f$ (such as @Expression@ for code generation).
The state types are as follows:

\begin{tabbing}
MMM \= MM \= MMMMMM \= M \= M \kill
$@Type@$ \> $=$ \> @Pull@     \>     \> $n$         \\
         \>     \> (Attempt to read from a named input) \\

         \> $~|$\> @Release@  \>     \> $n$         \\
         \>     \> (Values must be released before the next pull) \\

         \> $~|$\> @Close@    \>     \> $n$         \\
         \>     \> (An input may still have data, but we will not read it) \\

         \> $~|$\> @Out@      \> $f$ \> $n$         \\
         \>     \> (Emit the value computed by $f$ to output channel $n$) \\

         \> $~|$\> @OutDone@  \>     \> $n$         \\
         \>     \> (Signal that output channel $n$ is complete) \\

         \> $~|$\> @If@       \> $f$ \> $n$         \\
         \>     \> (If based on current state of output channel $n$) \\

         \> $~|$\> @Update@   \> $f$ \> $n$         \\
         \>     \> (Update output channel $n$'s current state) \\

         \> $~|$\> @Skip@     \>                    \\
         \>     \> (Simple goto, simplifies merge algorithm) \\

         \> $~|$\> @Done@     \>                    \\
         \>     \> (The program is finished) \\
\end{tabbing}

Note that @Out@, @OutDone@, @If@ and @Update@ are also annotated with the name of the output channel.
While a single combinator will only have one output, after merging multiple machines together there could be multiple outputs.
Each combinator has its own local mutable state, accessible by the functions given to @Update@, @Out@ and @If@.
These machine state types must be annotated with the name of the output channel to designate which local mutable state, if any.
During code generation, each local mutable state will be associated with a separate variable.

Some of these state types could be removed without affecting the expresivity or code generation of machines but are used in the merging algorithm, such as @Release@ which is used for synchronising shared reads.

Each of these state types requires a corresponding set of output transitions, or edges to other states, to be defined.
Most state types perform no flow control, and so require a single output transition denoting the state to be executed afterwards, like a semicolon in imperative code.
The states @Pull@ or @If@ need to perform different actions depending on whether the input is empty or not, or whether the predicate is true, so require multiple output transitions.
No further execution is possible after the @Done@ state, so it has no output transitions.
\begin{tabbing}
MMMMMMMMMMMM \= M \= \kill
$@StateOut@(@Pull @n)$
                            \> $=$ \> \{@Some@ n, @None@\} \\
$@StateOut@(@Release @n)$
                            \> $=$ \> \{@Unit@\} \\
$@StateOut@(@Close @n)$
                            \> $=$ \> \{@Unit@\} \\
$@StateOut@(@Out @   f~n)$
                            \> $=$ \> \{@Unit@\} \\
$@StateOut@(@OutDone @ n)$
                            \> $=$ \> \{@Unit@\} \\
$@StateOut@(@If @    f~n)$
                            \> $=$ \> \{@True@, @False@\} \\
$@StateOut@(@Update @f~n)$
                            \> $=$ \> \{@Unit@\} \\
$@StateOut@(@Skip@      )$
                            \> $=$ \> \{@Unit@\} \\
$@StateOut@(@Done@      )$
                            \> $=$ \> \{\} \\
\end{tabbing}

In the proceeding parts, we use $@Machine@~l$ to denote a machine with states labelled by $l$, where each state has a @Type@, and transitions as defined by the @StateOut@ of its type.

\subsection{Combinators}
\label{s:Machines:Combinators}
As an example of @filter p xs@ combinator expressed as a machine, see figure~\ref{fig:com:filter}.
Here, the initial state is a @Pull xs@, an attempt to read from the input stream @xs@.
If the read is successful, we take the @Some xs@ transition leading to $@If@~(p~xs_e)$ - the predicate applied to the currently read element of @xs@.
If the predicate is true, we execute $@Out@~o~xs_e$, writing the current element of @xs@ to the output channel named @o@; otherwise, we release the element of @xs@ and go back to the @Pull@.
Again, this release is not strictly necessary for expressivity or code generation, but is essential for synchronisation during merging.
Finally, if the @Pull xs@ fails and there is no data left to read, we signal that the output stream has finished, and then finish the entire machine.

Many interesting combinators can be described using these machines.
In figures~\ref{fig:com:map}-\ref{fig:com:groupby}, the machines for some important combinators are given.
To reduce clutter, these figures omit labels for @Unit@ transitions.

While writing combinators in this form is neither pleasant nor pretty, it only needs to be done once.
The implementation contains more combinators such as group by, append and indices of segment lengths.



%!TEX root = ../Main.tex


\begin{figure}
\centering
\Large
\begin{dot2tex}[scale=0.5]
digraph {
    rankdir=LR;
    mindist=0.1
    nodesep=0.1;
    ranksep=0.1;
    node [shape="circle", margin=0];
    10 [label="Pull xs"];
    20 [label="Out o (f x)"];
    30 [label="Release x"];

    90 [label="OutDone o"];
    91 [label="Done"];

    10 -> 20 [label="Some x"];
    10 -> 90 [label="None"];

    90 -> 91;

    20 -> 30;
    30 -> 10;
}
\end{dot2tex}
\caption{$o =$~map~$f$~$xs$}
\label{fig:com:map}
\end{figure}

\begin{figure}
\centering
\Large
\begin{dot2tex}[scale=0.5]
digraph {
    rankdir=LR;
    mindist=0.1
    nodesep=0.1;
    ranksep=0.1;
    node [shape="circle", margin=0];
    10 [label="Pull xs"];
    15 [label="If (p x)"];
    20 [label="Out o x"];
    30 [label="Release x"];

    90 [label="OutDone o"];
    91 [label="Done"];

    10 -> 15 [label="Some x"];
    15 -> 20 [label="True"];
    15 -> 30 [label="False"];
    10 -> 90 [label="None"];

    90 -> 91;

    20 -> 30;
    30 -> 10;
}
\end{dot2tex}
\caption{$o =$~filter~$p$~$xs$}
\label{fig:com:filter}
\end{figure}

\begin{figure}
\centering
\Large
\begin{dot2tex}[scale=0.5]
digraph {
    rankdir=LR;
    mindist=0.1
    nodesep=0.1;
    ranksep=0.1;
    node [shape="circle", margin=0];
    10 [label="Pull xs"];
    11 [label="Pull ys"];
    20 [label="Out o (x,y)"];
    30 [label="Release x"];
    31 [label="Release y"];

    80 [label="Close ys"];
    88 [label="Close xs"];
    89 [label="Release x"];

    90 [label="OutDone o"];
    91 [label="Done"];

    10 -> 11 [label="Some x"];
    11 -> 20 [label="Some y"];
    10 -> 80 [label="None"];
    80 -> 90;
    11 -> 88 [label="None"];
    88 -> 89;

    89 -> 90;

    90 -> 91;

    20 -> 30;
    30 -> 31;
    31 -> 10;
}
\end{dot2tex}
\caption{$o =$~zip~$xs$~$ys$}
\label{fig:com:zip}
\end{figure}

\begin{figure*}
\centering
\Large
\begin{dot2tex}[scale=0.45]
digraph {
    rankdir=LR;
    mindist=0.0
    nodesep=0.0;
    ranksep=0.4;
    node [shape="circle", margin=0];

    10 [label="Pull xs"];
    20 [label="Pull ys"];
    30 [label="If (x le y)"];

    40 [label="Out o x"];
    41 [label="Release x"];
    42 [label="Pull xs"];

    50 [label="Out o y"];
    51 [label="Release y"];
    52 [label="Pull ys"];

    100 [label="Pull ys"];
    101 [label="Out o y"];
    102 [label="Release y"];

    200 [label="Pull xs"];
    201 [label="Out o x"];
    202 [label="Release x"];

    900 [label="OutDone o"];
    901 [label="Done"];

    10 -> 20 [label="Some x"];
    10 -> 100 [label="None"];
    20 -> 30 [label="Some y"];
    20 -> 201 [label="None"];

    30 -> 40 [label="True"];
    30 -> 50 [label="False"];

    40 -> 41;
    41 -> 42;

    42 -> 30 [label="Some x"];
    42 -> 101 [label="None"];

    50 -> 51;
    51 -> 52;
    52 -> 30 [label="Some y"];
    52 -> 201 [label="None"];

    100 -> 101 [label="Some y"];
    100 -> 900 [label="None"];
    101 -> 102;
    102 -> 100;

    200 -> 201 [label="Some x"];
    200 -> 900 [label="None"];

    201 -> 202;
    202 -> 200;

    900 -> 901;
}
\end{dot2tex}
\caption{$o =$~merge~$xs$~$ys$}
\label{fig:com:merge}
\end{figure*}


\begin{figure}
{ \centering
\Large
\begin{dot2tex}[scale=0.35]
digraph {
    rankdir=LR;
    mindist=0.1
    nodesep=0.1;
    ranksep=0.1;
    node [shape="circle", margin=0];
    0 [label="Pull xs (init)"];
    1 [label="Update o_s = xs"];
    2 [label="Release xs"];
    3 [label="Pull xs"];

    4 [label="If eqfst"];

    10 [label="Update o_s = update"];

    20 [label="Out o_s"];

    30 [label="Out o_s"];

    40 [label="OutDone o"];

    50 [label="Done"];

    0 -> 1 [label="Some"];
    0 -> 40 [label="None"];
    2 -> 3;
    3 -> 4 [label="Some"];
    3 -> 30 [label="None"];

    4 -> 10 [label="True"];
    4 -> 20 [label="False"];

    10 -> 2;

    20 -> 1;

    30 -> 40;
    40 -> 50;
}
\end{dot2tex}

}

Note that @o_s@ denotes the state variable for output channel @o@, while the following functions check for equality of keys, and update their associated values respectively.

\begin{code}
eqfst  =  fst o_s == fst xs
update = (fst o_s, f (snd o_s) (snd xs))
\end{code}
\caption{$o =$~groupby~$f~xs$}
\label{fig:com:groupby}
\end{figure}



\subsection{Definitions}
\TODO{define some interesting functions:}

$@channel@~t$ takes a state type (eg $@Out@~f~n$) and returns the channel it is operating on ($n$), if applicable.
States like @Skip@ and @Done@ don't have an associated channel.

$@inputs@~m$ is the set of all channels used in @Pull@s, @Release@s and @Close@s.

$@outputs@~m$ is the set of all channels used in @Out@s, @OutDone@s, @Update@s and @If@s.
While @Update@s and @If@s do not perform outputs, they must belong to a particular output channel, and including them in @outputs@ simplifies handling of degenerate cases where @Update@s are performed but no @Out@s.

$@initial@~m$ gives the initial state of the machine.

$@states@~m$ gives the set of states of the machine, ignoring transitions.


