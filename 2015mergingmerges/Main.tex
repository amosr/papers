% Note:
% install dot2tex with
% $ pip install dot2tex
% then compile with shell escape enabled:
% $ latexmk --view=pdf --latexoption="-shell-escape"

\documentclass{sigplanconf}
\usepackage{amssymb}
\usepackage{graphicx}
\usepackage{amsmath}
\usepackage{mathptmx}
\usepackage{mathtools}
\usepackage{stmaryrd}
\usepackage{hyperref}
\usepackage{alltt}
\usepackage{url}
\usepackage{float}
\usepackage{style/utils}
\usepackage{style/code}
\usepackage{style/proof}
\usepackage{style/judgements}

% for combinator pictures
\usepackage{tikz}
\usetikzlibrary{shapes,arrows}
\usepackage[outputdir={out/}]{dot2texi}

% -----------------------------------------------------------------------------
\begin{document}

% \exclusivelicense
% \conferenceinfo{}{}
% \copyrightyear{2015}
% \copyrightdata{}
\doi{}
% \pagenumbering{gobble} 

\title{Merging merges, more or less}

\authorinfo{
  Amos Robinson$^\dagger$ 
  \and Ben Lippmeier$^\dagger$
  \and Gabriele Keller$^\dagger$ 
  \and Manuel Chakravarty$^\dagger$ 
}{
  \vspace{5pt}
  \shortstack{
    $^\dagger$Computer Science and Engineering \\
    University of New South Wales, Australia \\[2pt]
    \textsf{\{amosr,benl,keller,chak\}@cse.unsw.edu.au}
  }
}

\maketitle
\makeatactive

\begin{abstract}
Fusion is an essential part of optimising high-level array computations.
Fusion systems for combinators tend to focus on combinators with data-independent access patterns such as @zip@, @filter@ and @map@, while ignoring @merge@ as in merge sort.
In this paper, we show that describing combinators as deterministic finite automata allows us to express more combinators, while leading to a fusion algorithm based on intersection of automata.
\end{abstract}


\category
	{D.3.4}
	{Programming Languages}
	{Processors---Compilers; Optimization}

\terms
	Languages, Performance

\keywords
	Arrays; Fusion; Haskell

%!TEX root = ../Main.tex
\section{Introduction}

Data flow fusion~\cite{lippmeier2013flow} is a technique to compile a specific class of data flow programs into single, efficient imperative loops. This process of ``compilation'' is equivalent to performing array fusion on a combinator based functional array program, as per related work on stream fusion~\cite{coutts2007streamfusion} and delayed arrays~\cite{keller2010repa}. The key benefits of data flow fusion over this prior work are: 1) it fuses programs that use branching data flows where a produced array is consumed by several consumers, and 2) complete fusion into a single loop is guaranteed for all programs that operate on the same size input data, and contain no fusion-preventing dependencies between operators.

Fusion-preventing dependencies express the fact that some operators simply must wait for others to complete before they can produce their own output. For example, in the following:
\begin{code}
  normalize :: Array Int -> Array Int
  normalize xs = let sum = fold (+) 0 xs
                 in  map (/ sum) xs
\end{code}

If we wish to divide every element of an array by the sum of all elements, then it seems we are forever destined to compute the result using at least two loops: one to determine the sum, and one to divide the elements. The evaluation of @fold@ demands all elements of its source array, and we cannot produce any elements of the result array until we know the value of @sum@. 

However, many programs \emph{do} contain opportunities for fusion, if we only knew which opportunities to take. The following example offers \emph{several} unique, but mutually exclusive approaches to fusion. Figure~\ref{f:normalize2-cluterings} on the next page shows some of the possibilities.
\begin{code}
 normalize2 :: Array Int -> Array Int
 normalize2 xs
  = let sum1 = fold   (+)  0   xs
        gts  = filter (> 0)    xs
        sum2 = fold   (+)  0   gts
        ys1  = map    (/ sum1) xs
        ys2  = map    (/ sum2) xs
    in (ys1, ys2)
\end{code}

In Figure~\ref{f:normalize2-cluterings}, the dotted lines show possible clusterings of operators. Stream fusion implicitly choses the solution on the left as its compilation process cannot fuse a produced array into multiple consumers. The best existing ILP approach will chose the solution on the right as it cannot cluster operators that consume arrays of different lengths. Our system choses the solution in the middle, which is also optimal for this example. 

% NOTE: This set of bullets needs to fit on the first page, without spilling to the second.
Our contributions are as follows:
\begin{itemize}
\item   
We extend prior work by Megiddo~\cite{megiddo1998optimal} and Darte~\cite{darte2002contraction}, with support for length changing operators. Length changing operators can be clustered with the operators that generate their source arrays, and compiled naturally with data-flow fusion (\S\ref{s:ILP}).

\item
We present a simplification to constraint generation that is also applicable to some existing integer linear programming formulations such as Megiddo's,
where constraints between two nodes need not be generated if there exists a fusion-preventing path between the two (\S\ref{s:OptimisedConstraints}).

\item
Our constraint system also encodes a total ordering on the cost of clusterings, expressed using weights on the integer linear program. For example, we encode that memory traffic is more expensive than loop overheads, so given a choice between the two, the memory traffic will be reduced (\S\ref{s:ObjectiveFunction}).

\item
We present benchmarks of our algorithm applied to several common programming patterns, and to several pathological examples.
Our algorithm is complete and yields good results in practice, though if array sizes are unknown, an optimal solution is uncomputable in general. \TODO{ref}
\end{itemize}

The reduction of the clustering problem to integer linear programming was previously described by~\cite{megiddo1998optimal}, though they do not consider length changing operators.


% We must also decide which clustering is the `best' or most optimal. One obvious criterion for this is the minimum number of loops, but there may even be multiple clusterings with the minimum number of loops. In this case, the number of required manifest arrays must also be taken into account. 

% As real programs contain tens or hundreds of individual operators, performing an exhaustive search for an optimal clustering is not feasible, and greedy algorithms tend to produce poor solutions. 


%!TEX root = ../Main.tex
\section{Machines}
\label{s:Machines}

We use a particular type of deterministic finite automata (DFAs) where each state has a type that restricts the alphabet of output transitions.
We use state types such as @Pull a@ with output transitions @Some a@ and @None@, or type @Out b@ with single output transition @Unit@.

\subsection{Restricted DFA}

We describe a restricted automaton as a eight-tuple, extending the standard description by three elements: $(Q, \Sigma, \delta, q_0, F, \Gamma, \gamma, \sigma)$.

As usual, $Q$ is the set of states, $\Sigma$ the alphabet, but in addition $\Gamma$ is a set of state types.
We define the types of functions as follows:

\[ \gamma : Q \to \Gamma \]
Every state has a type.

\[ \sigma : \Gamma \to \{ \Sigma \} \]
Every type has some subset of the alphabet that transitions are required for.

\[ \delta : \forall q : Q. \sigma(\gamma(q)) \to Q \]
For each state, the transition function need only be defined for those letters of the state's type.

These restricted DFAs can be converted to normal automata by adding a rejecting sink state, and adding transitions for all missing letters of the alphabet to this rejecting state.

\subsection{State types}
We now describe a set of types and alphabet for describing combinator programs as a restricted DFA.
These are parameterised by both the names of sources, sinks and bindings $n$, and the type of worker functions $f$ (such as @Expression@ for code generation).

\begin{tabbing}
MM \= MM \= MMMMMM \= M \= M \kill
$\Gamma$ \> $=$ \> @Pull@     \>     \> $n$         \\
         \>     \> (Attempt to read from a named input) \\

         \> $~|$\> @Release@  \>     \> $n$         \\
         \>     \> (Read inputs must be released) \\

         \> $~|$\> @Out@      \> $f$ \> $n$         \\
         \>     \> (Emit the value computed by $f$ to output channel $n$) \\

         \> $~|$\> @OutDone@  \>     \> $n$         \\
         \>     \> (Signal that output channel $n$ is complete) \\

         \> $~|$\> @If@       \> $f$ \> $n$         \\
         \>     \> (If based on current state of output channel $n$) \\

         \> $~|$\> @Update@   \> $f$ \> $n$         \\
         \>     \> (Update output channel $n$'s current state) \\

         \> $~|$\> @Skip@     \>                    \\
         \>     \> (Simple goto, useful for code generation) \\

         \> $~|$\> @Done@     \>                    \\
         \>     \> (The program is finished) \\
\end{tabbing}

Each of these state types requires a corresponding set of output transitions to be defined.
\begin{tabbing}
MMMMMMMM \= M \= \kill
$\gamma(@Pull @n)$
                            \> $=$ \> \{@Some@ n, @None@\} \\
$\gamma(@Release @n)$
                            \> $=$ \> \{@Unit@\} \\
$\gamma(@Out @   f~n)$
                            \> $=$ \> \{@Unit@\} \\
$\gamma(@OutDone @ n)$
                            \> $=$ \> \{@Unit@\} \\
$\gamma(@If @    f~n)$
                            \> $=$ \> \{@True@, @False@\} \\
$\gamma(@Update @f~n)$
                            \> $=$ \> \{@Unit@\} \\
$\gamma(@Skip@      )$
                            \> $=$ \> \{@Unit@\} \\
$\gamma(@Done@      )$
                            \> $=$ \> \{\} \\
\end{tabbing}


\subsection{Combinators}
Many interesting combinators can be described using these machines.
In figures 1-4, the machines for some important combinators are given.
To reduce clutter, these figures omit labels for @Unit@ transitions.

While writing combinators in this form is neither pleasant nor pretty, it only needs to be done once.
The implementation contains more combinators such as group by, append and indices of segment lengths.



%!TEX root = ../Main.tex


\begin{figure}
\centering
\Large
\begin{dot2tex}[scale=0.5]
digraph {
    rankdir=LR;
    mindist=0.1
    nodesep=0.1;
    ranksep=0.1;
    node [shape="circle", margin=0];
    10 [label="Pull xs"];
    20 [label="Out o (f x)"];
    30 [label="Release x"];

    90 [label="OutDone o"];
    91 [label="Done"];

    10 -> 20 [label="Some x"];
    10 -> 90 [label="None"];

    90 -> 91;

    20 -> 30;
    30 -> 10;
}
\end{dot2tex}
\caption{$o =$~map~$f$~$xs$}
\label{fig:com:map}
\end{figure}

\begin{figure}
\centering
\Large
\begin{dot2tex}[scale=0.5]
digraph {
    rankdir=LR;
    mindist=0.1
    nodesep=0.1;
    ranksep=0.1;
    node [shape="circle", margin=0];
    10 [label="Pull xs"];
    15 [label="If (p x)"];
    20 [label="Out o x"];
    30 [label="Release x"];

    90 [label="OutDone o"];
    91 [label="Done"];

    10 -> 15 [label="Some x"];
    15 -> 20 [label="True"];
    15 -> 30 [label="False"];
    10 -> 90 [label="None"];

    90 -> 91;

    20 -> 30;
    30 -> 10;
}
\end{dot2tex}
\caption{$o =$~filter~$p$~$xs$}
\label{fig:com:filter}
\end{figure}

\begin{figure}
\centering
\Large
\begin{dot2tex}[scale=0.5]
digraph {
    rankdir=LR;
    mindist=0.1
    nodesep=0.1;
    ranksep=0.1;
    node [shape="circle", margin=0];
    10 [label="Pull xs"];
    11 [label="Pull ys"];
    20 [label="Out o (x,y)"];
    30 [label="Release x"];
    31 [label="Release y"];

    80 [label="Close ys"];
    88 [label="Close xs"];
    89 [label="Release x"];

    90 [label="OutDone o"];
    91 [label="Done"];

    10 -> 11 [label="Some x"];
    11 -> 20 [label="Some y"];
    10 -> 80 [label="None"];
    80 -> 90;
    11 -> 88 [label="None"];
    88 -> 89;

    89 -> 90;

    90 -> 91;

    20 -> 30;
    30 -> 31;
    31 -> 10;
}
\end{dot2tex}
\caption{$o =$~zip~$xs$~$ys$}
\label{fig:com:zip}
\end{figure}

\begin{figure*}
\centering
\Large
\begin{dot2tex}[scale=0.45]
digraph {
    rankdir=LR;
    mindist=0.0
    nodesep=0.0;
    ranksep=0.4;
    node [shape="circle", margin=0];

    10 [label="Pull xs"];
    20 [label="Pull ys"];
    30 [label="If (x le y)"];

    40 [label="Out o x"];
    41 [label="Release x"];
    42 [label="Pull xs"];

    50 [label="Out o y"];
    51 [label="Release y"];
    52 [label="Pull ys"];

    100 [label="Pull ys"];
    101 [label="Out o y"];
    102 [label="Release y"];

    200 [label="Pull xs"];
    201 [label="Out o x"];
    202 [label="Release x"];

    900 [label="OutDone o"];
    901 [label="Done"];

    10 -> 20 [label="Some x"];
    10 -> 100 [label="None"];
    20 -> 30 [label="Some y"];
    20 -> 201 [label="None"];

    30 -> 40 [label="True"];
    30 -> 50 [label="False"];

    40 -> 41;
    41 -> 42;

    42 -> 30 [label="Some x"];
    42 -> 101 [label="None"];

    50 -> 51;
    51 -> 52;
    52 -> 30 [label="Some y"];
    52 -> 201 [label="None"];

    100 -> 101 [label="Some y"];
    100 -> 900 [label="None"];
    101 -> 102;
    102 -> 100;

    200 -> 201 [label="Some x"];
    200 -> 900 [label="None"];

    201 -> 202;
    202 -> 200;

    900 -> 901;
}
\end{dot2tex}
\caption{$o =$~merge~$xs$~$ys$}
\label{fig:com:merge}
\end{figure*}


\begin{figure}
{ \centering
\Large
\begin{dot2tex}[scale=0.35]
digraph {
    rankdir=LR;
    mindist=0.1
    nodesep=0.1;
    ranksep=0.1;
    node [shape="circle", margin=0];
    0 [label="Pull xs (init)"];
    1 [label="Update o_s = xs"];
    2 [label="Release xs"];
    3 [label="Pull xs"];

    4 [label="If eqfst"];

    10 [label="Update o_s = update"];

    20 [label="Out o_s"];

    30 [label="Out o_s"];

    40 [label="OutDone o"];

    50 [label="Done"];

    0 -> 1 [label="Some"];
    0 -> 40 [label="None"];
    2 -> 3;
    3 -> 4 [label="Some"];
    3 -> 30 [label="None"];

    4 -> 10 [label="True"];
    4 -> 20 [label="False"];

    10 -> 2;

    20 -> 1;

    30 -> 40;
    40 -> 50;
}
\end{dot2tex}

}

Note that @o_s@ denotes the state variable for output channel @o@, while the following functions check for equality of keys, and update their associated values respectively.

\begin{code}
eqfst  =  fst o_s == fst xs
update = (fst o_s, f (snd o_s) (snd xs))
\end{code}
\caption{$o =$~groupby~$f~xs$}
\label{fig:com:groupby}
\end{figure}




\subsection{Code extraction}

Extracting imperative code from these machines is quite straightforward.
Variables are created for each 
Each state is converted to a basic block and given a unique identifier as its label.



%!TEX root = ../Main.tex
\section{Merging machines}
\label{s:Merging}

This section describes how machines can be merged together to form a single machine that computes both.

The idea is to create a new machine with the product of the two states; at each state, one or the other of the machines may be able to take a step.
If so, we update that machine's state and leave the other one as-is.
For shared inputs or outputs, both machines must take the steps at the same time: if they both pull from the same input, they must pull at the same time.
Similarly, if one machine produces an input and the other reads it, the first machine can only produce its output if the second is pulling on it.

If neither machine can make a move, the program is disallowed.
For example, if the first machine is trying to read from a shared input while the second tries to read from the first machine's input, executing it may require buffering the input until the second machine catches up.

However, the above explanation is too strict and outlaws programs such as @zip xs ys@ merged with @zip ys xs@, as neither machine can progress until the other does, leading to deadlock.
To resolve this, we add to the resulting machine's state a set of pending events for each machine.
The first machine may read from a shared input as long as there is not already a ``read'' event pending for the other machine to deal with.
Reading from a shared input adds a pending ``read'' event to the other machine, and the other machine may then skip past a @Pull@ for that source, as it is known that the other machine has already pulled on it.

\subsection{General merging}
Let us define the potential pending events on a machine.
One machine may have read from a shared input, but the other has not.
Similarly, if one machine has tried to read from a shared input but the input is empty, when the other machine attempts to read it can skip directly to the @None@ case.

Shared outputs, when one machine produces a value and the other consumes it.
Here, events are used to indicate that the producer has written a value, and cannot write another until the consumer has dealt with it, or that the producer has finished and the consumer's next read will skip to the @None@ case.

Finally, if one machine closes a shared input but the other is still reading, we do not close the input but instead treat the input as a local input, allowing the reading machine to continue reading at any rate. 

\begin{tabbing}
MM \= MM \= MMMMMM \= M\kill
$\Psi$ \> $=$  \> @Value@     \> $n$         \\
       \> $~|$ \> @Finished@  \> $n$         \\
       \> $~|$ \> @Closed@    \> $n$         \\
\end{tabbing}

The @merge@ function takes two machines with states $l_1$ and $l_2$, and produces a new machine with state $l_1 \times \{\Psi\} \times l_2 \times \{\Psi\}$.
It starts by creating an empty machine, and inserting states one by one, starting from the initial state as the initial state of each machine, and empty pending sets.
Then, for each output transition of the input machines, states are recursively added.

\begin{tabbing}
MMMMMMM \= MM \= MMMMMM \= M\kill
@merge@ \> $:$ \> $@Machine @l_1 \times @Machine @l_2$ \\
        \> $\to$ \> $@Machine @(l_1 \times \{\Psi\} \times l_2 \times \{\Psi\})$ \\
\\
@merge @$m_1~m_2$ \> $=$ \> $@merge@'~(@init @m_1)~(@init @m_2)~\{\}~@empty@$ \\
\\
$@merge@'$ \> $:$ \> $l_1 \times \{\Psi\} \times l_2 \times \{\Psi\}$               \\
           \> $\times$ \> $@Machine @(l_1 \times \{\Psi\} \times l_2 \times \{\Psi\})$ \\
           \> $\to$ \> $@Machine @(l_1 \times \{\Psi\} \times l_2 \times \{\Psi\})$ \\
\\
MMM \= M \= MMMMMMMMMM \= M \=\kill
$@merge@'~s_1~\psi_1~s_2~\psi_2~m$ \\
 \> $~|$ \> $(s_1, \psi_1, s_2, \psi_2) \in m$  \\
 \> $=$  \> $m$ \\
 \> $~|$ \> $(\gamma,\delta) \in @move@~s_1~\psi_1~s_2~\psi_2$ \\
 \> $=$  \> $@fold@~@merge@'~(m \cup \gamma \cup \delta)~\delta$ \\
\end{tabbing}

The $@merge@'$ function adds the given state and recursively adds the successor states to the given machine.
If the given state has already been added --- for example the state is reachable by multiple states --- the state need not be added.

\begin{tabbing}
MMM \= M \= MMMMMMM\kill
$@move@$ \> $:$ \> $l_1 \times \{\Psi\} \times l_2 \times \{\Psi\}$ \\
           \> $\to$ \> $\Gamma \times \{\Sigma \times (l_1 \times \{\Psi\} \times l_2 \times \{\Psi\})\}$ \\
\end{tabbing}

The @move@ function computes the states and transitions of the merged machine.
These are added to the machine by $@merge@'$ above.

%!TEX root = ../Main.tex

\begin{figure*}

$$
\ruleI
{
    \MoveOut
        {\sptp}
        {\gamma}
        {\MoveT{l}{s',\psi',t',\phi'};~ \MoveTU{\ldots} }
}
{
    \mathtt{move}~m_2~m_1~t~\phi~s~\psi
    ~=~\gamma,~\{\MoveT{l}{t',\phi',s',\psi'};~ \MoveTU{\ldots} \}
}
\textrm{(Commute)}
$$

\caption{Commutativity of merging}
\label{fig:merge:gen:comm}
\end{figure*}

\begin{figure*}


$$
\ruleI
{
    \arrLR
        { \CheckStateTypeM{m_1}{s}{\Update{f}{n}} }
        { \TransUM{s}{s'}{m_1} }
}
{
    \MoveOut
        {\sptp}
        {\Update{f}{n}}
        {\MoveTU{\sPptpC}}
}
\textrm{(Update)}
\quad
\ruleI
{
    \arrLR
        { \CheckStateTypeM{m_1}{s}{\Skip} }
        { \TransUM{s}{s'}{m_1} }
}
{
    \MoveOut
        {\sptp}
        {\Skip}
        {\MoveTU{\sPptpC}}
}
\textrm{(Skip)}
$$

$$
\ruleI
{
    \arrLCR
        { \CheckStateTypeM{m_1}{s}{\If{f}{n}} }
        { \TransM{s}{\True}{s_t}{m_1} }
        { \TransM{s}{\False}{s_f}{m_1} }
}
{
    \MoveOut
        {\sptp}
        {\If{f}{n}}
        { \MoveT{\True}{s_t,\psi,t,\phi}
         ;~
          \MoveT{\False}{s_f,\psi,t,\phi}
        }
}
\textrm{(If)}
\quad
\ruleI
{
    \arrLR
        { \CheckStateTypeM{m_1}{s}{\Done} }
        { \CheckStateTypeM{m_2}{s}{\Done} }
}
{
    \MoveOut
        {\sptp}
        {\Done}
        {}
}
\textrm{(DoneDone)}
$$

\caption{Non-interfering states}
\label{fig:merge:gen:noninter}
\end{figure*}

\begin{figure*}


$$
\ruleI
{
    \arrLCCR
        { \CheckStateTypeM{m_1}{s}{\Out{f}{n}} }
        { \TransUM{s}{s'}{m_1} }
        { n \in \inputs{m_2} }
        { \Closed{n} \in \phi }
}
{
    \MoveOut
        {\sptp}
        {\Out{f}{n}}
        {\MoveTU{\sPptpC}}
}
\textrm{(OutputClosed)}
$$

$$
\ruleI
{
    \arrLCCR
        { \CheckStateTypeM{m_1}{s}{\Out{f}{n}} }
        { \TransUM{s}{s'}{m_1} }
        { n \in \inputs{m_2} }
        { \Value{n} \not\in \phi }
}
{
    \MoveOut
        {\sptp}
        {\Out{f}{n}}
        {\MoveTU{s', \psi, t, (\phi \cup \sgl{\Value{n}}) }}
}
\textrm{(OutputReady)}
$$

$$
\ruleI
{
    \arrLCR
        { \CheckStateTypeM{m_1}{s}{\Out{f}{n}} }
        { \TransUM{s}{s'}{m_1} }
        { n \not\in \inputs{m_2} }
}
{
    \MoveOut
        {\sptp}
        {\Out{f}{n}}
        {\MoveTU{\sPptpC}}
}
\textrm{(OutputLocal)}
$$

$$
\ruleI
{
    \arrLCCCR
        { \CheckStateTypeM{m_1}{s}{\OutFinished{n}} }
        { \TransUM{s}{s'}{m_1} }
        { n \in \inputs{m_2} }
        { \Closed{n} \in \phi }
        { \Value{n} \not\in \phi }
}
{
    \MoveOut
        {\sptp}
        {\OutFinished{n}}
        {\MoveTU{\sPptpC}}
}
\textrm{(OutFinishedClosed)}
$$

$$
\ruleI
{
    \arrLCCCR
        { \CheckStateTypeM{m_1}{s}{\OutFinished{n}} }
        { \TransUM{s}{s'}{m_1} }
        { n \in \inputs{m_2} }
        { \Closed{n} \not\in \phi }
        { \Value{n} \not\in \phi }
}
{
    \MoveOut
        {\sptp}
        {\OutFinished{n}}
        {\MoveTU{s',\psi,t,(\phi \cup \sgl{\Finished{n}})}}
}
\textrm{(OutFinishedReady)}
$$

$$
\ruleI
{
    \arrLCR
        { \CheckStateTypeM{m_1}{s}{\OutFinished{n}} }
        { \TransUM{s}{s'}{m_1} }
        { n \not\in \inputs{m_2} }
}
{
    \MoveOut
        {\sptp}
        {\OutFinished{n}}
        {\MoveTU{\sPptpC}}
}
\textrm{(OutFinishedLocal)}
$$

\caption{Output and finishing output}
\label{fig:merge:gen:out}
\end{figure*}

\begin{figure*}


$$
\ruleI
{
    \arrLCCCR
        { \CheckStateTypeM{m_1}{s}{\Pull{n}} }
        { \TransM{s}{\Some{n}}{s_s}{m_1} }
        { \TransM{s}{\None}{s_n}{m_1} }
        { n \not\in \inputs{m_2} }
        { n \not\in \outputs{m_2} }
}
{
    \MoveOut
        {\sptp}
        {\Pull{n}}
        { \MoveT{\Some{n}}{s_s, \psi, t, \phi}
         ;~
          \MoveT{\None}{s_n, \psi, t, \phi} }
}
\textrm{(PullLocal)}
$$

$$
\ruleI
{
    \arrLCCR
        { \CheckStateTypeM{m_1}{s}{\Pull{n}} }
        { \TransM{s}{\Some{n}}{s_s}{m_1} }
        { \TransM{s}{\None}{s_n}{m_1} }
        { \Closed{n} \in \phi }
}
{
    \MoveOut
        {\sptp}
        {\Pull{n}}
        { \MoveT{\Some{n}}{s_s, \psi, t, \phi}
         ;~
          \MoveT{\None}{s_n, \psi, t, \phi} }
}
\textrm{(PullClosed)}
$$


$$
\ruleI
{
    \arrLCR
        { \CheckStateTypeM{m_1}{s}{\Pull{n}} }
        { \TransM{s}{\Some{n}}{s'}{m_1} }
        { \Value{n} \in \psi }
}
{
    \MoveOut
        {\sptp}
        {\Skip}
        { \MoveTU{\sPptpC} }
}
\textrm{(PullValue)}
$$

$$
\ruleI
{
    \arrLCCR
        { \CheckStateTypeM{m_1}{s}{\Pull{n}} }
        { \TransM{s}{\None}{s'}{m_1} }
        { \Finished{n} \in (\psi \cup \phi) }
        { \Value{n} \not\in (\psi \cup \phi) }
}
{
    \MoveOut
        {\sptp}
        {\Skip}
        { \MoveTU{\sPptpC} }
}
\textrm{(PullFinished)}
$$

$$
\ruleI
{
        \CheckStateTypeM{m_1}{s}{\Pull{n}}
        \quad
        \TransM{s}{\Some{n}}{s_s}{m_1}
        \quad
        \TransM{s}{\None}{s_n}{m_1}
        \quad
        \Finished{n} \not\in (\psi \cup \phi)
        \quad
        \Value{n} \not\in (\psi \cup \phi)
        \quad
        n \in \inputs{m_2}
}
{
    \MoveOut
        {\sptp}
        {\Pull{n}}
        { \MoveT{\Some{n}}{s_s, (\psi \cup \{ \Value{n} \}), t, (\phi \cup \{ \Value{n} \}) }
         ;~
          \MoveT{\None}{s_s, (\psi \cup \{ \Finished{n} \}), t, (\phi \cup \{ \Finished{n} \}) }
         }
}
$$
$$\textrm{(PullReady)}$$

\caption{Pulls}
\label{fig:merge:gen:pull}
\end{figure*}

\begin{figure*}


$$
\ruleI
{
        \CheckStateTypeM{m_1}{s}{\Release{n}}
        \quad
        \TransUM{s}{s'}{m_1}
        \quad
        n \not\in \inputs{m_2}
        \quad
        n \not\in \outputs{m_2}
}
{
    \MoveOut
        {\sptp}
        {\Release{n}}
        { 
          \MoveTU{s', \psi, t, \phi }
         }
}
\textrm{(ReleaseLocal)}
$$

$$
\ruleI
{
        \CheckStateTypeM{m_1}{s}{\Release{n}}
        \quad
        \TransUM{s}{s'}{m_1}
        \quad
        n \in \outputs{m_2}
        \quad
        \Value{n} \in \psi
}
{
    \MoveOut
        {\sptp}
        {\Skip}
        { 
          \MoveTU{s', (\psi \setminus \{\Value{n}\}), t, \phi }
         }
}
\textrm{(ReleaseOutput)}
$$

$$
\ruleI
{
        \CheckStateTypeM{m_1}{s}{\Release{n}}
        \quad
        \TransUM{s}{s'}{m_1}
        \quad
        n \in \inputs{m_2}
        \quad
        \Value{n} \in \phi
}
{
    \MoveOut
        {\sptp}
        {\Skip}
        { 
          \MoveTU{s', (\psi \setminus \{\Value{n}\}), t, \phi }
         }
}
\textrm{(ReleaseSharedFirst)}
$$

$$
\ruleI
{
        \CheckStateTypeM{m_1}{s}{\Release{n}}
        \quad
        \TransUM{s}{s'}{m_1}
        \quad
        n \in \inputs{m_2}
        \quad
        \Value{n} \in \psi
        \quad
        \Value{n} \not\in \phi
}
{
    \MoveOut
        {\sptp}
        {\Release{n}}
        { 
          \MoveTU{s', (\psi \setminus \{\Value{n}\}), t, \phi }
         }
}
\textrm{(ReleaseSharedSecond)}
$$


\caption{Releasing pulled values}
\label{fig:merge:gen:release}
\end{figure*}

\begin{figure*}

$$
\ruleI
{
    \arrLCCR
        { \CheckStateTypeM{m_1}{s}{\Close{n}} }
        { \TransUM{s}{s'}{m_1} }
        { n \not\in \inputs{m_2} }
        { n \not\in \outputs{m_2} }
}
{
    \MoveOut
        {\sptp}
        {\Close{n}}
        {\MoveTU{s',\psi, t, \phi}}
}
\textrm{(CloseLocal)}
$$

$$
\ruleI
{
    \arrLCR
        { \CheckStateTypeM{m_1}{s}{\Close{n}} }
        { \TransUM{s}{s'}{m_1} }
        { n \in \outputs{m_2} }
}
{
    \MoveOut
        {\sptp}
        {\Skip}
        {\MoveTU{s',(\psi \cup \Closed{n}), t, \phi}}
}
\textrm{(CloseOutput)}
$$


$$
\ruleI
{
    \arrLCCCR
        { \CheckStateTypeM{m_1}{s}{\Close{n}} }
        { \TransUM{s}{s'}{m_1} }
        { n \in \inputs{m_2} }
        { \Finished{n} \not\in \psi \cup \phi }
        { \Closed{n} \not\in \phi             }
}
{
    \MoveOut
        {\sptp}
        {\Skip}
        {\MoveTU{s',(\psi \cup \Closed{n}), t, \phi}}
}
\textrm{(CloseSharedOne)}
$$

$$
\ruleI
{
    \arrLCCR
        { \CheckStateTypeM{m_1}{s}{\Close{n}} }
        { \TransUM{s}{s'}{m_1} }
        { n \in \inputs{m_2} }
        { \Finished{n} \in \psi \cup \phi }
}
{
    \MoveOut
        {\sptp}
        {\Skip}
        {\MoveTU{s',(\psi \cup \Closed{n}), t, \phi}}
}
\textrm{(CloseSharedFinished)}
$$

$$
\ruleI
{
    \arrLCCCR
        { \CheckStateTypeM{m_1}{s}{\Close{n}} }
        { \TransUM{s}{s'}{m_1} }
        { n \in \inputs{m_2} }
        { \Finished{n} \not\in \psi \cup \phi }
        { \Closed{n} \in \phi             }
}
{
    \MoveOut
        {\sptp}
        {\Close{n}}
        {\MoveTU{s',(\psi \cup \Closed{n}), t, \phi}}
}
\textrm{(CloseSharedBoth)}
$$


\caption{Closing}
\label{fig:merge:gen:noninter}
\end{figure*}



\subsubsection{Non-interfering states}
If either of the machines are attempting to update their local state, this can be done easily without interfering with the other machine.

Skips are dealt with in the same way, affecting only one machine.

Similarly, @If@s only affect one of the machines, leaving the other in its original state.
If we had an oracle for checking functional equivalence, we could check whether two @If@s had the same function and inputs and potentially allow more programs.
However, lacking an oracle, even if both machines have @If@s with the same predicate, we generate a machine for all possible result combinations (all four of them).

If both machines are @Done@, the resulting machine will also be @Done@.

\subsubsection{Output and finishing}
The previous cases have been rather trivial as they require no synchronisation between the machines, but the remaining cases of @Out@, @OutFinished@, @Pull@, @Release@ and @Close@ are more complicated.

If the first machine is producing an output on channel $n$ and the second machine uses $n$, the first machine would usually have to wait until the second machine is ready to accept new data.
However, if the second machine has closed $n$, it will never pull on $n$ again, so the first machine may output to the channel as frequently as it likes.
Note that the output channels may have multiple consumers, so continuing to output after a consumer has finished is not as futile as it may seem at first.

Again, if the first machine is producing and the second is consuming, the first machine may only produce if there is no unhandled value on the channel.
This is what $\psi$ and $\phi$ are for.
If there is a value in the other machine's event set, $\phi$, a later case may allow the second machine to pull from it.

The final case for @Out@ is when the second machine does not use this output, allowing the first machine to output at any time.

The cases for @OutFinished@ proceed similarly to @Out@.

\subsubsection{Pull}
When the channel is a local input, @Pull@ing proceeds as normal.
Likewise with a shared input that has been closed by the other machine - the other machine will no longer attempt to pull, so no synchronisation is required.

If there is a @Value@ in the current machine's event set, the channel must either be a shared input, or an output channel on the other machine.
If it is a shared input, the presence of a @Value@ means the other machine has already executed a @Pull@ and has succeeded.
Otherwise, for an output channel, the @Value@ means the other machine has produced a value which must be used before another can be produced.
In either case, we simply proceed to the @Some@ branch of the @Pull@, safe in the knowledge that there is a value to use.

If there is a @Finished@ in either machine's event set, and neither machine has a pending @Value@s to deal with, we know that the other machine has had either an @OutFinished@ or an empty @Pull@.
We proceed to the @None@ case of the @Pull@.

Finally, if there are no pending @Value@ or @Finished@ events in either machine and it is a shared input, this machine can safely pull on the input, adding pending events for the other machine to deal with.


\subsubsection{Release}
Pulled values must be released.
This is used for synchronisation, so that one machine cannot pull on a shared input until after both machines have released the value.
Otherwise, one machine could pull while the other machine is still using the value, requiring a larger buffer.

For local channels, the release is performed as usual.

If the channel is an output of the other machine, the release is emitted as a skip, but the @Value@ is removed from the event set, which will allow the other machine's @Out@ to proceed.

For shared inputs, there will be two releases of the one value, but only the second should be emitted as an actual release.

For the first release, both machines will have a @Value@ event, and we emit a skip, removing this machine's @Value@.

For the second release, the other machine has already removed its @Value@ event, so we remove this machine's @Value@ and emit a real release.
At this stage, neither machine will have a @Value@ event, which will allow another @Pull@ to execute.

\subsubsection{Close}
For @Close@, if the channel is neither an input nor an output of the other machine, then closing the channel does not affect the other machine and acts as a normal @Close@.

If the channel is an output of the other machine, we skip past.
Because the channel may have other consumers, we do not stop it.
So that further outputs on this channel do not wait on this machine, we add a @Closed@ event to $\psi$.



If none of the above cases apply, it means the machines cannot be fused.
Either they require more than one element of buffering, or reasoning about function equality would be required to prove that they do not.
The only unhandled cases above are where the two machines share input or outputs, and there are existing values buffered in $\psi$.
In which case, we give a compile time error indicating that these two machines cannot be fused.

\subsubsection{Example}
Let us merge a @map@ (figure~\ref{fig:com:map}) and a @filter@ (figure~\ref{fig:com:filter}).
The actual program would be
\begin{code}
fun xs
 = let m  = map    (+1) xs
       f  = filter (>0) xs
   in (m, f)
\end{code}

We need to name each state.
While the graphical form is easier to understand, a textual form makes looking up a particular state's transitions easier.
The @map@ state types and transitions are below, with initial state being @m1@.

\begin{code}
m1 : Pull xs     { Some => m2; None => m4 }
m2 : Out m (x+1) { Unit => m3 }
m3 : Release x   { Unit => m1 }
m4 : OutFinished m{ Unit => m5 }
m5 : Done
\end{code}

Likewise, the filter has initial state @f1@.

\begin{code}
f1 : Pull xs     { Some => f2; None  => f5 }
f2 : If (x>0)    { True => f3; False => f4 }
f3 : Out f x     { Unit => f4 }
f4 : Release x   { Unit => f1 }
f5 : OutFinished f{ Unit => f6 }
f6 : Done
\end{code}

We start by computing $@merge@'$ of an empty machine, empty $\psi$ sets, and initial states of each machine:
$$
@merge@'~@m1@~\emptyset~@f1@~\emptyset~@empty@
$$

In this case, there are two possible moves: either @m@ may take a step, or @f@ may take a step, but both steps are @Pull xs@. When two moves are possible, we will take the @m@ one, but this decision has no effect on the outcome \TODO{(prove it)}.
\begin{code}
move m f m1 {} f1 {} =
    Pull xs,
    { Some => m2, {Value xs},    f1, {Value xs}
    ; None => m4, {Finished xs}, f1, {Finished xs} }
(Shared pull, no pending values)
\end{code}

We insert this state and transitions into the empty machine, noting that the transitions to undefined states will be defined by later moves.

\begin{code}
m1,{},f1,{}
    : Pull xs
    { Some => m2, {Value xs},    f1, {Value xs}
    ; None => m4, {Finished xs}, f1, {Finished xs} }
\end{code}

Now, calculate the first undefined move.
Again, either machine may move: @m@ could compute its output, or @f@ could skip past the @Pull xs@, since this has been performed by @m@, witnessed by the @Value xs@ in @f@'s $\psi$ set.

\begin{code}
move m f m2 {Value xs} f1 {Value xs} =
    Out m (x+1)
    { Unit => m3, {Value xs}, f1, {Value xs} }
(Local output, free to write)
\end{code}

This puts @m@ at a @Release@, which is changed to a skip as the later release by @f@ will be the real release.

\begin{code}
move m f m3 {Value xs} f1 {Value xs} =
    Skip
    { Unit => m1, {}, f1, {Value xs} }
(Shared release: first release is skip)
\end{code}

Now machine @m@ is back at a @Pull xs@, but it is unable to execute the pull itself as the other machine @f@ still has a pending @Value xs@ to deal with.
The machine @f@ is also at a @Pull xs@, and is able to skip past since it already has a value.

\begin{code}
move m f m1 {} f1 {Value xs} =
    Skip
    { Unit => m1, {}, f2, {Value xs} }
(Pull has value)
\end{code}

Now @f@ is at an @If@, and @m@ still cannot run.

\begin{code}
move m f m1 {} f2 {Value xs} =
    If (x>0)
    { True  => m1, {}, f3, {Value xs}
    , False => m1, {}, f4, {Value xs} }
(If)
\end{code}

The first case of the @If@ is a local output. 
\begin{code}
move m f m1 {} f3 {Value xs} =
    Out f x
    { Unit => m1, {}, f4, {Value xs} }
(Local output, free to write)
\end{code}

Next is a release.
In this case, the input is shared and the other machine has already released, since there is no @Value xs@ in its set.
We emit an actual release in this case, since we are sure both machines are finished using it.

\begin{code}
move m f m1 {} f4 {Value xs} =
    Release x
    { Unit => m1, {}, f1, {} }
(Shared release: other machine already released)
\end{code}

Now the only missing state is @m4, {Finished xs}, f1, {Finished xs}@ after the @m@'s pull is empty.

\begin{code}
move m f m4 {Finished xs} f1 {Finished xs} =
    OutFinished m
    { Unit => m5, {Finished xs}, f1, {Finished xs} }
(Finish local output)
\end{code}

Now @m@ is @Done@, and cannot move, while @f@ is at a @Pull xs@ that it already knows is finished.
\begin{code}
move m f m5 {Finished xs} f1 {Finished xs} =
    Skip
    { Unit => m5, {Finished xs}, f5, {Finished xs} }
(Pull of finished)
\end{code}

Now @m@ is @Done@, and cannot move, while @f@ is at a @Pull xs@ that it already knows is finished.
\begin{code}
move m f m5 {Finished xs} f5 {Finished xs} =
    OutFinished f
    { Unit => m5, {Finished xs}, f6, {Finished xs} }
(Finish local output)
\end{code}

Both machines are now done.
\begin{code}
move m f m5 {Finished xs} f6 {Finished xs} =
    Done
(Both machines done)
\end{code}

The entire machine at the end is:
\begin{code}
m1,{},f1,{}
    : Pull xs
    { Some => m2, {Value xs},    f1, {Value xs}
    ; None => m4, {Finished xs}, f1, {Finished xs} }
m2,{Value xs},f1,{Value xs}
    : Out m (x+1)
    { Unit => m3, {Value xs},    f1, {Value xs}    }
m3,{Value xs},f1,{Value xs}
    : Skip
    { Unit => m1, {},            f1, {Value xs}    }
m1,{},f1,{Value xs}
    : Skip
    { Unit => m1, {},            f2, {Value xs}    }
m1,{},f2,{Value xs}
    : If (x>0)
    { True => m1, {},            f3, {Value xs}
    , False=> m1, {},            f4, {Value xs}    }
m1,{},f3,{Value xs}
    : Out f x
    { Unit => m1, {},            f4, {Value xs}    }
m1,{},f4,{Value xs}
    : Release x
    { Unit => m1, {},            f1, {}            }
m4,{Finished xs},f1,{Finished xs}
    : OutFinished m
    { Unit => m5, {Finished xs}, f1, {Finished xs} }
m5,{Finished xs},f1,{Finished xs}
    : Skip
    { Unit => m5, {Finished xs}, f5, {Finished xs} }
m5,{Finished xs},f5,{Finished xs}
    : OutFinished f
    { Unit => m5, {Finished xs}, f6, {Finished xs} }
m5,{Finished xs},f6,{Finished xs}
    : Done
\end{code}

The @Skip@s can then be removed quite easily, leading to the graph below.

\begin{dot2tex}[scale=0.5]
digraph {
    rankdir=LR;
    mindist=0.1
    nodesep=0.1;
    ranksep=0.1;
    node [shape="circle", margin=0];
    10 [label="Pull xs"];
    20 [label="Out m (x+1)"];
    50 [label="If"];
    60 [label="Out f x"];
    70 [label="Release x"];
    900[label="OutFinished m"];
    920[label="OutFinished f"];
    930[label="Done"];

    10 -> 20 [label="Some x"];
    10 -> 900 [label="None"];

    50 -> 60 [label="True"];
    50 -> 70 [label="False"];

    20 -> 50;
    60 -> 70;
    70 -> 10;
    900 -> 920;
    920 -> 930;
}
\end{dot2tex}



\subsection{Theorems we want to prove}

As we do not wish to handle recursive combinators, we also do not need to handle merging cyclic machines.
This is one case that regular dataflow handles but we do not.
The predicate below checks if either of the machines does not use the other's input; if both used the other's input, it would create a cycle.

\begin{tabbing}
MMMMMMM \= MM \= \kill
$@acyclic@~a~b$
\> $:=$
\> $\outputs{a}~\cap~\inputs{b}~=~\emptyset$
\\
\> $~\vee$ \> $\outputs{b}~\cap~\inputs{a}~=~\emptyset$
\end{tabbing}

Two machines cannot be merged if they both provide the same output, or write to the same output channel.

\begin{tabbing}
MMMMMMM \= MM \= \kill
$@distinct@~a~b$
\> $:=$
\> $\outputs{a}~\cap~\outputs{b}~=\emptyset$
\end{tabbing}

\subsubsection{Preservation}
The result of merging two machines should maintain the pull invariants (\S\ref{s:Machines:Invariants}), provided that the input machines also maintain their invariants.

\begin{tabbing}
MMMMMMM \= MM \= MM \= \kill
$\forall a~b.$
\>
\> $a~@ok@~\wedge~b~@ok@$
\\
\> $\wedge$
\> $@acyclic@~a~b$
\\
\> $\wedge$
\> $@distinct@~a~b$
\\
\> $\implies$
\> $\forall c \in (@merge@~a~b).\ c~@ok@$
\end{tabbing}

\paragraph{Lemma values:} 
for each state $u$ of $@merge@~a~b$, if either $\psi$ or $\phi$ contain a @Value n@, then the machine's invariant set will contain @Value n@ at state $u$.
(Note that the converse is not true, as @Value n@ will appear in the invariant set, but not $\psi$ or $\phi$ for local @Out@s)

We proceed by induction over the @move@ rules.

The only way that @Value n@ can be added to $\psi$ or $\phi$ is through the rules (Shared output, ready to read) and (Shared pull, no pending values).
These rules themselves produce state types of @Out@ and @Pull@ respectively, and by the definition of invariant sets @Value@ is inserted for these transitions.

For rule (Local release), the state type is @Release n@ which removes @Value n@ from the invariant set.
Due to the locality preconditions $n~\not\in~@inputs@~m_2$ and $n~\not\in~@outputs@~m_2$, we know that neither of the rules to add @n@ could apply, so $\psi$ and $\phi$ must not contain @Value n@.

For the cases that remove a @Value n@ from the $\psi$ set, the first two, (Release of output) and (Shared release: first release is skip) both only emit skips, so do not affect the invariant set.
This means that while the invariant set may still contain a @Value n@, it is not obliged to.
We appeal to the induction hypothesis for the input transitions, since earlier input transitions must have added the @Value n@ to both sets.

The last case that removes a @Value n@ from the $\psi$ set is (Shared release: other machine already released).
Here, we know $\psi$ contains a @Value n@ and that $\phi$ does not.
This means that after removing the @Value n@ from $\psi$, neither set will have one.
The produced state type also means that the invariant set will no longer have a @Value n@.

For the remaining cases, the @Value@s inside the $\psi$ and $\phi$ sets are left alone, and their state types do not explicitly remove any @Value n@s from the invariant set either.
We can conclude that any @Value@s in the sets have been added by other states, and not removed.
The last thing is that @Value@s can be implicitly removed at join-points of states.
However, in order for a state to have @Value@s in its sets, all transitions into this state must have either inserted @Value@s itself, or kept @Value@s from its predecessors in turn; otherwise it would not have a transition to a state with @Value@s in its sets.

\paragraph{Lemma closes and finishes:} 
for each state $u$ of $@merge@~a~b$, if \emph{either} event set contains a @Finish n@, the invariant set at state $u$ will also contain a @Finish n@.
If \emph{both} event sets contain a @Close n@, the invariant set will contain a @Close n@.

This proof proceeds very similarly to the above lemma.
For (Finish shared output, no pending events), a @Finished n@ is added to $\phi$, while @OutFinished n@ adds a @Finish n@ to the invariant set.
For (Shared pull, no pending values), a @Pull@ is emitted, and the @None@ case adds @Finished n@ to both sets, as with the invariant set.


\paragraph{Lemma locals:} for any reads, outs or closes in one machine that are not in @inputs@ or @outputs@ of the other machine, no state in the other machine will affect the invariant set relating to these channels.
By definition of @inputs@ and @outputs@.

\paragraph{Lemma transitions:} for each state $u$ of $@merge@~a~b$ and corresponding states $s$ and $t$ of $a$ and $b$, any output transitions from $u$ either move along $s$'s output transitions or $t$'s output transitions, but not both.
Similarly, the state type of $u$ will either be the type of $s$, the type of $t$, or a @Skip@.
By inspection of each rule of @move@.

\paragraph{Lemma paths:} for two given states $u$ and $v$ of $@merge@~a~b$, and their corresponding input machine states $s$ and $t$, for each path between $u$ and $v$, there exists a path between $s$ and $t$ such that all states in the $st$ path are in the $uv$ path, \emph{except}:
\begin{itemize}
\item @Pull@s can be omitted if every predecessor path of the @Pull@ has a @Pull@, @Out@ or @OutFinished@ on the same channel;
\item @Release@s can be omitted if the channel is an output of the other machine;
\item @Release@s can be omitted if every successor path has a @Release@ on the same channel;
\item @Close@s can be omitted if every successor path has a @Close@ or @Pull@ on the same channel.
\end{itemize}

It is easy to see that for non-interfering transitions @Update@, @Skip@, @If@ and @Done@, this holds.
For the other cases, it is necessary to perform induction over the @move@ rules, and inspect the event sets.
If one of the input machine's states is a @Pull@, then either a @Pull@ or a @Skip@ may be produced.
@Skip@s are only produced if the event set already contains a @Value@ or a @Finished@, which can only be produced by an earlier @Pull@, @Out@ or @OutFinished@.
Likewise for the remaining cases.


\paragraph{Proof sketch:} 
We can use the above lemmas to show that if the two input machines are valid, the result of @merge@ will also be valid.
For non-shared or local channels, it is easy to see that the invariant set for these channels will be exactly as it was in the input machine (lemma locals).

For shared inputs, since each input machine is valid, we know all @Pull@s are paired with @Release@s.
Given lemma values, and that @Release@s remove @Value@s from the invariant set, we can see that each generated @Pull@ on a shared resource will only be generated when the invariant set has no @Value@s, and will be similarly paired for @Release@s.

For shared outputs, we just need to verify that the only way the reading machine can reach a reading state is while there is a @Value n@ in the invariant set, and this also follows easily from lemma values.

Finally, we can see as a result of lemma closes and finishes and paths, that since both machines finish or close their inputs and outputs by their @Done@ state, then by the @Done@ of the merged machine, where both machines are also @Done@, all channels must be similarly closed at the end.


%!TEX root = ../Main.tex
\section{Invariants}
\label{s:Machines:Invariants}
Not all machines are valid or meaningful, and we wish to rule them out.
For example, a machine with an initial state of $@If@~(xs_e~>~5)~n$ is meaningless because there is no element of $xs$ that has been read from an input. 
On the other hand, some programs could be meaningful, but requiring them to be in a particular form simplifies fusion, later.
An example of such is a program that does not release its input before pulling again; a sympathetic code extraction would be to simply release before every pull, but having explicit releases makes synchronising two machines easier.

The invariants are:
\begin{itemize}
\item each function mentioned in @Out@, @If@ or @Update@ can only refer to previously read values;
\item each successful @Pull@ must be @Released@ before another @Pull@ can be made;
\item all inputs must either be finished by an empty @Pull@ or @Close@d at @Done@;
\item all outputs must be finished with @OutDone@ at @Done@;
\item inputs cannot be read after an empty @Pull@ or after they are @Close@d;
\item outputs cannot be written after they are finished with @OutDone@.
\end{itemize}

To perform this invariant checking, we annotate each state with a set of available information at that state.
The DFA machines simplify the merging algorithm, but lose binding and scope information which would be implicit in a structured language.
We need to reconstruct and check the scope information, similar to a type state system\CITE.

The available event information can be that there exists some value that can be used, or that a channel is finished or closed.

\begin{tabbing}
MMMM \= M \= \kill
@Event@ \> $=$  \> $@Value@~n$ \\
       \> $~|$ \> $@Finished@~n$ \\
       \> $~|$ \> $@Closed@~n$ \\
\end{tabbing}

To begin, we annotate the initial state with an empty set of information: at the start of the machine, we have no read values available, nor is it known (or likely!) that any channels are finished.
Then, for each output transition, we annotate the output state according to the current state type and the transition label: @Pull n@ adds a @Value n@ or @Finished n@ to its @Some n@ and @None@ transitions respectively.
If the output state has already been annotated, we check that the two annotations are the same, modulo values produced by @Out@.
The reason that annotations may be different in the values produced by @Out@ is that the @Out@ values are not explicitly released, and may be referred to by other functions, for example after fusing two combinators together.
These @Out@ values are then implicitly released.

For each state, we check that its annotation allows its execution: a @Pull n@ cannot run if $n$ is already finished; @Release n@ can only run if there is a @Value n@ to release, functions of @Out@ and @If@ require their values to be present, and so on.

%!TEX root = ../Main.tex

\begin{figure*}

$$
\ruleI
{
    \begin{array}{lr}
        \CheckOutGSA                    &
        \CheckStateType{s}{\Pull{n}} 
    \end{array}
}
{ 
    \CheckOutTrans
        { \Some{n} }
        { \cupsgl{a}{\Value{n}} }
}
\quad
\textrm{(Pull Some)}
\quad
\ruleI
{
    \begin{array}{lr}
        \CheckOutGSA                    &
        \CheckStateType{s}{\Pull{n}}
    \end{array}
}
{ 
    \CheckOutTrans
        { \None }
        { \cupsgl{a}{\Finished{n}} }
}
\quad
\textrm{(Pull None)}
$$

$$
\ruleI
{
    \begin{array}{lr}
        \CheckOutGSA                        &
        \CheckStateType{s}{\Release{n}}
    \end{array}
}
{ 
    \CheckOutTransU
        { a \setminus \sgl{\Value{n}} }
}
\quad
\textrm{(Release)}
\quad
\ruleI
{
    \begin{array}{lr}
        \CheckOutGSA                    &
        \CheckStateType{s}{\Close{n}}
    \end{array}
}
{ 
    \CheckOutTransU
        { \cupsgl{a}{\Closed{n}} }
}
\quad
\textrm{(Close)}
$$

$$
\ruleI
{
    \begin{array}{lr}
        \CheckOutGSA                        &
        \CheckStateType{s}{\Update{f}{n}}
    \end{array}
}
{ 
    \CheckOutTransU
        { a }
}
\quad
\textrm{(Update)}
\quad
\ruleI
{
    \begin{array}{lr}
        \CheckOutGSA                    &
        \CheckStateType{s}{\If{f}{n}}
    \end{array}
}
{ 
    \CheckOutTransU
        { a }
}
\quad
\textrm{(If)}
$$

$$
\ruleI
{
    \begin{array}{lr}
        \CheckOutGSA                      &
        \CheckStateType{s}{\Out{f}{n}}
    \end{array}
}
{ 
    \CheckOutTransU
        { \cupsgl{a}{\Value{n}} }
}
\quad
\textrm{(Out)}
\quad
\ruleI
{
    \begin{array}{lr}
        \CheckOutGSA                           &
        \CheckStateType{s}{\OutFinished{n}}
    \end{array}
}
{ 
    \CheckOutTransU
        { \cupsgl{a}{\Finished{n}} }
}
\quad
\textrm{(OutFinished)}
$$

$$
\ruleI
{
    \begin{array}{lr}
        \CheckOutGSA                 &
        \CheckStateType{s}{\Skip}
    \end{array}
}
{ 
    \CheckOutTransU
        { a }
}
\quad
\textrm{(Skip)}
$$


\caption{Generating available set for transition}
\label{fig:inv:generation}
\end{figure*}

\begin{figure*}

$$
\ruleI
{
    \begin{array}{lcccr}
        \CheckOutGSA                    &
        \CheckStateType{s}{\Pull{n}}    &
        \Value{n} \not\in a             &
        \Finished{n} \not\in a          &
        \Closed{n} \not\in a
    \end{array}
}
{ 
    \StateOK
}
\quad
\textrm{(Pull)}
$$

$$
\ruleI
{
    \begin{array}{lcr}
        \CheckOutGSA                        &
        \CheckStateType{s}{\Release{n}}    &
        \Value{n} \in a
    \end{array}
}
{ 
    \StateOK
}
\quad
\textrm{(Release)}
\quad
\ruleI
{
    \begin{array}{lccr}
        \CheckOutGSA                    &
        \CheckStateType{s}{\Close{n}}    &
        \Value{n} \not\in a             &
        \Finished{n} \not\in a
    \end{array}
}
{ 
    \StateOK
}
\quad
\textrm{(Close)}
$$

$$
\ruleI
{
    \begin{array}{lcr}
        \CheckOutGSA                        &
        \CheckStateType{s}{\Update{f}{n}}    &
        \fvs{f} \subset a
    \end{array}
}
{ 
    \StateOK
}
\quad
\textrm{(Update)}
\quad
\ruleI
{
    \begin{array}{lcr}
        \CheckOutGSA                        &
        \CheckStateType{s}{\If{f}{n}}    &
        \fvs{f} \subset a
    \end{array}
}
{ 
    \StateOK
}
\quad
\textrm{(If)}
$$

$$
\ruleI
{
    \begin{array}{lccr}
        \CheckOutGSA                      &
        \CheckStateType{s}{\Out{f}{n}}    &
        \fvs{f} \subset a                   &
        \Finished{n} \not\in a
    \end{array}
}
{ 
    \StateOK
}
\quad
\textrm{(Out)}
\quad
\ruleI
{
    \begin{array}{lcr}
        \CheckOutGSA                            &
        \CheckStateType{s}{\OutFinished{n}}     &
        \Finished{n}    \not\in a
    \end{array}
}
{ 
    \StateOK
}
\quad
\textrm{(OutFinished)}
$$

$$
\ruleI
{
    \begin{array}{lr}
        \CheckOutGSA                &
        \CheckStateType{s}{\Skip}
    \end{array}
}
{ 
    \StateOK
}
\quad
\textrm{(Skip)}
$$

$$
\ruleI
{
    \begin{array}{lcr}
        \CheckOutGSA                &
        \CheckStateType{s}{\Done}    &
        \forall c \in @inputs@~m \cup @outputs@~m.~(\Finished{c} \in a \vee \Closed{c} \in a)
    \end{array}
}
{ 
    \StateOK
}
\quad
\textrm{(Done)}
$$

\caption{Checking single state}
\label{fig:inv:checking}
\end{figure*}

\begin{figure*}

$$
\EnvGrow{\Gamma}{\emptyset}{\Gamma}
\quad
\textrm{(Finished)}
$$

$$
\ruleI
{
    \CheckOutGSA
    \quad
    s~:_\psi~b~\in~p
    \quad
    a~=~b
    \quad
    \EnvGrow{\Gamma}{p \setminus \{s\}}{\Gamma'}
}
{
    \EnvGrow{\Gamma}{p}{\Gamma'}
}
\quad
\textrm{(State already computed)}
$$

$$
\ruleI
{
    \CheckOutGSA
    \quad
    s~:_\psi~b~\in~p
    \quad
    \forall n \in (a \setminus b) \cup (b \setminus a).~ n \in @outputs@~m
    \quad
    \EnvGrow{\Gamma \setminus s}{p, s~:_\psi~(a~\cup~b)}{\Gamma'}
}
{
    \EnvGrow{\Gamma}{p}{\Gamma'}
}
\quad
\textrm{(Allow different output values)}
$$

$$
\ruleI
{
    s~:_\psi~a~\in~p
    \quad
    s~\not\in~\Gamma
    \quad
    p'~=~\{ \CheckOutTrans{l}{b} ~|~ s~\xRightarrow{l}~ t~\in~ m\}
    \quad
    \EnvGrow{\Gamma,s~:_\psi~a}{p~\setminus~s~\cup~ p' }{\Gamma'}
}
{
    \EnvGrow{\Gamma}{p}{\Gamma'}
}
\quad
\textrm{(Compute transitions)}
$$

\caption{Environment closure}
\label{fig:inv:closure}
\end{figure*}


\begin{figure*}

$$
\ruleI
{
    \EnvGrow{\emptyset}{@initial@~m~:_\psi~\emptyset}{\Gamma'}
    \quad
    \forall s \in m.\ \StateOK
}
{
    m~@ok@
}
\quad
\textrm{(Check entire machine)}
$$


\caption{Check entire machine}
\label{fig:inv:entire}
\end{figure*}


\subsection{Generation}
Figure~\ref{fig:inv:generation}.

Each transition modifies its input's annotation, depending on the state type and the transition label.
For example, the @Some@ output edge of a @Pull@ adds new information that a @Value@ is available.

$$ \CheckOut{\Gamma}{s}{a} $$
Under environment $\Gamma$ and machine $m$, the state $s$ has available information $a$.

$$ \CheckOutTrans{l}{a} $$
Under environment $\Gamma$ and machine $m$, the transition from state $s$ to $t$ under label $l$ has available information $a$.

$$ \CheckStateType{s}{T} $$
In machine $m$, state $s$ has state type $T$.


\subsection{Checking}
Figure~\ref{fig:inv:checking}.

We check each state type against its final annotation. For example, @Release@ requires that there exists a @Value@ to release.

$$ \StateOK $$
Under environment $\Gamma$ and machine $m$, the state $s$ has a valid annotation.

$$ \fvs{f} \subset a $$
Any free variables mentioned in a function must exist as @Value@s in the annotation, in order for the function to use these values.

\subsection{Environment closure}
Figure~\ref{fig:inv:closure}.

We must compute the transitive closure of each state's annotation.
Given two environments, the final environment and unvetted environment, states are pulled from the unvetted environment and checked against the final environment.
If the state already exists in the final environment and the annotations are equal, that state has been computed and can be removed from the unvetted environment.
If the state exists in both environments but they are different, either the only differences are output values, and we proceed with the intersection of the two, or if there are other differences, the machine is invalid.
Finally, if the state does not exist in the final environment, we add it to the final environment, compute all output transitions of the state, and add them to the unvetted environment.

\TODO{Perhaps this would be described better as a function than an inference rule.}
The problem, of course, is that state machines are not trees, so a derivation tree for typing does not make much sense.

$$ \EnvGrow{\Gamma}{p}{\Gamma'} $$
Under environment $\Gamma$ and machine $m$, with $p$ as an unvetted environment to be merged into $\Gamma$, the transitive closure of adding transitions is $\Gamma'$.

$$ \EnvMachine{m}{\Gamma} $$
The environment closure for an entire machine is computed by starting with the initial state of the machine having no available information.

\subsection{Entire machine}
Figure~\ref{fig:inv:entire}.

Every state of the machine is checked to be valid against the environment closure.

$$ m~@ok@ $$
The entire machine is valid.


\subsection{Code extraction}

Extracting imperative code from these machines is quite straightforward.
Variables are created for each @Pull@ source, to store a single value read from the source, and each output channel has a variable to store its latest output and its current state.
Each state is converted to a basic block and given a unique identifier as its label.

For example, the @filter@ in Figure~\ref{fig:com:filter} is compiled to something like the Figure~\ref{fig:extract:filter}.

\begin{figure}
\begin{code}
run :: IO Int -> (Maybe Int -> IO ()) -> IO ()
run pull_xs when_o = l10
 where
  l10 = do  x  <- pull_xs
            case x of
             Nothing -> l90
             Just x' -> l15 x'

  l15 x | p x       = l20 x
        | otherwise = l30 x

  l20 x =   when_o (Just x)  >> l30 x

  l30 x =   l10

  l90   =   when_o Nothing  >> l91

  l91   =   return ()
\end{code}
\caption{Extracted code for filter}
\label{fig:extract:filter}
\end{figure}


%!TEX root = ../Main.tex
\section{Proofs}
\label{s:Proofs}

As we do not wish to handle recursive combinators, we also do not need to handle merging cyclic machines.
This is one case that regular dataflow handles but we do not.
The predicate below checks if either of the machines does not use the other's input; if both used the other's input, it would create a cycle.

\begin{tabbing}
MMMMMMM \= MM \= \kill
$@acyclic@~a~b$
\> $:=$
\> $\outputs{a}~\cap~\inputs{b}~=~\emptyset$
\\
\> $~\vee$ \> $\outputs{b}~\cap~\inputs{a}~=~\emptyset$
\end{tabbing}

Two machines cannot be merged if they both provide the same output, or write to the same output channel.

\begin{tabbing}
MMMMMMM \= MM \= \kill
$@distinct@~a~b$
\> $:=$
\> $\outputs{a}~\cap~\outputs{b}~=\emptyset$
\end{tabbing}

The final requirement for correct fusion is the two machines to satisfy their invariants.
\begin{tabbing}
MMMMMMM \= MM \= \kill
$@oks@~a~b$
\> $:=$
\> $@acyclic@~a~b~\wedge~@distinct@~a~b$ \\
\> $\wedge$ \> $@ok@~a~\wedge~@ok@~b$ \\
\end{tabbing}

We also define a predicate for checking whether there exists a path of transitions between two states of a machine.
\begin{tabbing}
MMMMMMM \= MM \= \kill
$@reachable@~m~u~v$ \\
\> $:=$
\> $u~=~v$ \\
\> $\wedge$ \> $\bigvee_{u'|u~\Rightarrow~u'}~@reachable@~m~u'~v$ \\
\end{tabbing}

\subsection{Closure transitions}
The first lemma we require is to show that the environment closure computes the intersection of all transitions into a given state.

We would like to show that this holds for computing the closure for an entire machine:

\begin{tabbing}
MM \= MM \= MM \= \kill
$\forall...$ \\
\> \> $\EnvMachine{m}{\Gamma}$ \\
\> $\wedge$ \> $u~:_E~e~\in~\Gamma$ \\
\> $\implies$\> $\bigcap \{e'~|~n~\Rightarrow~u~\in~m$ \\
\>           \>         \> $\wedge~\CheckOutTransUP{\Gamma'}{n}{u}{e'} \}$ \\
\> $=$       \> $e$
\end{tabbing}

However this is not quite true, as the environment closure is only computed for reachable transitions.
We also need a weaker postcondition in order to perform induction over the closure environment.
We add the extra constraint that only transitions reachable from the pending state are considered.
We also take into account the intersection of the initial set and the pending set.

The actual lemma to prove, then, is:

\begin{lemma}
Closure is intersection.

\begin{tabbing}
MM \= MM \= MM \= \kill
$\forall...$ \\
\>          \> $\EnvGrow{\Gamma}{p}{\Gamma'}$ \\
\> $\wedge$ \> $u~:_E~e~\in~\Gamma'$ \\
\> $\implies$\> $\bigcap\{e'~|~n~\Rightarrow~u~\in~m$ \\
\>           \>         \> $\wedge~\CheckOutTransUP{\Gamma'}{n}{u}{e'}$ \\
\>           \>         \> $\wedge~\bigvee_{v~\in~p}@reachable@~m~v~n\}$ \\
\> $\cap$ \> $\bigcap\{e' ~|~ u~:_E~e'~\in~\Gamma \} $ \\
\> $\cap$ \> $\bigcap\{e' ~|~ u~:_E~e'~\in~p \} $ \\
\> $=$       \> $e$
\end{tabbing}
\end{lemma}

\begin{proof}
By induction over the closure.

\begin{itemize}
\item \textbf{Case (CloComputed):}

regardless of whether $s~=~u$ or not, we simply use the induction hypothesis; since $a~=~b$, removing $s$ from $p$ does not change the result, as $s~:_E~a$ is left in $\Gamma$.

\item \textbf{Case (CloTransitions):}

if $s~=~u$, the intersection is not affected as it is moved from $p$ to $\Gamma$.
If $u$ is one of the transitions from $s$, it is already in the first part of the intersection, so adding it to $p$ will not change the intersection.

\item \textbf{Case (CloRemoveOut):}

again, we just use the induction hypothesis;
$s$ is removed from $\Gamma$ but the intersection of $\Gamma$'s $s$ and $p$'s $s$ is inserted into $p$.


\item \textbf{Case (CloFinished):}

here, $p$ is empty and there is no path from any elements of $p$ to $u$, so neither the first nor the last intersections have any effect, and the result is just $\Gamma$.
\end{itemize}

\end{proof}

Before we can prove that the result of merging satisfies invariants, we must show that the result of merging has an environment closure.
Computing the environment closure requires all transitions into a state to have the same event set, except for values of outputs.

\begin{lemma}
Likeness implies closure
\begin{tabbing}
MM \= MM \= MM \kill
 $~($
\> $\forall u ~|~ \bigcup_{q~\in~p~\cup~\Gamma}(@reachable@~m~q~u).$ \\
\> $\forall p_0 \in p~\cup~\Gamma.$ \\
\> $\forall p_1 \in p~\cup~\Gamma.$ \\
\> $\forall path_0 \in @path@~p_0~u.$ \\
\> $\forall path_1 \in @path@~p_1~u.$ \\
\> \> $path_0~:_E~\tau_0$ \\
\> $\wedge$ \> $path_1~:_E~\tau_1$ \\
\> $\implies$ \> $@like@~\tau_0~\tau_1)$ \\
$\implies$ \> $\exists \Gamma'.$ \> $\EnvGrow{\Gamma}{p}{\Gamma'}$ \\
\end{tabbing}
\end{lemma}

\begin{proof}
Induction over $\langle \card{@states@~m~\setminus~\Gamma},~\Sigma_{s~:_E~\tau~\in~\Gamma}\card{\tau},~\card{p} \rangle$, where $\langle \cdots \rangle$ is a lexicographic ordering.

\begin{description}
\item[Case $\langle 0,~0,~0 \rangle$]

$\card{p}~=~0 \implies p=\emptyset$

apply (CloFinished)

\item[Case $\langle g,~t,~y \rangle$]

destruct $p$
\begin{description}
\item[Case $\emptyset$]

apply (CloFinished)

\item[Case $s~:_E~\tau, p$]

destruct $s \in \Gamma$
\begin{description}
\item[Case false]

apply (CloTransitions).

apply i.h:

$g$ decreases, as
$\card{@states@~m~\setminus~(s~:_E~\tau, \Gamma)} < \card{@states@~m~\setminus~\Gamma}$

\item[Case $s~:_E~a~\in~\Gamma$]

$@like@~a~\tau$ from hypothesis.

destruct $a~\cap~\tau~=~a$
\begin{description}
\item[Case $a~\cap~\tau~=~a$]

apply (CloComputed).

apply i.h.
$g$ same. $t$ same. $y$ decreases, as $\card{p~\setminus~s}~<~\card{p}$

\item[Case $a~\cap~\tau~\not=~a$]

apply (CloUpdateTransitions).

apply i.h.

$g$ same. $t$ decreases, as
$a~\cap~\tau~\not=~a~\implies~\card{a~\cap~\tau}~<~\card{a}$

\end{description}
\end{description}
\end{description}
\end{description}
    


\end{proof}

\subsection{Merged event set}
Assuming that the two inputs to a merged program are valid according to the invariants, we must show that the merged result, if it exists, is also valid.
We can compute the event set for a merge state from the event sets of the corresponding input machines' states.

For each input or output of either machine, if it is closed and local, it is closed.
If it is shared and only closed in one machine, the other machine must still be using it.
If either machine has a value, the result machine has a value.
We must also check the state sets of the output machine, $e_1$ and $e_2$ here, because the input machine may have made an output and silently discarded it, but if that value is yet to be read by the other machine, it cannot be silently discarded by the result machine.
If either machine is finished, the result machine is finished.



\begin{tabbing}
M \= MMMMMM \= M \= MM \kill
$@mergeT@~(s_1,~e_1,~s_2,~e_2)$ \\
$\bigcup_{n~\in~ns}$ \\
 \> $\Closed{n}$ \> $~|~$ \> $\Closed{n} \in e_a \cup e_1 ~\wedge~ @notin@ b n$ \\
 \>              \> $\vee$\> $\Closed{n} \in e_b \cup e_2 ~\wedge~ @notin@ a n$ \\
 \>              \> $\vee$\> $\Closed{n} \in e_a \cup e_1 ~\wedge~ \Closed{n} \in e_b \cup e_2$ \\
\\
 \> $\Value{n} $ \> $~|~$ \> $\Value{n} \in e_a \cup e_1  ~\vee~   \Value{n} \in e_b \cup e_2$ \\
\\
 \> $\Finished{n}$\>$~|~$ \> $\Finished{n} \in e_a \cup e_1  ~\vee~   \Finished{n} \in e_b \cup e_2$ \\
where \\
 \> $\EnvMachine{a}{\Gamma_a}$ \\
 \> $\EnvMachine{b}{\Gamma_b}$ \\
 \> $s_1~:_E~e_a~\in~\Gamma_a$ \\
 \> $s_2~:_E~e_b~\in~\Gamma_b$ \\
 \> $ns~=~@inputs@~a~\cup~@outputs@~a~\cup~@inputs@~b~\cup~@outputs@~b$ \\
 \> $@notin@~m~n~=~n~\not\in~@inputs@~m~\cup~@outputs@~m$ \\
\end{tabbing}

\subsection{Values}
For each state $u$ of $@merge@~a~b$, if either event set contains a @Value n@, then the machine's invariant set will contain @Value n@ at state $u$.
(Note that the converse is not true, as @Value n@ will appear in the invariant set, but neither event set for local @Out@s)



\begin{tabbing}
MMMMMMM \= MM \= MM \= \kill
$\forall a~b.\ \forall u \in (@merge@~a~b).$
\\
\>
\> $@oks@~a~b$
\\
\> $\wedge$
\> $\EnvMachine{@merge@~a~b}{\Gamma}$
\\
\> $\wedge$
\> $u~:_E~e~\in~\Gamma$
\\
\> $\wedge$
\> $\Value{n}~\in~(u \cdot e_1~\cup~u \cdot e_2)$
\\
\> $\implies$
\> $\Value{n}~\in~e$
\end{tabbing}

Let $m~=~@merge@~a~b$.

We first apply rule (CloMachine) to the assumption.

$$
\EnvMachine{@merge@~a~b}{\Gamma}
\implies
\EnvGrow{\emptyset}{@initial@~m~:_E~\emptyset}{\Gamma}
$$

We now proceed by induction over the closure.

$$
\EnvGrow{\Gamma}{p}{\Gamma'}
$$

(CloComputed): by the induction hypothesis.

We proceed by induction over the @move@ rules.

The only way that @Value n@ can be added to the event set is through the rules (Shared output, ready to read) and (Shared pull, no pending values).
These rules themselves produce state types of @Out@ and @Pull@ respectively, and by the definition of invariant sets @Value@ is inserted for these transitions.

For rule (Local release), the state type is @Release n@ which removes @Value n@ from the invariant set.
Due to the locality preconditions $n~\not\in~@inputs@~m_2$ and $n~\not\in~@outputs@~m_2$, we know that neither of the rules to add @n@ could apply, so neither event set can contain @Value n@.

For the cases that remove a @Value n@ from the event set, the first two, (Release of output) and (Shared release: first release is skip) both only emit skips, so do not affect the invariant set.
This means that while the invariant set may still contain a @Value n@, it is not obliged to.
We appeal to the induction hypothesis for the input transitions, since earlier input transitions must have added the @Value n@ to both sets.

The last case that removes a @Value n@ from the event set is (Shared release: other machine already released).
Here, we know the first machine's event set, $e_1$, contains a @Value n@ and that the second machine's event set, $e_2$, does not.
This means that after removing the @Value n@ from $e_1$, neither set will have one.
The produced state type also means that the invariant set will no longer have a @Value n@.

For the remaining cases, the @Value@s inside the event sets are left alone, and their state types do not explicitly remove any @Value n@s from the invariant set either.
We can conclude that any @Value@s in the sets have been added by other states, and not removed.
The last thing is that @Value@s can be implicitly removed at join-points of states.
However, in order for a state to have @Value@s in its sets, all transitions into this state must have either inserted @Value@s itself, or kept @Value@s from its predecessors in turn; otherwise it would not have a transition to a state with @Value@s in its sets.


\subsection{Preservation}
The result of merging two machines should maintain the pull invariants (\S\ref{s:Machines:Invariants}), provided that the input machines also maintain their invariants.

\begin{tabbing}
MMMMMMM \= MM \= MM \= \kill
$\forall a~b.$
\>
\> $a~@ok@~\wedge~b~@ok@$
\\
\> $\wedge$
\> $@acyclic@~a~b$
\\
\> $\wedge$
\> $@distinct@~a~b$
\\
\> $\implies$
\> $\forall c \in (@merge@~a~b).\ c~@ok@$
\end{tabbing}

\paragraph{Lemma closes and finishes:} 
for each state $u$ of $@merge@~a~b$, if \emph{either} event set contains a @Finish n@, the invariant set at state $u$ will also contain a @Finish n@.
If \emph{both} event sets contain a @Close n@, the invariant set will contain a @Close n@.

This proof proceeds very similarly to the above lemma.
For (Finish shared output, no pending events), a @Finished n@ is added to $e_2$, while @OutFinished n@ adds a @Finish n@ to the invariant set.
For (Shared pull, no pending values), a @Pull@ is emitted, and the @None@ case adds @Finished n@ to both sets, as with the invariant set.


\paragraph{Lemma locals:} for any reads, outs or closes in one machine that are not in @inputs@ or @outputs@ of the other machine, no state in the other machine will affect the invariant set relating to these channels.
By definition of @inputs@ and @outputs@.

\paragraph{Lemma transitions:} for each state $u$ of $@merge@~a~b$ and corresponding states $s$ and $t$ of $a$ and $b$, any output transitions from $u$ either move along $s$'s output transitions or $t$'s output transitions, but not both.
Similarly, the state type of $u$ will either be the type of $s$, the type of $t$, or a @Skip@.
By inspection of each rule of @move@.

\paragraph{Lemma paths:} for two given states $u$ and $v$ of $@merge@~a~b$, and their corresponding input machine states $s$ and $t$, for each path between $u$ and $v$, there exists a path between $s$ and $t$ such that all states in the $st$ path are in the $uv$ path, \emph{except}:
\begin{itemize}
\item @Pull@s can be omitted if every predecessor path of the @Pull@ has a @Pull@, @Out@ or @OutFinished@ on the same channel;
\item @Release@s can be omitted if the channel is an output of the other machine;
\item @Release@s can be omitted if every successor path has a @Release@ on the same channel;
\item @Close@s can be omitted if every successor path has a @Close@ or @Pull@ on the same channel.
\end{itemize}

It is easy to see that for non-interfering transitions @Update@, @Skip@, @If@ and @Done@, this holds.
For the other cases, it is necessary to perform induction over the @move@ rules, and inspect the event sets.
If one of the input machine's states is a @Pull@, then either a @Pull@ or a @Skip@ may be produced.
@Skip@s are only produced if the event set already contains a @Value@ or a @Finished@, which can only be produced by an earlier @Pull@, @Out@ or @OutFinished@.
Likewise for the remaining cases.


\paragraph{Proof sketch:} 
We can use the above lemmas to show that if the two input machines are valid, the result of @merge@ will also be valid.
For non-shared or local channels, it is easy to see that the invariant set for these channels will be exactly as it was in the input machine (lemma locals).

For shared inputs, since each input machine is valid, we know all @Pull@s are paired with @Release@s.
Given lemma values, and that @Release@s remove @Value@s from the invariant set, we can see that each generated @Pull@ on a shared resource will only be generated when the invariant set has no @Value@s, and will be similarly paired for @Release@s.

For shared outputs, we just need to verify that the only way the reading machine can reach a reading state is while there is a @Value n@ in the invariant set, and this also follows easily from lemma values.

Finally, we can see as a result of lemma closes and finishes and paths, that since both machines finish or close their inputs and outputs by their @Done@ state, then by the @Done@ of the merged machine, where both machines are also @Done@, all channels must be similarly closed at the end.




%!TEX root = ../Main.tex
\section{Conclusion}
\label{s:Conclusion}

In this paper we have described a fusion algorithm that handles combinators with value-dependent access patterns, such as @merge@.
This is achieved by converting combinators to deterministic finite automata, then combining machines in a way that is similar to parallel execution of the machines.

As our algorithm only takes two machines, an entire combinator program must be fused by repeatedly fusing pairs of machines.
However, the order in which machines are fused can affect whether or not fusion is possible.
For example, given two inputs, maps of the inputs, and zipping the results:

\begin{code}
zipf xs ys
 = let xs' = map (+1) xs
       ys' = map (+1) ys
       zs  = zip xs' ys'
   in  zs
\end{code}

In this example, if the machines for @xs'@ and @ys'@ are fused together first, the fusion will prematurely decide on an ordering of both machines: perhaps all @xs'@ will be computed followed by all @ys'@, perhaps the other way around, or maybe they will be interspersed as we wish.
Then, when this arbitrarily-ordered machine is fused with the @zip@, it will fail as the @zip@ would need to read all the @xs'@ before finding a @ys'@.
It is important to note, however, that the ordering of fusion does not affect the semantics of the result, just whether a result is found.

To ensure a fusion result in this case, we must first fuse @zs@ with one of its inputs, then with the other input.
It is possible to attempt all possible permutations of the order and take the first that succeeds.
We are confident that there exists an algorithm to find a valid ordering, but this is left to future work.


Most dataflow language optimisations tend to focus on fully static networks where the exact access pattern is known at compile time\cite{thies2002streamit}, but disallowing combinators like merge join, append and filter.
At the other side of the spectrum, some dataflow languages such as Lucid\cite{stephens1997survey} focus on expressivity, forgoing any kind of static analyses for optimisations.

Merge joins, appends, and filters are not typically necessary for the bulk kind of operations such as audio transforms, video compression and so on, that regular dataflow has found its applications in\cite{johnston2004advances}.

For querying large data sets merge-like combinators are essential.
The ``MapReduce'' framework has been touted as a solution to easy distribution of workloads, but has been found to be lacking in flexibility\cite{vrba2009kahn}, in favour of Kahn process networks.
Kahn process networks alone are too flexible: many interesting properties we would like to assure, such as the absence of deadlocks, and ability to run in bounded memory, are undecidable.
Regular dataflow languages correspond to a subset of Kahn process networks, restricted to the point of keeping these desired properties\cite{thies2009language}.

In functional languages, fusion systems such as stream fusion\cite{coutts2007stream} and fold/build fusion\cite{jones2001playing} rely on the inliner to move producers into their consumers, after which rewrite rules can remove intermediate allocations.
This short-cut fusion works well for vertical fusion when producers have only single consumers, but when a producer is used by multiple consumers it cannot be inlined without duplicating work, so fusion will not occur.
Our earlier work on flow fusion\cite{lippmeier2013data} is able to fuse these multiple consumer cases, but only for a small set of combinators; neither @append@ nor @merge@ are handled.

Our goal is to extend the regular dataflow languages enough to allow this subset of dynamic combinators, while keeping the desired properties, for the purpose of compilation and optimisation.




% %!TEX root = ../Main.tex
\section{Related work}
\label{s:Related}

\subsection{StreamIt}
StreamIt\cite{thies2002streamit} uses synchronous dataflow (SDF), which means completely static rates, so the exact number of input elements and output elements for each combinator is always known.
For example, an audio lowpass filter may have an input of one, an output of one, and lookbehind, or peek, of ten.
Static rates make scheduling with bounded buffers easier, but disallow rate-changing operations like predicate-matching filters, or merges.
Translating from StreamIt parlance to functional programming combinators, a ``stateless filter'' corresponds to a map, and a ``stateful filter'' corresponds to a scan.

I think StreamIt extends SFD with asynchronous peeks?

StreamIt's peeking lets you write some stateful computations without state, which can then be parallelised.
It feels a little bit like how requiring an associative operator to fold allows parallelisation that still looks like state.
I don't think the details of these optimisations are particularly important right now.
\cite{gordon2010compiler}



``We plan to support dynamically changing rates in the next version of StreamIt''\cite{thies2002streamit} (I have found no recent evidence of this)

\subsubsection{Dynamic Expressivity with Static Optimization for Streaming Languages (streamit slides)}
Talks about dynamic scheduling: the graph is partitioned into static subgraphs, so that all dynamic edges cross partition boundaries. These static subgraphs are then compiled into kernels as normal, and an overall dynamic scheduler is used.
Nothing about detecting deadlocks or buffer overruns.

\subsubsection{W. Thies, Language and compiler support for stream programs, 2009}
Kahn Process Networks...
``It is undecidable to statically determine the amount of buffering ... or to check whether the computation might deadlock.''\cite{thies2009language}

Synchronous dataflow has static rates, and can therefore find a valid scheduling at compiletime.

CSP has synchronous messages, blocking until a reply message is sent.

StreamIt has a structured way of creating graphs: only splitjoins, vertical (sequential) pipelines, and feedback loops. A bit like a series-parallel graph, with feedbacks added.

\begin{quote}
Boolean dataflow [HL97] is a compromise between these two extremes; it computes a parameterized schedule
of the graph at compile time, and substitutes runtime conditions to decide which paths are taken.
The performance is nearly that of synchronous dataflow while keeping some flexibility of dynamic
dataflow.
\end{quote}


\subsection{Lava}
I don't think Lava is relevant, but Obsidian might be.

\subsection{Advances in Dataflow Programming Languages, W Johnston, Hanna}
Starts mainly about dataflow hardware\cite{johnston2004advances}.
Most disadvantages and criticisms of early dataflow, and hence its decline, can be attributed to the hardware models themselves, rather than the languages.

Static dataflow architecture:
seems like it would be sufficient for us. Check references on what's possible:

Synchronous dataflow (SDF) requires statically known input/output rates, which is not sufficient for us, although I think if you could annotate rates with ``Maybes'' it would be (but suspect that would lose the nice properties of SDF).

I had these citations highlighted but I don't remember why.
\begin{enumerate}
\item
DAVIS, A. L. 1978. The architecture and system method of DDM1: A recursively structured data driven machine. In Proceedings of the 5th Annual Symposion on Computer Architecture (New York). 210–215.

\item
DENNIS, J. B. AND MISUNAS, D. P. 1975. A prelimi- nary architecture for a basic data-flow processor. In Proceedings of the Second Annual Symposium on Computer Architecture. 126–132.

\item
DENNIS, J. B. 1980. Data flow supercomputers. IEEE Comput. 13, 11 (Nov.), 48–56.

\item
DENNIS, J. B. 1974. First version of a data flow pro- cedure language. In Proceedings of the Sympo- sium on Programming (Institut de Programma- tion, University of Paris, Paris, France). 241– 271.

\item
SILC, J., ROBIC, B., AND UNGERER, T. 1998. Asyn- chrony in parallel computing: from dataflow to multithreading. Parallel Distrib. Comput. Pract. 1, 1, 3–30.
\end{enumerate}

\subsection{A survey of stream processing, Robert Stephens, 1997}
This is looking for a general theory of stream processing - sounds useful\cite{stephens1997survey}.
Highlights that dataflow SPSs can be seen as an implementation of abstract STs.
One of dataflow's main aims has always been to avoid the ``von Neumann bottleneck'' by allowing parallelism.

Classifies SPSs along three axes: asynchronous or synchronous; deterministic or non; and uni-directional or bidirectional channels. StreamIt, for example, fits into SDU.
(Although another paper said that one way to implement asynchronous was to use bidirectional channels?)

Functional stream languages:
``several specialized stream orientated functional languages have been developed including ARTIC (see [60]), HOPE (see [52]) and RUTH (see [98])''

FOCUS: ``Within such networks data is exchanged via unbounded FIFO channels that are modelled as streams.''
Unbounded channels, of course, is exactly what we don't want.


Reactive systems: it talks about real time systems such as operating systems, which seem like they would actually be closest to what we need.

\subsubsection{Section 8}
Section 8.2 is a particularly confusing definition of standard list combinators.

Section 8.3 has some prolog definitions of more list combinators, and then some higher-order versions of the combinators.
There is a combinator called ``Merge'', which is actually an interleaving of every second element:

\begin{code}
merge :: [a] -> [a] -> [a]
merge (a:a':as) (b:b':bs)
 = a : b' : merge as bs
\end{code}

However, this and append are the only combinators with multiple stream inputs.

\subsubsection{Lucid}
Section 9.4 is about Lucid. Lucid has some more interesting combinators which take multiple streams.

\begin{code}
attime :: (Time -> a) -> (Time -> Time) -> (Time -> a)
attime as ts t
 = as (ts t)

upon :: [a] -> [Bool] -> [a]
upon (a:as) bs
 = a : rest (a:as) bs

rest (a:as) (b:bs)
 | b
 = head as : rest as bs
 | otherwise
 = a : rest (a:as) bs
\end{code}

So I imagine that these combinators could actually be used for interesting merges.
However, there is no mention of whether compilation without unbounded buffers can be achieved.
It looks like primitive operators such as @nor@ are essentially @zipWith (nor)@.


\subsubsection{LUSTRE}
LUSTRE is related to Lucid, but requires causality and synchronicity.
Their deadlock checking does reject some valid programs.
The primitive combinators are: previous (called Z in signal processing?), followed by (something like @head a ++ tail b@), when (find the next\emph{[sic?]} true value in bool stream and use value then) and current (like a past-looking when, except for initial value find future).
These seem to destroy causality.

This looks promising, but I don't know whether our merges will be allowed by the deadlock checker.

\subsubsection{SIGNAL}
Could be worth looking into. Synchronous. Not sure about deadlock and buffer detection etc.


\subsection{A Co-iterative Characterization of Synchronous Stream Functions}
This is \cite{caspi1998co}, Paul Caspi and Marc Pouzet.

Coiteration for streams: giving a stream an initial state, and a transition function from state to pair of value and new state.

First, looking at length-preserving functions, then non-length preserving such as filter.
Starts off very similar to stream fusion, but the catch is that most streams based on other streams don't actually care about the input stream's guts (transition function), but actually just care about the values!

So instead of

\begin{code}
data Stream a s = (s, s -> (s,a))
map :: Stream a s -> Stream a s
map = ...
\end{code}
we have
\begin{code}
data Stream i o s = (s, i -> s -> (s,o))
map :: (a -> b) -> Stream a b s
map = ...
\end{code}
but these can only handle length-preserving functions.


Definition of a synchronous stream function:
\begin{code}
co_apply :: CStream (a -> Maybe s -> (b, Maybe s))
         -> CStream a
         -> CStream b

 f : Stream a -> Stream b
is synchronous iff there exists
 f' : Stream (a -> s -> (b, s))
such that
 f == co_apply f'
\end{code}

Lambda and recursion are a bit confusing. Read again.

Now, the result type of each CStream becomes @F a s@ instead of just @(a, s)@ where @F a s@ is defined:
\begin{code}
data F a s
 = P a s
 | S   s
\end{code}
which is equivalent to @(Maybe a, s)@. They then write an @extend@ (@map apply@) combinator which consumes two inputs at once, but throws an error if one argument yields when the other doesn't. Put another way, ``if the clocks of the two arguments are the same'', where the clock is @map isJust@.

Their definition of merge is interesting. The list version they give is
\begin{code}
merge (False:cs) xs (y:ys) = y : merge cs xs ys
merge (True :cs) (x:xs) ys = x : merge cs xs ys
\end{code}
which only pulls from one of the true or false streams at a time.
However, the co-iterative version pulls from all streams at each iteration, but requires that only one of the true or false streams actually produce a value: if both produce a value, it is a runtime error.
It feels like there is a step missing between the two.

My initial thought is that this clock test is a bit too restrictive for us, because we could store \emph{one} value in a buffer if the two clocks don't exactly line up.

The actual clock calculus looks very similar to a regular type system, with fairly minor and understandable differences.


\subsection{Lucy-n: a n-Synchronous Extension of Lustre}
This is Pouzet et al again\cite{mandel2010lucy}.
While the synchronous model requires no buffering, n-synchrony relaxes this by allowing communication through buffers whose size is known at compile-time.
They use a similar clock calculus to before, but extend it with subtyping. Buffering is introduced through an explicit @buffer@ primitive, but the required buffer size is inferred.


What is ``Cyclo Static Data Flow''?
Apparently ``synchronous languages'' like Lustre and Lucid Synchrone are not quite as flexible as SDF.
In synchronous languages, two streams can be composed (zipped?) only if their clocks are \emph{exactly} the same. 
N-synchronous relaxes this, and allows composition of streams with unequal clocks, but that can be synchronised with a finite, statically known sized, buffer.

``In this paper, we restrict the clock language ce to define ultimately periodic boolean sequences only''
ie infinite repetitions with some finite prefix.

It looks like the kernel is missing the @on@ operator: @s on e@ which is like a filter. It does have @e when ce@ but here @ce@ is a clock, so can only be a repeating boolean sequence. No - it does have @on@, but it takes a clock instead of an expression.
For some clocks @a@ and @b@, @a on b@ is a sub-clock of @a@.

Section 5 is where it gets interesting: ``In order to overcome this complexity, we propose in this paper not to consider exact periodic clocks but their abstraction''.
They abstract over the clocks with $\langle b^0, b^1 \rangle (r)$, where $b$s are the minimum and maximum shift of @1@s, and $r$ is the proportion of @1@s to @0@s.
So, basically, $r$ is the slope of the line of @1@s. The resulting abstract slope of @a on b@ can be found quite easily.
Two clocks can only be scheduled together if their $r$s are the same, that is they have the same slope.
The size of the buffer needed is the difference between two of the $b$s.

\subsection{Compile-Time Scheduling of Dynamic Constructs in Dataflow Program Graphs}
This is Ha and Lee\cite{ha1997compile}.
``The main purpose of this paper is to show how we can define the profiles of dynamic constructs at compile-time''.

They use simulation or programmer pragmas to approximate the runtime statistics.
It looks like this is actually just about dynamic length/runtimes of certain nodes, rather than dynamic dependencies.
I don't think this is relevant.


\subsection{Scheduling dynamic dataflow graphs with bounded memory using the token flow model}
This is Buck\cite{buck1993scheduling}.

Petri nets: Petri nets are directed graphs. Vertices are split into \emph{places} and \emph{transitions}.

Safeness: a Petri net with an initial marking $\mu$ is safe if it is not possible, by any sequence of transition firings, to reach a new marking $\mu'$ where any place has more than one token.
Adding backwards acknowledgement arcs can force safeness (but I suspect can ruin liveness).

Boundedness: a generalisation of safety. A place is \emph{k-bounded} if the number of tokens never exceeds $k$.

Liveness: the avoidance of deadlock (where nothing can fire).

Conservativeness: strict conservatism is when the number of tokens is never changed by firing. Conservative with respect to a weight vector $w$ where the weighted sum of number of tokens never changes.

Karp and Miller computation graphs: nodes are operations and arcs are queues of data.
Each arc $d_p$ has four natural numbers: $A_p$ size of initial queue, $U_p$ output size of arc's in, $W_p$ input size of arc's out, $T_p$ minimum queue length necessary for out to execute (peek?).
These computation graphs are determinate (if their functions are). They specify the conditions for termination (rather than deadlock-free!).

Marked graphs: a subset of Petri nets. Every place has exactly one input transition and one output transition. No parallel arcs. I guess a transition can have multiple output places though.
For a given place, only one output transition exists, so if the place is full there is only one choice of what to execute.
Marked graphs are much easier to analyse than general Petri nets: deadlocks cannot occur on cycles with at least one token going through them.

Homogeneous dataflow graphs: where all nodes, in each firing, consume exactly one token from each input arc and produce one token on each output arc. This is corresponds to a marked graph.

Synchronous, or regular, dataflow: generalisation of homogeneous where consumption or production can be different to one, but still constant and known.

Dynamic dataflow: where the number of tokens consumed or produced depends on values of certain input tokens (eg SWITCH/SELECT). This is Turing complete, so more powerful than Petri nets.
Nondeterministic merge is another extension.

Kahn: blocking reads etc, so no nondeterministic merge.

Regular dataflow techniques cannot handle dynamic dependencies, but sometimes conditional branching can be emulated with conditional assignment (executing both sides) but this is not always appropriate.
There are some extensions to regular dataflow to allow certain dynamic dependencies.

Control flow / dataflow hybrid: Turing complete, so hard to analyse.
Standard compiler program block DAGs.

Controlled use of dynamic actors: dynamic actors in general are Turing complete, but restricting dynamic actors can make analysis feasible. For example, where all conditionals are later merged with the same flag.

Clock calculus: as in LUSTRE etc.

Token flow model: an extension of regular (synchronous) dataflow allowing dynamic actors such as SWITCH and SELECT.
Boolean-controlled dataflow (BDF): each port is annotated with a symbolic expression (rather than just a constant) denoting how many values are consumed or produced.
For example, the SWITCH inputs are both still annotated with $1$, but the outputs are changed to $p_i$ and $1 - p_i$, meaning only one of the outputs will produce one value at any time.
As with regular graphs, a topology matrix is created with these symbolic expressions instead of constants, as a function of the $p$s.
If the topology matrix function has a nontrivial solution that \emph{does not} depend on the values of $p$s, it is strongly consistent: no matter what, it will be able to be scheduled.
If there are nontrivial solutions that exist only for particular $p$s, it is weakly consistent: schedules would work for only some sets of data.
(However, this analysis is undecidable in general)
The actual value of these $p_i$ values is not important, it's just some abstract symbolic probability, however they can be treated as the fraction of iterations that have values.
Strong consistency alone does not assure bounded memory.

Chapter 3 may be worth rereading, but chapter 4 is just implementation details.

Chapter 5 is extending BDF: an extension to boolean-controlled dataflow that generalises the boolean controls to integers (IDF). Not too worried.

\subsection{Generic Programming with Adjunctions}
This is Hinze\cite{hinze2012generic}.

``For instance, append does not have the form of a fold as it takes a second argument that is used later in the base case.''
An adjoint fold is a generalisation, by somehow allowing the argument of a fold to be wrapped in a functor application.
The functor must have a right adjoint, or left adjoint for unfolds.

Skipping to S3.



\subsection{Conjugate Hylomorphisms}
This is Gibbons\cite{gibbons2015conjugate}.


\subsection{Coroutines and networks of parallel processes}
This is the original Kahn network paper\cite{kahn1976coroutines}.
It's not actually particularly interesting, and just explains processes as coroutines, then goes through a few example programs.
Interesting historically as an early (1976?) example of yearning for purity and such.

\subsection{Bounded scheduling of process networks}
This is \cite{parks1995bounded}.
About finding a bounded execution of Kahn processes, which is undecidable, but this choice quote is hardly illuminating:
``Fortunately, because we are interested in programs that never terminate, our scheduler has infinite time and can guarantee that programs execute forever with bounded buffering whenever
possible.''

``We can let the scheduler work as the program executes. Because the program is designed to run forever without terminating, the scheduler has an infinite amount of time to arrive at an answer''
I don't quite understand this. Presumably you use a default execution to start with, but if the default execution is fast enough, why bother.
It seems like it just chooses some bound and executes a slightly modified graph with that bound, and if it fails, tries with a larger bound.

\subsection{Kahn process networks are a flexible alternative to MapReduce}
MapReduce alone is a bit restrictive, but Kahn processes are more flexible and still composable and whatnot\cite{vrba2009kahn}.
Does runtime deadlock detection when a processes blocks on a send.
Not much really conceptual work here.




\section*{Acknowledgements}
The audience of the inaugural SAWDAP workshop, especially Thomas Sutton and Patryk Zadarnowski, were very helpful in an early version of this work.

\bibliographystyle{plain}
\bibliography{Main}

\end{document}


