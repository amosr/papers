\documentclass{sigplanconf}
\usepackage{amssymb}
\usepackage{graphicx}
\usepackage{amsmath}
\usepackage{mathptmx}
\usepackage{stmaryrd}
\usepackage{hyperref}
\usepackage{alltt}
\usepackage{url}
\usepackage{float}
\usepackage{style/utils}
\usepackage{style/code}
\usepackage{style/proof}
\usepackage{style/judgements}

% -----------------------------------------------------------------------------
\begin{document}

% \exclusivelicense
% \conferenceinfo{}{}
% \copyrightyear{2015}
% \copyrightdata{}
\doi{}
% \pagenumbering{gobble} 

\title{Merging merges, more or less}

\authorinfo{
  Amos Robinson$^\dagger$ 
  \and Ben Lippmeier$^\dagger$
  \and Gabriele Keller$^\dagger$ 
  \and Manuel Chakravarty$^\dagger$ 
}{
  \vspace{5pt}
  \shortstack{
    $^\dagger$Computer Science and Engineering \\
    University of New South Wales, Australia \\[2pt]
    \textsf{\{amosr,benl,keller,chak\}@cse.unsw.edu.au}
  }
}

\maketitle
\makeatactive

\begin{abstract}
Fusion is an essential part of optimising high-level array computations.
Fusion systems for combinators tend to focus on combinators with data-independent access patterns such as @zip@, @filter@ and @map@, while ignoring @merge@ as in merge sort.
In this paper, we show that describing combinators as deterministic finite automata allows us to express more combinators, while leading to a fusion algorithm based on intersection of automata.
\end{abstract}


\category
	{D.3.4}
	{Programming Languages}
	{Processors---Compilers; Optimization}

\terms
	Languages, Performance

\keywords
	Arrays; Fusion; Haskell

%!TEX root = ../Main.tex
\section{Introduction}

Data flow fusion~\cite{lippmeier2013flow} is a technique to compile a specific class of data flow programs into single, efficient imperative loops. This process of ``compilation'' is equivalent to performing array fusion on a combinator based functional array program, as per related work on stream fusion~\cite{coutts2007streamfusion} and delayed arrays~\cite{keller2010repa}. The key benefits of data flow fusion over this prior work are: 1) it fuses programs that use branching data flows where a produced array is consumed by several consumers, and 2) complete fusion into a single loop is guaranteed for all programs that operate on the same size input data, and contain no fusion-preventing dependencies between operators.

Fusion-preventing dependencies express the fact that some operators simply must wait for others to complete before they can produce their own output. For example, in the following:
\begin{code}
  normalize :: Array Int -> Array Int
  normalize xs = let sum = fold (+) 0 xs
                 in  map (/ sum) xs
\end{code}

If we wish to divide every element of an array by the sum of all elements, then it seems we are forever destined to compute the result using at least two loops: one to determine the sum, and one to divide the elements. The evaluation of @fold@ demands all elements of its source array, and we cannot produce any elements of the result array until we know the value of @sum@. 

However, many programs \emph{do} contain opportunities for fusion, if we only knew which opportunities to take. The following example offers \emph{several} unique, but mutually exclusive approaches to fusion. Figure~\ref{f:normalize2-cluterings} on the next page shows some of the possibilities.
\begin{code}
 normalize2 :: Array Int -> Array Int
 normalize2 xs
  = let sum1 = fold   (+)  0   xs
        gts  = filter (> 0)    xs
        sum2 = fold   (+)  0   gts
        ys1  = map    (/ sum1) xs
        ys2  = map    (/ sum2) xs
    in (ys1, ys2)
\end{code}

In Figure~\ref{f:normalize2-cluterings}, the dotted lines show possible clusterings of operators. Stream fusion implicitly choses the solution on the left as its compilation process cannot fuse a produced array into multiple consumers. The best existing ILP approach will chose the solution on the right as it cannot cluster operators that consume arrays of different lengths. Our system choses the solution in the middle, which is also optimal for this example. 

% NOTE: This set of bullets needs to fit on the first page, without spilling to the second.
Our contributions are as follows:
\begin{itemize}
\item   
We extend prior work by Megiddo~\cite{megiddo1998optimal} and Darte~\cite{darte2002contraction}, with support for length changing operators. Length changing operators can be clustered with the operators that generate their source arrays, and compiled naturally with data-flow fusion (\S\ref{s:ILP}).

\item
We present a simplification to constraint generation that is also applicable to some existing integer linear programming formulations such as Megiddo's,
where constraints between two nodes need not be generated if there exists a fusion-preventing path between the two (\S\ref{s:OptimisedConstraints}).

\item
Our constraint system also encodes a total ordering on the cost of clusterings, expressed using weights on the integer linear program. For example, we encode that memory traffic is more expensive than loop overheads, so given a choice between the two, the memory traffic will be reduced (\S\ref{s:ObjectiveFunction}).

\item
We present benchmarks of our algorithm applied to several common programming patterns, and to several pathological examples.
Our algorithm is complete and yields good results in practice, though if array sizes are unknown, an optimal solution is uncomputable in general. \TODO{ref}
\end{itemize}

The reduction of the clustering problem to integer linear programming was previously described by~\cite{megiddo1998optimal}, though they do not consider length changing operators.


% We must also decide which clustering is the `best' or most optimal. One obvious criterion for this is the minimum number of loops, but there may even be multiple clusterings with the minimum number of loops. In this case, the number of required manifest arrays must also be taken into account. 

% As real programs contain tens or hundreds of individual operators, performing an exhaustive search for an optimal clustering is not feasible, and greedy algorithms tend to produce poor solutions. 


%!TEX root = ../Main.tex
\section{Streams and combinators}
\label{s:Streams}
First, we define an abstraction of two-input combinators, very similar to stream fusion\CITE or co-iterative stream functions of Caspi and Pouzet\cite{caspi1998co}.
We present a simplified stateless version here, which would not be able to express all useful combinators (such as segmented scan) but suffices for merge, append and zip.
We also only treat combinators that consume both input streams linearly, which excludes combinators like cross product or gather.
In Coq\footnote{The definitions have been reordered for clarity}:

\begin{code}
Definition Stream2 (a b c : Set)
 := option a -> option b -> Step2 c

Inductive Step2 c :=
 | MkStep2 : Move -> Move -> option c -> Step2 c
 | Done2   : Step2 c.

Inductive Move :=
 | Forward
 | Stay.
\end{code}

At each evaluation step, a @Stream2@ is called with the value at each input stream, or @None@ if the input stream is finished.
The return of each @Stream2@ evaluation step can be either a @MkStep2@ indicating that further processing is possible, or @Done2@, in which case the combinator has finished.
A @MkStep2@ contains two @Move@s, which designate which input streams should be consumed and the next value pulled, and which should stay where they are.
@MkStep2@ also contains an optional output value.

The main difference from stream fusion is that the coordination between streams is lifted out.
In stream fusion, each stream transformer takes a stream as input and evaluates it itself.
This means that, since the stream evaluation is local to a stream, scheduling decisions can only be made locally.
These local schedules cannot guarantee desired properties like bounded memory, as multiple streams may need to be stepped together if they use the same input.
Instead, the co-iterative version is evaluated by an outside scheduler, which can find an evaluation order that satisfies these properties.

For single-input combinators such as map, filter and scan, @Stream1@ is defined below.
\begin{code}
Definition Stream1 (a b : Set)
 := option a -> Step2 b

Inductive Step1 b :=
 | MkStep1 : Move -> option b -> Step1 b
 | Done1   : Step1 b.
\end{code}


\subsection{Examples}

Below, we show example combinators.
The first combinator pairwise zips two streams together, consuming both streams in lock-step.

\begin{code}
Definition Zip (A B : Set)
 :  Stream2 A B (A*B)
 := fun a b =>
    match (a,b) with
    | (Some a, Some b)
    => MkStep2 Forward Forward (Some (a,b))
    | _               
    => Done2
    end.
\end{code}

Next, append consumes from the left stream until it is empty, then consumes the right stream.
\begin{code}
Definition Append (A : Set)
 :  Stream2 A A A
 := fun a b =>
    match (a,b) with
    | (Some a, _)
    => MkStep2 Forward Stay    (Some a)
    | (None,   Some b)
    => MkStep2 Stay    Forward (Some b)
    | (None,   None)               
    => Done2
    end.
\end{code}

As a final example, the merge join inspects both its input elements, and consumes the one with the smallest value.
There are many variations on this theme, but this particular version that outputs all values in order is actually equivalent to a segmented append.

\begin{code}
Definition MergeJoin (A : Set)
 :  Stream2 (nat*A) (nat*A) (nat*A)
 := fun a b =>
    match (a,b) with
    | (Some (ia,a), Some (ib, b))
    => if   le_dec ia ib
       then MkStep2 Forward Stay (Some (ia, a))
       else MkStep2 Stay Forward (Some (ib, b))
    | (Some (ia,a), None)
    =>      MkStep2 Forward Stay (Some (ia,a))
    | (None, Some (ib,b))
    =>      MkStep2 Stay Forward (Some (ib,b))
    | (None, None)
    =>      Done2
    end.
\end{code}



\subsection{Kahn process networks}

We define a simplified Kahn process network that contains no cycles (recursion). 

\begin{code}
Inductive Graph
 : list Set -> list Set -> list Set -> Type :=
\end{code}
The graph is parameterised by the list of inputs, the list of interior nodes, and the list of outputs.

\begin{code}
 | Empty  : Graph [] [] []

 | Inp    : forall a i n o,
            Graph i n o
         -> Graph (a::i) (a::n) o

 | Out    : forall a i n o,
            Index n a
         -> Graph i n o
         -> Graph i n (a::o)
\end{code}

A @Graph@ can be @Empty@, in which case there are no inputs, nodes or outputs.
The @Inp@ constructor takes an existing graph, and adds an input node to the start.
Because the graph cannot be recursive, the inner graph of @Inp@ cannot refer to the newly created @Inp@.
Similarly, @Out@ takes an existing graph, and an index into the list of interior nodes, and marks that node as an output.

\begin{code}
 | Trans1 : forall a b i n o,
            Index n a
         -> Stream1 a b
         -> Graph i n o
         -> Graph i (b::n) o.

 | Trans2 : forall a b c i n o,
            Index n a
         -> Index n b
         -> Stream2 a b c
         -> Graph i n o
         -> Graph i (c::n) o.
\end{code}
The two transition constructors, @Trans1@ and @Trans2@ take @Stream1@ and @Stream2@ respectively, which defines the actual computation of the transition. They also extend existing graphs, and their inputs must be in the existing graph, again disallowing recursive edges or cycles.



%!TEX root = ../Main.tex
\section{Merging machines}
\label{s:Merging}

This section describes how machines can be merged together to form a single machine that computes both.

The idea is to create a new machine with the product of the two states; at each state, one or the other of the machines may be able to take a step.
If so, we update that machine's state and leave the other one as-is.
For shared inputs or outputs, both machines must take the steps at the same time: if they both pull from the same input, they must pull at the same time.
Similarly, if one machine produces an input and the other reads it, the first machine can only produce its output if the second is pulling on it.

If neither machine can make a move, the program is disallowed.
For example, if the first machine is trying to read from a shared input while the second tries to read from the first machine's input, executing it may require buffering the input until the second machine catches up.

However, the above explanation is too strict and outlaws programs such as @zip xs ys@ merged with @zip ys xs@, as neither machine can progress until the other does, leading to deadlock.
To resolve this, we add to the resulting machine's state a set of pending events for each machine.
The first machine may read from a shared input as long as there is not already a ``read'' event pending for the other machine to deal with.
Reading from a shared input adds a pending ``read'' event to the other machine, and the other machine may then skip past a @Pull@ for that source, as it is known that the other machine has already pulled on it.

\subsection{General merging}
Let us define the potential pending events on a machine.
One machine may have read from a shared input, but the other has not.
Similarly, if one machine has tried to read from a shared input but the input is empty, when the other machine attempts to read it can skip directly to the @None@ case.

The other cases are when one machine produces a value, and another consumes it.
Here, the producer may have produced a value, or may have finished producing on that channel.

\begin{tabbing}
MM \= MM \= MMMMMM \= M\kill
$\Psi$ \> $=$  \> @Value@     \> $n$         \\
       \> $~|$ \> @Finished@  \> $n$         \\
\end{tabbing}

The @merge@ function takes two machines with states $l_1$ and $l_2$, and produces a new machine with state $l_1 \times l_2 \times \{\Psi\}$.
It starts by creating an empty machine, and inserting states one by one, starting from the initial state as the initial state of each machine, and empty pending sets.
Then, for each output transition of the input machines, states are recursively added.

\begin{tabbing}
MMMMMMM \= MM \= MMMMMM \= M\kill
@merge@ \> $:$ \> $@Machine @l_1 \times @Machine @l_2$ \\
        \> $\to$ \> $@Machine @(l_1 \times l_2 \times \{\Psi\})$ \\
\\
@merge @$m_1~m_2$ \> $=$ \> $@merge@'~(@init @m_1)~(@init @m_2)~\{\}~@empty@$ \\
\\
$@merge@'$ \> $:$ \> $l_1 \times l_2 \times \{\Psi\} \times @Machine @(l_1 \times l_2 \times \{\Psi\})$ \\
           \> $\to$ \> $@Machine @(l_1 \times l_2 \times \{\Psi\})$ \\
\\
MMM \= M \= MMMMMMMMMM \= M \=\kill
$@merge@'~s_1~s_2~\psi~m$ \\
 \> $~|$ \> $(s_1, s_2, \psi) \in m$  \\
 \> $=$  \> $m$ \\
 \> $~|$ \> $(\gamma,\delta) \in @move@~s_1~s_2~\psi$ \\
 \> $=$  \> $@fold@~@merge@'~(m \cup \gamma \cup \delta)~\delta$ \\
\end{tabbing}

The $@merge@'$ function adds the given state and recursively adds the successor states to the given machine.
If the given state has already been added --- for example the state is reachable by multiple states --- the state need not be added.

\begin{tabbing}
MMM \= M \= MMMMMMM\kill
$@move@$ \> $:$ \> $l_1 \times l_2 \times \{\Psi\}$ \\
           \> $\to$ \> $\Gamma \times \{\Sigma \times (l_1 \times l_2 \times \{\Psi\})\}$ \\
\end{tabbing}

The @move@ function computes the states and transitions of the merged machine.
These are added to the machine by $@merge@'$ above.

\begin{tabbing}
MMM \= M \= MMMMMMM \= M \=\kill
 \> $~|$ \> $@Update@~f~n$ \> $=$ \> $@state@~m_1~s_1$ \\
 \> $\wedge$ \> $s_1'$     \> $=$ \> $@trans@~m_1~s_1~@Unit@$ \\
 \> $=$ \> $(@Update@~f~n,~\{(@Unit@, (s_1',s_2,\psi))\})$ \\
\\
 \> $~|$ \> $@Update@~f~n$ \> $=$ \> $@state@~m_2~s_2$ \\
 \> $\wedge$ \> $s_2'$     \> $=$ \> $@trans@~m_2~s_2~@Unit@$ \\
 \> $=$ \> $(@Update@~f~n,~\{(@Unit@, (s_1,s_2',\psi))\})$ \\
\end{tabbing}

If either of the machines are attempting to update their local state, this can be done easily without interfering with the other machine.

\begin{tabbing}
MMM \= M \= MMMMMMM \= M \=\kill
 \> $~|$ \> $@Skip@$ \> $=$ \> $@state@~m_1~s_1$ \\
 \> $\wedge$ \> $s_1'$     \> $=$ \> $@trans@~m_1~s_1~@Unit@$ \\
 \> $=$ \> $(@Skip@,~\{(@Unit@, (s_1',s_2,\psi))\})$ \\
\\
 \> $~|$ \> $@Skip@$ \> $=$ \> $@state@~m_2~s_2$ \\
 \> $\wedge$ \> $s_2'$     \> $=$ \> $@trans@~m_2~s_2~@Unit@$ \\
 \> $=$ \> $(@Skip@,~\{(@Unit@, (s_1,s_2',\psi))\})$ \\
\end{tabbing}

Skips are dealt with in the same way, affecting only one machine.

\begin{tabbing}
MMM \= M \= MMMMMMM \= M \=\kill
 \> $~|$ \> $@If@~f~n$ \> $=$ \> $@state@~m_1~s_1$ \\
 \> $\wedge$ \> $s_1'$     \> $=$ \> $@trans@~m_1~s_1~@True@$ \\
 \> $\wedge$ \> $s_1''$     \> $=$ \> $@trans@~m_1~s_1~@False@$ \\
 \> $=$ \> $(@If@~f~n,~\{(@True@, (s_1',s_2,\psi)), (@False@, (s_1'', s_2, \psi))\})$ \\
\\
 \> $~|$ \> $@If@~f~n$ \> $=$ \> $@state@~m_2~s_2$ \\
 \> $\wedge$ \> $s_2'$     \> $=$ \> $@trans@~m_2~s_2~@True@$ \\
 \> $\wedge$ \> $s_2''$     \> $=$ \> $@trans@~m_2~s_2~@False@$ \\
 \> $=$ \> $(@If@~f~n,~\{(@True@, (s_1,s_2',\psi)), (@False@, (s_1, s_2'', \psi))\})$ \\
\end{tabbing}

Similarly, @If@s only affect one of the machines, leaving the other in its original state.
If we had an oracle for checking functional equivalence, we could check whether two @If@s had the same function and inputs and potentially allow more programs.
However, lacking an oracle, even if both machines have @If@s with the same predicate, we generate a machine for all possible result combinations (all four of them).

\begin{tabbing}
MMM \= M \= MMMMMMM \= M \=\kill
 \> $~|$ \> $@Done@$ \> $=$ \> $@state@~m_1~s_1$ \\
 \> $\wedge$ \> $@Done@$ \> $=$ \> $@state@~m_2~s_2$ \\
 \> $=$ \> $(@Done@,~\{\})$ \\
\end{tabbing}

If both machines are @Done@, the resulting machine will also be @Done@.

The previous cases have been rather trivial as they require no synchronisation between the machines, but the remaining cases of @Pull@, @Release@, @Out@ and @OutDone@ are more complicated.

\begin{tabbing}
MMM \= M \= MMMMMMM \= M \=\kill
 \> $~|$ \> $@Out@~f~n$ \> $=$ \> $@state@~m_1~s_1$ \\
 \> $\wedge$ \> $n$      \> $\in$ \> $@inputs@~m_2$ \\
 \> $\wedge$ \> $@Done@$ \> $=$ \> $@state@~m_2~s_2$ \\
 \> $\wedge$ \> $s_1'$     \> $=$ \> $@trans@~m_1~s_1~@Unit@$ \\
 \> $=$ \> $(@Out@~f~n,~\{(@Unit@, (s_1',s_2,\psi \setminus \{@Value@~n\} ))\})$ \\
\end{tabbing}

If the first machine is producing an output on channel $n$ and the second machine uses $n$, the first machine would usually have to wait until the second machine is ready to accept new data.
However, if the second machine is @Done@, the first machine may output to the channel as frequently as it likes.
Note that the output channel $n$ may have multiple consumers, so continuing to output after the second machine consumer has finished is not as futile as it may seem at first.
\TODO{Checking for @Done@ of machine is too coarse-grained, need to check if $n$ is closed instead.}

\begin{tabbing}
MMM \= M \= MMMMMMM \= M \=\kill
 \> $~|$ \> $@Out@~f~n$ \> $=$ \> $@state@~m_1~s_1$ \\
 \> $\wedge$ \> $n$      \> $\in$ \> $@inputs@~m_2$ \\
 \> $\wedge$ \> $@Value@~n$ \> $\not\in$ \> $\psi$ \\
 \> $\wedge$ \> $s_1'$     \> $=$ \> $@trans@~m_1~s_1~@Unit@$ \\
 \> $=$ \> $(@Out@~f~n,~\{(@Unit@, (s_1',s_2,\psi \cup \{@Value@~n\} ))\})$ \\
\end{tabbing}

Again, if the first machine is producing and the second is consuming, the first machine may only produce if there is no unhandled value on the channel.
This is what $\psi$ is for.
If there is a value in $\psi$, a later case may allow the second machine to pull from it.

\begin{tabbing}
MMM \= M \= MMMMMMM \= M \=\kill
 \> $~|$ \> $@Out@~f~n$ \> $=$ \> $@state@~m_1~s_1$ \\
 \> $\wedge$ \> $n$      \> $\not\in$ \> $@inputs@~m_2$ \\
 \> $\wedge$ \> $s_1'$     \> $=$ \> $@trans@~m_1~s_1~@Unit@$ \\
 \> $=$ \> $(@Out@~f~n,~\{(@Unit@, (s_1',s_2,\psi))\})$ \\
\end{tabbing}

The final case for @Out@ is when the second machine does not use this output, allowing the first machine to output at any time.
The cases where the second machine is on an @Out@ are symmetric to these.

The cases for @OutDone@ proceed similarly, and are also symmetric for the first and second machines.

\begin{tabbing}
MMM \= M \= MMMMMMM \= M \=\kill
 \> $~|$ \> $@OutDone@~n$ \> $=$ \> $@state@~m_1~s_1$ \\
 \> $\wedge$ \> $n$      \> $\in$ \> $@inputs@~m_2$ \\
 \> $\wedge$ \> $@Done@$ \> $=$ \> $@state@~m_2~s_2$ \\
 \> $\wedge$ \> $s_1'$     \> $=$ \> $@trans@~m_1~s_1~@Unit@$ \\
 \> $=$ \> $(@OutDone@~n,~\{(@Unit@, (s_1',s_2,\psi))\})$ \\
 \\
 \> $~|$ \> $@OutDone@~n$ \> $=$ \> $@state@~m_1~s_1$ \\
 \> $\wedge$ \> $n$      \> $\in$ \> $@inputs@~m_2$ \\
 \> $\wedge$ \> $@Value@~n$ \> $\not\in$ \> $\psi$ \\
 \> $\wedge$ \> $s_1'$     \> $=$ \> $@trans@~m_1~s_1~@Unit@$ \\
 \> $=$ \> $(@OutDone@~n,~\{(@Unit@, (s_1',s_2,\psi \cup \{@Finished@~n\}))\})$ \\
 \\
 \> $~|$ \> $@OutDone@~n$ \> $=$ \> $@state@~m_1~s_1$ \\
 \> $\wedge$ \> $n$      \> $\not\in$ \> $@inputs@~m_2$ \\
 \> $\wedge$ \> $s_1'$     \> $=$ \> $@trans@~m_1~s_1~@Unit@$ \\
 \> $=$ \> $(@OutDone@~n,~\{(@Unit@, (s_1',s_2,\psi))\})$ \\
\end{tabbing}

Pulls now.

Releases.

If none of the above cases apply, it means the machines cannot be fused.
Either they require more than one element of buffering, or reasoning about function equality would be required to prove that they do not.
The only unhandled cases above are where the two machines share input or outputs, and there are existing values buffered in $\psi$.
In which case, we give a compile time error indicating that these two machines cannot be fused.




\subsection{Producer-consumer}
The general case above is sufficient, but in some cases can generate larger machines than necessary.
Talk about special case for producer-consumers.

\begin{code}
data Which = LeftExec | RightExec

mergeV :: Machine l1 -> Machine l2
       -> Machine (l1, l2, Which)
\end{code}


%!TEX root = ../Main.tex
\section{Related work}
\label{s:Related}

\subsection{StreamIt}
StreamIt\cite{thies2002streamit} uses synchronous dataflow (SDF), which means completely static rates, so the exact number of input elements and output elements for each combinator is always known.
For example, an audio lowpass filter may have an input of one, an output of one, and lookbehind, or peek, of ten.
Static rates make scheduling with bounded buffers easier, but disallow rate-changing operations like predicate-matching filters, or merges.
Translating from StreamIt parlance to functional programming combinators, a ``stateless filter'' corresponds to a map, and a ``stateful filter'' corresponds to a scan.

I think StreamIt extends SFD with asynchronous peeks?

StreamIt's peeking lets you write some stateful computations without state, which can then be parallelised.
It feels a little bit like how requiring an associative operator to fold allows parallelisation that still looks like state.
I don't think the details of these optimisations are particularly important right now.
\cite{gordon2010compiler}



``We plan to support dynamically changing rates in the next version of StreamIt''\cite{thies2002streamit} (I have found no recent evidence of this)

\subsubsection{Dynamic Expressivity with Static Optimization for Streaming Languages (streamit slides)}
Talks about dynamic scheduling: the graph is partitioned into static subgraphs, so that all dynamic edges cross partition boundaries. These static subgraphs are then compiled into kernels as normal, and an overall dynamic scheduler is used.
Nothing about detecting deadlocks or buffer overruns.

\subsubsection{W. Thies, Language and compiler support for stream programs, 2009}
Kahn Process Networks...
``It is undecidable to statically determine the amount of buffering ... or to check whether the computation might deadlock.''\cite{thies2009language}

Synchronous dataflow has static rates, and can therefore find a valid scheduling at compiletime.

CSP has synchronous messages, blocking until a reply message is sent.

StreamIt has a structured way of creating graphs: only splitjoins, vertical (sequential) pipelines, and feedback loops. A bit like a series-parallel graph, with feedbacks added.

\begin{quote}
Boolean dataflow [HL97] is a compromise between these two extremes; it computes a parameterized schedule
of the graph at compile time, and substitutes runtime conditions to decide which paths are taken.
The performance is nearly that of synchronous dataflow while keeping some flexibility of dynamic
dataflow.
\end{quote}


\subsection{Lava}
I don't think Lava is relevant, but Obsidian might be.

\subsection{Advances in Dataflow Programming Languages, W Johnston, Hanna}
Starts mainly about dataflow hardware\cite{johnston2004advances}.
Most disadvantages and criticisms of early dataflow, and hence its decline, can be attributed to the hardware models themselves, rather than the languages.

Static dataflow architecture:
seems like it would be sufficient for us. Check references on what's possible:

Synchronous dataflow (SDF) requires statically known input/output rates, which is not sufficient for us, although I think if you could annotate rates with ``Maybes'' it would be (but suspect that would lose the nice properties of SDF).

I had these citations highlighted but I don't remember why.
\begin{enumerate}
\item
DAVIS, A. L. 1978. The architecture and system method of DDM1: A recursively structured data driven machine. In Proceedings of the 5th Annual Symposion on Computer Architecture (New York). 210–215.

\item
DENNIS, J. B. AND MISUNAS, D. P. 1975. A prelimi- nary architecture for a basic data-flow processor. In Proceedings of the Second Annual Symposium on Computer Architecture. 126–132.

\item
DENNIS, J. B. 1980. Data flow supercomputers. IEEE Comput. 13, 11 (Nov.), 48–56.

\item
DENNIS, J. B. 1974. First version of a data flow pro- cedure language. In Proceedings of the Sympo- sium on Programming (Institut de Programma- tion, University of Paris, Paris, France). 241– 271.

\item
SILC, J., ROBIC, B., AND UNGERER, T. 1998. Asyn- chrony in parallel computing: from dataflow to multithreading. Parallel Distrib. Comput. Pract. 1, 1, 3–30.
\end{enumerate}

\subsection{A survey of stream processing, Robert Stephens, 1997}
This is looking for a general theory of stream processing - sounds useful\cite{stephens1997survey}.
Highlights that dataflow SPSs can be seen as an implementation of abstract STs.
One of dataflow's main aims has always been to avoid the ``von Neumann bottleneck'' by allowing parallelism.

Classifies SPSs along three axes: asynchronous or synchronous; deterministic or non; and uni-directional or bidirectional channels. StreamIt, for example, fits into SDU.
(Although another paper said that one way to implement asynchronous was to use bidirectional channels?)

Functional stream languages:
``several specialized stream orientated functional languages have been developed including ARTIC (see [60]), HOPE (see [52]) and RUTH (see [98])''

FOCUS: ``Within such networks data is exchanged via unbounded FIFO channels that are modelled as streams.''
Unbounded channels, of course, is exactly what we don't want.


Reactive systems: it talks about real time systems such as operating systems, which seem like they would actually be closest to what we need.

\subsubsection{Section 8}
Section 8.2 is a particularly confusing definition of standard list combinators.

Section 8.3 has some prolog definitions of more list combinators, and then some higher-order versions of the combinators.
There is a combinator called ``Merge'', which is actually an interleaving of every second element:

\begin{code}
merge :: [a] -> [a] -> [a]
merge (a:a':as) (b:b':bs)
 = a : b' : merge as bs
\end{code}

However, this and append are the only combinators with multiple stream inputs.

\subsubsection{Lucid}
Section 9.4 is about Lucid. Lucid has some more interesting combinators which take multiple streams.

\begin{code}
attime :: (Time -> a) -> (Time -> Time) -> (Time -> a)
attime as ts t
 = as (ts t)

upon :: [a] -> [Bool] -> [a]
upon (a:as) bs
 = a : rest (a:as) bs

rest (a:as) (b:bs)
 | b
 = head as : rest as bs
 | otherwise
 = a : rest (a:as) bs
\end{code}

So I imagine that these combinators could actually be used for interesting merges.
However, there is no mention of whether compilation without unbounded buffers can be achieved.
It looks like primitive operators such as @nor@ are essentially @zipWith (nor)@.


\subsubsection{LUSTRE}
LUSTRE is related to Lucid, but requires causality and synchronicity.
Their deadlock checking does reject some valid programs.
The primitive combinators are: previous (called Z in signal processing?), followed by (something like @head a ++ tail b@), when (find the next\emph{[sic?]} true value in bool stream and use value then) and current (like a past-looking when, except for initial value find future).
These seem to destroy causality.

This looks promising, but I don't know whether our merges will be allowed by the deadlock checker.

\subsubsection{SIGNAL}
Could be worth looking into. Synchronous. Not sure about deadlock and buffer detection etc.


\subsection{A Co-iterative Characterization of Synchronous Stream Functions}
This is \cite{caspi1998co}, Paul Caspi and Marc Pouzet.

Coiteration for streams: giving a stream an initial state, and a transition function from state to pair of value and new state.

First, looking at length-preserving functions, then non-length preserving such as filter.
Starts off very similar to stream fusion, but the catch is that most streams based on other streams don't actually care about the input stream's guts (transition function), but actually just care about the values!

So instead of

\begin{code}
data Stream a s = (s, s -> (s,a))
map :: Stream a s -> Stream a s
map = ...
\end{code}
we have
\begin{code}
data Stream i o s = (s, i -> s -> (s,o))
map :: (a -> b) -> Stream a b s
map = ...
\end{code}
but these can only handle length-preserving functions.


Definition of a synchronous stream function:
\begin{code}
co_apply :: CStream (a -> Maybe s -> (b, Maybe s))
         -> CStream a
         -> CStream b

 f : Stream a -> Stream b
is synchronous iff there exists
 f' : Stream (a -> s -> (b, s))
such that
 f == co_apply f'
\end{code}

Lambda and recursion are a bit confusing. Read again.

Now, the result type of each CStream becomes @F a s@ instead of just @(a, s)@ where @F a s@ is defined:
\begin{code}
data F a s
 = P a s
 | S   s
\end{code}
which is equivalent to @(Maybe a, s)@. They then write an @extend@ (@map apply@) combinator which consumes two inputs at once, but throws an error if one argument yields when the other doesn't. Put another way, ``if the clocks of the two arguments are the same'', where the clock is @map isJust@.

Their definition of merge is interesting. The list version they give is
\begin{code}
merge (False:cs) xs (y:ys) = y : merge cs xs ys
merge (True :cs) (x:xs) ys = x : merge cs xs ys
\end{code}
which only pulls from one of the true or false streams at a time.
However, the co-iterative version pulls from all streams at each iteration, but requires that only one of the true or false streams actually produce a value: if both produce a value, it is a runtime error.
It feels like there is a step missing between the two.

My initial thought is that this clock test is a bit too restrictive for us, because we could store \emph{one} value in a buffer if the two clocks don't exactly line up.

The actual clock calculus looks very similar to a regular type system, with fairly minor and understandable differences.


\subsection{Lucy-n: a n-Synchronous Extension of Lustre}
This is Pouzet et al again\cite{mandel2010lucy}.
While the synchronous model requires no buffering, n-synchrony relaxes this by allowing communication through buffers whose size is known at compile-time.
They use a similar clock calculus to before, but extend it with subtyping. Buffering is introduced through an explicit @buffer@ primitive, but the required buffer size is inferred.


What is ``Cyclo Static Data Flow''?
Apparently ``synchronous languages'' like Lustre and Lucid Synchrone are not quite as flexible as SDF.
In synchronous languages, two streams can be composed (zipped?) only if their clocks are \emph{exactly} the same. 
N-synchronous relaxes this, and allows composition of streams with unequal clocks, but that can be synchronised with a finite, statically known sized, buffer.

``In this paper, we restrict the clock language ce to define ultimately periodic boolean sequences only''
ie infinite repetitions with some finite prefix.

It looks like the kernel is missing the @on@ operator: @s on e@ which is like a filter. It does have @e when ce@ but here @ce@ is a clock, so can only be a repeating boolean sequence. No - it does have @on@, but it takes a clock instead of an expression.
For some clocks @a@ and @b@, @a on b@ is a sub-clock of @a@.

Section 5 is where it gets interesting: ``In order to overcome this complexity, we propose in this paper not to consider exact periodic clocks but their abstraction''.
They abstract over the clocks with $\langle b^0, b^1 \rangle (r)$, where $b$s are the minimum and maximum shift of @1@s, and $r$ is the proportion of @1@s to @0@s.
So, basically, $r$ is the slope of the line of @1@s. The resulting abstract slope of @a on b@ can be found quite easily.
Two clocks can only be scheduled together if their $r$s are the same, that is they have the same slope.
The size of the buffer needed is the difference between two of the $b$s.

\subsection{Compile-Time Scheduling of Dynamic Constructs in Dataflow Program Graphs}
This is Ha and Lee\cite{ha1997compile}.
``The main purpose of this paper is to show how we can define the profiles of dynamic constructs at compile-time''.

They use simulation or programmer pragmas to approximate the runtime statistics.
It looks like this is actually just about dynamic length/runtimes of certain nodes, rather than dynamic dependencies.
I don't think this is relevant.


\subsection{Scheduling dynamic dataflow graphs with bounded memory using the token flow model}
This is Buck\cite{buck1993scheduling}.

Petri nets: Petri nets are directed graphs. Vertices are split into \emph{places} and \emph{transitions}.

Safeness: a Petri net with an initial marking $\mu$ is safe if it is not possible, by any sequence of transition firings, to reach a new marking $\mu'$ where any place has more than one token.
Adding backwards acknowledgement arcs can force safeness (but I suspect can ruin liveness).

Boundedness: a generalisation of safety. A place is \emph{k-bounded} if the number of tokens never exceeds $k$.

Liveness: the avoidance of deadlock (where nothing can fire).

Conservativeness: strict conservatism is when the number of tokens is never changed by firing. Conservative with respect to a weight vector $w$ where the weighted sum of number of tokens never changes.

Karp and Miller computation graphs: nodes are operations and arcs are queues of data.
Each arc $d_p$ has four natural numbers: $A_p$ size of initial queue, $U_p$ output size of arc's in, $W_p$ input size of arc's out, $T_p$ minimum queue length necessary for out to execute (peek?).
These computation graphs are determinate (if their functions are). They specify the conditions for termination (rather than deadlock-free!).

Marked graphs: a subset of Petri nets. Every place has exactly one input transition and one output transition. No parallel arcs. I guess a transition can have multiple output places though.
For a given place, only one output transition exists, so if the place is full there is only one choice of what to execute.
Marked graphs are much easier to analyse than general Petri nets: deadlocks cannot occur on cycles with at least one token going through them.

Homogeneous dataflow graphs: where all nodes, in each firing, consume exactly one token from each input arc and produce one token on each output arc. This is corresponds to a marked graph.

Synchronous, or regular, dataflow: generalisation of homogeneous where consumption or production can be different to one, but still constant and known.

Dynamic dataflow: where the number of tokens consumed or produced depends on values of certain input tokens (eg SWITCH/SELECT). This is Turing complete, so more powerful than Petri nets.
Nondeterministic merge is another extension.

Kahn: blocking reads etc, so no nondeterministic merge.

Regular dataflow techniques cannot handle dynamic dependencies, but sometimes conditional branching can be emulated with conditional assignment (executing both sides) but this is not always appropriate.
There are some extensions to regular dataflow to allow certain dynamic dependencies.

Control flow / dataflow hybrid: Turing complete, so hard to analyse.
Standard compiler program block DAGs.

Controlled use of dynamic actors: dynamic actors in general are Turing complete, but restricting dynamic actors can make analysis feasible. For example, where all conditionals are later merged with the same flag.

Clock calculus: as in LUSTRE etc.

Token flow model: an extension of regular (synchronous) dataflow allowing dynamic actors such as SWITCH and SELECT.
Boolean-controlled dataflow (BDF): each port is annotated with a symbolic expression (rather than just a constant) denoting how many values are consumed or produced.
For example, the SWITCH inputs are both still annotated with $1$, but the outputs are changed to $p_i$ and $1 - p_i$, meaning only one of the outputs will produce one value at any time.
As with regular graphs, a topology matrix is created with these symbolic expressions instead of constants, as a function of the $p$s.
If the topology matrix function has a nontrivial solution that \emph{does not} depend on the values of $p$s, it is strongly consistent: no matter what, it will be able to be scheduled.
If there are nontrivial solutions that exist only for particular $p$s, it is weakly consistent: schedules would work for only some sets of data.
(However, this analysis is undecidable in general)
The actual value of these $p_i$ values is not important, it's just some abstract symbolic probability, however they can be treated as the fraction of iterations that have values.
Strong consistency alone does not assure bounded memory.

Chapter 3 may be worth rereading, but chapter 4 is just implementation details.

Chapter 5 is extending BDF: an extension to boolean-controlled dataflow that generalises the boolean controls to integers (IDF). Not too worried.

\subsection{Generic Programming with Adjunctions}
This is Hinze\cite{hinze2012generic}.

``For instance, append does not have the form of a fold as it takes a second argument that is used later in the base case.''
An adjoint fold is a generalisation, by somehow allowing the argument of a fold to be wrapped in a functor application.
The functor must have a right adjoint, or left adjoint for unfolds.

Skipping to S3.



\subsection{Conjugate Hylomorphisms}
This is Gibbons\cite{gibbons2015conjugate}.


\subsection{Coroutines and networks of parallel processes}
This is the original Kahn network paper\cite{kahn1976coroutines}.
It's not actually particularly interesting, and just explains processes as coroutines, then goes through a few example programs.
Interesting historically as an early (1976?) example of yearning for purity and such.

\subsection{Bounded scheduling of process networks}
This is \cite{parks1995bounded}.
About finding a bounded execution of Kahn processes, which is undecidable, but this choice quote is hardly illuminating:
``Fortunately, because we are interested in programs that never terminate, our scheduler has infinite time and can guarantee that programs execute forever with bounded buffering whenever
possible.''

``We can let the scheduler work as the program executes. Because the program is designed to run forever without terminating, the scheduler has an infinite amount of time to arrive at an answer''
I don't quite understand this. Presumably you use a default execution to start with, but if the default execution is fast enough, why bother.
It seems like it just chooses some bound and executes a slightly modified graph with that bound, and if it fails, tries with a larger bound.

\subsection{Kahn process networks are a flexible alternative to MapReduce}
MapReduce alone is a bit restrictive, but Kahn processes are more flexible and still composable and whatnot\cite{vrba2009kahn}.
Does runtime deadlock detection when a processes blocks on a send.
Not much really conceptual work here.




\section*{Acknowledgements}
The audience of the inaugural SAWDAP workshop, especially Thomas Sutton and Patryk Zadarnowski, were very helpful in an early version of this work.

\bibliographystyle{plain}
\bibliography{Main}

\end{document}


