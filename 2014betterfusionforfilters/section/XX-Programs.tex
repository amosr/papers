\section{Motivating programs}
\subsection{Quickselect}
We can't quite do @quickselect@, sadly; it requires @head@.
\begin{code}
select :: Vec a -> Int -> a
select xs k
 | length xs > 0
 = let p    = head xs 
       xslt = filter  (<p) xs
       xsge = filter (>=p) xs
   in  if     length xslt > k
       then   select xsge (k - length xslt)
       else   select xslt  k
 | length xs == 1
 = head xs
 | length xs == 0
 = error "Empty array!"

median :: Vec a -> a
median xs
 = select xs (length xs `div` 2)
\end{code}

\subsection{Closest pairs}
Closest pairs makes use of fast-ish @median@ to ensure balanced division of work.
\begin{code}
closest :: Vec Pt -> (Pt,Pt)
closest pts
 | length pts < 250
 = naive pts
 | otherwise
 = divide pts

divide :: Vec Pt -> (Pt,Pt)
divide pts
 = let p      = median pts
       aboves = filter (above p) pts
       belows = filter (below p) pts
       above' = closest aboves
       below' = closest belows

       border = min (distance above') (distance below')

       aboveB = filter (above (p - border)) pts
       belowB = filter (below (p + border)) pts

       cs     = cross aboveB belowB
       bord   = minBy distance cs
   in  above' `minDist` below' `minDist` bord


naive :: Vec Pt -> (Pt,Pt)
naive pts
 = let c = cross pts pts
   in  minBy distance c

minBy :: Ord b => (a -> b) -> Vec a -> a
minBy f xs
 = fold (min . f...) ... xs
\end{code}

Let's translate @divide@ to a program in CNF.
\begin{code}
divide :: Vec Pt -> (Pt,Pt)
divide pts
 = let p      = external pts

       aboves = filter (... p) pts          -- A
       belows = filter (... p) pts          -- A

       above' = external aboves
       below' = external belows
       border = external above' below'

       aboveB = filter (... p border) pts   -- C
       belowB = filter (... p border) pts   -- B

       cs     = cross  aboveB belowB        -- C
       bord   = fold   (...) cs             -- C

       min'   = external above' below' bord
   in  min'
\end{code}
where @A@, @B@ and @C@ are distinct clusters.

What happens if we do this using stream fusion?
\begin{code}
divide :: Vec Pt -> (Pt,Pt)
divide pts
 = let p      = external pts

       aboves = filter (... p) pts          -- A
       belows = filter (... p) pts          -- B

       above' = external aboves
       below' = external belows
       border = external above' below'


       aboveB = filter (... p border) pts   -- D
       belowB = filter (... p border) pts   -- C

       cs     = cross  aboveB belowB        -- D
       bord   = fold   (...) cs             -- D

       min'   = external above' below' bord
   in  min'
\end{code}
So, we have four clusters instead of three, and the same number of manifest arrays. Not particularly impressive.

\subsection{Quickhull}
Is Quickhull any better? Seems like it's just one cluster; @filterMax@. And we don't have append (@++@), anyway.
\begin{code}
quickhull :: Vec Pt -> Vec Pt
quickhull pts
 = let top  = fold getTop pts
       bot  = fold getBot pts
       tops = hull (top,bot) pts
       bots = hull (bot,top) pts
   in  tops ++ bots

hull :: (Pt,Pt) -> Vec Pt -> Vec Pt
hull line@(l,r) pts
 = let pts' = filter (above   line) pts
       ma   = fold   (maxFrom line) pts'
       hl   = hull   (l, ma)        as
       hr   = hull   (ma, r)        as
   in  hl  ++ hr
\end{code}
Yep. @hull@ is pretty boring.

Something with a \emph{fold}, and then filtering or mapping based on that fold would be good.
Or something with
\begin{code}
let xs' = filter f    xs
    i   = fold   g    xs'
    xs''= map   (h i) xs'
\end{code}
would be really good.

FFT - not really.

\TODO{Look at, I dunno, k-d tree etc..}
