%!TEX root = ../Main.tex
\section{Introduction}

Data flow fusion~\cite{lippmeier2013flow} is a technique to compile a specific class of data flow programs into single, efficient imperative loops. This process of ``compilation'' is equivalent to performing array fusion on a combinator based functional array program, as per related work on stream fusion~\cite{coutts2007streamfusion} and delayed arrays~\cite{keller2010repa}. The key benefits of data flow fusion over this prior work are: 1) it fuses programs that use branching data flows where a produced array is consumed by several consumers, and 2) complete fusion into a single loop is guaranteed for all programs that operate on the same size input data, and contain no fusion-preventing dependencies between operators.

Fusion-preventing dependencies express the fact that some operators simply must wait for others to complete before they can produce their own output. For example, in the following:
\begin{code}
  normalize :: Array Int -> Array Int
  normalize xs = let sum = fold (+) 0 xs
                 in  map (/ sum) xs
\end{code}

If we wish to divide every element of an array by the sum of all elements, then it seems we are forever destined to compute the result using at least two loops: one to determine the sum, and one to divide the elements. The evaluation of @fold@ demands all elements of its source array, and we cannot produce any elements of the result array until we know the value of @sum@. 

However, many programs \emph{do} contain opportunities for fusion, if we only knew which opportunities to take. The following example offers \emph{several} unique, but mutually exclusive approaches to fusion. Figure~\ref{f:normalize2-cluterings} on the next page shows some of the possibilities.
\begin{code}
 normalize2 :: Array Int -> Array Int
 normalize2 xs
  = let sum1 = fold   (+)  0   xs
        gts  = filter (> 0)    xs
        sum2 = fold   (+)  0   gts
        ys1  = map    (/ sum1) xs
        ys2  = map    (/ sum2) xs
    in (ys1, ys2)
\end{code}

In Figure~\ref{f:normalize2-cluterings}, the dotted lines show possible clusterings of operators. Stream fusion implicitly choses the solution on the left as its compilation process cannot fuse a produced array into multiple consumers. The best existing ILP approach will chose the solution on the right as it cannot cluster operators that consume arrays of different lengths. Our system choses the solution in the middle, which is also optimal for this example. 

% NOTE: This set of bullets needs to fit on the first page, without spilling to the second.
Our contributions are as follows:
\begin{itemize}
\item   
We extend prior work by Megiddo~\cite{megiddo1998optimal} and Darte~\cite{darte2002contraction}, with support for length changing operators. Length changing operators can be clustered with the operators that generate their source arrays, and compiled naturally with data-flow fusion (\S\ref{s:ILP}).

\item
We present a simplification to constraint generation that is also applicable to some existing integer linear programming formulations such as Megiddo's,
where constraints between two nodes need not be generated if there exists a fusion-preventing path between the two (\S\ref{s:OptimisedConstraints}).

\item
Our constraint system also encodes a total ordering on the cost of clusterings, expressed using weights on the integer linear program. For example, we encode that memory traffic is more expensive than loop overheads, so given a choice between the two, the memory traffic will be reduced (\S\ref{s:ObjectiveFunction}).

\item
We present benchmarks of our algorithm applied to several common programming patterns, and to several pathological examples.
Our algorithm is complete and yields good results in practice, though if array sizes are unknown, an optimal solution is uncomputable in general. \TODO{ref}
\end{itemize}

The reduction of the clustering problem to integer linear programming was previously described by~\cite{megiddo1998optimal}, though they do not consider length changing operators.


% We must also decide which clustering is the `best' or most optimal. One obvious criterion for this is the minimum number of loops, but there may even be multiple clusterings with the minimum number of loops. In this case, the number of required manifest arrays must also be taken into account. 

% As real programs contain tens or hundreds of individual operators, performing an exhaustive search for an optimal clustering is not feasible, and greedy algorithms tend to produce poor solutions. 

