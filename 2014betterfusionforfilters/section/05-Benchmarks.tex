%!TEX root = ../Main.tex
\section{Benchmarks}
\label{s:Benchmarks}

\begin{figure*}
$$\begin{array}{c}

\begin{tabular}{lrrrrrrrr}
                &   \multicolumn{2}{c}{Unfused}         & \multicolumn{2}{c}{Stream}
                & \multicolumn{2}{c}{Megiddo} &\multicolumn{2}{c}{\textbf{Ours}} \\
                & Time & Loops   & Time & Loops      & Time & Loops & Time & Loops   \\
\hline
Normalize2      & 1.88s & 5      & 1.64s & 4          & 1.82s & 3  & \textbf{1.59s} & \textbf{2}\\
Closest pair    & 3.83s & 6      & 3.33s & 5          & 2.92s & 3  & \textbf{2.92s} & \textbf{3}\\
QuadTree        & 5.22s & 8      & 5.22s & 8          & 4.72s & 2  & \textbf{4.72s} & \textbf{2}\\
\end{tabular}

\end{array}$$
\caption{Benchmark results}
\label{f:BenchResults}
\end{figure*}

This section discusses three representative benchmarks, and gives the full ILP program of the first. These benchmarks highlight the main differences between our fusion mechanism and related work. The runtimes of each benchmark are summarized in Figure~\ref{f:BenchResults}. We report times for: the unfused case where each operator is assigned to its own cluster; the clustering implied by stream fusion~\cite{coutts2007streamfusion}; the clustering chosen by Megiddo~\cite{megiddo1998optimal}, and the clustering chosen by our system. 

For each benchmark we report the runtimes of hand-fused C code based on the clustering determined by each algorithm. Although we also have an implementation of our Data Flow Fusion system in terms of a GHC plugin~\cite{lippmeier2013flow}, we report on hand-fused C code to provide a fair comparison to related work. As mentioned in~\cite{lippmeier2013flow}, the current Haskell stream fusion mechanism introduces overhead in terms of a large number of duplicate loop counters, which increases register pressure unnecessarily. Hand fusing all code and compiling it with the same compiler (GCC) isolates the true cost of the various clusterings from low level differences in code generation.

The benchmark programs are at \url{https://github.com/amosr/papers/tree/master/2014betterfusionforfilters/benches}.

% The three benchmarks are the @normalize2@ example, finding the closest pair of points, and quadtree. The benchmark programs are all hand-written, hand-fused C code based on the clustering. Each program was run with the same input five times, and the minimum runtime was used. Runtimes and the number of loops for each clustering are shown in Figure~\ref{f:BenchResults}. In all cases, our clustering performs better than or as good as Megiddo's, and better than stream fusion and unfused. Interestingly, stream fusion's clustering for @normalize2@ performs better than Megiddo's, despite having more loops, as stream fusion is able to remove the intermediate array.

\subsection{Normalize2}
To demonstrate the ILP formulation, the clustering of @normalize2@ is derived as follows.
\begin{code}
 normalize2 :: Array Int -> Array Int
 normalize2 xs
  = let sum1 = fold   (+)  0   xs
        gts  = filter (>   0)  xs
        sum2 = fold   (+)  0   gts
        ys1  = map    (/ sum1) xs
        ys2  = map    (/ sum2) xs
    in (ys1, ys2)
\end{code}

To make the program shorter and easier to work with, we use the optimised version.
First, we must calculate $possible$ -- that is, those nodes which have no fusion-preventing path between them.
Any elements in the following sets have no fusion-preventing paths.
\[ \{ \{sum1, gts, sum2\}
 , \{sum1, ys2\}
 , \{gts, sum2, ys1\}
 , \{ys1, ys2\} \} \]

Note that, in the objective function, the weights for $x_{sum1, sum2}$ and $x_{sum2, ys1}$ are both only 1, because they do not share any input arrays.

\begin{tabbing}
MMMMM   \= MMMMMMM \= M \= MMMMMMM \= M \= MMMMMMM \= \kill
Minimise   \> $25 \cdot x_{sum1, gts} + 1 \cdot x_{sum1,sum2} + 25 \cdot x_{sum1, ys2} +$ \\
           \> $25 \cdot x_{gts, sum2} + 25 \cdot x_{gts, ys1} + 1 \cdot x_{sum2, ys1} +$ \\
           \> $ 25 \cdot x_{ys1, ys2} + 5 \cdot c_{gts} + 5 \cdot c_{ys1} + 5 \cdot c_{ys2} $\\
Subject to \\
    \> $-5 \cdot x_{sum1, gts}$  \> $\le$ \> $\pi_{gts} - \pi_{sum1}$  \> $\le$ \> $5 \cdot x_{sum1, gts}$  \\
    \> $-5 \cdot x_{sum1, sum2}$ \> $\le$ \> $\pi_{sum2} - \pi_{sum1}$ \> $\le$ \> $5 \cdot x_{sum1, sum2}$ \\
    \> $-5 \cdot x_{sum1, ys2 }$ \> $\le$ \> $\pi_{ys2 } - \pi_{sum1}$ \> $\le$ \> $5 \cdot x_{sum1, ys2 }$ \\
    \> $-5 \cdot x_{gts,  ys1 }$ \> $\le$ \> $\pi_{ys1 } - \pi_{gts }$ \> $\le$ \> $5 \cdot x_{gts, ys1  }$ \\
    \> $-5 \cdot x_{sum2, ys1 }$ \> $\le$ \> $\pi_{ys1 } - \pi_{sum2}$ \> $\le$ \> $5 \cdot x_{sum2, ys1 }$ \\
    \> $-5 \cdot x_{ys1, ys2  }$ \> $\le$ \> $\pi_{ys2 } - \pi_{ys1 }$ \> $\le$ \> $5 \cdot x_{ys1, ys2  }$ \\
\\
    \> $   x_{gts, sum2 }$ \> $\le$ \> $\pi_{sum2} - \pi_{gts }$ \> $\le$ \> $5 \cdot x_{gts, sum2 }$ \\
\\
    \>                     \>       \> $\pi_{sum1} < \pi_{ys1}$ \\
    \>                     \>       \> $\pi_{sum2} < \pi_{ys2}$ \\
\\
    \> $ x_{gts,sum2} $    \> $\le$ \> $c_{gts}$ \\
\\
    \> $x_{gts, sum2}$     \> $\le$ \> $x_{sum1, sum2}$ \\
    \> $x_{sum1,sum1}$     \> $\le$ \> $x_{sum1, sum2}$ \\
    \> $x_{sum1, gts}$     \> $\le$ \> $x_{sum1, sum2}$
\end{tabbing}

One minimal solution to this is:
\begin{tabbing}
MMMMMMMMMMMMMMMMMMMMMMMMMM \= M \= \kill
$x_{sum1, gts}, x_{sum1, sum1}, x_{sum1, sum2}, x_{gts, sum2}, x_{ys1,  ys2}$
    \> $=$ \> $0$ \\
$x_{sum1, ys2}, x_{gts, ys1 }, x_{sum2, ys1}$
    \> $=$ \> $1$ \\
\\
$\pi_{sum1}, \pi_{gts }, \pi_{sum2}$
    \> $=$ \> $0$ \\
$\pi_{ys1 }, \pi_{ys2 }$
    \> $=$ \> $1$ \\
\\
$c_{gts}, c_{ys1}, c_{ys2}$           
    \> $=$ \> $0$
\end{tabbing}
This minimal solution is not unique, however in this case the only other minimal solutions use different $\pi$ values, and denote the same clustering.

Looking at just the non-zero variables in the objective function, the value is
$25 \cdot x_{sum1,ys2} + 25 \cdot x_{gts,ys1} + 1 \cdot x_{sum2, ys1} = 51$.
It is worth noting that, for example, the objective could be reduced by fusing $x_{sum1,ys2}$ together.
However, this conflicts with other constraints. Since $x_{sum1, sum2} = 0$, the constraint requires that $\pi_{sum1} = \pi_{sum2}$, and another constraint requires $\pi_{sum2} < \pi_{ys2}$.
These constraints may not all hold, so a clustering that fused sum1 and ys2 together would not allow sum1 and sum2 to be fused together.

We can also calculate the objective function for the clustering used by stream fusion, to contrast and show that the constraints are valid.
To do this, the values of $x_{ij}$ variables are:
\begin{tabbing}
MMMMMMMMMMMMMMMMMMMMMMMMMM \= M \= \kill
$x_{gts, sum2}$
    \> $=$ \> $0$ \\
$x_{sum1, gts}, x_{sum1, sum2}, x_{ys1,  ys2}, x_{sum1, ys2}, x_{gts, ys1 }, x_{sum2, ys1}$
    \> $=$ \> $1$
\end{tabbing}
If we add these extra equality constraints to the integer linear program and solve it, one possible solution is:
\begin{tabbing}
MMMMMMMMMMMMMMMMMMMMMMMMMM \= M \= \kill
$\pi_{sum1}, \pi_{gts }, \pi_{sum2}$
    \> $=$ \> $0$ \\
$\pi_{ys1 }, \pi_{ys2 }$
    \> $=$ \> $1$ \\
$c_{gts}, c_{ys1}, c_{ys2}$           
    \> $=$ \> $0$
\end{tabbing}
It is interesting to note that two nodes having equal $\pi_i$ values does not imply that the two nodes are fused together.
Conversely, however, different $\pi_i$ values do imply that two nodes are not fused together.

The objective function's value is
$25 \cdot x_{sum1, gts} + 1 \cdot x_{sum1,sum2} + 25 \cdot x_{sum1, ys2} + 25 \cdot x_{gts, ys1} + 1 \cdot x_{sum2, ys1} + 25 \cdot x_{ys1, ys2} = 102$.

\subsection{Closest Points}
The closest points benchmark is a divide-and-conquer algorithm that, given an array of points, finds the closest pair of points.
A midpoint is found, and the points are filtered into those above the midpoint, and those below.
The closest points in the filtered arrays are then recursively computed, and the two results merged.
As the filters are operated on recursively, there are no opportunities for fusing the filters, making our clustering the same as Megiddo's.
Our clustering is able to generate both filtered arrays in a single loop, unlike stream fusion, which requires a separate loop for each.

\subsection{QuadTree}
The QuadTree benchmark recursively computes a 2-dimensional space partitioning tree of an array of points.
At each step, the array of points is filtered into four boxes.
As with closest points, there are no opportunities for fusing the filters, and our clustering is the same as Megiddo's.
Our clustering produces all four filtered results in a single loop, whereas stream fusion requires four loops.


\subsection{QuickHull}
QuickHull is another program that is worth mentioning, as it highlights some fusion opportunities with filters.
The core of the QuickHull algorithm is shown below: given a line and an array of points, the points are filtered to those above the line, and the farthest point above the line is also found.

\begin{code}
hull :: (Point,Point) -> Array Point -> Array Point
hull line@(l,r) pts
 = let pts' = filter (above   line) pts
       ma   = fold   (maxFrom line) pts'
   in (hull (l, ma) pts') ++ (hull (ma, r) pts')
\end{code}

Stream fusion cannot fuse @pts'@ and @ma@ together, as @pts'@ is used multiple times and cannot be inlined into @ma@.
Megiddo cannot fuse them either, as their iteration sizes are different.
Our formulation is able to fuse these together.
If the @fold@ is changed to operate over @pts@ instead of @pts'@, Megiddo's formulation \emph{is} able to fuse the two.
However, a programmer unaware of fusion would expect the original version to be faster, because @pts'@ is smaller than @pts@.

