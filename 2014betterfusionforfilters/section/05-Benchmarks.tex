%!TEX root = ../Main.tex
\section{Benchmarks}
\label{s:Benchmarks}
\ben{Use the larger programs as benchmarks, or as running examples. We only need small, simple programs to demonstrate how the algorithm works.}

\begin{figure*}
$$\begin{array}{c}

\begin{tabular}{lrrrrrrrr}
                &   \multicolumn{2}{c}{Unfused}         & \multicolumn{2}{c}{Stream}
                & \multicolumn{2}{c}{Megiddo} &\multicolumn{2}{c}{\textbf{Ours}} \\
                & Time & Loops   & Time & Loops      & Time & Loops & Time & Loops   \\
\hline
Normalize2      & 1.88s & 5      & 1.64s & 4          & 1.82s & 3  & \textbf{1.59s} & \textbf{2}\\
Closest pair    & 3.83s & 6      & 3.33s & 5          & 2.92s & 3  & \textbf{2.92s} & \textbf{3}\\
QuadTree        & 5.22s & 8      & 5.22s & 8          & 4.72s & 2  & \textbf{4.72s} & \textbf{2}\\
\end{tabular}

\end{array}$$
\caption{Benchmark results}
\label{f:BenchResults}
\end{figure*}

Notes:
\begin{itemize}
% \item
% Normalize2 tested with around 1gb input data. This should be small enough that all intermediates fit in memory, but large enough not to fit in cache.
% \item
% ClosestPoints only tested with 80mb data.
\item
In some cases, the clusterings were the same for different methods. In this case, the results are the same. 
\item
The stream fusion benchmarks use the clustering that stream fusion \emph{would} use, but hand-fused and written in C.
\item
Programs were run five times with the same input data, and the fastest run was used.
\item
ILP solutions were created by hand-fusing based on the clustering of the implementation. In the future, this will be integrated to use data flow fusion.
\item
Benchmark programs are available at \url{https://github.com/amosr/papers/tree/master/2014betterfusionforfilters/benches}
\end{itemize}


\subsection{Closest pairs}
Closest pairs makes use of fast-ish @median@ to ensure balanced division of work.
\begin{code}
closest :: Vec Pt -> (Pt,Pt)
closest pts
 | length pts < 250
 = naive pts
 | otherwise
 = divide pts

divide :: Vec Pt -> (Pt,Pt)
divide pts
 = let p      = median pts
       aboves = filter (above p) pts
       belows = filter (below p) pts
       above' = closest aboves
       below' = closest belows

       border = min (distance above') (distance below')

       aboveB = filter (above (p - border)) pts
       belowB = filter (below (p + border)) pts

       cs     = cross aboveB belowB
       bord   = minBy distance cs
   in  above' `minDist` below' `minDist` bord


naive :: Vec Pt -> (Pt,Pt)
naive pts
 = let c = cross pts pts
   in  minBy distance c

minBy :: Ord b => (a -> b) -> Vec a -> a
minBy f xs
 = fold (min . f...) ... xs
\end{code}

Let's translate @divide@ to a program in CNF.
\begin{code}
divide :: Vec Pt -> (Pt,Pt)
divide pts
 = let p      = external pts

       aboves = filter (... p) pts          -- A
       belows = filter (... p) pts          -- A

       above' = external aboves
       below' = external belows
       border = external above' below'

       aboveB = filter (... p border) pts   -- C
       belowB = filter (... p border) pts   -- B

       cs     = cross  aboveB belowB        -- C
       bord   = fold   (...) cs             -- C

       min'   = external above' below' bord
   in  min'
\end{code}
where @A@, @B@ and @C@ are distinct clusters.

What happens if we do this using stream fusion?
\begin{code}
divide :: Vec Pt -> (Pt,Pt)
divide pts
 = let p      = external pts

       aboves = filter (... p) pts          -- A
       belows = filter (... p) pts          -- B

       above' = external aboves
       below' = external belows
       border = external above' below'


       aboveB = filter (... p border) pts   -- D
       belowB = filter (... p border) pts   -- C

       cs     = cross  aboveB belowB        -- D
       bord   = fold   (...) cs             -- D

       min'   = external above' below' bord
   in  min'
\end{code}
So, we have four clusters instead of three, and the same number of manifest arrays. Not particularly impressive.

\subsection{Quickhull}
Is Quickhull any better? Seems like it's just one cluster; @filterMax@. And we don't have append (@++@), anyway.
\begin{code}
quickhull :: Vec Pt -> Vec Pt
quickhull pts
 = let top  = fold getTop pts
       bot  = fold getBot pts
       tops = hull (top,bot) pts
       bots = hull (bot,top) pts
   in  tops ++ bots

hull :: (Pt,Pt) -> Vec Pt -> Vec Pt
hull line@(l,r) pts
 = let pts' = filter (above   line) pts
       ma   = fold   (maxFrom line) pts'
       hl   = hull   (l, ma)        as
       hr   = hull   (ma, r)        as
   in  hl  ++ hr
\end{code}
Yep. @hull@ is pretty boring.

Something with a \emph{fold}, and then filtering or mapping based on that fold would be good.
Or something with
\begin{code}
let xs' = filter f    xs
    i   = fold   g    xs'
    xs''= map   (h i) xs'
\end{code}
would be really good.

FFT - not really.

\subsection{QuadTree}
\begin{code}
bounds pts
 = let x1 = fold minX pts
       y1 = fold minY pts
       x2 = fold maxX pts
       y2 = fold maxY pts
   in ((x1,y1), (x2,y2))

%splitbox ((x1,y1), (x2,y2))
% = let xm = mid x1 x2
%       ym = mid y1 y2
%   in ( ((x1, y1), (xm, ym))
%      , ((x1, ym), (xm, y2))
%      , ((xm, y1), (x2, ym))
%      , ((xm, ym), (x2, y2)))
%
quadtree pts
 = go (bounds pts)
 where
  go bounds pts
   | length pts > 0
   = let (b1,b2,b3,b4) = splitbox bounds
         pts1  = filter (inbox b1) pts
         pts2  = filter (inbox b2) pts
         pts3  = filter (inbox b3) pts
         pts4  = filter (inbox b4) pts
         tree1 = go b1 pts1
         tree2 = go b2 pts2
         tree3 = go b3 pts3
         tree4 = go b4 pts4
      in Tree tree1 tree2 tree3 tree4
   | otherwise
   = Empty
\end{code}
It's easy to see that @bounds@ should only require one loop, but stream fusion requires four, as there is no inlining that can occur.
The same is true of @go@ in @quadtree@. Our implementation only requires one loop for each of these.
