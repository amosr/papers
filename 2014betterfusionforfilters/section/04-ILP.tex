\section{Integer linear program formulation}

\subsection{Cost function}
We argue that our cost function makes intuitive sense, by reducing the number of manifest arrays and loops.
However, due to the nature of our heuristic cost function we have no strong guarantees that the clustering with the lowest cost will actually be the clustering with the shortest runtime.
In general, since we do not know the expected sizes of the arrays, this is impossible to guarantee correctly.
It is possible that for certain restricted subsets of the language, a more sophisticated cost function may be able to guarantee optimality.
This is left to future work.

\subsection{Proof}
To prove correctness of our linear program formulation, we need to prove two different things.
Firstly, the formulation's constraints must always be satisfiable; that is, there must exist a variable assignment that satisfies all constraints.
This is rather simple to show, but guarantees that the linear program will always give an answer.
The next thing to show is that any produced clustering is legal: if a variable assignment satisfies the constraints, then it is a valid and legal clustering.
This means that, not only do we get \emph{an} answer, we also get the \emph{right} answer.

\subsubsection{Satisfiability}
For any program $p$, there exists a trivial clustering with no fusion at all.
We can use this as the variable assignment of $ilp(p)$.
For each pair of nodes $m,n \in p$, $x_{mn} = 1$ - no fusion is possible.
For the $\pi$ variables, we must find a topographical ordering of the nodes in $p$, which is simple since we are assured it is a dag.

Now, prove that this assignment actually satisfies the constraints.


\subsubsection{Legality}
For any program $p$ and variable asignment $v$, if $v$ satisfies the constraints for $ilp(p)$, the clustering denoted by $x_ij$ in $v$ is legal.

For a clustering to be legal, it must satisfy three constraints:
\begin{description}
\item[Acyclic]
after merging nodes of same cluster together, the resulting graph must be a dag
\item[Precedence preserving]
if there is an edge between two nodes $i$ and $j$, and they are not merged together, then we require $\pi_j > \pi_i$
\item[Type constraint]
if two nodes $i$ and $j$ are in the same cluster, then $\tau_i = \tau_j$, or if $\tau_i$ is a subtype of $\tau_j$ (or $\tau_j$ is a subtype of $\tau_i$), then the \emph{generator} for $\tau_i$ (or $\tau_j$) must also be in the same cluster as $i$ and $j$.
    \\
    \TODO This is wrong.
\end{description}





