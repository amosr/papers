\section{Combinator normal form}
We can accept functions written in \emph{combinator normal form}, which is a specialised form of first-order array programs detailed in Figure~\ref{f:CombinatorNormalForm}.
This form is a list of combinator bindings, where variables are split into scalar and array variables.
The main restriction is that worker functions may only reference scalar variables, and thus do not perform array combinators.

The only combinators we fuse are @map@, @filter@, @fold@ and @gather@.
Most of these are standard combinators except for @gather@, which is equivalent to @gather xs ys = map (index xs) ys@.
However, as we support no @index@ operation, @gather@ is implemented as a primitive.

Since it is unlikely that an entire function will be comprised of these few combinators, we support one additional binding type: @external@. This signifies that the referenced variables are used by a computation that is not a primitive combinator, and must be materialised fully in memory at this stage.
The @external@ also signifies variables produced by a non-primitive combinators.
Without knowing the nature of the computation expressed by @external@, we must naturally take a conservative view, and allow no fusion to occur between at these points. They are, in effect, fusion barriers, forcing arrays and scalars to be fully computed before continuing.

It is important to note that because of the purity of Haskell, we are free to take certain liberties when reordering the program.
None of the worker functions, nor any @external@ computations may produce visible side-effects; the only observable effect must be to produce their output.
%!TEX root = ../Main.tex
\begin{figure}
\begin{tabbing}
MMMM        \= MM \= MMMMMMMMM \= \kill
$scalar$    \> $\to$ \> (scalar variable) \\
$array$     \> $\to$ \> (array variable)  \\
$f$         \> $\to$ \> (worker function) \\
$fun$       \> $\to$ \> $f~scalar\ldots$
\\[2ex]
$bind$      \> @::=@ \> $scalar$ \> $=~sbind$ \\
            \> $~|$  \> $array$  \> $=~abind$ \\
            \> $~|$  \> $scalar\ldots,array\ldots$ \> $=~@external@~scalar\ldots~array\ldots$
\end{tabbing}

\begin{tabbing}
MMMM        \= MM \= MMMMM \= MMMMMM \= M \= MMMM \= MMMMMM \= \kill
$sbind$     \> @::=@ \> $@fold@$     \> $fun~~ array$
\\[1ex]

$abind$     \> @::=@ \> $@map@_n$    \> $fun~~ array^n$ 
            \> $~|$  \> $@filter@$   \> $fun~~ array$   \\
            \> $~|$  \> $@generate@$ \> $scalar~~ fun$  
            \> $~|$  \> $@gather@$   \> $array~~ array$ \\
            \> $~|$  \> $@cross@$    \> $array~~ array$
\\[1ex]
$function$  \> @::=@ \> $\lambda scalar\ldots~array\ldots~\to$ \\
            \>          \> $@let@~bind\ldots$                  \\
            \>          \> $@in@~(scalar\ldots,~array\ldots)$
\\[3ex]
$@fold@$     \> $:~ (a \to a \to a) \to @Array@~~ a \to a$     \\
$@map@_n$    \> $:~ (\{a_i          \to\}^{\;i\; \gets 1 \dots n}~~ b)  \to
                       \{@Array@~~ a_i \to\}^{\;i\; \gets 1 \dots n}~~ @Array@~~ b$ \\
$@filter@$   \> $:~ (a \to @Bool@) \to @Array@~~ a \to @Array@~~ a$      \\
$@generate@$ \> ~~ $:~ @Nat@ \to (@Nat@ \to a) \to @Array@~~ a$          \\
$@gather@$   \> ~~ $:~ @Array@~~ a \to @Array@~~ @Nat@  \to @Array@~~ a$ \\
$@cross@$    \> ~~ $:~ @Array@~~ a \to @Array@~~ a ~~~~ \to @Array@~~ a$
\end{tabbing}
\caption{Combinator normal form}
\label{f:CombinatorNormalForm}
\end{figure}




