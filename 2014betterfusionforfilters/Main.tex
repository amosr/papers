\documentclass[preprint]{sigplanconf}
\usepackage{amssymb}
\usepackage{graphicx}
\usepackage{amsmath}
\usepackage{mathptmx}
\usepackage{stmaryrd}
\usepackage{hyperref}
\usepackage{alltt}
\usepackage{url}
\usepackage{float}
\usepackage{style/utils}
\usepackage{style/code}
\usepackage{style/proof}
\usepackage{style/judgements}

% -----------------------------------------------------------------------------
\begin{document}

\title{Fusing Filters with Integer Linear Programming}
% Better title?
% \title{Better fusion for filters}

\authorinfo{ 
  Amos Robinson$^\dagger$ 
  \and Ben Lippmeier$^\dagger$
  \and Manuel M. T. Chakravarty$^\dagger$
  \and Gabriele Keller$^\dagger$ 
}{
  \vspace{5pt}
  \shortstack{
    $^\dagger$Computer Science and Engineering \\
    University of New South Wales, Australia \\[2pt]
    \textsf{\{amosr,benl,chak,keller\}@cse.unsw.edu.au}
  }
}

\maketitle
\makeatactive

\begin{abstract}
Minimising memory traffic is key to compiling functional, collection oriented array programs into efficient code. To achieve this, it is not sufficient to just fuse subsequent array operations into a single, more complex computation, we also have to also \emph{cluster} separate traversals of the same array into one single loop. Previous work demonstrated how Integer Linear Programming (ILP) can be used to cluster the operators in a general data-flow graph into subgraphs, which can be individually fused. However, these approaches have one serious restrictions: they can only handle operations which preserve the size of the array. The technique we present in this paper extends the ILP approach with support for size-changing operations, using an external ILP server to find good clusterings.

%The original abstract:
%Recent work on array fusion shows how to extract data-flow graphs from functional programs and compile them into efficient imperative loops. However, the compilation process only handles graphs that can be compiled into \emph{single} loops, and most programs require more. Prior work shows how use Integer Linear Programming (ILP) to \emph{cluster} the operators in a general data-flow graph into subgraphs that can be individually fused, but until now this approach did not handle operations like filter, which produce arrays of a different size than their input. We extend the existing ILP approach with support for size changing operators, using an external ILP solver to find good clusterings.
\end{abstract}


\category
	{D.3.3}
	{Programming Languages}
	{Language Constructs and Features---Concurrent programming structures; Control structures; Abstract data types}

\terms
	Languages, Performance

\keywords
	Arrays, Fusion, Haskell

%!TEX root = ../Main.tex
\section{Introduction}
\label{s:Introduction}

Suppose we have two input streams of numeric identifiers, and wish to perform some analysis on these identifiers. The identifiers from both streams arrive sorted, but may include duplicates. We wish to produce an output stream of unique identifiers from the first input stream, as well as produce the unique union of identifiers from both streams. Can we perform both of these tasks at once, without needing to read through the stream data multiple times, and without needing unbounded buffering? Here is how we might write the source code, where @S@ is for @S@-tream.
\begin{code}
  uniquesUnion : S Nat -> S Nat -> (S Nat, S Nat)
  uniquesUnion sIn1 sIn2
   = let  sUnique = group sIn1
          sMerged = merge sIn1 sIn2
          sUnion  = group sMerged
     in   (sUnique, sUnion)
\end{code}

In this implementation the @group@ operator filters out consecutive duplicates, while @merge@ combines two sorted streams so that the output remains sorted. This example has a few interesting properties. Firstly, the data-access pattern of @merge@ is \emph{value-dependent}, meaning that the order in which this operator pulls values from @sIn1@ and @sIn2@ depends on the values themselves. If all the values from @sIn1@ are smaller than the values in @sIn2@, then @merge@ will pull all values from @sIn1@ before pulling the rest from @sIn2@, and vice versa. Secondly, although @sIn1@ occurs twice in the program, at runtime we only want to handle the elements of each stream once. To achieve this, the compiled program must coordinate between the two uses of @sIn1@, so that values are only read when both the @group@ and @merge@ operators are ready to receive a new value. Finally, as the stream length is assumed to be unbounded, we cannot buffer an arbitrary number of elements read from either stream, or risk running out of local storage space.

For an implementation which does \emph{not} use stream fusion, we might implement each of the operators as a separate concurrent process, and send each identifier value using an intra-process communication mechanism. Developing such an implementation could be easy or hard, depending on what language features are available for concurrency. However, worrying about the \emph{performance tuning} of such a system, such as whether we need back-pressure, or how to chunk the stream data to reduce the amount of communication overhead, is invariably a headache. 

We might instead define some sort of uniform interface for data sources, with a single `pull' function that provides the next value in each stream. Each operator could be given this interface, so that the next value from each result stream is computed on demand. This is approach is commonly taken with implementations of physical operators in data base systems. However, this `pull only' model does not support operators with multiple outputs, such as our derived @uniquesUnion@ operator, at least not without unbounded buffering. Suppose a consumer pulls many elements from the result @sUnique@ stream. The implementation needs to pull the corresponding source elements from @sIn1@ \emph{as well} as buffering an arbitrary number of matching elements from @sIn2@. It needs to buffer an aribrary number of elements from @sIn2@ because there is no guarantee of when a consumer will also pull from the @sUnion@ result stream. Once that happens the elements from @sIn2@ no longer need to be retained, but not before.

Instead, for a single threaded program, we want to perform \emph{stream fusion}, which takes the dataflow network and produces a simple sequential loop that gets the job done without requiring extra process-control abstractions and without requiring unbounded buffering. Sadly, existing stream fusion transformations cannot handle our example. As observed by \citet{kay2009you}, both pull-based and push-based fusion have fundamental limitations. Pull-based systems such as short cut stream fusion~\cite{coutts2007stream} cannot handle cases where a particular stream or intermediate result is used by multiple consumers. We refer to this situation as a \mbox{\emph{split} --- in the} dataflow network the flow from input stream @sIn1@ is split into both the @group@ and @merge@ consumers. 

% Leave this to related work. We've already mentioned a canonical pull-based system.
% Recent work on stream fusion by \citet{kiselyov2016stream} uses staged computation to ensure all combinators are inlined, but for splits this causes excessive inlining which duplicates work, due to values of the source arrays being read multiple times.

Push-based systems such as foldr/build fusion~\cite{gill1993short} also cannot fuse our example because they do not support operators with multiple inputs. We refer to such a situation as a \emph{join} --- in our example the @merge@ operator expresses a join in the data-flow graph. Some systems support both pull and push: data flow inspired array fusion~\cite{lippmeier2013data} allows both splits and joins but only for a limited, predefined set of operators. More recent work on polarized data flow fusion~\cite{lippmeier2016polarized} \emph{is} able to fuse our example, but requires the program to be rewritten to use explicitly polarized stream types. 

% The mechanism that combines the implementations of both operators, to yield efficient imperative code also depends on the general purpose compiler optimisations implemented by GHC, and it can be difficult to tell if these have ``worked'' without inspecting the intermediate representations of the compiler.

Synchronous dataflow languages such as Lucy-n~\cite{mandel2010lucy} reject value-dependent operators such as @merge@, while general dataflow languages fall back on less performant dynamic scheduling for these cases \cite{bouakaz2013real}. The polyhedral array fusion model~\cite{feautrier2011polyhedron} is used for loop transformations in imperative programs, but operates at a much lower level. The polyhedral model is based around affine loops, which makes it difficult to support filter-like operators such as @group@ and @merge@.

In our new system we still view the program as a concurrent process network. Each operator is a separate process, and the stream data flows through communication channels between the processes. Each operator is expressed as a restricted, sequential imperative program with commands that include both @pull@ for reading from an input stream and @push@ for writing to an output stream. The fusion transform takes the concurrent process network and \emph{sequentializes} it into a single process by choosing a particular evaluation order that requires no unbounded intermediate buffers. When the fusion transformation succeeds we know it has worked. There is no need to inspect intermediate representations of the compiler to debug poor performance, which is a common problem in systems based on general purpose program transformations \cite{lippmeier2012:guiding}.

In summary, we make the following contributions:
\begin{itemize}
\item a process calculus for encoding infinite streaming programs (\S\ref{s:Processes});
\item an algorithm for fusing these processes, the first to support arbitrary splits and joins (\S\ref{s:Fusion});
\item numerical results that demonstrate that the algorithm is well behaved when the number of fused processes is large. The size of the fused result program is not excessive. \TODO{Ref}
\item a formalization and proof of soundness for the core fusion algorithm in Coq (\S\ref{s:Proofs});
\end{itemize}

Our fusion transformation for infinite stream programs could also serve as the basis for an \emph{array} fusion system, using a natural extension to finite streams. We discuss this extension in \S\ref{s:Finite}.

% TODO: We can't make the appendix a contribution because the reviewers are not required to read appendices.
% \item and show our processes are general enough for many combinators, including segmented operations (\S\ref{s:Combinators}).

% \ben{Add a few more sentences on related work. Explain how this work extends the old flow fusion paper. It is not short-cut fusion like Oleg's recent work. We are not in the same space as Fortran style array fusion transformations like polyhedral}

% BL: describe this later.
% Furthermore, the data-flow fusion system of~\cite{lippmeier2013data} only deals with a fixed set of baked-in combinators. 

% BL: Shift the detailed description into a later section.
% The example above has three combinators, so the process network has three processes.
% The two @writeFile@s outputs are treated as sinks that values can be pushed to at any time, and are not converted to processes.
% During code generation, any output values from the @uniques@ and @union@ streams are sent to the corresponding @writeFile@ sink, but we do not address code generation in this paper.

% The process for @uniques@ is defined by the @group@ combinator, and can be thought of as an imperative loop: first it reads from its input stream @file1@ and stores that in a local variable.
% It also keeps track of the last pulled value, and compares that against the newly read value.
% If they are different, it pushes the new value to its output stream @uniques@.
% In either case, it updates the last pulled value and loops back to the start to pull from @file1@ again.

% The process for @merged@ is defined by the @merge@ combinator, which starts by reading from both @file1@ and @file2@ and storing these in local variables.
% It then compares its two values to see which is the smaller.
% If the value from @file1@ is smaller, it pushes that value and pulls a new value from @file1@, otherwise it pushes the value from @file2@ and pulls from @file2@.
% This is performed in a loop.

% We fuse these two processes by interleaving the two such that the shared input @file1@ is only pulled from when both processes agree.
% The new process pulls from @file1@, which is copied to the variables for both processes.
% The @uniques@ process now has all it needs to execute, so it checks the value against the last pulled value, pushes if necessary, and goes back to try to pull from @file1@ again.
% At this stage the @merged@ process still has a value from @file1@ that it has pulled but not used, so @uniques@ cannot pull from @file1@ again.
% We now let @merged@ run, pulling from @file2@ and checking which is smaller.
% If the value from @file1@ is smaller, the value is emitted and @merged@ wishes to pull a new value from @file1@.
% Both processes now agree on pulling from @file1@ again, so the new value is pulled and @uniques@ can run again.
% Otherwise if the value from @file1@ is not smaller, the value from @file2@ is emitted and @merged@ pulls from @file2@ with no coordination required.

% If we wish to ensure that each value is only read from the file once, we must coordinate between the two use sites: when @uniques@ requires a new value it must ensure that @union@ is ready to receive a new value, and vice versa. Note that we cannot just execute @uniques@ while storing the read values in a buffer, as this may require more memory than is available.
% In order to fuse this example, we require both pull \emph{and} push streams.
% The input streams must be pull streams since the order values are required is determined by the @merge@ combinator.
% For the same reason, the outputs sent to each @writeFile@ must be push streams.

% Fusion for array programs is important for removing intermediate arrays, reducing memory traffic and reducing allocations.
% However, when dealing with data too large to fit in memory such as tables on disk, removing intermediate arrays becomes essential rather than just desirable.
% Attempting to create an intermediate array of such amounts of data would lead to thrashing and swapping to disk, or perhaps even running out of swap.
% For these situations, some sort of assurance of total fusion is required: either the program can be fused with no intermediate arrays or unbounded buffers, or it will not compile at all.


% Fusion eliminates intermediate array buffers and converts pipelines of array combinators into low-level iteration based loop code. Different fusion systems can handle 

% When comparing fusion systems, three important criteria to consider are: whether the system supports splits, where a stream is used multiple times; whether it supports joins, where a combinator has multiple inputs; and whether arbitrary combinators such as @merge@ and segmented appends can be encoded. Existing fusion systems support one or two of these, but not three. We present a fusion system based on process calculus that supports all three: splits, joins and arbitrary combinators.

% Our system has been formalised in Coq where we have proved soundness of the fusion algorithm. It is expressive enough to encode a wide range of combinators including operations on segmented arrays.

% \amos{``Arbitrary combinators'' is not quite true. How can we distinguish combinators we support from 2013 data flow fusion paper? Perhaps by mentioning value-dependent input / access patterns.}

% Leave this to the description of the algorithm, not the abstract.
% We encode each combinator as a separate process with any number of input and output channels. Each process is sequential but multiple processes can be executed concurrently. We give a concurrent execution semantics for multiple processes, but these are used only as a specification for how the fused program must behave. The fused program itself is sequential and can easily be converted to simple imperative code.

% Our fusion algorithm takes two concurrently executable processes and creates a sequential interleaving of the two such that they execute with no unbounded buffers.
% If fusion would require unbounded buffers (or the fusion algorithm wrongly infers that it would) then fusion fails.
% If fusion fails, the user can be presented with an error message telling them which combinators could not be fused.
% For scenarios where fusion is required, this is a great advantage over fragile shortcut fusion systems.

% BL: leave the apologies to the conclusion.
% The version presented here deals with infinite streams, and we informally describe the extensions required to support finite streams.

% BL: leave this to the main intro.
% optimising high-level array and streaming computations, as it reduces memory traffic and intermediate arrays. The benefits of removing intermediate arrays are even more important as data sizes approach the size of memory.

%!TEX root = ../Main.tex
%\eject

\begin{figure*}[ht!]
\begin{center}
\includegraphics[scale=0.5]{figures/ex1-compare.pdf}
\end{center}

\caption{Clusterings for normalize2 example: with stream fusion; our system; best imperative system}
\label{f:normalize2-cluterings}
\end{figure*}


\section{Combinator normal form}
\TODO{Mention that in the data flow graph we draw materialized vectors with a box around them.}
\TODO{Say a program in this form is just a textual representation of its data-flow graph. This is similar to Loop Communication Graphs (LCG) from \cite{sarkar1991optimization}, except that we avoid needing ``array access vectors'' that track which direction an array is accessed in. All our operators access elements from front to back, and we don't consider reversals. },  
We can accept functions written in \emph{combinator normal form}, which is a specialised form of first-order array programs detailed in Figure~\ref{f:CombinatorNormalForm}.
This form is a list of combinator bindings, where variables are split into scalar and array variables.
The main restriction is that worker functions may only reference scalar variables, nor may they include any array combinators.

We support the following combinators: $@map@_n$, @filter@, @fold@, @gather@ and @cross@.
Most of these are standard combinators except for @gather@ and @cross@, however $@map@_n$ differs slightly from usual @map@ in that it takes $n$ arrays of the same length, and applies the worker function to all elements of the same index.
The @gather@ combinator is equivalent to @gather xs ys = map (index xs) ys@.
However, as we support no @index@ operation, and worker functions may not refer to arrays, @gather@ is implemented as a primitive.
The @cross@ combinator takes two arrays, and returns the cartesian product of the two.

Since it is unlikely that an entire function will be comprised of these few combinators, we support one additional binding type: @external@.
This signifies that the referenced variables are used by a computation that is not a primitive combinator, and must be materialised fully in memory at this stage.
The @external@ also lists the variables produced by the external computation.
Without knowing the nature of the computation expressed by @external@, we must naturally take a conservative view, and allow no fusion to occur between at these points. They are, in effect, fusion barriers, forcing arrays and scalars to be fully computed before continuing.
This also allows us to support combinators other than those here, but as they must be treated conservatively will likely be less optimal than if they were implemented as primitives.

It is important to note that because of the purity of Haskell, we are free to take certain liberties when fusing the program.
None of the worker functions, nor any @external@ computations may produce visible side-effects; the only observable effect must be to produce their output.
This means that any reordering of the program is allowed, as long as mentioned variables are bound beforehand.
\begin{figure}
\begin{tabbing}
MMMM        \= MM \= \kill
$scalar$    \> $\to$ \> (scalar variable)      \\
$array$     \> $\to$ \> (array variable)       \\
$f$         \> $\to$ \> (worker function)      \\
$fun$       \> $\to$ \> $f~~ scalar...$          \\
\end{tabbing}
\begin{tabbing}
MMMM        \= MM \= MMMMMMMM \= \kill
$bind$      \> \tt{::=} \> $scalar$ \> $=~~ sbind$ \\
            \> $~~|$    \> $array$  \> $=~~ abind$ \\
            \> $~~|$    \> $scalar...,array...$  \> $=~~ \tt{external}~~ scalar...~~ array...$
\end{tabbing}
\begin{tabbing}
MMMM        \= MM \= MMMMM \= MM \= \kill
$sbind$     \> \tt{::=} \> $\tt{fold}$  \> $fun$ \> $array$ \\
\\[2ex]

$abind$     \> \tt{::=} \> $\tt{map}_n$ \> $fun$ \> $array^n$ \\
            \> $~~|$    \> $\tt{filter}$\> $fun$ \> $array$ \\
            \> $~~|$    \> $\tt{gather}$\>       \> $array~~ array$ \\
\end{tabbing}
\begin{tabbing}
MMMM        \= MM \= \kill
$function$  \> \tt{::=} \> $\lambda scalar...~~array...~~ \to$      \\
            \>          \> $\tt{let}~~bind...$                  \\
            \>          \> $\tt{in}~~(scalar...,~~array...)$    \\
\end{tabbing}
\caption{Combinator normal form}
\label{f:CombinatorNormalForm}
\end{figure}





%!TEX root = ../Main.tex
\section{Size inference}

Knowing the relative sizes of loops is important for fusion.
While it \emph{is} technically possible to fuse loops of different sizes, it requires extra complexity to find the maximum of the loop sizes, and additional branches are required to only execute the smaller loop fewer times. For simplicity, we do away with this added complexity and only support fusion of equal sized loops. \emph{Size inference} is performed on the combinators to infer as much information as possible about the sizes of the resulting loops.

It is important to emphasise that we are only interested in relative sizes of arrays; which arrays are equal, and which may be smaller.
The $@map@_n$ combinators require all input arrays to be the same size, and their output is the same.
Since @fold@s do not produce arrays, they have no constraints.
@gather@ takes two arrays; the data and the indices. Its output size is the size of the indices.
The size of a @filter@'s output is most interesting; the exact size is not known until after it has been executed, only that it is less than or equal to the input size.
Similarly, the size of @external@ outputs is not known at all, and thus cannot be constrained.

Size inference has been explored before in the context of fusion by Chatterjee~\cite{chatterjee1991size}, but the only constraints they support are equality.
Their formulation has no filtering operations, and all array operations in their case generate the same size as at least one of the operation's inputs.
A more in-depth analysis has also been explored by Jay~\cite{jay1996shape} in the form of shape inference, which is built from primitives such as @cons@ and @nil@.
Since we have fewer primitives and simpler goals, our size inference does not need to be as complicated as Jay's shape inference.


Our formulation of size inference is very similar to type inference, a la \CITE.
First, constraints are generated for the bindings, such as equality among two rates, the conjunction of two constraints, and existentials.
If these constraints can be solved, the equality constraints are used to group the rates into equivalence classes.
Otherwise if the constraints are unable to be solved, the program cannot be assured to require no runtime size checks and will not be fused by our system.
After the constraints are generated and solved, each combinator is given an iteration size.
Two loops with the same iteration size can be fused together.

\newcommand{\constr}[1]{\llbracket #1 \rrbracket}
\newcommand{\hole}[0]{[]}
\newcommand{\fillhole}[2]{#1\left[#2\right]}

\subsection{Constraint generation}
Rates may either be some rate variable, $k_n$ where $n$ is the name of an array, or they may be a cross product of two rates.

\begin{tabbing}
MMMMM       \= MM \= MMMMMMMMMMM \= \kill
$rate$      \> @::=@ \> $k_n$               \> (rate variable)\\
            \> $~|$  \> $rate \times rate$  \> (cross product) \\
%            \> $~|$  \> $Filtered_n$        \> (output of @filter@s) \\
%            \> $~|$  \> $External_n$        \> (output of @external@s) \\
\end{tabbing}

The most important constraint is equality on two rates.
Context holes are used to simplify later definitions, and the notation $\fillhole{c}{d}$ is used to denote replacing all holes in $c$ with $d$.
Constraints may be joined together with $c \wedge d$, which requires both $c$ and $d$ to be satisfiable.
Forall quantifiers may be given an upper bound, and mean the constraint must be satisfiable for every rate.
Existential quantifiers simply state that there must exist a rate that satisfies the constraint.


\begin{tabbing}
MMMMM       \= MM \= MMMMMMMMMMM \= \kill
$C$          \> @::=@ \> $true$                                 \> (trivially true) \\
             \> $~|$  \> $\hole$                                \> (constraint hole) \\
             \> $~|$  \> $rate = rate$                          \> (equality constraint) \\
             \> $~|$  \> $C \wedge C$                           \> (conjunction) \\
             \> $~|$  \> $\forall k_n.\ C$                      \> (forall quantification) \\
             \> $~|$  \> $\exists k_n.\ C$                      \> (existential quantification)\\
%             \> $~|$  \> $\exists k_n \le rate.\ C$             \> (exists with upper bound)    \\
\end{tabbing}


Constraints are generated for a list of bindings.
As folds produce no constraints, they simply return a hole to be filled later.
Maps require their inputs and output to all have the same rate, so if there exists some rate that is equal to all its inputs, then that is the rate of the output.
As the output rate of a filter is unknown and cannot be constrained except for the upper bound, it is a forall: for all possible rates, the subconstraint must be satisfiable.
Similarly, external must work for any possible output, so the subconstraint must be satisfiable for all rates.

The final clauses show how the constraint holes are filled by constraints of subsequent bindings.
If there no bindings, the constraint is trivially true.

$$\begin{array}{lcrcl}
&&\constr{\_} & :: & binds \rightarrow C \\
\\
% MMMMMMMMMMMMMMMMMM \= MM \= \kill
% $\constr{\lambda scalars~arrays \to @let @binds@ in @x}$ \> $=$ \> $\forall_{a \in arrays} k_a. \constr{binds}$ \\

\constr{o &=& @fold @f@ @n}       &  = &  \hole \\
\constr{o &=& @map@_n~f~ns}       &  = &  \exists k_o.\ \bigwedge_{n \in ns}\{k_o = k_n\} \wedge \hole \\
\constr{o &=& @filter@~f~n}       &  = &  \forall k_o.\ \hole \\
\constr{o &=& @gather@~i~d}       &  = &  \exists k_o.\ k_o = k_i \wedge \hole \\
\constr{o &=& @cross@~a~b}        &  = &  \exists k_o.\ k_o = k_a \times k_b \wedge \hole \\
\constr{outs &=& @external@~ins}  &  = &  \forall_{o \in outs} k_o.\ \hole \\
\\
&&\constr{b;~bs}  &  = &  \fillhole{\constr{b}}{\constr{bs}}       \\
&&\constr{nil}    &  = &  true                          \\
\end{array}$$

\subsubsection{Examples}

As an example of constraint generation, here is the @normalize2@ program from earlier.
\begin{tabbing}
@MMMMMMMMMMMMMMMMMMMMMMMMMMMMMMMM@  \= \kill
@normalize2 us@                     \> $\exists k_{us}.$      \\
@ = let sum1 = fold   (+) 0 us@     \>                      \\
@       gts  = filter (>0)  us@     \> $\forall k_{gts}.$ \\
@       sum2 = fold   (+) 0 gts@    \> \\
@       nor1 = map  (/sum1) us@     \> $\exists k_{nor1}.\ k_{nor1} = k_{us} \wedge$ \\
@       nor2 = map  (/sum2) us@     \> $\exists k_{nor2}.\ k_{nor2} = k_{us} \wedge$ \\
@   in (nor1, nor2)@                \> $true$ \\
\end{tabbing}
This constraint is satisfiable, with the equivalence classes being:
\newcommand{\eqclasses}[1]{
    \begin{tabbing}
        MM \= M \= \kill
        #1
    \end{tabbing}}
\newcommand{\eqclass}[2]{$#1$ \> $\in$ \> $\{#2\}$ \\}
\eqclasses{
    \eqclass{k_{us}}{k_{us}, k_{nor1}, k_{nor2}}
    \eqclass{k_{gts}}{k_{gts}}
}

The next example involves two filters using the same predicate.
Despite using the same predicate and input data, we produce different output rates for each filter.
\begin{tabbing}
@MMMMMMMMMMMMMMMMMMMMMMMMMMMMMMMM@  \= \kill
@diff xs@                           \> $\exists k_{xs}.$ \\
@ = let ys1 = filter p xs@          \> $\forall k_{ys1}.$       \\
@       ys2 = filter p xs@          \> $\forall k_{ys2}.$       \\
@   in (ys1, ys2)@                  \> $true$                   \\
\end{tabbing}

This constraint is satisfiable, with the equivalence classes being:
\eqclasses{
    \eqclass{k_{xs}}    {k_{xs}}
    \eqclass{k_{ys1}}   {k_{ys1}}
    \eqclass{k_{ys2}}   {k_{ys2}}
}

This example is disallowed, as it would require a runtime check on the size of the @flt@ array.
\begin{tabbing}
@MMMMMMMMMMMMMMMMMMMMMMMMMMMMMMMM@  \= \kill
@bad1 vs@                           \> $\exists k_{vs}.$ \\
@ = let flt   = filter p vs@        \> $\forall k_{flt}.$ \\
@       wrong = map2   f flt vs@    \> $\exists k_{wrong}.\ k_{wrong} = k_{flt}$ \\
                                    \> $\wedge k_{wrong} = k_{vs} \wedge$ \\
@   in  wrong@                      \> $true$
\end{tabbing}
This constraint is unsatisfiable, as by transitivity it can be simplified to $\exists k_{vs}.\ \forall k_{flt}.\ k_{vs} = k_{flt}$.
Obviously, there exists no sole size such that all sizes are equal to it.
As a result, no equivalence classes are generated for this example, and no fusion is performed.


Similarly, this example:
\begin{tabbing}
@MMMMMMMMMMMMMMMMMMMMMMMMMMMMMMMM@  \= \kill
@bad2 vs@                           \> $\exists k_{vs}.$ \\
@ = let flt  = filter p  vs@        \> $\forall k_{flt}.$ \\
@       flt2 = filter p' vs@        \> $\forall k_{flt2}.$ \\
@       mix  = map2   f  flt flt2@  \> $\exists k_{mix}.\ k_{mix} = k_{flt}$ \\
                                    \> $\wedge k_{mix} = k_{flt2} \wedge$ \\
@   in  mix@                        \> $true$                               \\
\end{tabbing}
Again, transitivity here requires that $Filtered_{flt} = Filtered_{flt2}$, but since these are both concrete constructors, we cannot have such a constraint.
Basically, we are constraining two different filtered outputs to be the same size.
Operationally, implementing this would require to buffer both @flt@ and @flt2@, before checking the two lengths.



\subsection{Iteration size}
After the constraints are solved, each combinator is assigned a $rate$ as an iteration size -- the size of the loop required to generate the output.
It is important to note that for filters, the iteration size is different to the output size.
The iteration sizes, $tau_n$, are used to check whether two loops may be fused.
Any two iteration sizes in the same equivalence class are the same size, and so are fusible.
External computations are treated separately, as they cannot be fused with any other nodes.

\begin{tabbing}
MMMMM       \= MM \= MMMMMMMMMMM \= \kill
$\tau$       \> @::=@ \> $rate$                                  \> (loop size of some rate) \\
             \> $~|$  \> @external@                              \> (external and unfusible) \\
\end{tabbing}

Once the constraints are solved, known to be satisfiable, and sorted into equivalence classes, each combinator is assigned a rate.
Note that for a filter, the size of the output array $k_o$ is some existential that is less than or equal to $k_n$, but the actual loop size of the \emph{combinator} is equal to $k_n$.
This is because, in order to produce the filtered output, all elements of the input $n$ must be considered.


\begin{tabbing}
MM \= MM \= MMMMMMMMM \= MMMM \= MM \= \kill
$\tau$  \>$::$\> $binds \rightarrow name \rightarrow \tau$ \\
\\
$\tau_{bs,o}$    
            \> $|$ \> $o = @fold@~f~n$      \> $\in bs$ \> $=$ \> $k_n$ \\
            \> $|$ \> $o = @map@_n~f~ns$    \> $\in bs$ \> $=$ \> $k_o$ \\
            \> $|$ \> $o = @filter@~f~n$    \> $\in bs$ \> $=$ \> $k_n$ \\
            \> $|$ \> $o = @gather@~i~d$    \> $\in bs$ \> $=$ \> $k_i$ \\
            \> $|$ \> $o = @cross@~a~b$     \> $\in bs$ \> $=$ \> $k_a \times k_b$ \\
            \> $|$ \> $o = @external@~ins$  \> $\in bs$ \> $=$ \> $@external@$ \\
\end{tabbing}

Solving the generated equality constraints allows each rate to be grouped into equivalence classes.
All combinators whose constraints are in the same equivalence class have the same iteration size, and may be fused together.
These constraints are also used to rule out any programs that require runtime checks on array lengths during the program.

\subsection{Runtime checks on array lengths}
The $@map@_n$ combinator requires all its input arrays to be the same size.
We do not, however, want to perform runtime checks on the size of arrays before every combinator.
We have decided to only fuse programs that can be statically determined to require only array length checks at the start of execution. 

\TODO{We assign different rates to both @ys1@ and @ys2@ in this example.}
\begin{code}
 diff xs = let ys1 = filter p xs
               ys2 = filter p xs
           in  (ys1, ys2)
\end{code}


% \ben{This point is only relavant in the context of our specific implementation, which is no the focus of this paper. }
% Any other programs are simply left alone by our fusion system, potentially falling back to stream fusion.


We use the constraints to rule out any programs that require runtime checks on array lengths during the program. Firstly, filter and external constraints are only allowed to be mentioned after their binding site, as it is only at that stage that the lengths are actually known.

\ben{Refactor presentation to use the actual $\exists k. $ quantifier, as in ``The essence of ML type inference'', from ATAPL}

Secondly, it is not possible to constrain one constructor to equal another: $Filtered_n = External_n$ is invalid. Finally, any initial equality constraints, such as two input arrays being the same length, or one input array being the length of the cross product of two others, are dealt with by inserting runtime checks only at the start of the function.


The constraints $Filtered_n$, $External_n$ are treated as \emph{existential} constructors, bound at $n$. This means that such constraints are not able to be referred to before their bindings.
Intuitively, we know that the output of a @filter@ or an @external@ has \emph{some} length, but we don't know what its value is until after they have executed. Thus, constraining the input of something to be some exact length, before we know what the length is, is impossible.

Similarly, these constructors cannot be constrained to equal any other constructor. 
For example, $Filtered_n = External_n$ is invalid, as is $a \times b = Filtered_n$.
This corresponds to each equivalence class having \emph{at most} one distinct constructor associated with it. Any other elements of the equivalence class must be either the same constructor, or simply rate variables.

\subsection{Generators}

\begin{tabbing}
MMM       \= MM \= MMMMMM \= MM \= MMM \= MM \= \kill
$gen$   \> @::@  \> $\{constraint\}$  \> $\to$ \> $array$ \> $\to$ \> $\{array\}$ \\
MMMMMMM                 \= M  \= MMMMMMMM \= MM \= \kill
$gen(constrs, a)$ \> $|$ \> $a \le_c b \in constrs$ \> $=$ \> $\{b\}$                        \\
$gen(constrs, a)$ \> $|$ \> $a =_c b \in constrs$   \> $=$ \> $\bigcup gen(constrs, b)$                        \\
$gen(constrs, a)$ \> $|$ \> $otherwise$             \> $=$ \> $\emptyset$                        \\
\end{tabbing}

%\begin{lemma}
\textbf{Lemma: unique generation of filters}
For some bindings $bs$ and associated constraints $cs$, then
\[
valid(cs) \implies \forall x. (\exists y \in bs. gen(cs, x) = \{y\}) \vee gen(cs, x) = \emptyset
\]
%\end{lemma}


%!TEX root = ../Main.tex
\section{Integer linear program formulation}
\label{s:ILP}

\begin{figure}[ht!]
\begin{center}
\includegraphics[scale=0.5]{figures/ex2-normalizeInc.pdf}
\end{center}
\caption{Possible clusterings for \texttt{normalizeInc}}
\end{figure}

\TODO{More justification of why we want to use a heavyweight technology like ILP to do the clustering. Use normalizeInc or one like it to argue that plenty of real programs will admit multiple valid clusterings, so we really want to do this via constraints and an objective function. Compare with prior work on typed fusion that only tries to minimise the number of clusters.} \TODO{Also give a quick rundown of what ILP actually is.}
\begin{code}
 normalizeInc :: Array Int -> Array Int
 normalizeInc xs
  = let incs = map  (+1)    us      (A1) (B2)
        sum  = fold (+) 0   us      (A1) (B1)
        norm = map  (/ sum) incs    (A2) (B2)
    in  norm
\end{code}

For @normalizeInc@, the first approach is to fuse the computation of @incs@ and @sum@ into a single loop (indicated by @A1@), then use a second loop to compute @norm@ (indicated by @A2@). The other approach is to compute @sum@ first (@B1@), and fuse the computation of @incs@ and @norm@ into the second loop (@B2@). For this specific function the second approach seems preferable because it avoids creating the intermediate array @incs@, but for large programs with many operators the optimal way to cluster operators into loops can be entirely non-obvious. 

While the different combinators are all implemented differently, certain aspects are shared between all; they operate on input arrays, use scalar variables in their worker functions, and produce output.
To simplify creation of the integer linear program formulation, we first convert a function to a dependency graph.
The dependency graph has a node for each combinator, and edges between two combinators when one uses another's output.
After the graph is created, it is converted to an integer linear program with an objective of the least memory traffic and loops, solved, and the solution converted to a clustering.

When creating the graph, it is important to note that not all loops can be fused together; firstly, loops of different size cannot be fused. Secondly, two loops cannot be fused together if an iteration in one loop relies on the output of a later iteration in another loop. For example, a @map@ may not be fused with a @fold@ if the @map@'s worker function uses the result of the @fold@.
This fusion restriction is encoded as a \emph{fusion-preventing} edge between two combinators. Other edges are \emph{fusible}, and may be fused together. If they are not fused together for whatever reason, the \emph{from} combinator must be scheduled before the \emph{to} combinator.

The dependency graph is translated to an integer linear program. The integer linear program has some integral, boolean and real variables, an objective function to minimise, and a set of constraints that the variables must conform to. Finding a variable assignment that satisfies the constraints and is the minimal objective function is NP-hard, but existing ILP solvers tend to be sufficient for realistic problem sizes. For larger problems, we can find an approximate answer, within say $10\%$ of the optimal answer, which still gives the exact answer for small problems.


% -----------------------------------------------------------------------------
\subsection{Function $\to$ graph}
\ben{This is a different sort of graph than before. This is a dependency graph where the nodes are intermediate variables, verses the previous ``program graph'' where the nodes can also be operators. We need to contrast these properly and add a diagram of a dependency graph.}
\TODO{Put this somewhere: Fusion-preventing dependencies are akin to barrier synchronisations in the Bulk Synchronous model of parallel computation. }


Converting a function in \emph{combinator normal form} to a graph --- in fact, a DAG --- is quite simple.
Each binding becomes a node in the graph. When a binding references other arrays or scalars, there is an edge between those two nodes. Edges may be either \emph{fusible} or \emph{fusion-preventing}. Fusion-preventing edges mean that the entire input node must completely finish its execution before the output node can start executing. For example, @fold@s consume their entire input array before producing a single result, so any references to folds must be fusion-preventing. Conversely, @map@s produce output data for every input element, so may be fused.

The @gather@ operation is interesting; it takes an indices array and a data array, and for each element in indices returns that element in data.
Thus, @gather@ requires random-access in its data array, and is not fusible, but consumes indices linearly and may fuse.

\begin{tabbing}
MMMM        \= MM   \= \kill
$nodes$     \> @::@ \> $program \to V$          \\
$edges$     \> @::@ \> $program \to E$          \\
$edge$      \> @::@ \> $\{bind\} \times bind \to E$\\
\\
MMMMM       \= \kill
$nodes(bs)$ \> $= \{(name(b), \tau_b) | b \in bs\}$       \\
\end{tabbing}
Each node may simply be the name of its output binding (or bindings, in the case of @external@); as we require names to only be bound once, this is assured to be unique.
Creating edges between these nodes is simply when one binding references an earlier one. The only complication is designating edges as \emph{fusible} or \emph{fusion-preventing}.

\begin{tabbing}
MMMMM       \= \kill
$edges(bs)$ \> $= \bigcup_{b \in bs}edge(b)$    \\
\\
MMMM             \= M \= \kill
$edge(bs, out = @fold@~f~in)$ \\
                              \> $=$    \> $\{inedge(bs,out,s) | s \in f\}$ \\
                              \> $\cup$ \> $\{inedge(bs, out, in) \}$       \\
$edge(bs, out = @map@~f~in)$  \\
                              \> $=$    \> $\{inedge(bs,out,s) | s \in f\}$ \\
                              \> $\cup$ \> $\{inedge(bs, out, in) \}$       \\
$edge(bs, out = @filter@~f~in)$  \\
                              \> $=$    \> $\{inedge(bs,out,s) | s \in f\}$ \\
                              \> $\cup$ \> $\{inedge(bs, out, in) \}$       \\
$edge(bs, out = @gather@~data~indices)$  \\
                              \> $=$    \> $\{(out,data, fusion~preventing) \}$ \\
                              \> $\cup$ \> $\{inedge(bs, out, indices) \}$       \\
$edge(bs, out = @cross@~a~b)$            \\
                              \> $=$    \> $\{inedge(bs, out, a) \}$       \\
                              \> $\cup$ \> $\{(out, b, fusion~preventing) \}$ \\
$edge(bs, outs = @external@~ins)$  \\
                              \> $=$    \> $\{(outs,i, fusion~preventing) | i \in ins \}$ \\
\\
$inedge(bs,to,from)$ \\
                     \> $|$ \> $(from = @fold@~f~s) \in bs$     \\
                     \> $=$ \> $(to, from, fusion~preventing)$  \\
                     \> $|$ \> $(\ldots,from,\ldots = @external@ \ldots) \in bs$     \\
                     \> $=$ \> $(to, from, fusion~preventing)$  \\
                     \> $|$ \> $otherwise$                      \\
                     \> $=$ \> $(to, from, fusible)$            \\
\end{tabbing}


% -----------------------------------------------------------------------------
\subsection{Graph $\to$ ILP}
With the dependency graph now constructed, it must be converted to an integer linear program. The integer linear program will contain some variables, an objective function, and constraints that the variables must conform to. An external solver will find and return a variable assignment that minimises the objective function. With the variable assignment, the graph's nodes are partitioned into a clustering, and then flow fusion\cite{lippmeier2013flow} is used to extract imperative loops for each cluster.

The variables in the ILP formulation can be split into three groups:

\begin{tabbing}
M   \= MM \= MMMMMMM \= MM \= \kill
$x$   \> @:@  \> $node \times node$ \> $\to$ \> $\mathbb{B}$
\end{tabbing}
The first and most important group is created for every pair of nodes, and has a boolean value indicating whether the two nodes can be fused. If $x_{ij} = 0$, then $i$ and $j$ are fused together. While it may seem counterintuitive for 0 to mean fused and 1 unfused, it means the objective function can simply minimise a weighted $x_{ij}$, and also makes later precedence constraints simpler. The values of these variables are used to construct the partitioning at the end, such that $\forall i,j.\ x_{ij} = 0 \iff cluster(i) = cluster(j)$. \ben{When we're talking about the dependency graph, when we say ``nodes can be fused'', it means that the operators that produce these values can be fused -- or that the values are produced by the same cluster of operators.}

\begin{tabbing}
M   \= MM \= MMMMMMM \= MM \= \kill
$\pi$ \> @:@  \> $node$             \> $\to$ \> $\mathbb{R}$
\end{tabbing}
The second group of variables is used to ensure that the clustering is acyclic. For each node $i$, we associate a real $\pi_i$ such that every node $j$ after $i$ we have $\pi_j > \pi_i$. This has the effect of requiring an acyclic clustering, as if one node is after another its $\pi$ must be strictly greater than its predecessor, and the predecessor must be strictly less than the successor. If two nodes are to be fused together, their $\pi$ values must be the same. \ben{I don't understand the ``acyclic'' part. If we don't add these constraints, then how might the clustering turn out to be ``cyclic''?}

\begin{tabbing}
M   \= MM \= MMMMMMM \= MM \= \kill
$c$   \> @:@  \> $node$             \> $\to$ \> $\mathbb{B}$
\end{tabbing}
The third, and final, group of variables are purely for minimisation, indicating whether a node's array is \emph{fully contracted}.
They are not required for correctness of the clustering, but are used in the objective function.
A node's array will be fully contracted --- that is, it will be replaced by a scalar --- if all nodes that use this array are fused together: $c_i = 0 \iff \forall (i,j) \in E. x_{ij} = 0$. \ben{Using $\forall$ like this implies that the quantified $i$ isn't the same as the one attached to $c_i$. Write this with a list comprehension instead.}


% -----------------------------------------------------------------------------
\subsubsection{Unoptimised version}
Before showing the optimised version with certain constraints removed (\S\ref{s:OptimisedConstraints}), this simpler, unoptimised version is shown. The only difference is that fewer constraints and variables are required in the optimised version, but both versions are equivalent. \ben{What does ``equivalent'' mean when the constraints are different?} \ben{Say what ``optimised'' means in this case.} \ben{Give some hints about why the unoptimised version has more constraints. }


\paragraph{Acyclic and precedence-preserving}

\begin{tabbing}
MMMMM   \= MMM \= M \= MMM \= M \= MMM \= \kill
Minimise   \> \ldots \\
Subject to \> \ldots \\
           \>    $x_{ij}$ \> $\le$ \> $\pi_j - \pi_i$ \> $\le$ \> $N x_{ij}$ \\
           \>             (an edge from $i$ to $j$)            \\
           \> $-N x_{ij}$ \> $\le$ \> $\pi_j - \pi_i$ \> $\le$ \> $N x_{ij}$ \\
           \>             (no edge from $i$ to $j$)            \\
           \> \ldots
\end{tabbing}
As per Megiddo\cite{megiddo1998optimal}, we look at every pair of nodes $i$ and $j$.

Whether there is an edge between $i$ and $j$ or not, if the two are fused together, then $x_{ij} = 0$. If $x_{ij} = 0$, then both cases may be simplified to $0 \le \pi_j - \pi_i \le 0$, which is equivalent to $\pi_i = \pi_j$. \ben{Highlight that ``fused together'' means ``in the same cluster'', and that operators can be fused even if there are no direct edges between them.}

Otherwise, if the two nodes are not fused together, then $x_{ij} = 1$. If there is an edge between $i$ and $j$, the constraint simplifies to $1 \le \pi_j - \pi_i \le N$, where $N$ is the number of nodes in the graph. This constraint means that the difference between the two $\pi$s must be at least 1, and less than $N$. Since there are $N$ nodes, the maximum difference between any two $\pi$s would be at most $N$, so the upper bound of $N$ is large enough to be safely ignored. This means the constraint can roughly be translated to $\pi_i < \pi_j$, which enforces the acyclic constraint.

If there is no edge between $i$ and $j$ and the two are not fused together, then $x_{ij} = 1$ and the constraint simplifies to $-N \le \pi_j - \pi_i \le N$, which effectively puts no constraint on the $\pi$ values.
Note, however, that other edges may still constrain $i$ or $j$.


\paragraph{Fusion-preventing edges}
\begin{tabbing}
MMMMM   \= MMM \= M \= MMM \= M \= MMM \= \kill
Minimise   \> \ldots \\
Subject to \> \ldots \\
           \> $x_{ij}$    \> $=$   \> $1$             \>       \>            \\
           \> (fusion-preventing edges from $i$ to $j$)      \\
           \> \ldots
\end{tabbing}
As per Megiddo\cite{megiddo1998optimal}, for every fusion-preventing edge, we add the constraint $x_{ij} = 1$, so that no fusion can occur. This, together with the precedence constraints above, has the effect of enforcing $\pi_i < \pi_j$.


\paragraph{Fusion between different iteration sizes}

\begin{tabbing}
MMMMM   \= MMM \= M \= MMM \= M \= MMM \= \kill
Minimise   \> \ldots \\
Subject to \> \ldots \\
           \> $x_{ij}$    \> $=$   \> $1$             \>       \>            \\
           \> (if $\tau(i) \not= \tau(j) \wedge parents(i,j) = \emptyset$)  \\
           \\
           \> $x_{i'j}$   \> $\le$ \> $x_{ij}$        \>       \>            \\
           \> $x_{ij'}$   \> $\le$ \> $x_{ij}$        \>       \>            \\
           \> $x_{i'j'}$   \> $\le$ \> $x_{ij}$        \>       \>            \\
           \> (if $\tau(i) \not= \tau(j) \wedge parents(i,j) = \{i',j'\}$) \\
           \> \ldots
\end{tabbing}
If two nodes have different iteration sizes, they may not necessarily be fused.
However, if they have parent transducers of the same size, they may be fused if both their parents are fused together, and each is fused with its respective parent.

The main difference to existing integer linear programming solutions is that, while nodes with different loop sizes cannot generally be fused together, we do support some fusion of filtered sizes. If two nodes have different sizes or rates, but are both sub-rates of another rate, they \emph{can} be fused together if they are both fused with their generators. \ben{What is a subrate? How do we determine the subrates of each rate?} \ben{Highlight why we want to allow this sort of fusion, and why it is specific to data flow fusion.} \ben{Explain how we are using the idea of transducers}.

As a degenerate example, consider fusing an operation on filtered data with its generating filter:
\begin{code}
sum1 = fold (+) 0  us
gts  = filter (>0) us
sum2 = fold (+) 0  gts
\end{code}
Here, $parents(gts,sum2) = \{gts, gts\}$.
This generates spurious, but still valid, constraints that $gts$ must be fused with $gts$ and $gts$ must be fused with $sum2$, in order for $gts$ and $sum2$ to be fused together.
While these constraints are unnecessary in this case, they are harmless.

In the same example, $sum1$ and $sum2$ also have different iteration sizes.
We have that $parents(sum1,sum2) = \{sum1, gts\}$.
This means that $sum1$ and $sum2$ may only be fused if $sum1$ is fused with $sum1$ (trivial), $sum2$ is fused with $gts$, and $sum1$ is fused with $gts$.
While it may seem like these constraints should be implied by transitivity of clustering, all clustering variables are used for the objective function.


\paragraph{Array contraction}
\begin{tabbing}
MMMMM   \= MMM \= M \= MMM \= M \= MMM \= \kill
Minimise   \> \ldots \\
Subject to \> \ldots \\
           \> $x_{ij}$    \> $\le$ \> $c_i$           \>       \>            \\
           \> (for all edges from i)            \\
           \> \ldots \\
\end{tabbing}
As per Darte's work on optimising for array contraction\cite{darte2002contraction}, we define a variable $c_i$ for each array.
An array is fully contracted if all its output edges are fused with it. Thus, $c_i=0$ only if $\forall j | (i,j) \in E. x_{ij} = 0$. By minimising $c_i$ in the objective function, we favour solutions that reduce the number of required intermediate arrays. \ben{Say what array contraction is, and why we want this constraint.}


\subsubsection{The objective function}
\label{s:ObjectiveFunction}

To find the objective function, note that fusing loops can have three main benefits:
\begin{itemize}
\item
reducing memory traffic, such as multiple loops reading from the same array;
\item
removing intermediate arrays, thus reducing the amount of memory required;
\item
and reducing loop overhead, such as when two loops operate on different arrays of the same size.
\end{itemize}
However, these benefits cannot be considered in isolation; for example, fusing two loops to reduce loop overhead may remove potential fusion opportunities that reduce memory traffic.
When operating on large arrays that do not fit in cache, memory traffic dominates execution time.
An excessive number of intermediate arrays can also cause issues if all are live in memory at once, potentially leading to \emph{thrashing}.
The benefits of removing loop overhead are least of all; it should be performed if possible, but must never remove opportunities for other fusion. \TODO{Refer to example code that demonstrates these different opportunities.}

This total ordering can be encoded into an ILP objective function as weights.
If the program graph contains $N$ combinators, then there are at most $N$ opportunities for fusion.
The encoding of loop overhead is weight $1$, removing intermediate arrays is weight $N$, and reducing memory traffic is weight $N^2$.
This ensures that no amount of loop overhead reduction can outweigh the benefit of removing an intermediate array,
and likewise no number of removed intermediate arrays can outweigh a reduction in memory traffic.
Although, it is worth noting that reducing memory traffic does \emph{tend} to remove intermediate arrays, and vice versa.

\begin{tabbing}
MMMMM   \= MMMM \= M \= \kill
Minimise   \>     \> $\Sigma_{(i,j) \in E} W_{ij} x_{ij}$   \\
           \> \> \> (memory traffic and loop overhead)         \\
           \> $+$ \> $\Sigma_{i \in V} N c_i$  \\
           \> \> \> (removing intermediate arrays)         \\
Subject to \> \ldots                                \\
Where      \>                                       \\
           \> $W_{ij} = N^2$ \> $~|$ \> $(i,j) \in E $         \\
           \> \> \> (fusing $i$ and $j$ will reduce memory traffic)         \\
           \> $W_{ij} = N^2$ \> $~|$ \> $\exists k. (k,i) \in E \wedge (k,j) \in E $     \\
           \> \> \> ($i$ and $j$ share an input array)                                         \\
           \> $W_{ij} = 1$   \> $~|$ \> $@otherwise@$                                                  \\
           \> \> \> (the only benefit is loop overhead)                                        \\
           \\
           \> $N = |V|$
\end{tabbing}



\subsection{A note on transitivity}
It may seem that we could generate far fewer constraints and rely on transitivity of clustering equality $x_{ij}$.
This means that for each pair $i,j$ we wouldn't generate an $x_{ij}$ and thus not generate those constraints, unless $i$ or $j$ is a @filter@, or arc between $i$ and $j$.
This would help, but also means we can't \emph{minimise} on these removed $x_{ij}$, so the solution won't count the (rather smaller) benefits of fusing two non-connected nodes.

\subsubsection{Fusion-preventing path optimisation}
\label{s:OptimisedConstraints}
Alternatively:
we can say, only generate constraints if there is no \emph{fusion preventing path} between $i$ and $j$.
Here, we generate fewer constraints than originally, but still more than above.
We also must preprocess the graph to find such blocking paths.
But we still get the minimisation to count non-connected nodes.

\begin{tabbing}
MMMMMM      \= MM   \=  MMMMMMMMMMMM    \=  \kill
$split$     \> @::@ \> $V \times E \to \{ \{name\} \}$      \\
MMMM        \= M    \= \kill
$split(vs,es)$ \\
    \> $=$  \> $\{clusterable(v,vs,es) | v \in vs\}$      \\
$clusterable(v,vs,es)$  \\
    \>$=$       \>$\{v' | v' \in vs$                                \\
    \>$\wedge$  \>$\forall p. p = path(v,v') \vee p = path(v',v)$   \\
    \>$\wedge$  \>$fusion~preventing \not\in p\}$\\
\end{tabbing}


We can simplify the version above and remove many constraints and variables by noting that if there is a fusion-preventing edge between $i$ and $j$, $x_{ij} = 1$, so the precedence constraint can be simplified to just $\pi_i < \pi_j$, and the $x_{ij}$ variable removed.
Likewise with clusters of different, incompatible types.
The contraction variable $c_i$ can be removed, if $i$ has any fusion-preventing output edges: any such edges use the array and cannot be fused, thus there is no possibility of the array being contracted away.

\begin{tabbing}
MMMMM   \= MMM \= M \= MMM \= M \= MMM \= \kill
Minimise   \> \ldots \\
Subject to \> $-N x_{ij}$ \> $\le$ \> $\pi_j - \pi_i$ \> $\le$ \> $N x_{ij}$ \\
           \>             (no edge from i to j, but there is a fusion benefit)            \\
           \>    $x_{ij}$ \> $\le$ \> $\pi_j - \pi_i$ \> $\le$ \> $N x_{ij}$ \\
           \>             (an edge from i to j)            \\
           \>             \>       \> $\pi_i < \pi_j$ \>       \>            \\
           \>             (a \emph{fusion-preventing} edge from i to j)            \\
\\
           \> $x_{ij}$    \> $\le$ \> $c_i$           \>       \>            \\
           \> (for all edges from i with no fusion-preventing outputs)      \\
\\
           \> $x_{i'j}$   \> $\le$ \> $x_{ij}$        \>       \>            \\
           \> (if $\tau_i \not= \tau_j \wedge gen_i=\{i'\}$) \\
           \> $x_{ij'}$   \> $\le$ \> $x_{ij}$        \>       \>            \\
           \> (if $\tau_i \not= \tau_j \wedge gen_j=\{j'\}$) \\
\end{tabbing}


\subsection{Proof}
To prove correctness of our linear program formulation, we need to prove two different things.
Firstly, the formulation's constraints must always be satisfiable; that is, there must exist a variable assignment that satisfies all constraints.
This is rather simple to show, but guarantees that the linear program will always give an answer.
The next thing to show is that any produced clustering is legal: if a variable assignment satisfies the constraints, then it is a valid and legal clustering.
This means that, not only do we get \emph{an} answer, we also get the \emph{right} answer.

%!TEX root = ../Main.tex
\subsubsection{Satisfiability}
For any program $p$, there exists a trivial clustering with no fusion at all.
We can use this as the variable assignment of $ilp(p)$.
For each pair of nodes $m,n \in p$, $x_{mn} = 1$ --- no fusion is possible.
For the $\pi$ variables, we must find a topographical ordering of the nodes in $p$, which is simple since we are assured it is a dag.

\TODO{Now, prove that this assignment actually satisfies the constraints.}


\subsubsection{Soundness}
For any program $p$ and variable asignment $v$, if $v$ satisfies the constraints for $ilp(p)$, the clustering denoted by $x_{ij}$ in $v$ is legal.

For a clustering to be legal, it must satisfy three constraints:
\begin{description}
\item[Acyclic]
after merging nodes of same cluster together, the resulting graph must be a dag
\item[Precedence preserving]
if there is an edge between two nodes $i$ and $j$, and they are not merged together, then we require $\pi_j > \pi_i$
\item[Fusion preventing]
likewise, if there is a fusion-preventing edge between two nodes $i$ and $j$, then we require $\pi_j > \pi_i$, which implies that they are not merged together
\item[Type constraint]
if two nodes $i$ and $j$ are in the same cluster, then $\tau_i = \tau_j$, or if $\tau_i$ is a subtype of $\tau_j$ (or $\tau_j$ is a subtype of $\tau_i$), then the \emph{generator} for $\tau_i$ (or $\tau_j$) must also be in the same cluster as $i$ and $j$.
    \\
    \TODO Actually, let us say $x_{ij} = 0 \implies check_{ij}$

\end{description}
where
\begin{tabbing}
MMMMM      \= M \= MMMMMMM \= MM \= \kill
$check$ \> @::@  \> $array \times array$ \> $\to$ \> $\mathbb{B}$ \\
MMMMM      \= M \= MMMMMM \= MM \= \kill
$check(i, j)$     \> $|$ \> $tau_i = tau_j$ \> $=$ \> $x_{i,j} = 0$                        \\
$check(i, j)$     \> $|$ \> $i' \in gen(i) $ \> $=$ \> $x_{i',j} = 0 \wedge x_{i,i'} = 0 \wedge check(i', j)$                        \\
$check(i, j)$     \> $|$ \> $j' \in gen(j) $ \> $=$ \> $x_{i,j'} = 0 \wedge x_{j,j'} = 0 \wedge check(i, j')$                        \\
$check(i, j)$     \> $|$ \> $tau_i \not= tau_j$\> $=$ \> $\bot$
\end{tabbing}







%!TEX root = ../Main.tex
\section{Benchmarks}
\label{s:Benchmarks}
\ben{Use the larger programs as benchmarks, or as running examples. We only need small, simple programs to demonstrate how the algorithm works.}

\begin{figure*}
$$\begin{array}{c}

\begin{tabular}{lrrrrrrrr}
                &   \multicolumn{2}{c}{Unfused}         & \multicolumn{2}{c}{Stream}
                & \multicolumn{2}{c}{Megiddo} &\multicolumn{2}{c}{\textbf{Ours}} \\
                & Time & Loops   & Time & Loops      & Time & Loops & Time & Loops   \\
\hline
Normalize2      & 1.88s & 5      & 1.64s & 4          & 1.82s & 3  & \textbf{1.59s} & \textbf{2}\\
Closest pair    & 3.83s & 6      & 3.33s & 5          & 2.92s & 3  & \textbf{2.92s} & \textbf{3}\\
QuadTree        & 5.22s & 8      & 5.22s & 8          & 4.72s & 2  & \textbf{4.72s} & \textbf{2}\\
\end{tabular}

\end{array}$$
\caption{Benchmark results}
\label{f:BenchResults}
\end{figure*}

Notes:
\begin{itemize}
% \item
% Normalize2 tested with around 1gb input data. This should be small enough that all intermediates fit in memory, but large enough not to fit in cache.
% \item
% ClosestPoints only tested with 80mb data.
\item
In some cases, the clusterings generated by the different methods were the same. In this case, the results are also the same. 
\item
The stream fusion benchmarks use the clustering that stream fusion \emph{would} use, but hand-fused and written in C.
\item
Programs were run five times with the same input data, and the fastest run was used.
\item
ILP solutions were created by hand-fusing based on the clustering of the implementation. In the future, this will be integrated to use data flow fusion.
\item
Benchmark programs are available at \url{https://github.com/amosr/papers/tree/master/2014betterfusionforfilters/benches}
\end{itemize}

\subsection{Example}
The normalize2 example is already in combinator normal form.
\begin{code}
 normalize2 :: Array Int -> Array Int
 normalize2 xs
  = let sum1 = fold   (+)  0   xs
        gts  = filter (> 0)    xs
        sum2 = fold   (+)  0   gts
        ys1  = map    (/ sum1) xs
        ys2  = map    (/ sum2) xs
    in (ys1, ys2)
\end{code}

First, we must calculate $split$ -- that is, those nodes which have no fusion-preventing path between them.
\begin{code}
split(sum1, gts)    = T
split(sum1, sum2)   = T
split(sum1, ys2)    = T
split(gts,  sum2)   = T
split(gts,  ys1)    = T
split(sum2, ys1)    = T
split(ys1,  ys2)    = T
\end{code}
We only need to create cluster variables for these.

Note that, in the objective function, the weights for $x_{sum1, sum2}$ and $x_{sum2, ys1}$ are both only 1, because they do not share any input arrays.

\begin{tabbing}
MMMMM   \= MMMMMMM \= M \= MMMMMMM \= M \= MMMMMMM \= \kill
Minimise   \> $25 \cdot x_{sum1, gts} + 1 \cdot x_{sum1,sum2} + 25 \cdot x_{sum1, ys2} +$ \\
           \> $25 \cdot x_{gts, sum2} + 25 \cdot x_{gts, ys1} + 1 \cdot x_{sum2, ys1} +$ \\
           \> $ 25 \cdot x_{ys1, ys2} + 5 \cdot c_{gts} + 5 \cdot c_{ys1} + 5 \cdot c_{ys2} $\\
Subject to \\
    \> $-5 \cdot x_{sum1, gts}$  \> $\le$ \> $\pi_{gts} - \pi_{sum1}$  \> $\le$ \> $5 \cdot x_{sum1, gts}$  \\
    \> $-5 \cdot x_{sum1, sum2}$ \> $\le$ \> $\pi_{sum2} - \pi_{sum1}$ \> $\le$ \> $5 \cdot x_{sum1, sum2}$ \\
    \> $-5 \cdot x_{sum1, ys2 }$ \> $\le$ \> $\pi_{ys2 } - \pi_{sum1}$ \> $\le$ \> $5 \cdot x_{sum1, ys2 }$ \\
    \> $-5 \cdot x_{gts,  ys1 }$ \> $\le$ \> $\pi_{ys1 } - \pi_{gts }$ \> $\le$ \> $5 \cdot x_{gts, ys1  }$ \\
    \> $-5 \cdot x_{sum2, ys1 }$ \> $\le$ \> $\pi_{ys1 } - \pi_{sum2}$ \> $\le$ \> $5 \cdot x_{sum2, ys1 }$ \\
    \> $-5 \cdot x_{ys1, ys2  }$ \> $\le$ \> $\pi_{ys2 } - \pi_{ys1 }$ \> $\le$ \> $5 \cdot x_{ys1, ys2  }$ \\
\\
    \> $   x_{gts, sum2 }$ \> $\le$ \> $\pi_{sum2} - \pi_{gts }$ \> $\le$ \> $5 \cdot x_{gts, sum2 }$ \\
\\
    \>                     \>       \> $\pi_{sum1} < \pi_{ys1}$ \\
    \>                     \>       \> $\pi_{sum2} < \pi_{ys2}$ \\
\\
    \> $ x_{gts,sum2} $    \> $\le$ \> $c_{gts}$ \\
\\
    \> $x_{sum1,sum2}$     \> $\le$ \> $x_{sum1, sum2}$ \\
    \> $x_{sum1, gts}$     \> $\le$ \> $x_{sum1, sum2}$ \\
    \> $x_{sum1, gts}$     \> $\le$ \> $x_{sum1, sum2}$ \\
\end{tabbing}
$parents(sum1,sum2)  = \{(sum1, gts)\}$

A minimal solution to this is
\begin{tabbing}
MMMMMMMMMMMMMMMMMMMMMMMMMM \= M \= \kill
$x_{sum1, gts}, x_{sum1, sum2}, x_{gts, sum2}, x_{ys1,  ys2}$
    \> $=$ \> $0$ \\
$x_{sum1, ys2}, x_{gts, ys1 }, x_{sum2, ys1}$
    \> $=$ \> $1$ \\
\\
$\pi_{sum1}, \pi_{gts }, \pi_{sum2}$
    \> $=$ \> $0$ \\
$\pi_{ys1 }, \pi_{ys2 }$
    \> $=$ \> $1$ \\
\\
$c_{gts}, c_{ys1}, c_{ys2}$           
    \> $=$ \> $0$ \\
\end{tabbing}
Looking at just the non-zero variables in the objective function, the value is
$25 \cdot x_{sum1,ys2} + 25 \cdot x_{gts,ys1} + 1 \cdot x_{sum2, ys1} = 51$.
It is worth noting that, for example, the objective could be reduced by fusing $x_{sum1,ys2}$ together.
However, this conflicts with other constraints. Since $x_{sum1, sum2} = 0$, the constraint requires that $\pi_{sum1} = \pi_{sum2}$, and another constraint requires $\pi_{sum2} < \pi_{ys2}$.
These constraints may not all hold, so a clustering that fused sum1 and ys2 together would not allow sum1 and sum2 to be fused together.

We can also calculate the objective function for the clustering used by stream fusion, to contrast and show that the constraints are valid.
To do this, the values of $x_{ij}$ variables are:
\begin{tabbing}
MMMMMMMMMMMMMMMMMMMMMMMMMM \= M \= \kill
$x_{gts, sum2}$
    \> $=$ \> $0$ \\
$x_{sum1, gts}, x_{sum1, sum2}, x_{ys1,  ys2}, x_{sum1, ys2}, x_{gts, ys1 }, x_{sum2, ys1}$
    \> $=$ \> $1$ \\
\end{tabbing}
If we add these extra equality constraints to the integer linear program and solve it, one possible solution is:
\begin{tabbing}
MMMMMMMMMMMMMMMMMMMMMMMMMM \= M \= \kill
$\pi_{sum1}, \pi_{gts }, \pi_{sum2}$
    \> $=$ \> $0$ \\
$\pi_{ys1 }, \pi_{ys2 }$
    \> $=$ \> $1$ \\
\\
$c_{gts}, c_{ys1}, c_{ys2}$           
    \> $=$ \> $0$ \\
\end{tabbing}
It is interesting to note that two nodes having equal $\pi_i$ values does not imply that the two nodes are fused together.
Conversely, however, different $\pi_i$ values do imply that two nodes are not fused together.

The objective function for this solution is
$25 \cdot x_{sum1, gts} + 1 \cdot x_{sum1,sum2} + 25 \cdot x_{sum1, ys2} + 25 \cdot x_{gts, ys1} + 1 \cdot x_{sum2, ys1} + 25 \cdot x_{ys1, ys2} = 102$.


\subsection{QuickHull}
The core of the QuickHull algorithm is very similar to @hull@, below.
Given a line and an array of points, the points are filtered to those above the line,
and the farthest point above the line is also found.

\begin{code}
hull :: (Point,Point) -> Array Point -> Array Point
hull line@(l,r) pts
 = let pts' = filter (above   line) pts
       ma   = fold   (maxFrom line) pts'
       hl   = hull   (l, ma)        pts'
       hr   = hull   (ma, r)        pts'
   in  hl  ++ hr
\end{code}

Stream fusion is unable to fuse the @pts'@ and @ma@ together, as @pts'@ is used multiple times and cannot be inlined into @ma@.
Similarly, Megiddo~\cite{megiddo1998optimal} is unable to fuse them together, as their iteration sizes are different.
Our formulation is able to fuse these together.

If the @fold@ is changed to operate over @pts@ instead of @pts'@, Megiddo's formulation \emph{is} able to fuse the two.
Interestingly, a programmer unaware of fusion would expect the original version to be faster, because @pts'@ is smaller than @pts@.

\begin{code}
 = let pts' = filter (above   line) pts
       ma   = fold   (maxFrom line) pts
\end{code}



\input{section/06-Related.tex}
\section{Conclusion}

\section*{Acknowledgements}
Many thanks are due to
Robert Clifton-Everest,
Kai Engelhardt,
Bill Kroon,
Frederik Madsen,
Abdallah Saffidine,
Carter Schonwald,
and Jingling Xue
for enlightening discussions relating to this work.

\bibliographystyle{plain}
\bibliography{Main}

\end{document}


